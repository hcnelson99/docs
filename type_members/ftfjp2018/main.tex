\documentclass[sigplan, anonymous, review]{acmart}

\usepackage{booktabs} % For formal tables
\usepackage{mathpartir}
\usepackage{common/common}
\usepackage{listings}
\usepackage{amsmath}
\usepackage{mathpartir}
\usepackage{color,soul}
\usepackage{graphicx}
\usepackage{float}
\usepackage{xcolor,colortbl}
\usepackage{mathtools}
\usepackage{tikz}
\usepackage{xcolor}
\usetikzlibrary{arrows}
\usepackage[]{algorithm2e}



%definitions
        \newtheorem{case}{Case}
        \newtheorem{subcase}{Subcase}

\definecolor{lightYellow}{rgb}{1,0.98,0.92}
\newcolumntype{a}{>{\columncolor{lightYellow}}l}

\newcolumntype{b}{>{\columncolor{lightYellow}}r}


\lstdefinestyle{custom_lang}{
    backgroundcolor=\color{lightYellow},%
  xleftmargin=\parindent,
  showstringspaces=false,
  basicstyle={\footnotesize \ttfamily},
  morekeywords={return, type, def, extends, if, else, class, interface, new, val}
}

\lstdefinestyle{custom_lang2}{
%  backgroundcolor=\color[rgb]{1,1,1},%
  xleftmargin=\parindent,
  showstringspaces=false,
  basicstyle= {\footnotesize \ttfamily},
  keywordstyle={\tt \bfseries},
  morekeywords={return, type, def, extends, if, else, class, interface, new, val}
}

\lstset{%
    basicstyle=\small\ttfamily,%
    numbers=left, numberstyle=\tiny, stepnumber=1, numbersep=5pt,
    frame=tlbr, framesep=0.2cm, framerule=0pt
    }%

\lstset{emph={%  
    val, def, type, new, z%
    },emphstyle={\bfseries \tt}%
}

\SetKwFor{Case}{case}{}{end case}%


% Copyright
%\setcopyright{none}
%\setcopyright{acmcopyright}
%\setcopyright{acmlicensed}
\setcopyright{rightsretained}
%\setcopyright{usgov}
%\setcopyright{usgovmixed}
%\setcopyright{cagov}
%\setcopyright{cagovmixed}


% DOI
\acmDOI{10.475/123_4}

% ISBN	
\acmISBN{123-4567-24-567/08/06}

%Conference
\acmConference[FTfJP'18]{Formal Techniques for Java-like Programs}{July 2018}{Amsterdam, The Netherlands}
\acmYear{1997}
\copyrightyear{2016}

\acmPrice{15.00}

%\acmBadgeL[http://ctuning.org/ae/ppopp2016.html]{ae-logo}
%\acmBadgeR[http://ctuning.org/ae/ppopp2016.html]{ae-logo}


\begin{document}
\title{An even Simpler Soundness Proof for Dependent Object Types}
%\titlenote{Produces the permission block, and
%  copyright information}
%\subtitle{Extended Abstract}
%\subtitlenote{The full version of the author's guide is available as
%  \texttt{acmart.pdf} document}

\author{Julian Mackay}
\affiliation{%
  \institution{Victoria University of Wellington}
  \streetaddress{P.O. Box 1212}
  \city{Dublin}
  \state{Ohio}
  \postcode{43017-6221}
}
\email{trovato@corporation.com}


% The default list of authors is too long for headers.
\renewcommand{\shortauthors}{B. Trovato et al.}


\begin{abstract}
In recent years there has been several important efforts in formalizing sound calculi for Dependent Object Types (DOT). Perhaps the most notable characteristic of these formalisms is their complexity. The causes of this complexity can be attributed to a mix of path dependent types, intersection types, bounded type members and enviroment narrowing and recursive self types. Environment narrowing in particular makes it difficult to even derive subtype transitivity due to a series of mutually dependent concepts. We demonstrate that removal of environment narrowing and intersection types from this mix can give us a sound calculus that still retains considerable expressiveness.
\end{abstract}

%
% The code below should be generated by the tool at
% http://dl.acm.org/ccs.cfm
% Please copy and paste the code instead of the example below.
%
\begin{CCSXML}
<ccs2012>
 <concept>
  <concept_id>10010520.10010553.10010562</concept_id>
  <concept_desc>Computer systems organization~Embedded systems</concept_desc>
  <concept_significance>500</concept_significance>
 </concept>
 <concept>
  <concept_id>10010520.10010575.10010755</concept_id>
  <concept_desc>Computer systems organization~Redundancy</concept_desc>
  <concept_significance>300</concept_significance>
 </concept>
 <concept>
  <concept_id>10010520.10010553.10010554</concept_id>
  <concept_desc>Computer systems organization~Robotics</concept_desc>
  <concept_significance>100</concept_significance>
 </concept>
 <concept>
  <concept_id>10003033.10003083.10003095</concept_id>
  <concept_desc>Networks~Network reliability</concept_desc>
  <concept_significance>100</concept_significance>
 </concept>
</ccs2012>
\end{CCSXML}

\ccsdesc[500]{Computer systems organization~Embedded systems}
\ccsdesc[300]{Computer systems organization~Redundancy}
\ccsdesc{Computer systems organization~Robotics}
\ccsdesc[100]{Networks~Network reliability}


\keywords{ACM proceedings, \LaTeX, text tagging}

\maketitle

\section{Introduction}

\section{Simpler forms of Soundness}

\subsection{Environment Narrowing and Transitivity}
A central difficulty in achieving a proof of soundness is the interplay between environment narrowing (the modification of an environment to a more specific one) and subtype transitivity. The proofs of soundness for DOT deal with this by pushing back on transitivity using \emph{tight} and \emph{inert} typing.
In this paper we demonstrate that a much simplified type system can easily achieve soundness while still containing a variety of interesting features. Primarily we remove environment narrowing altogether by using a double headed subtyping relation ($\Gamma_1 \vdash \tau_1 <: \tau_2 \dashv \Gamma_2$). \citeauthor{nada 2014} used a similar subtyping relation when developing soundness using a big step semantics. We remove environment narrowing in the absence of intersection types.

\subsection{Path Dependent Types}

DOT only allows type selections on variables ($x$). This disqualifies the use of more complex type selections that include field access in the path ($x.f_1 \ldots .f_n$). Such paths are required for modelling modules as objects. They hold some complexity in the evaluation of the paths. Consider the following example.
\begin{lstlisting}[mathescape, style=custom_lang]
val x = new {
        def m(x : {z $\Rightarrow$
                   type L = $\top$
          }) = { x } : x.L
        }
\end{lstlisting}
In this case we define a method that takes a parameter \verb|x| and returns a term of type \verb|x.L|. Now if we attempt to type the following code in this context.
\begin{lstlisting}[mathescape, style=custom_lang]
val z = new {type L = $\top$}
val y = new {
         val f = z
         }
x.m(y)
\end{lstlisting}
\verb|x.m(y.f)| has type \verb|y.f.L|. Further reduction eventually results in \verb|z.L|. This is not the same type and so violates the type preservation portion of type safety.

We aim to introduce complex paths as part of selection types. The strategy involved here is to include paths as values, that is paths are not evaluated.



\section{Syntax}

\begin{figure}[t]
\begin{minipage}{\linewidth}
\small
\[
\begin{array}{l}
\begin{syntax}
\syntaxElement{t}{Term}
	{p}	{path}
	{new\; \{z \Rightarrow \overline{d_v}\}}{object initialization}
	{t.f}{field selection}
	{t.m(t)}{method selection}
\endSyntaxElement\\
\syntaxElement{p}{Path}
	{x,y}	{variable/store location}
	{p.f}{field selection}
\endSyntaxElement\\
\syntaxElement{\tau}{Type}
	{\{z \Rightarrow \overline{\sigma}\}}{record type}
	{p.L}{type selection}
	{\top} {top}
	{\bot}{bottom}
\endSyntaxElement\\
\syntaxElement{d}{Declaration}
	{\typeInit{L}{\tau}}{}
	{\defInit{m}{x}{\tau}{t}}{}
	{\valInit{f}{t}}{}
\endsyntaxElement\\
\syntaxElement{d_v}{Declaration Values}
	{\vdots}{}
%	{\defInit{m}{x}{\tau}{t}}{}
	{\valInit{f}{y}}{}
\endsyntaxElement\\
\syntaxElement{\sigma}{Declaration Type}
	{\typeDefSup{L}{\tau}}{}
	{\typeDefSub{L}{\tau}}{}
	{\typeDefExt{L}{\tau}}{}
	{\valType{f}{\tau}}{}
	{\defType{m}{x}{\tau}{\tau}}{}
\endsyntaxElement\\
\syntaxElement{E}{Evaluation Context}
	{\bigcirc}{}
	{E.f}{}
	{E.m(t)}{}
	{p.m(E)}{}
\endsyntaxElement\\
\syntaxElement{\Gamma}{Environment}
	{\varnothing}{}
	{\Gamma, x : \tau}{}
\endsyntaxElement\\
\syntaxElement{\mu}{Store}
	{\varnothing}{}
	{\mu, y : \{z \Rightarrow \overline{d_v}\}}{}
\endsyntaxElement\\
\end{syntax}
\end{array}
\]
\end{minipage}
\caption{Wyvern Syntax}
\label{f:Wyv:Syntax}
\end{figure}

We present the syntax of Wyvern in Figure \ref{f:Wyv:Syntax}. \textbf{Terms} are paths ($p$), object intializations ($new\; \{z \Rightarrow \overline{d_v}\}$), field accesses ($t.f$), and method selections ($t.m(t)$). \textbf{Paths} are variables ($x$), field selections on paths ($p.f$) and locations in the store ($y$). \textbf{Types} are recursive self types ($\{z \Rightarrow \overline{\sigma}\}$), selection types ($p.L$), the top type ($\top$) and the bottom type ($\bot$). \textbf{Declarations} are abstract type members ($\typeInit{L}{\tau}$), method definitions ($\defInit{m}{x}{\tau}{t}$) and value assignments ($\valInit{f}{t}$). \textbf{Declaration Types} are upper ($\typeDefSub{L}{\tau}$) and lower ($\typeDefSup{L}{\tau}$) bound abstract type members ($\typeDefExt{L}{\tau}$), type member assignments ($\typeDefExt{L}{\tau}$), value types ($\valType{f}{\tau}$) and method types ($\defType{m}{x}{\tau}{\tau}$).


\section{Semantics}


%\begin{figure}[h]
%\begin{mathpar}
%\inferrule
%  {}
%  {p \equiv p}
%  \quad (\textsc{Eq-Refl})
%  \and
%\inferrule
%  {p_1 \equiv p_2}
%  {p_2 \equiv p_1}
%  \quad (\textsc{Eq-Sym})
%  \and
%\inferrule
%  {p_1 \equiv p_2 \\
%   p_2 \equiv p_3}
%  {p_1 \equiv p_3}
%  \quad (\textsc{Eq-Trans})
%  \and
%\inferrule
%  {p_1 \equiv p_2}
%  {p_1 \equiv p_2 \unlhd T}
%  \quad (\textsc{Eq-Path})
%\end{mathpar}
%\caption{Path Equivalence}
%\label{f:path_equiv}
%\end{figure}


%\begin{figure}[h]
%\begin{mathpar}
%\inferrule
%  {
%  \overline{\sigma}_1 \equiv \overline{\sigma}_2
%  }
%  {\{z \Rightarrow \overline{\sigma}_1\} \equiv \{z \Rightarrow \overline{\sigma}_2\}}
%  \quad (\textsc{Eq-Struct})
%  \and
%\inferrule
%  {
%  	p_1 \equiv p_2
%   }
%  {p_1.L \equiv p_2.L}
%  \quad (\textsc{Eq-Select})
%  \and
%\inferrule
%  {}
%  {\vdash \top \equiv \top}
%  \quad (\textsc{Eq-Top})
%  \and
%\inferrule
%  {}
%  {\bot \equiv \bot}
%  \quad (\textsc{Eq-Bottom})
%\end{mathpar}
%\caption{Type Path Equivalence}
%\label{f:type_equiv}
%\end{figure}

\begin{figure}[h]
\hfill \fbox{$\Gamma_1 \vdash \tau_1 <: \tau_2 \dashv \Gamma_2$}
\begin{mathpar}
\inferrule
	{}
	{\Gamma_1 \vdash x.L\ <:\ x.L \dashv \Gamma_2}
	\quad (\textsc {S-Refl})
	\and
\inferrule
	{\Gamma_1, z : \{z \Rightarrow \overline{\sigma}_1\} \vdash \overline{\sigma}_1 <:\ \overline{\sigma}_2 \dashv \Gamma_2, z : \{z \Rightarrow \overline{\sigma}_2\}}
	{\Gamma_1 \vdash \{z \Rightarrow \overline{\sigma}_1\}\ <:\ \{z \Rightarrow \overline{\sigma}_2\} \dashv \Gamma_2}
	\quad (\textsc {S-Rec})
%	\and
%\inferrule
%	{\Gamma_2 \vdash \tau_2\ <:\ \tau_1 \dashv \Gamma_1 \\
%	 \Gamma, x : \tau_1 \vdash \tau_1'\ <:\ \tau_2' \dashv \Gamma_2, x : \tau_2}
%	{\Gamma_1 \vdash \arrowType{\tau_1}{x}{\tau_1'} <:\ \arrowType{\tau_2}{x}{\tau_2'} \dashv \Gamma_2}
%	\quad (\textsc {S-Arrow})
	\and
\inferrule
	{\Gamma_1 \vdash x \ni \text{\texttt{type}} \ L : \tau_1 \ldots  \tau_2\\
	 \Gamma_1 \vdash \tau_2 <: \tau \dashv \Gamma_2}
	{\Gamma_1 \vdash p.L\ <:\ \tau \dashv \Gamma_2}
	\quad (\textsc {S-Select-Upper})
	\and
\inferrule
	{\Gamma_2 \vdash x \ni \text{\texttt{type}} \ L : \tau_1 \ldots  \tau_2\\
	 \Gamma_1 \vdash \tau <: \tau_1 \dashv \Gamma_2}
	{\Gamma_1 \vdash \tau \ <:\ p.L \dashv \Gamma_2}
	\quad (\textsc {S-Select-Lower})
	\and
\inferrule
	{}
	{\Gamma_1 \vdash \tau\ <:\ \top \dashv \Gamma_2}
	\quad (\textsc {S-Top})
	\and
\inferrule
	{}
	{\Gamma_1 \vdash \bot\ <:\ \tau \dashv \Gamma_2}
	\quad (\textsc {S-Bottom})
\end{mathpar}
\hfill \fbox{$\Gamma_1 \vdash \sigma <: \sigma' \dashv \Gamma_2$}
\begin{mathpar}
\inferrule
	{\Gamma_1 \vdash \tau_1 <: \tau_2 \dashv \Gamma_2}
	{\Gamma_1 \vdash \valType{f}{\tau_1} <: \valType{f}{\tau_2} \dashv \Gamma_2}
	\quad (\textsc {S-Decl-Val})	
	\and
\inferrule
	{\Gamma_2 \vdash \tau_2 <: \tau_1 \dashv \Gamma_1 \\
	\Gamma_1, x : \tau_1 \vdash \tau_1' <: \tau_2' \dashv \Gamma_2, x : \tau_2}
	{\Gamma_1 \vdash \defType{m}{x}{\tau_1}{\tau_1'} <: \defType{m}{x}{\tau_2}{\tau_2'} \dashv \Gamma_2}
	\quad (\textsc {S-Decl-Def})	
	\and
\inferrule
	{\Gamma_1 \vdash \tau_1 <: \tau_2 \dashv \Gamma_2}
	{\Gamma_1 \vdash \typeDefSub{L}{\tau_1} <:\ \typeDefSub{L}{\tau_2} \dashv \Gamma_2}
	\quad (\textsc {S-Decl-Upper})
	\and
\inferrule
	{\Gamma_2 \vdash \tau_2 <: \tau_1 \dashv \Gamma_1}
	{\Gamma_1 \vdash \typeDefSup{L}{\tau_1} <:\ \typeDefSup{L}{\tau_2} \dashv \Gamma_2}
	\quad (\textsc {S-Decl-Lower})
	\and
\inferrule
	{\Gamma_1 \vdash \tau_1 <: \tau_2 \dashv \Gamma_1}
	{\Gamma_1 \vdash \typeDefExt{L}{\tau_1} <:\ \typeDefSub{L}{\tau_2} \dashv \Gamma_2}
	\quad (\textsc {S-Decl-Type-1})
	\and
\inferrule
	{\Gamma_2 \vdash \tau_2 <: \tau_1 \dashv \Gamma_1}
	{\Gamma_1 \vdash \typeDefExt{L}{\tau_1} <:\ \typeDefSup{L}{\tau_2} \dashv \Gamma_2}
	\quad (\textsc {S-Decl-Type-2})
\end{mathpar}
\caption{Subtyping}
\label{f:subtype}
\end{figure}
The subtype semantics are given in Figure \ref{f:subtype}. The only non-standard aspect of subtyping is the presence of a second environment. This is captured in the rules for recursive self types (\textsc{S-Rec}) and method declarations \textsc{S-Decl-Def}. The use of a single environment in these cases would introduce environment narrowing, altering the context of the larger types during subtyping.

%\begin{figure}[h]
%\hfill \fbox{$A\Gamma \vdash T \  \textbf{wf}$}
%\begin{mathpar}
%\inferrule
%  {A\Gamma \vdash p \ni \texttt{type} \ L : S \ldots  U \\
%  	A\Gamma \vdash \texttt{type} \ L : S \ldots  U \ \textbf{wf} }
%  {A\Gamma \vdash p.L \ \textbf{wf}}
%  \quad (\textsc {WF-Sel})
%	\and
%\inferrule
%  {A\Gamma,z:\{z \Rightarrow \overline{\sigma}\} \vdash \overline{\sigma} \ \textbf{wf} \\
%  	\forall j \neq i, \ dom(\sigma_j) \neq dom(\sigma_i)}
%  {A\Gamma \vdash \{z \Rightarrow \overline{\sigma}\} \ \textbf{wf}}
%  \quad (\textsc {WF-Struct})
%%	\and
%%\inferrule
%%  {\Gamma \vdash S \  \textbf{wf} \\
%%  	\Gamma \vdash T \  \textbf{wf}}
%%  {\Gamma \vdash S \rightarrow T \  \textbf{wf}}
%%  \quad (\textsc {WF-Func})
%	\and
%\inferrule
%  {}
%  {A\Gamma \vdash \top \  \textbf{wf}}
%  \quad (\textsc {WF-Top})
%	\and
%\inferrule
%  {}
%  {A\Gamma \vdash \bot \  \textbf{wf}}
%  \quad (\textsc {WF-Bottom})
%\end{mathpar}
%\hfill \fbox{$A\Gamma \vdash \sigma \  \textbf{wf}$}
%\begin{mathpar}
%\inferrule
%  {A\Gamma \vdash T : \textbf{wf}}
%  {A\Gamma \vdash \texttt{val} \ f:T \  \textbf{wf}}
%  \quad (\textsc {WF-Val})
%	\and
%\inferrule
%  {A\Gamma \vdash T : \textbf{wf} \\
%  	A\Gamma \vdash S : \textbf{wf}}
%  {A\Gamma \vdash \texttt{def} \ m:S \rightarrow T \  \textbf{wf}}
%  \quad (\textsc {WF-Def})
%	\and
%\inferrule
%  {A\Gamma \vdash S : \textbf{wfe} \ \vee \ S = \bot\\
%  	A\Gamma \vdash U : \textbf{wfe} \\
%  	A\Gamma \vdash S <: U}
%  {A\Gamma \vdash \texttt{type} \ L : S \ldots  U \ \textbf{wf}}
%  \quad (\textsc {WF-Type})
%\end{mathpar}
%\hfill \fbox{$A \vdash \Gamma \  \textbf{wf}$}
%\begin{mathpar}
%\inferrule
%  {\forall x \in dom(\Gamma), A\Gamma \vdash \Gamma(x) \ \textbf{wf}}
%  { \vdash \Gamma \  \textbf{wf}}
%  \quad (\textsc {WF-Environment})
%\end{mathpar}
%\hfill \fbox{$ \  \textbf{wf}$}
%\begin{mathpar}
%\inferrule
%  {\forall l \in dom(), \varnothing\varnothing \vdash (l) \ \textbf{wf}}
%  { \  \textbf{wf}}
%  \quad (\textsc {WF-Store-Context})
%\end{mathpar}
%\begin{mathpar}
%\inferrule
%  {\forall l \in dom(\mu), \varnothing\varnothing \vdash \mu(l) : (l)}
%  {\mu : }
%  \quad (\textsc {WF-Store})
%\end{mathpar}
%\caption{Well-Formedness}
%\label{f:wf}
%\end{figure}
%
%\begin{figure}[h]
%\hfill \fbox{$\Gamma \vdash T \  \textbf{wfe}$}
%\begin{mathpar}
%\inferrule
%  {\Gamma \vdash T \ \textbf{wf} \\
%  	\Gamma \vdash T \prec \overline{\sigma}}
%  {\Gamma \vdash T \ \textbf{wfe}}
%  \quad (\textsc {WFE})
%\end{mathpar}
%\caption{Well-Formed and Expanding Types}
%\label{f:wfe}
%\end{figure}

\begin{figure}[h]
\hfill \fbox{$\Gamma \vdash \sigma^z \in \tau$}
\begin{mathpar}
\inferrule
	{\sigma \in \overline{\sigma}}
	{\Gamma \vdash \sigma^z \in \{z \Rightarrow \overline{\sigma}\}}
	\quad (\textsc {MT-Struct})
	\and
\inferrule
	{\Gamma \vdash p \ni \typeDefSub{L}{\tau_2} \\
	 \Gamma \vdash \sigma^z \in \tau_2}
	{\Gamma \vdash \sigma^z \in p.L}
	\quad (\textsc {MT-Sel-1})
	\and
\inferrule
	{\Gamma \vdash p \ni \typeInit{L}{\tau_2} \\
	 \Gamma \vdash \sigma^z \in \tau_2}
	{\Gamma \vdash \sigma^z \in p.L}
	\quad (\textsc {MT-Sel-2})
\end{mathpar}
\caption{Type Membership}
\label{f:exp}
\hfill \fbox{$\Gamma \vdash t \ni \sigma$}
\begin{mathpar}
\inferrule
  {\Gamma \vdash p : \tau \\
  	\Gamma \vdash \sigma^z \in \tau}
  {\Gamma \vdash p \ni [p/z]\sigma}
  \quad (\textsc {M-Path})
	\and
\inferrule
  {\Gamma \vdash t : \tau \\
  	\Gamma \vdash \sigma^z \in \tau \\
  	z \notin \sigma}
  {\Gamma \vdash t \ni \sigma}
  \quad (\textsc {M-Exp})
\end{mathpar}
\caption{Term Membership}
\label{f:mem}
\end{figure}
Term and type membership is relatively straightforward and is given in Figure \ref{f:mem}. One thing to note, type membership is only ever done on paths, and never non-path terms. This simplifies the term typing need during subtyping to only enviroment lookups. This means that during the proof of transitivity, we need only consider the typing a small subset of relatively simple terms.

%\subsubsection{Typing}

\begin{figure}[h]
\hfill \fbox{$\Gamma \vdash e:T$}
\begin{mathpar}
\inferrule
  {x \in dom(\Gamma)}
  {\Gamma \vdash x : \Gamma(x)}
  \quad (\textsc {T-Var})
	\and
\inferrule
  {\Gamma, z : \{z \Rightarrow \overline{\sigma}\} 
  \vdash \overline{d} : \overline{\sigma}}
  {\Gamma \vdash \texttt{new} \ \{z \Rightarrow \overline{d}\} : 
  \{z \Rightarrow \overline{\sigma}\}}
  \quad (\textsc {T-New})
	\and
\inferrule
  {\Gamma \vdash t \ni \valType{f}{\tau}}
  {\Gamma \vdash t.f : \tau}
  \quad (\textsc {T-Field})
	\and
\inferrule
  {\Gamma \vdash t \ni \defType{m}{x}{\tau_1}{\tau_2} \\
   \Gamma \vdash t' : \tau_1 \\
   \Gamma \vdash t' <: t_1 \dashv \Gamma}
  {\Gamma \vdash t.m : \tau_2}
  \quad (\textsc {T-Meth})
\end{mathpar}
\caption{Term Typing}
\label{f:t_typ}
\end{figure}
\begin{figure}[h]
\hfill \fbox{$\Gamma \vdash d:\sigma$}
\begin{mathpar}
\inferrule
  {\Gamma \vdash t : \tau}
  {\Gamma \vdash \valInit{f}{t} : \valType{f}{\tau}}
  \quad (\textsc {T-Decl-Val})
	\and
\inferrule
  {\Gamma, x : \tau_1 \vdash t : \tau_2}
  {\Gamma \vdash \defInit{m}{x}{\tau_1}{t} : \defType{m}{x}{\tau_1}{\tau_2}}
  \quad (\textsc {T-Decl-Def})
	\and
\inferrule
  {}
  {\Gamma \vdash \typeInit{L}{\tau} : \typeInit{L}{\tau}}
  \quad (\textsc {T-Decl-Type})
\end{mathpar}
\caption{Declaration Typing}
\label{f:d_typ}
\end{figure}
Figure \ref{f:t_typ} gives the full term typing relation, while \ref{f:d_typ} gives the typing relation for declarations. As previously mentioned, in a world on only path term, typing reduces to a very simple environment lookup, as only the rules \textsc{T-Var} and \textsc{T-Field} are used.

%\begin{figure}[h]
%\hfill \fbox{$\mu \vdash v \leadsto l$}
%\begin{mathpar}
%\inferrule
%  {}
%  {\mu \vdash l \leadsto l }
%  \quad (\textsc {P-Loc})
%	\and
%\inferrule
%  {\mu \vdash v \leadsto v'}
%  {\mu \vdash v \unlhd T \leadsto v' \unlhd T}
%  \quad (\textsc {L-Type})
%	\and
%\inferrule
%  {\mu \vdash v \leadsto v' \\
%   \mu \vdash v' \leadsto_{f} v_f}
%  {\mu \vdash v.f \leadsto v_f}
%  \quad (\textsc {L-Path})
%\end{mathpar}
%\hfill \fbox{$\mu \vdash d_v \leadsto d$}
%\begin{mathpar}
%\inferrule
%  {\mu \vdash v \leadsto v'}
%  {\mu \vdash \texttt{val} \ f : T = v \leadsto \texttt{def} \ m : S \rightarrow T}
%  \quad (\textsc {L-Val})
%	\and
%\inferrule
%  {}
%  {\mu \vdash \texttt{def} \ m : S(x:T) = e \leadsto \texttt{def} \ m(x:S) = e : T}
%  \quad (\textsc {L-Def})
%	\and
%\inferrule
%  {}
%  {\mu \vdash \texttt{type} \ L : S \ldots  U \leadsto \texttt{type} \ L : S \ldots  U}
%  \quad (\textsc {L-Type})
%\end{mathpar}
%\caption{Path Leads-to Judgement}
%\label{f:path}
%\end{figure}
%
%\begin{figure}[h]
%\hfill \fbox{$\mu \vdash v \leadsto_{f} v$}
%\begin{mathpar}
%\inferrule
%  {}
%  {\mu \vdash l \leadsto l }
%  \quad (\textsc {P-Loc})
%	\and
%\inferrule
%  {\mu \vdash v \leadsto v'}
%  {\mu \vdash v \unlhd T \leadsto v' \unlhd T}
%  \quad (\textsc {L-Type})
%	\and
%\inferrule
%  {\mu \vdash v \leadsto v' \\
%   \mu \vdash v' \leadsto_{f} v_f}
%  {\mu \vdash v.f \leadsto v_f}
%  \quad (\textsc {L-Path})
%\end{mathpar}
%\caption{Path Leads-to Judgement}
%\label{f:path}
%\end{figure}
%
% 
%
%\begin{figure}[h]
%\hfill \fbox{$\mu \vdash v_1 \leadsto v_2$}
%\begin{mathpar}
%\inferrule
%  {}
%  {\mu \vdash l \leadsto l}
%  \quad (\textsc {L-Loc})
%	\and
%\inferrule
%  {\mu \vdash {v_1}^{p'_z} \leadsto v_2}
%  {\mu \vdash v_1^{p'_z} \unlhd \{z \Rightarrow \overline{\sigma}\}_{p_z} \leadsto v_2 \unlhd \{z \Rightarrow [p_z/z]\overline{\sigma}\}}
%  \quad (\textsc {L-Type})
%	\and
%\inferrule
%  {\varnothing\varnothing \vdash v_1 : p'.L \\
%  	\varnothing\varnothing \vdash p \ni \texttt{type} \ L : S \ldots  U \\
%  	p \not\equiv p' \\
%  	\mu \vdash v_1 \unlhd S \unlhd U \leadsto v_2}
%  {\mu \vdash v_1 \unlhd p.L \leadsto v_2}
%  \quad (\textsc {L-Type-Select-Lower})
%	\and
%\inferrule
%  {\varnothing\varnothing \vdash v_1 : p'.L \\
%  	\varnothing\varnothing \vdash p \ni \texttt{type} \ L : S \ldots  U \\
%  	p \equiv p' \\
%  	\mu \vdash v_1 \unlhd U \leadsto v_2}
%  {\mu \vdash v_1 \unlhd p.L \leadsto v_2}
%  \quad (\textsc {L-Type-Select-Upper})
%%	\and
%%\inferrule
%%  {\varnothing\varnothing \vdash v_1 : p'.L \\
%%   p' \equiv p \\
%%  	\varnothing\varnothing \vdash p \ni \texttt{type} \ L : S \ldots  U \\
%%  	\mu \vdash v_1 \unlhd U \leadsto v_2}
%%  {\mu \vdash v_1 \unlhd p.L \leadsto v_2}
%%  \quad (\textsc {L-Type-Select-Refl})
%\end{mathpar}
%\caption{Leadsto Judgement}
%\label{f:leadsto}
%\end{figure}
%
%\begin{figure}[h]
%\hfill \fbox{$\mu \vdash v_1 \leadsto_{f} v_2$}
%\begin{mathpar}
%\inferrule
%  {\mu(l) = \{z \Rightarrow \ldots , \texttt{val} \ f : T = v, \ldots \}}
%  {\mu \vdash l \leadsto_{f} [l/z]v \unlhd T}
%  \quad (\textsc {L\textsubscript{$f$}-Loc})
%	\and
%\inferrule
%  {\mu \vdash v_1 \leadsto_{f} v_2 \\
%  \texttt{val} \ f:T \in \overline{\sigma}}
%  {\mu \vdash v_1 \unlhd \{z \Rightarrow \overline{\sigma}\} \leadsto_{f} v_2 \unlhd [v_1 \unlhd \{z \Rightarrow \overline{\sigma}\} / z]T}
%  \quad (\textsc {L\textsubscript{$f$}-Type})
%	\and
%\inferrule
%  {\mu \vdash v_1 \unlhd p.L \leadsto v_2 \\
%   \mu \vdash v_2 \leadsto_{f} v_3}
%  {\mu \vdash v_1 \unlhd p.L \leadsto_{f} v_3}
%  \quad (\textsc {L\textsubscript{$f$}-Type-Select})
%%	\and
%%\inferrule
%%  {\varnothing\varnothing \vdash v_1 : T \\
%%  	\varnothing\varnothing \vdash p \ni \texttt{type} \ L : S \ldots  U \\
%%  	\varnothing\varnothing \not\vdash T <: U \\
%%  	\mu \vdash v_1 \unlhd S \unlhd U \leadsto v_2}
%%  {\mu \vdash v_1 \unlhd p.L \leadsto v_2}
%%  \quad (\textsc {L\textsubscript{$f$}-Type-Select-Lower})
%%	\and
%%\inferrule
%%  {\varnothing\varnothing \vdash v_1 : T \\
%%  	\varnothing\varnothing \vdash p \ni \texttt{type} \ L : S \ldots  U \\
%%  	\varnothing\varnothing \vdash T <: U \\
%%  	\mu \vdash v_1 \unlhd U \leadsto v_2}
%%  {\mu \vdash v_1 \unlhd p.L \leadsto v_2}
%%  \quad (\textsc {L\textsubscript{$f$}-Type-Select-Upper})
%%	\and
%%\inferrule
%%  {\mu \vdash v_1 \leadsto_{f_1} v_2 \\
%%   v_2 \neq v_1.f_1 \\
%%  	\mu \vdash v_2 \leadsto_{f_2} v_3}
%%  {\mu \vdash v_1.f_1 \leadsto_{f_2} v3}
%%  \quad (\textsc {L\textsubscript{$f$}-Field})
%%	\and
%%\inferrule
%%  {\mu \vdash v_1 \leadsto_{f_1} v_1.f_1}
%%  {\mu \vdash v_1.f_1 \leadsto_{f_2} v_1.f_1.f_2}
%%  \quad (\textsc {L\textsubscript{$f$}-Field-Stop})
%\end{mathpar}
%\caption{Field Leadsto Judgement}
%\label{f:field_leadsto}
%\end{figure}
%
%\begin{figure}[h]
%\hfill \fbox{$\mu \vdash v_1 \leadsto_{m(v_2)} e$}
%\begin{mathpar}
%\inferrule
%  {\mu(l) = \{z \Rightarrow \ldots , \texttt{def} \ m (x : S) = e : T, \ldots \}}
%  {\mu \vdash l \leadsto_{m(v_2)} [v_2 \unlhd S/x, l/z]e \unlhd T}
%  \quad (\textsc {L\textsubscript{$m$}-Loc})
%	\and
%\inferrule
%  {\mu \vdash v_1 \leadsto_{m(v_2 \unlhd S)} e \\
%  \texttt{def} \ m : S \rightarrow T \in \overline{\sigma}}
%  {\mu \vdash v_1 \unlhd \{z \Rightarrow \overline{\sigma}\} \leadsto_{m(v_2)} e \unlhd T}
%  \quad (\textsc {L\textsubscript{$m$}-Type})
%	\and
%\inferrule
%  {\mu \vdash v_1 \unlhd p.L \leadsto v_2 \\
%   \mu \vdash v_2 \leadsto_{m(v_2)} v_3}
%  {\mu \vdash v_1 \unlhd p.L \leadsto_{m} v_3}
%  \quad (\textsc {L\textsubscript{$m$}-Type-Select})
%%	\and
%%\inferrule
%%  {\mu \vdash v_1 \leadsto_{f} v_2 \\
%%  	\mu \vdash v_2 \leadsto_{m} v_3}
%%  {\mu \vdash v_1.f \leadsto_{m} v3}
%%  \quad (\textsc {L\textsubscript{$m$}-Field})
%\end{mathpar}
%\caption{Method Leadsto Judgement}
%\label{f:meth_leadsto}
%\end{figure}

%\begin{figure}[h]
%\hfill \fbox{$ \vdash v \leadsto v'$}
%\begin{mathpar}
%\inferrule
%	{}
%	{ \vdash l \leadsto l}
%	\quad(\textsc{L-Loc})
%	\and
%\inferrule
%	{ \vdash v \leadsto v'}
%	{ \vdash v \unlhd \{z \Rightarrow \overline{\sigma}\} \leadsto v' \unlhd \{z \Rightarrow \overline{\sigma}\}}
%	\quad(\textsc{L-Struct})
%	\and
%\inferrule
%	{\emptyset \vdash v : \tau \\
%	 \tau \not\equiv p.L \\
%	 \emptyset \vdash p \ni \text{\texttt{type}}\ L: \tau_1 \ldots \tau_2 \\
%	  \vdash v \unlhd \tau_1 \unlhd \tau_2 \leadsto v'}
%	{ \vdash v \unlhd p.L \leadsto v'}
%	\quad(\textsc{L-Select})
%	\and
%\inferrule
%	{\emptyset \vdash v : \tau \\
%	 \tau \equiv p.L \\
%	 \emptyset \vdash p \ni \text{\texttt{type}}\ L: \tau_1 \ldots \tau_2 \\
%	  \vdash v \unlhd \tau_2 \leadsto v'}
%	{ \vdash v \unlhd p.L \leadsto v'}
%	\quad(\textsc{L-Select-Path-Equiv})
%\end{mathpar}
%\caption{Leadsto}
%\label{f:leadsto}
%\end{figure}

%\begin{figure}[h]
%\hfill \fbox{$\mu \vdash v \leadsto_m t$}
%\begin{mathpar}
%\inferrule
%	{\mu(l) = \{z \Rightarrow \ldots, \text{\texttt{def}}\ m : \tau = t, \ldots\}}
%	{\mu \vdash l \leadsto_m [l/z](t \unlhd \tau)}
%	\quad(\textsc{ML-Loc})
%	\and
%\inferrule
%	{\mu \vdash v \leadsto_m t \\
%	 \text{\texttt{def}}\ m : \tau \in \overline{\sigma}}
%	{\mu \vdash v \unlhd \{z \Rightarrow \overline{\sigma}\} \leadsto_m t \unlhd [v \unlhd \{z \Rightarrow \overline{\sigma}\}/z]\tau}
%	\quad(\textsc{ML-Struct})
%\end{mathpar}
%\caption{Method Leadsto}
%\label{f:mleadsto}
%\end{figure}

\begin{figure}[h]
\hfill \fbox{$p \mapsto_\mu y$}
\begin{mathpar}
\inferrule
	{}
	{y \mapsto_\mu y}
	\quad(\textsc{MAP-Loc})
	\and
\inferrule
	{p \mapsto_\mu y \\
	\mu(y) = \{z \Rightarrow \overline{d_v}\} \\
	\valInit{f}{y_f} \in \overline{d_v}}
	{p.f \mapsto_\mu y_f}
	\quad(\textsc{MAP-Field})
\end{mathpar}
\caption{Path Mapping}
\label{f:mapsto}
\end{figure}

\begin{figure}[h]
\hfill \fbox{$\mu \ | \ t \ \rightarrow \mu' \ | \ t'$}
\begin{mathpar}
\inferrule
  {y \notin dom(\mu) \\
  	\mu' = \mu, y \mapsto \{\texttt{z} \Rightarrow \overline{d_v}\}}
  {\mu \ | \ \texttt{new} \ \{\texttt{z} \Rightarrow \overline{d_v}\} \ \rightarrow \mu' \ | \ y}
  \quad (\textsc {R-New})
  \and
\inferrule
  {p \mapsto_\mu y \\
  \mu(y) = \{z \Rightarrow \overline{d_v}\} \\
  \defInit{m}{x}{\tau}{t} \in \overline{d_v}}
  {\mu \ | \ p.m(p') \rightarrow \mu \ | [p,p'/z,x]t}
  \quad (\textsc {R-Meth})
%  \and
%\inferrule
%  {p \mapsto_\mu y \\
%  \mu(y) = \{z \Rightarrow \overline{d_v}\} \\
%  \valInit{f}{p'} \in \overline{d_v}}
%  {\mu \ | \ p.f \rightarrow \mu \ | [p/z]p'}
%  \quad (\textsc {R-Field})
  \and
\inferrule
  {	\mu \ | \ t \ \rightarrow \ \mu' \ | \ t'}
  {\mu \ | \ E[t] \ \rightarrow \mu' \ | \ E[t']}
  \quad (\textsc {R-Context})
\end{mathpar}
\caption{Term Reduction}
\label{f:red}
\end{figure}
The typing relation is given in Figure \ref{f:red}. This makes use of a helper relation in Figure \ref{f:mapsto} for Path Mapping. As we do not reduce paths in order to maintain type preservation, we use Path Mapping to determine the ultimate location in the store of a path in order to do method lookup.


























%\section{Syntax}

% 
\documentclass{llncs}

\usepackage{listings}
\usepackage{amssymb}
\usepackage[margin=.9in]{geometry}
\usepackage{amsmath}
%\usepackage{amsthm}
\usepackage{mathpartir}
\usepackage{color,soul}
\usepackage{graphicx}

\setcounter{secnumdepth}{5}

%\theoremstyle{definition}
%%\newtheorem{case1}{Case1}
\spnewtheorem{casethm}{Case}[theorem]{\itshape}{\rmfamily}
\spnewtheorem{subcase}{Subcase}{\itshape}{\rmfamily}
\numberwithin{subcase}{casethm}
\numberwithin{casethm}{theorem}
\numberwithin{casethm}{lemma}





\lstdefinestyle{custom_lang}{
  xleftmargin=\parindent,
  showstringspaces=false,
  basicstyle=\ttfamily,
  keywordstyle=\bfseries
}

\lstset{emph={%  
    val, def, type, new, z%
    },emphstyle={\bfseries \tt}%
}

\begin{document}

\section{Syntax}

\begin{figure}[h]
\[
\begin{array}{lll}
\begin{array}{lllr}
e & ::= & x & expression \\
& | & \texttt{new} \; \{z \Rightarrow \overline{d}\}&\\
& | & e.m(e) &\\
& | & e.f &\\
& | & l &\\
&&\\
p & ::= & x & paths \\
& | & l &\\
&&\\
v & ::= & l & value \\
%& | & v \unlhd T &\\
&&\\
d & ::= & \texttt{val} \; f : \tau = e & declaration \\
  & |   & \texttt{def} \; m(x:\tau) = e : \tau &\\
  & |   & \texttt{type} \; M \; L : \tau .. \tau&\\
&&\\
M & :: = & Material \; | \; Shape &  Type \; Modifier \\
&&\\
\tau & ::= & N_M\{z \Rightarrow \overline{\sigma}\} & type \\
& | & p.L \\
& | & \tau \wedge \tau & \\
& | & \tau \vee \tau & \\
 \end{array}
& ~~~~~~
&
\begin{array}{lllr}
\sigma & ::= & \texttt{val} \; f:\tau & decl \; type\\
       & |   & \texttt{def} \; m:\tau \rightarrow \tau \\
		 & |   & \texttt{type} \; M \; L : \tau .. \tau &\\
&&\\
E & :: = & \bigcirc & eval \; context\\
       & | & E.m(e)\\
       & | & p.m(E)\\
       & | & E.f\\
&&\\
d_v & ::= & \texttt{val} \; f : \tau = e & declaration \; value \\
  & |   & \texttt{def} \; m(x:\tau) = e : \tau &\\
  & |   & \texttt{type} \; M \; L : \tau .. \tau &\\
&&\\
\Gamma & :: = & \varnothing \; | \; \Gamma,\; x : \tau & Environment \\
&&\\
%A & :: = & \varnothing \; | \; A,\; \tau <: \tau & Assumption \; Context \\
%&&\\
A & :: = & \varnothing \; | \; A,\; N <:: N & Subtype \; Table \\
&&\\
\mu & :: = & \varnothing \; | \; \mu,\; l \mapsto \{z \Rightarrow \overline{d}\} & store \\
&&\\
\Sigma & :: = & \varnothing \; | \; \Sigma,\; l : \{\texttt{z} \Rightarrow \overline{\sigma}\} & store \; type \\
\end{array}
\end{array}
\]
\caption{Syntax}
\label{f:syntax}
\end{figure}


\begin{figure}[h]
\begin{mathpar}
\inferrule
  {}
  {\mathcal{M}(N_M\{z \Rightarrow \overline{\sigma}\}) = M}
  \and
\inferrule
  {}
  {label(\texttt{def} \, m : S \rightarrow T) = m}
  \and
\inferrule
  {}
  {label(\texttt{type} \, L : S .. U) = L}
\end{mathpar}
\caption{Declaration Label Function}
\label{f:label}
\end{figure}


%\begin{figure}[h]
%\begin{mathpar}
%\inferrule
%  {}
%  {p \equiv p}
%  \quad (\textsc{Eq-Refl})
%  \and
%\inferrule
%  {p_1 \equiv p_2}
%  {p_2 \equiv p_1}
%  \quad (\textsc{Eq-Sym})
%  \and
%\inferrule
%  {p_1 \equiv p_2 \\
%   p_2 \equiv p_3}
%  {p_1 \equiv p_3}
%  \quad (\textsc{Eq-Trans})
%  \and
%\inferrule
%  {p_1 \equiv p_2}
%  {p_1 \equiv p_2 \unlhd T}
%  \quad (\textsc{Eq-Path})
%%  \and
%%\inferrule
%%  {p_1 \equiv p_2}
%%  {p_1.f \equiv p_2.f}
%%  \quad (\textsc{Eq-Field})
%\end{mathpar}
%\caption{Path Equivalence}
%\label{f:path_equiv}
%\end{figure}


%\begin{figure}[h]
%\begin{mathpar}
%%\inferrule
%%  {
%%  T_1 \equiv T_2 \in \Delta
%%  }
%%  {A; \Sigma; \Gamma \vdash T_1 \equiv T_2}
%%  \quad (\textsc{Eq-Assume})
%%  \and
%%\inferrule
%%  {
%%  \Delta \vdash \overline{\sigma}_1 \equiv \overline{\sigma}_2
%%  }
%%  {\Delta \vdash \{z \Rightarrow \overline{\sigma}_1\} \equiv \{z \Rightarrow \overline{\sigma}_2\}}
%%  \quad (\textsc{Eq-Struct})
%%  \and
%\inferrule
%  {
%  A; \Sigma; \Gamma, z : \{z \Rightarrow \overline{\sigma}_1\} \vdash \overline{\sigma}_1 \equiv [z \unlhd \{z \Rightarrow \overline{\sigma}_2\}/z]\overline{\sigma}_2
%  }
%  {A; \Sigma; \Gamma \vdash \{z \Rightarrow \overline{\sigma}_1\} \equiv \{z \Rightarrow \overline{\sigma}_2\}}
%  \quad (\textsc{Eq-Struct})
%%  \and
%%\inferrule
%%  {
%%  	p_1 \equiv p_2 \\
%%  	T_1 \equiv T_2 
%%%   A; \Sigma; \Gamma \vdash p_1 : T, p_2 : T
%%   }
%%  {(p_1 \unlhd T_1).L \equiv (p_2 \unlhd T_2).L}
%%  \quad (\textsc{Eq-Select})
%  \and
%\inferrule
%  {
%  	p_1 \equiv p_2 \\
%   A; \Sigma; \Gamma \vdash p_1 \ni \texttt{type} L : S_1 .. U_1\\
%   A; \Sigma; \Gamma \vdash p_2 \ni \texttt{type} L : S_2 .. U_2\\
%   A; \Sigma; \Gamma \vdash S_2 \equiv S_1  \\
%   A; \Sigma; \Gamma \vdash U_1 \equiv U_2 
%   }
%  {A; \Sigma; \Gamma \vdash p_1.L \equiv p_2.L}
%  \quad (\textsc{Eq-Select})
%  \and
%\inferrule
%  {}
%  {A; \Sigma; \Gamma \vdash \top \equiv \top}
%  \quad (\textsc{Eq-Top})
%  \and
%\inferrule
%  {}
%  {A; \Sigma; \Gamma \vdash \bot \equiv \bot}
%  \quad (\textsc{Eq-Bottom})
%\end{mathpar}
%\caption{Type Path Equivalence}
%\label{f:type_equiv}
%\end{figure}

%\begin{figure}[h]
%\begin{mathpar}
%\end{mathpar}
%\caption{Declaration Path Function}
%\label{f:path}
%\end{figure}
%\begin{figure}[h]
%\begin{mathpar}
%\inferrule
%  {}
%  {narrow(x) = x \\ narrow(v \unlhd T) = narrow(v)}
%\end{mathpar}
%\caption{Narrow Function}
%\label{f:narrow}
%\end{figure}

%\newpage

\section{Semantics}

%\begin{figure}[h]
%\hfill \fbox{$\mu; \Sigma \vdash v \leadsto l$}
%\begin{mathpar}
%\inferrule
%  {}
%  {\mu; \Sigma \vdash l \leadsto l }
%  \quad (\textsc {P-Loc})
%	\and
%\inferrule
%  {\mu; \Sigma \vdash v \leadsto v'}
%  {\mu; \Sigma \vdash v \unlhd T \leadsto v' \unlhd T}
%  \quad (\textsc {L-Type})
%	\and
%\inferrule
%  {\mu; \Sigma \vdash v \leadsto v' \\
%   \mu; \Sigma \vdash v' \leadsto_{f} v_f}
%  {\mu; \Sigma \vdash v.f \leadsto v_f}
%  \quad (\textsc {L-Path})
%\end{mathpar}
%\hfill \fbox{$\mu; \Sigma \vdash d_v \leadsto d$}
%\begin{mathpar}
%\inferrule
%  {\mu; \Sigma \vdash v \leadsto v'}
%  {\mu; \Sigma \vdash \texttt{val} \; f : T = v \leadsto \texttt{def} \; m : S \rightarrow T}
%  \quad (\textsc {L-Val})
%	\and
%\inferrule
%  {}
%  {\mu; \Sigma \vdash \texttt{def} \; m : S(x:T) = e \leadsto \texttt{def} \; m(x:S) = e : T}
%  \quad (\textsc {L-Def})
%	\and
%\inferrule
%  {}
%  {\mu; \Sigma \vdash \texttt{type} \; L : S .. U \leadsto \texttt{type} \; L : S .. U}
%  \quad (\textsc {L-Type})
%\end{mathpar}
%\caption{Path Leads-to Judgement}
%\label{f:path}
%\end{figure}

%\begin{figure}[h]
%\hfill \fbox{$\mu; \Sigma \vdash v \leadsto_{f} v$}
%\begin{mathpar}
%\inferrule
%  {}
%  {\mu; \Sigma \vdash l \leadsto l }
%  \quad (\textsc {P-Loc})
%	\and
%\inferrule
%  {\mu; \Sigma \vdash v \leadsto v'}
%  {\mu; \Sigma \vdash v \unlhd T \leadsto v' \unlhd T}
%  \quad (\textsc {L-Type})
%	\and
%\inferrule
%  {\mu; \Sigma \vdash v \leadsto v' \\
%   \mu; \Sigma \vdash v' \leadsto_{f} v_f}
%  {\mu; \Sigma \vdash v.f \leadsto v_f}
%  \quad (\textsc {L-Path})
%\end{mathpar}
%\caption{Path Leads-to Judgement}
%\label{f:path}
%\end{figure}
%
%\newpage

%\begin{figure}[h]
%\hfill \fbox{$\mu; \Sigma \vdash v_1 \leadsto v_2$}
%\begin{mathpar}
%\inferrule
%  {}
%  {\mu; \Sigma \vdash l \leadsto l}
%  \quad (\textsc {L-Loc})
%	\and
%\inferrule
%  {\mu; \Sigma \vdash {v_1}^{p'_z} \leadsto v_2}
%  {\mu; \Sigma \vdash v_1^{p'_z} \unlhd \{z \Rightarrow \overline{\sigma}\}_{p_z} \leadsto v_2 \unlhd \{z \Rightarrow [p_z/z]\overline{\sigma}\}}
%  \quad (\textsc {L-Type})
%	\and
%\inferrule
%  {\varnothing; \Sigma; \varnothing \vdash v_1 : p'.L \\
%  	\varnothing; \Sigma; \varnothing \vdash p \ni \texttt{type} \; L : S .. U \\
%  	p \not\equiv p' \\
%  	\mu; \Sigma \vdash v_1 \unlhd S \unlhd U \leadsto v_2}
%  {\mu; \Sigma \vdash v_1 \unlhd p.L \leadsto v_2}
%  \quad (\textsc {L-Type-Select-Lower})
%	\and
%\inferrule
%  {\varnothing; \Sigma; \varnothing \vdash v_1 : p'.L \\
%  	\varnothing; \Sigma; \varnothing \vdash p \ni \texttt{type} \; L : S .. U \\
%  	p \equiv p' \\
%  	\mu; \Sigma \vdash v_1 \unlhd U \leadsto v_2}
%  {\mu; \Sigma \vdash v_1 \unlhd p.L \leadsto v_2}
%  \quad (\textsc {L-Type-Select-Upper})
%	\and
%\inferrule
%  {\varnothing; \Sigma; \varnothing \vdash v_1 : p'.L \\
%   p' \equiv p \\
%  	\varnothing; \Sigma; \varnothing \vdash p \ni \texttt{type} \; L : S .. U \\
%  	\mu; \Sigma \vdash v_1 \unlhd U \leadsto v_2}
%  {\mu; \Sigma \vdash v_1 \unlhd p.L \leadsto v_2}
%  \quad (\textsc {L-Type-Select-Refl})
%\end{mathpar}
%\caption{Leadsto Judgement}
%\label{f:leadsto}
%\end{figure}
%
%\begin{figure}[h]
%\hfill \fbox{$\mu; \Sigma \vdash v_1 \leadsto_{f} v_2$}
%\begin{mathpar}
%\inferrule
%  {\mu(l) = \{z \Rightarrow ..., \texttt{val} \; f : T = v, ...\}}
%  {\mu; \Sigma \vdash l \leadsto_{f} [l/z]v \unlhd T}
%  \quad (\textsc {L\textsubscript{$f$}-Loc})
%	\and
%\inferrule
%  {\mu; \Sigma \vdash v_1 \leadsto_{f} v_2 \\
%  \texttt{val} \; f:T \in \overline{\sigma}}
%  {\mu; \Sigma \vdash v_1 \unlhd \{z \Rightarrow \overline{\sigma}\} \leadsto_{f} v_2 \unlhd [v_1 \unlhd \{z \Rightarrow \overline{\sigma}\} / z]T}
%  \quad (\textsc {L\textsubscript{$f$}-Type})
%	\and
%\inferrule
%  {\mu; \Sigma \vdash v_1 \unlhd p.L \leadsto v_2 \\
%   \mu; \Sigma \vdash v_2 \leadsto_{f} v_3}
%  {\mu; \Sigma \vdash v_1 \unlhd p.L \leadsto_{f} v_3}
%  \quad (\textsc {L\textsubscript{$f$}-Type-Select})
%	\and
%\inferrule
%  {\varnothing; \Sigma; \varnothing \vdash v_1 : T \\
%  	\varnothing; \Sigma; \varnothing \vdash p \ni \texttt{type} \; L : S .. U \\
%  	\varnothing; \Sigma; \varnothing \not\vdash T <: U \\
%  	\mu; \Sigma \vdash v_1 \unlhd S \unlhd U \leadsto v_2}
%  {\mu; \Sigma \vdash v_1 \unlhd p.L \leadsto v_2}
%  \quad (\textsc {L\textsubscript{$f$}-Type-Select-Lower})
%	\and
%\inferrule
%  {\varnothing; \Sigma; \varnothing \vdash v_1 : T \\
%  	\varnothing; \Sigma; \varnothing \vdash p \ni \texttt{type} \; L : S .. U \\
%  	\varnothing; \Sigma; \varnothing \vdash T <: U \\
%  	\mu; \Sigma \vdash v_1 \unlhd U \leadsto v_2}
%  {\mu; \Sigma \vdash v_1 \unlhd p.L \leadsto v_2}
%  \quad (\textsc {L\textsubscript{$f$}-Type-Select-Upper})
%	\and
%\inferrule
%  {\mu; \Sigma \vdash v_1 \leadsto_{f_1} v_2 \\
%   v_2 \neq v_1.f_1 \\
%  	\mu; \Sigma \vdash v_2 \leadsto_{f_2} v_3}
%  {\mu; \Sigma \vdash v_1.f_1 \leadsto_{f_2} v3}
%  \quad (\textsc {L\textsubscript{$f$}-Field})
%	\and
%\inferrule
%  {\mu; \Sigma \vdash v_1 \leadsto_{f_1} v_1.f_1}
%  {\mu; \Sigma \vdash v_1.f_1 \leadsto_{f_2} v_1.f_1.f_2}
%  \quad (\textsc {L\textsubscript{$f$}-Field-Stop})
%\end{mathpar}
%\caption{Field Leadsto Judgement}
%\label{f:field_leadsto}
%\end{figure}

%\begin{figure}[h]
%\hfill \fbox{$\mu; \Sigma \vdash v_1 \leadsto_{m(v_2)} e$}
%\begin{mathpar}
%\inferrule
%  {\mu(l) = \{z \Rightarrow ..., \texttt{def} \; m (x : S) = e : T, ...\}}
%  {\mu; \Sigma \vdash l \leadsto_{m(v_2)} [v_2 \unlhd S/x, l/z]e \unlhd T}
%  \quad (\textsc {L\textsubscript{$m$}-Loc})
%	\and
%\inferrule
%  {\mu; \Sigma \vdash v_1 \leadsto_{m(v_2 \unlhd S)} e \\
%  \texttt{def} \; m : S \rightarrow T \in \overline{\sigma}}
%  {\mu; \Sigma \vdash v_1 \unlhd \{z \Rightarrow \overline{\sigma}\} \leadsto_{m(v_2)} e \unlhd T}
%  \quad (\textsc {L\textsubscript{$m$}-Type})
%	\and
%\inferrule
%  {\mu; \Sigma \vdash v_1 \unlhd p.L \leadsto v_2 \\
%   \mu; \Sigma \vdash v_2 \leadsto_{m(v_2)} v_3}
%  {\mu; \Sigma \vdash v_1 \unlhd p.L \leadsto_{m} v_3}
%  \quad (\textsc {L\textsubscript{$m$}-Type-Select})
%%	\and
%%\inferrule
%%  {\mu; \Sigma \vdash v_1 \leadsto_{f} v_2 \\
%%  	\mu; \Sigma \vdash v_2 \leadsto_{m} v_3}
%%  {\mu; \Sigma \vdash v_1.f \leadsto_{m} v3}
%%  \quad (\textsc {L\textsubscript{$m$}-Field})
%\end{mathpar}
%\caption{Method Leadsto Judgement}
%\label{f:meth_leadsto}
%\end{figure}

%\newpage

\begin{figure}[h]
\hfill \fbox{$A; \Sigma; \Gamma \vdash \tau <: \tau'$}
\begin{mathpar}
\inferrule
	{}
	{A, \Sigma; \Gamma \vdash \tau \; \texttt{<:}\; \tau}
	\quad (\textsc {S-Refl})
	\and
\inferrule
	{N_M <:: {N'}_M \in A \\
	 A, \Sigma; \Gamma, z : \{z \Rightarrow \overline{\sigma}_1\} \vdash \overline{\sigma}_1 <:\; \overline{\sigma}_2}
	{A, \Sigma; \Gamma \vdash \{z \Rightarrow \overline{\sigma}_1\}\; <:\; \{z \Rightarrow \overline{\sigma}_2\}}
	\quad (\textsc {S-Struct})
	\and
\inferrule
  {A, \Sigma; \Gamma \vdash x \ni \texttt{type} \; L : \tau_1 .. \tau_2\\
   A, \Sigma; \Gamma \vdash x' \ni \texttt{type} \; L : \tau_1' .. \tau_2'\\
   A, \Sigma; \Gamma \vdash \tau_1' <: \tau_1, \tau_2 <: \tau_2'}
  {A; \Sigma; \Gamma \vdash x.L <: x'.L}
  \quad (\textsc {S-Path})
	\and
\inferrule
	{A, \Sigma; \Gamma \vdash x \ni \texttt{type} \; L : \tau_1 .. \tau_2\\
	 A, \Sigma; \Gamma \vdash \tau_2 <: \tau}
	{A, \Sigma; \Gamma \vdash x.L\; <:\; \tau}
	\quad (\textsc {S-Select-Upper})
	\and
\inferrule
	{A, \Sigma; \Gamma \vdash x \ni \texttt{type} \; L : \tau_1 .. \tau_2\\
	 A, \Sigma; \Gamma \vdash \tau <: \tau_1}
	{A, \Sigma; \Gamma \vdash \tau \; <:\; x.L}
	\quad (\textsc {S-Select-Lower})
	\and
\inferrule
	{A, \Sigma; \Gamma \vdash \tau \; \texttt{<:}\; \tau_1}
	{A, \Sigma; \Gamma \vdash \tau \; \texttt{<:}\; \tau_1 \vee \tau_2}
	\quad (\textsc {S-Union-L})
	\and
\inferrule
	{A, \Sigma; \Gamma \vdash \tau \; \texttt{<:}\; \tau_2}
	{A, \Sigma; \Gamma \vdash \tau \; \texttt{<:}\; \tau_1 \vee \tau_2}
	\quad (\textsc {S-Union-R})
	\and
\inferrule
	{A, \Sigma; \Gamma \vdash \tau \; \texttt{<:}\; \tau_1 \\
	 A, \Sigma; \Gamma \vdash \tau \; \texttt{<:}\; \tau_2}
	{A, \Sigma; \Gamma \vdash \tau \; \texttt{<:}\; \tau_1 \wedge \tau_2}
	\quad (\textsc {S-Intersection})
	\and
\inferrule
	{}
	{A, \Sigma; \Gamma \vdash \tau \; \texttt{<:}\; \top}
	\quad (\textsc {S-Top})
	\and
\inferrule
	{}
	{A, \Sigma; \Gamma \vdash \bot \; \texttt{<:}\; \tau}
	\quad (\textsc {S-Bottom})
\end{mathpar}
\hfill \fbox{$A; \Sigma; \Gamma \vdash \sigma <: \sigma'$}
\begin{mathpar}
\inferrule
	{}
	{A; \Sigma; \Gamma \vdash \texttt{val} \; f:T <: \texttt{val} \; f:T}
	\quad (\textsc {S-Decl-Val})
	\and
\inferrule
	{A; \Sigma; \Gamma \vdash S' <: S \\
	 A; \Sigma; \Gamma \vdash T <: T'}
	{A; \Sigma; \Gamma \vdash \texttt{def} \; m:S \rightarrow T <: \texttt{def} \; m:S' \rightarrow T'}
	\quad (\textsc {S-Decl-Def})
	\and
\inferrule
	{A; \Sigma; \Gamma \vdash S' <: S \\
	 A; \Sigma; \Gamma \vdash U <: U'}
	{A; \Sigma; \Gamma \vdash \texttt{type} \; L : S .. U \; <:\; \texttt{type} \; L : S' .. U'}
	\quad (\textsc {S-Decl-Type})
\end{mathpar}
\caption{Subtyping}
\label{f:subtype}
\end{figure}

\begin{figure}[h]
\hfill \fbox{$A; \Sigma; \Gamma \vdash T \;  \textbf{wf}$}
\begin{mathpar}
\inferrule
  {A; \Sigma; \Gamma \vdash p \ni \texttt{type} \; L : S .. U \\
  	A; \Sigma; \Gamma \vdash \texttt{type} \; L : S .. U \; \textbf{wf} }
  {A; \Sigma; \Gamma \vdash p.L \; \textbf{wf}}
  \quad (\textsc {WF-Sel})
	\and
\inferrule
  {A; \Sigma; \Gamma,z:\{z \Rightarrow \overline{\sigma}\} \vdash \overline{\sigma} \; \textbf{wf} \\
  	\forall j \neq i, \; dom(\sigma_j) \neq dom(\sigma_i)}
  {A; \Gamma; \Sigma \vdash \{z \Rightarrow \overline{\sigma}\} \; \textbf{wf}}
  \quad (\textsc {WF-Struct})
%	\and
%\inferrule
%  {\Gamma \vdash S \;  \textbf{wf} \\
%  	\Gamma \vdash T \;  \textbf{wf}}
%  {\Gamma \vdash S \rightarrow T \;  \textbf{wf}}
%  \quad (\textsc {WF-Func})
	\and
\inferrule
  {}
  {A; \Sigma; \Gamma \vdash \top \;  \textbf{wf}}
  \quad (\textsc {WF-Top})
	\and
\inferrule
  {}
  {A; \Sigma; \Gamma \vdash \bot \;  \textbf{wf}}
  \quad (\textsc {WF-Bottom})
\end{mathpar}
\hfill \fbox{$A; \Sigma; \Gamma \vdash \sigma \;  \textbf{wf}$}
\begin{mathpar}
\inferrule
  {A; \Sigma; \Gamma \vdash T : \textbf{wf}}
  {A; \Sigma; \Gamma \vdash \texttt{val} \; f:T \;  \textbf{wf}}
  \quad (\textsc {WF-Val})
	\and
\inferrule
  {A; \Sigma; \Gamma \vdash T : \textbf{wf} \\
  	A; \Sigma; \Gamma \vdash S : \textbf{wf}}
  {A; \Sigma; \Gamma \vdash \texttt{def} \; m:S \rightarrow T \;  \textbf{wf}}
  \quad (\textsc {WF-Def})
	\and
\inferrule
  {A; \Sigma; \Gamma \vdash S : \textbf{wfe} \; \vee \; S = \bot\\
  	A; \Sigma; \Gamma \vdash U : \textbf{wfe} \\
  	A; \Sigma; \Gamma \vdash S <: U}
  {A; \Sigma; \Gamma \vdash \texttt{type} \; L : S .. U \; \textbf{wf}}
  \quad (\textsc {WF-Type})
\end{mathpar}
\hfill \fbox{$A; \Sigma \vdash \Gamma \;  \textbf{wf}$}
\begin{mathpar}
\inferrule
  {\forall x \in dom(\Gamma), A; \Sigma; \Gamma \vdash \Gamma(x) \; \textbf{wf}}
  {\Sigma \vdash \Gamma \;  \textbf{wf}}
  \quad (\textsc {WF-Environment})
\end{mathpar}
\hfill \fbox{$\Sigma \;  \textbf{wf}$}
\begin{mathpar}
\inferrule
  {\forall l \in dom(\Sigma), \varnothing; \Sigma; \varnothing \vdash \Sigma(l) \; \textbf{wf}}
  {\Sigma \;  \textbf{wf}}
  \quad (\textsc {WF-Store-Context})
\end{mathpar}
\begin{mathpar}
\inferrule
  {\forall l \in dom(\mu), \varnothing; \Sigma; \varnothing \vdash \mu(l) : \Sigma(l)}
  {\mu : \Sigma}
  \quad (\textsc {WF-Store})
\end{mathpar}
\caption{Well-Formedness}
\label{f:wf}
\end{figure}

\begin{figure}[h]
\hfill \fbox{$A; \Sigma; \Gamma \vdash T \;  \textbf{wfe}$}
\begin{mathpar}
\inferrule
  {A; \Sigma; \Gamma \vdash T \; \textbf{wf} \\
  	A; \Sigma; \Gamma \vdash T \prec \overline{\sigma}}
  {A; \Sigma; \Gamma \vdash T \; \textbf{wfe}}
  \quad (\textsc {WFE})
\end{mathpar}
\caption{Well-Formed and Expanding Types}
\label{f:wfe}
\end{figure}

\begin{figure}[h]
\hfill \fbox{$A; \Sigma; \Gamma \vdash T \prec \overline{\sigma}$}
\begin{mathpar}
\inferrule
  {}
  {A; \Sigma; \Gamma \vdash \{z \Rightarrow \overline{\sigma}\} \prec_z \overline{\sigma}}
  \quad (\textsc {E-Struct})
	\and
\inferrule
  {A; \Sigma; \Gamma \vdash p \ni \texttt{type} \; L : S..U \\
  	A; \Sigma; \Gamma \vdash U \prec_z \overline{\sigma}}
  {A; \Sigma; \Gamma \vdash p.L \prec_z [z \unlhd U/z]\overline{\sigma}}
  \quad (\textsc {E-Sel})
%	\and
%\inferrule
%  {A; \Sigma; \Gamma \vdash T_1 \prec_z \overline{\sigma}_1 \\
%   A; \Sigma; \Gamma \vdash T_2 \prec_z \overline{\sigma}_2}
%  {A; \Sigma; \Gamma \vdash T_1 \wedge T_2 \prec_z \overline{\sigma}_1 \wedge\overline{\sigma}_2}
%  \quad (\textsc {E-Int})
%	\and
%\inferrule
%  {A; \Sigma; \Gamma \vdash T_1 \prec_z \overline{\sigma}_1 \\
%   A; \Sigma; \Gamma \vdash T_2 \prec_z \overline{\sigma}_2}
%  {A; \Sigma; \Gamma \vdash T_1 \vee T_2 \prec_z \overline{\sigma}_1 \vee \overline{\sigma}_2}
%  \quad (\textsc {E-Union})
	\and
\inferrule
  {}
  {A; \Sigma; \Gamma \vdash \top \prec_z \varnothing}
  \quad (\textsc {E-Top})
\end{mathpar}
\caption{Expansion}
\label{f:exp}
\end{figure}
\begin{figure}[h]
\hfill \fbox{$A; \Sigma; \Gamma \vdash e \ni \sigma$}
\begin{mathpar}
\inferrule
  {A; \Sigma; \Gamma \vdash p : T \\
  	A; \Sigma; \Gamma \vdash T \prec_z \overline{\sigma}\\
  	\sigma_i \in \overline{\sigma}}
  {A, \Sigma; \Gamma \vdash p \ni [p/z]\sigma_i}
  \quad (\textsc {M-Path})
	\and
\inferrule
  {A; \Sigma; \Gamma \vdash e : T \\
  	A; \Sigma; \Gamma \vdash T \prec_z \overline{\sigma}\\
  	\sigma_i \in \overline{\sigma} \\
  	z \notin \sigma_i}
  {A; \Sigma; \Gamma \vdash e \ni \sigma_i}
  \quad (\textsc {M-Exp})
\end{mathpar}
\caption{Membership}
\label{f:mem}
\end{figure}

%\subsubsection{Typing}

\begin{figure}[h]
\hfill \fbox{$A; \Sigma; \Gamma \vdash e:T$}
\begin{mathpar}
\inferrule
  {x \in dom(\Gamma)}
  {	A; \Sigma; \Gamma \vdash x : \Gamma(x)}
  \quad (\textsc {T-Var})
	\and
\inferrule
  {	l \in dom(\Sigma)}
  {	A; \Sigma; \Gamma \vdash l : \Sigma(l)}
  \quad (\textsc {T-Loc})
	\and
\inferrule
  {A; \Sigma; \Gamma, z : \{z \Rightarrow \overline{\sigma}\} 
  \vdash \overline{d} : \overline{\sigma}}
  {A; \Sigma; \Gamma \vdash \texttt{new} \; \{z \Rightarrow \overline{d}\} : 
  \{z \Rightarrow \overline{\sigma}\}}
  \quad (\textsc {T-New})
	\and
\inferrule
  {A; \Sigma; \Gamma \vdash e_0 \ni \texttt{def} \; m:S \rightarrow T \\
  	A; \Sigma; \Gamma \vdash e_0 : T_0 \\
  	A; \Sigma; \Gamma \vdash e_1 : S' \\
  	A; \Sigma; \Gamma \vdash S' <: S}
  {A; 	\Sigma; \Gamma \vdash e_0.m(e_1) : T}
  \quad (\textsc {T-Meth})
	\and
\inferrule
  {	A; \Sigma; \Gamma \vdash e : S \\
  	A; \Sigma; \Gamma \vdash e \ni \texttt{val} \; f:T}
  {	A; \Sigma; \Gamma \vdash e.f : T}
  \quad (\textsc {T-Acc})
	\and
\inferrule
  {A; \Sigma; \Gamma \vdash e : S \\
   A; \Sigma; \Gamma \vdash S <: T}
  {A; \Sigma; \Gamma \vdash e \unlhd T : T}
  \quad (\textsc {T-Type})
\end{mathpar}
\caption{Expression Typing}
\label{f:e_typ}
\end{figure}
\begin{figure}[h]
\hfill \fbox{$A; \Gamma; E \vdash d:\sigma$}
\begin{mathpar}
\inferrule
  {A; \Sigma; \Gamma \vdash p : T' \\
   A; \Sigma; \Gamma\vdash T' <: T \dashv \Gamma}
  {A; \Sigma; \Gamma \vdash \texttt{val} \; f:T = p : \texttt{val} \; f:T}
  \quad (\textsc {T-Decl-Var})
	\and
\inferrule
  {A; \Sigma; \Gamma, x : S \vdash e_0 : T' \\
   A; \Sigma; \Gamma, x : S \vdash T' <: T \dashv \Gamma, x : S}
  {A; \Sigma; \Gamma \vdash \texttt{def} \; m(x:S) = e : T : \texttt{def} \; m:S \rightarrow T}
  \quad (\textsc {T-Decl-Def})
	\and
\inferrule
  {A; \Sigma; \Gamma \vdash \texttt{type} \; L : S .. U \; \textbf{wf} }
  {A; \Sigma; \Gamma \vdash \texttt{type} \; L : S .. U : \texttt{type} \; L : S .. U}
  \quad (\textsc {T-Decl-Type})
\end{mathpar}
\caption{Declaration Typing}
\label{f:d_typ}
\end{figure}
%\begin{figure}[h]
%\hfill \fbox{$\Gamma \vdash \mu:\Sigma$}
%\begin{mathpar}
%\inferrule
%  {\forall x \in dom(\mu), \; \mu(x)= \{\texttt{z} \Rightarrow \overline{d}\} \\
%  	\Gamma(x) = \{\texttt{z} \Rightarrow \overline{\sigma}\} \\
%  	\Gamma \vdash \overline{d} : \overline{\sigma}}
%  {\Gamma \vdash \mu}
%  \quad (\textsc {T-Store})
%\end{mathpar}
%\caption{Store Typing}
%\label{f:s_typ}
%\end{figure}

\begin{figure}[h]
\hfill \fbox{$\mu \; | \; e \; \rightarrow \mu' \; | \; e'$}
\begin{mathpar}
\inferrule
  {l \notin dom(\mu) \\
  	\mu' = \mu, l \mapsto \{\texttt{z} \Rightarrow \overline{d_v}\}}
  {\mu \; | \; \texttt{new} \; \{\texttt{z} \Rightarrow \overline{d_v}\} \; \rightarrow \mu' \; | \; l}
  \quad (\textsc {R-New})
  \and
\inferrule
  {\mu : \Sigma \\
   \mu; \Sigma \vdash v_1 \leadsto_{m(v_2)} e}
  {\mu \; | \; v_1.m(v_2) \;\rightarrow \mu \; | \; e}
  \quad (\textsc {R-Meth})
  \and
\inferrule
  {\mu : \Sigma \\
   \mu; \Sigma \vdash v_1 \leadsto_{f} v_2}
  {\mu \; | \; v_1.f \;\rightarrow \mu \; | \; v_2}
  \quad (\textsc {R-Field})
  \and
\inferrule
  {	\mu \; | \; e \; \rightarrow \; \mu' \; | \; e'}
  {\mu \; | \; E[e] \; \rightarrow \mu' \; | \; E[e']}
  \quad (\textsc {R-Context})
\end{mathpar}
\caption{Reduction}
\label{f:red}
\end{figure}




\bibliographystyle{plain}
\bibliography{bib}

\end{document}
\section{Soundness}

\subsection{Transitivity}

Once environment narrowing, fully bounded types and intersection types have been removed, transitivity is relatively easy to achieve.

\begin{theorem}[Transitivity]
If $\Gamma_1 \vdash \tau_1 <: \tau_2 \dashv \Gamma_2$ and $\Gamma_2 \vdash \tau_2 <: \tau_3 \dashv \Gamma_3$ then $\Gamma_1 \vdash \tau_1 <: \tau_3 \dashv \Gamma_3$
\end{theorem}
\begin{proof}
We proceed by structural induction on $\tau_2$.
\begin{case}[$\tau_2 = \{z \Rightarrow \overline{\sigma}\}$]
By inversion on the derivation of $\Gamma_1 \vdash \tau_1 <: \tau_2 \dashv \Gamma_2$  and subsequently the derivation of $\Gamma_2 \vdash \tau_2 <: \tau_3 \dashv \Gamma_3$ ...
\begin{subcase}[\textsc{S-Bottom}: $\tau_1 = \bot$]
Trivial.
\end{subcase}
\begin{subcase}[\textsc{S-Top}: $\tau_3 = \top$]
Trivial.
\end{subcase}
\begin{subcase}[$\tau_1 = \{z \Rightarrow \overline{\sigma}\}_1$, $\tau_3 = \{z \Rightarrow \overline{\sigma}\}_3$]
By the inductive hypothesis.
\end{subcase}
\begin{subcase}[$\tau_1 = p.L_1$, $\tau_2 = p.L_3$]
It is easy to demostrate that there exists some type $\{z \Rightarrow \overline{\sigma}\}_1$ such that 
$\forall \Gamma \; \tau, \Gamma_1 \vdash \tau_1 <: \tau \dashv \Gamma$\ iff\ $\Gamma_1 \vdash \{z \Rightarrow \overline{\sigma}\}_1 <: \tau \dashv \Gamma$. This is similarly the case for $p.L_3$, only for the lower bound. Then by the inductive hypothesis we get the desired result.
\end{subcase}
\end{case}
\begin{case}[$\tau_2 = p.L$]
T
\end{case}
\end{proof}


\subsection{Preservation}




\bibliographystyle{ACM-Reference-Format}
\bibliography{sample-bibliography}

\end{document}
