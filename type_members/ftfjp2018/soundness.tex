\section{Soundness}

\subsection{Transitivity}

Once environment narrowing, fully bounded types and intersection types have been removed, transitivity is relatively easy to achieve.

\begin{theorem}[Transitivity]
If $\Gamma_1 \vdash \tau_1 <: \tau_2 \dashv \Gamma_2$ and $\Gamma_2 \vdash \tau_2 <: \tau_3 \dashv \Gamma_3$ then $\Gamma_1 \vdash \tau_1 <: \tau_3 \dashv \Gamma_3$
\end{theorem}
\begin{proof}
We proceed by structural induction on $\tau_2$.
\begin{case}[$\tau_2 = \{z \Rightarrow \overline{\sigma}\}$]
By inversion on the derivation of $\Gamma_1 \vdash \tau_1 <: \tau_2 \dashv \Gamma_2$  and subsequently the derivation of $\Gamma_2 \vdash \tau_2 <: \tau_3 \dashv \Gamma_3$ ...
\begin{subcase}[\textsc{S-Bottom}: $\tau_1 = \bot$]
Trivial.
\end{subcase}
\begin{subcase}[\textsc{S-Top}: $\tau_3 = \top$]
Trivial.
\end{subcase}
\begin{subcase}[$\tau_1 = \{z \Rightarrow \overline{\sigma}\}_1$, $\tau_3 = \{z \Rightarrow \overline{\sigma}\}_3$]
By the inductive hypothesis.
\end{subcase}
\begin{subcase}[$\tau_1 = p.L_1$, $\tau_2 = p.L_3$]
It is easy to demostrate that there exists some type $\{z \Rightarrow \overline{\sigma}\}_1$ such that 
$\forall \Gamma \; \tau, \Gamma_1 \vdash \tau_1 <: \tau \dashv \Gamma$\ iff\ $\Gamma_1 \vdash \{z \Rightarrow \overline{\sigma}\}_1 <: \tau \dashv \Gamma$. This is similarly the case for $p.L_3$, only for the lower bound. Then by the inductive hypothesis we get the desired result.
\end{subcase}
\end{case}
\begin{case}[$\tau_2 = p.L$]
T
\end{case}
\end{proof}


\subsection{Preservation}


