\documentclass{llncs}

\usepackage{listings}
\usepackage{amssymb}
\usepackage[margin=.9in]{geometry}
\usepackage{amsmath}
%\usepackage{amsthm}
\usepackage{mathpartir}
\usepackage{color,soul}


\newtheorem{subcase}{SubCase}
\numberwithin{subcase}{case}
\numberwithin{case}{theorem}
\numberwithin{case}{lemma}




\lstdefinestyle{custom_lang}{
  xleftmargin=\parindent,
  showstringspaces=false,
  basicstyle=\ttfamily,
  keywordstyle=\bfseries
}

\lstset{emph={%  
    val, def, type, new, z%
    },emphstyle={\bfseries \tt}%
}

\begin{document}

\section{Syntax}

\begin{figure}[h]
\[
\begin{array}{lllll}
\begin{array}{llr}
t 		& ::= 														&\text{\bf{Term}} \\
		& x 														&\text{\emph{variable}}\\
		& \texttt{new} \ \{z \Rightarrow \overline{d}\}	&\text{\emph{new instance}}\\
		& \lambda x:\tau.t 										&\text{\emph{lambda}}\\
		& t\ t 													&\text{\emph{application}}\\
		& t.m 														&\text{\emph{method invocation}}\\
		& t \unlhd \tau 										&\text{\emph{upcast}}\\
		& l  														&\text{\emph{location}}\\
&&\\
v 		& ::= 														&\text{\bf{Value}} \\
		& l 														&\\
		& \lambda x:\tau.t										&\\
		& v \unlhd \tau 										&\\
&&\\
p 		& ::= 														&\text{\bf{Path}} \\
		& x 														&\\
		& l 														&\\
		& p \unlhd \tau 										&\\
 \end{array}
& ~~~~~~
&
\begin{array}{llr}
\tau 	& ::= 														&\text{\bf{Type}} \\
		& \{z \Rightarrow \overline{\sigma}\}				&\text{\emph{record}}\\
		& \tau \rightarrow \tau^x 							&\text{\emph{arrow}}\\
		& p.L 														&\text{\emph{selection}}\\
		& \tau \wedge \tau 										&\text{\emph{intersection}}\\
		& \tau \vee \tau 										&\text{\emph{union}}\\
		& \top 													&\text{\emph{top}}\\
		& \bot 													&\text{\emph{bottom}}\\
&&\\
d 		& ::= 														&\text{\bf{Declaration}} \\
  		& \text{\texttt{def}} \ m : \tau = t 				&\text{\emph{method}}\\
  		& \text{\texttt{type}} \ L = \tau					&\text{\emph{type}}\\
&&\\
\sigma	& ::= 														&\text{\bf{Decl Type}}\\
		& \text{\texttt{def}} \ m:\tau						&\text{\emph{method}}\\
  		& \text{\texttt{type}} \ L : \tau \ldots  \tau			&\text{\emph{type}}\\
\end{array}
& ~~~~~~
&
\begin{array}{llr}
\Gamma	& :: = 													&\text{\bf{Environment}} \\
		& \varnothing 											&\\
		& \Gamma,\ x : \tau 									&\\
&&\\
\mu 	& :: = 													& \text{\bf{Store}}\\
		& \varnothing 											&\\
		& \mu,\ l \mapsto \{z \Rightarrow \overline{d}\}	&\\
&&\\
\Sigma	& :: = 													&\text{\bf{Store Type}}\\
		& \varnothing 											&\\
		& \Sigma,\ l : \{\texttt{z} \Rightarrow \overline{\sigma}\}	&\\
&&\\
E 		& :: = 													& \text{\bf{Eval Context}}\\
		& \bigcirc 												&\\
		& E.m 														&\\
		& E\ t 													&\\
		& p\ E 													&\\
		& E \unlhd \tau											&\\
\end{array}
\end{array}
\]
\caption{Syntax}
\label{f:syntax}
\end{figure}

%\begin{figure}[h]
%\[
%\begin{array}{lll}
%\begin{array}{llr}
%e 		& ::= 														&\text{\bf{Expression}} \\
%		& x 														&\text{\emph{variable}}\\
%		& \texttt{new} \ \{z \Rightarrow \overline{d}\}	&\text{\emph{new instance}}\\
%		& \lambda x:\tau.e 										&\text{\emph{lambda}}\\
%		& e\ e 													&\text{\emph{application}}\\
%		& e.m 														&\text{\emph{method invocation}}\\
%		& e \unlhd \tau 										&\text{\emph{upcast}}\\
%		& l  														&\text{\emph{location}}\\
%&&\\
%p 		& ::= 														&\text{\bf{Path}} \\
%		& x 														&\\
%		& l 														&\\
%		& p \unlhd \tau 										&\\
%&&\\
%v 		& ::= 														&\text{\bf{Value}} \\
%		& l 														&\\
%		& v \unlhd \tau 										&\\
%&&\\
%d 		& ::= 														&\text{\bf{Declaration}} \\
%  		& \text{\texttt{def}} \ m : \tau = e 				&\text{\emph{method}}\\
%  		& \text{\texttt{type}} \ L = \tau					&\text{\emph{type}}\\
%&&\\
%\sigma	& ::= 														&\text{\bf{Decl Type}}\\
%		& \text{\texttt{def}} \ m:\tau						&\text{\emph{method}}\\
%  		& \text{\texttt{type}} \ L : \tau \ldots  \tau			&\text{\emph{type}}\\
% \end{array}
%& ~~~~~~
%&
%\begin{array}{llr}
%\tau 	& ::= 														&\text{\bf{Type}} \\
%		& \{z \Rightarrow \overline{\sigma}\}				&\text{\emph{record}}\\
%		& (x: \tau) \rightarrow \tau 								&\text{\emph{arrow}}\\
%		& p.L 														&\text{\emph{selection}}\\
%		& \tau \wedge \tau 										&\text{\emph{intersection}}\\
%		& \tau \vee \tau 										&\text{\emph{union}}\\
%		& \top 													&\text{\emph{top}}\\
%		& \bot 													&\text{\emph{bottom}}\\
%&&\\
%E 		& :: = 													& \text{\bf{Eval Context}}\\
%		& \bigcirc 												&\\
%		& E.m 														&\\
%		& E\ e 													&\\
%		& p\ E 													&\\
%		& E \unlhd \tau											&\\
%&&\\
%\Gamma	& :: = 													&\text{\bf{Environment}} \\
%		& \varnothing 											&\\
%		& \Gamma,\ x : \tau 									&\\
%&&\\
%\mu 	& :: = 													& \text{\bf{Store}}\\
%		& \varnothing 											&\\
%		& \mu,\ l \mapsto \{z \Rightarrow \overline{d}\}	&\\
%&&\\
%\Sigma	& :: = 													&\text{\bf{Store Type}}\\
%		& \varnothing 											&\\
%		& \Sigma,\ l : \{\texttt{z} \Rightarrow \overline{\sigma}\}	&\\
%\end{array}
%\end{array}
%\]
%\caption{Syntax}
%\label{f:syntax}
%\end{figure}


%\begin{figure}[h]
%\begin{mathpar}
%\inferrule
%  {}
%  {label(\texttt{def} \, m : \tau) = m}
%  \and
%\inferrule
%  {}
%  {label(\texttt{type} \, L : \tau_1 \ldots  \tau_2) = L}
%\end{mathpar}
%\caption{Declaration Label Function}
%\label{f:label}
%\end{figure}


In Figure \ref{f:leadsto}:
\begin{itemize}
\item
$v = l$: return the location.
\item
$v = v' \unlhd \{\ldots\}$: find the structural equivalent path of $v'$ and keep the upcast.
\item
$v = v' \unlhd p.L$: if the type of $v'$ is not equivalent to $p.L$, then our structural equivalent path is just the structural equivalent of $v'$ upcast to the lower bound and then the upper bound.
\item
$v = v' \unlhd p.L$: if the type of $v'$ ($\tau$) is equivalent to $p.L$, then the structural equivalent path is the structural equivalent path of $v'$ upcast to the upper bound. This is because we know (by \textsc{S-Path}) that the upper bound of $p.L$ must be a super type of the upper bound of $\tau$.
\end{itemize}
Since in the two cases for selection types, the subderivations are not strictly smaller, the finiteness of this derivation is based upon the finiteness of the expansion of all types. (Problem for induction?)


\section{Semantics}


\begin{figure}[h]
\begin{mathpar}
\inferrule
  {}
  {p \equiv p}
  \quad (\textsc{Eq-Refl})
  \and
\inferrule
  {p_1 \equiv p_2}
  {p_2 \equiv p_1}
  \quad (\textsc{Eq-Sym})
  \and
\inferrule
  {p_1 \equiv p_2 \\
   p_2 \equiv p_3}
  {p_1 \equiv p_3}
  \quad (\textsc{Eq-Trans})
  \and
\inferrule
  {p_1 \equiv p_2}
  {p_1 \equiv p_2 \unlhd T}
  \quad (\textsc{Eq-Path})
\end{mathpar}
\caption{Path Equivalence}
\label{f:path_equiv}
\end{figure}


\begin{figure}[h]
\begin{mathpar}
\inferrule
  {
  \overline{\sigma}_1 \equiv \overline{\sigma}_2
  }
  {\{z \Rightarrow \overline{\sigma}_1\} \equiv \{z \Rightarrow \overline{\sigma}_2\}}
  \quad (\textsc{Eq-Struct})
  \and
\inferrule
  {
  	p_1 \equiv p_2
   }
  {p_1.L \equiv p_2.L}
  \quad (\textsc{Eq-Select})
  \and
\inferrule
  {}
  {\vdash \top \equiv \top}
  \quad (\textsc{Eq-Top})
  \and
\inferrule
  {}
  {\bot \equiv \bot}
  \quad (\textsc{Eq-Bottom})
\end{mathpar}
\caption{Type Path Equivalence}
\label{f:type_equiv}
\end{figure}

\begin{figure}[h]
\hfill \fbox{$\Sigma; \Gamma \vdash \tau_1 <: \tau_2$}
\begin{mathpar}
\inferrule
	{\Sigma; \Gamma, z : \{z \Rightarrow \overline{\sigma}_1\} \vdash \overline{\sigma}_1 <:\ [z \unlhd \{z \Rightarrow \overline{\sigma}_2\} / z]\overline{\sigma}_2}
	{\Sigma; \Gamma \vdash \{z \Rightarrow \overline{\sigma}_1\}\ <:\ \{z \Rightarrow \overline{\sigma}_2\}}
	\quad (\textsc {S-Struct})
	\and
\inferrule
	{\Sigma; \Gamma \vdash \tau_1'\ <:\ \tau_1 \\
	 \Sigma; \Gamma, x : \tau_1' \vdash [x \unlhd \tau_1/x]\tau_2\ <:\ \tau_2'}
	{\Sigma; \Gamma \vdash \tau_1 \rightarrow\ \tau_2^x <:\ \tau_1' \rightarrow\ \tau_2'^x}
	\quad (\textsc {S-Arrow})
	\and
\inferrule
	{p_1 \equiv p_2 \\
	 \Sigma; \Gamma \vdash p_1 \ni \text{\texttt{type}}\ L: \tau_1 \ldots \tau_1' \\
	 \Sigma; \Gamma \vdash p_2 \ni \text{\texttt{type}}\ L: \tau_2 \ldots \tau_2' \\
	 \Sigma; \Gamma \vdash \tau_2 <: \tau_1
	 \Sigma; \Gamma \vdash \tau_1' <: \tau_2'
	 }
	{\Sigma; \Gamma \vdash p_1.L\ <:\ p_2.L}
	\quad (\textsc {S-Path})
	\and
\inferrule
	{\Sigma; \Gamma \vdash p \ni \text{\texttt{type}} \ L : \tau_1 \ldots  \tau_2\\
	 \Sigma; \Gamma \vdash \tau_2 <: \tau}
	{\Sigma; \Gamma \vdash p.L\ <:\ \tau}
	\quad (\textsc {S-Select-Upper})
	\and
\inferrule
	{\Sigma; \Gamma \vdash p \ni \text{\texttt{type}} \ L : \tau_1 \ldots  \tau_2\\
	 \Sigma; \Gamma \vdash \tau <: \tau_1}
	{\Sigma; \Gamma \vdash \tau \ <:\ p.L}
	\quad (\textsc {S-Select-Lower})
	\and
\inferrule
	{\Sigma; \Gamma \vdash \tau_1\ <:\ \tau \\
	 \Sigma; \Gamma \vdash \tau_2\ <:\ \tau}
	{\Sigma; \Gamma \vdash \tau_1 \vee \tau_2\ <:\ \tau}
	\quad (\textsc {S-Union})
	\and
\inferrule
	{\Sigma; \Gamma \vdash \tau\ <:\ \tau_1}
	{\Sigma; \Gamma \vdash \tau\ <:\ \tau_1 \vee \tau_2}
	\quad (\textsc {S-Union-L})
	\and
\inferrule
	{\Sigma; \Gamma \vdash \tau\ <:\ \tau_2}
	{\Sigma; \Gamma \vdash \tau\ <:\ \tau_1 \vee \tau_2}
	\quad (\textsc {S-Union-R})
	\and
\inferrule
	{\Sigma; \Gamma \vdash \tau\ <:\ \tau_1 \\
	 \Sigma; \Gamma \vdash \tau\ <:\ \tau_2}
	{\Sigma; \Gamma \vdash \tau\ <:\ \tau_1 \wedge \tau_2}
	\quad (\textsc {S-Intersection})
	\and
\inferrule
	{\Sigma; \Gamma \vdash \tau_1\ <:\ \tau}
	{\Sigma; \Gamma \vdash \tau_1 \wedge \tau_2\ <:\ \tau}
	\quad (\textsc {S-Intersection-L})
	\and
\inferrule
	{\Sigma; \Gamma \vdash \tau_2\ <:\ \tau}
	{\Sigma; \Gamma \vdash \tau_1 \wedge \tau_2\ <:\ \tau}
	\quad (\textsc {S-Intersection-R})
	\and
\inferrule
	{}
	{\Sigma; \Gamma \vdash \tau\ <:\ \top}
	\quad (\textsc {S-Top})
	\and
\inferrule
	{}
	{\Sigma; \Gamma \vdash \bot\ <:\ \tau}
	\quad (\textsc {S-Bottom})
\end{mathpar}
\hfill \fbox{$\Sigma; \Gamma \vdash \sigma <: \sigma'$}
\begin{mathpar}
\inferrule
	{\Sigma; \Gamma \vdash \tau_1 <: \tau_2}
	{\Sigma; \Gamma \vdash \texttt{def} \ m:\tau_1 <: \texttt{def} \ m:\tau_2}
	\quad (\textsc {S-Decl-Def})
	\and
\inferrule
	{\Sigma; \Gamma \vdash \tau_1' <: \tau_1 \\
	 \Sigma; \Gamma \vdash \tau_2 <: \tau_2'}
	{\Sigma; \Gamma \vdash \text{\texttt{type}} \ L : \tau_1 \ldots \tau_2 \ <:\ \text{\texttt{type}} \ L : \tau_1' \ldots \tau_2'}
	\quad (\textsc {S-Decl-Type})
\end{mathpar}
\caption{Subtyping}
\label{f:subtype}
\end{figure}

%\begin{figure}[h]
%\hfill \fbox{$A; \Sigma; \Gamma \vdash T \  \textbf{wf}$}
%\begin{mathpar}
%\inferrule
%  {A; \Sigma; \Gamma \vdash p \ni \texttt{type} \ L : S \ldots  U \\
%  	A; \Sigma; \Gamma \vdash \texttt{type} \ L : S \ldots  U \ \textbf{wf} }
%  {A; \Sigma; \Gamma \vdash p.L \ \textbf{wf}}
%  \quad (\textsc {WF-Sel})
%	\and
%\inferrule
%  {A; \Sigma; \Gamma,z:\{z \Rightarrow \overline{\sigma}\} \vdash \overline{\sigma} \ \textbf{wf} \\
%  	\forall j \neq i, \ dom(\sigma_j) \neq dom(\sigma_i)}
%  {A; \Gamma; \Sigma \vdash \{z \Rightarrow \overline{\sigma}\} \ \textbf{wf}}
%  \quad (\textsc {WF-Struct})
%%	\and
%%\inferrule
%%  {\Gamma \vdash S \  \textbf{wf} \\
%%  	\Gamma \vdash T \  \textbf{wf}}
%%  {\Gamma \vdash S \rightarrow T \  \textbf{wf}}
%%  \quad (\textsc {WF-Func})
%	\and
%\inferrule
%  {}
%  {A; \Sigma; \Gamma \vdash \top \  \textbf{wf}}
%  \quad (\textsc {WF-Top})
%	\and
%\inferrule
%  {}
%  {A; \Sigma; \Gamma \vdash \bot \  \textbf{wf}}
%  \quad (\textsc {WF-Bottom})
%\end{mathpar}
%\hfill \fbox{$A; \Sigma; \Gamma \vdash \sigma \  \textbf{wf}$}
%\begin{mathpar}
%\inferrule
%  {A; \Sigma; \Gamma \vdash T : \textbf{wf}}
%  {A; \Sigma; \Gamma \vdash \texttt{val} \ f:T \  \textbf{wf}}
%  \quad (\textsc {WF-Val})
%	\and
%\inferrule
%  {A; \Sigma; \Gamma \vdash T : \textbf{wf} \\
%  	A; \Sigma; \Gamma \vdash S : \textbf{wf}}
%  {A; \Sigma; \Gamma \vdash \texttt{def} \ m:S \rightarrow T \  \textbf{wf}}
%  \quad (\textsc {WF-Def})
%	\and
%\inferrule
%  {A; \Sigma; \Gamma \vdash S : \textbf{wfe} \ \vee \ S = \bot\\
%  	A; \Sigma; \Gamma \vdash U : \textbf{wfe} \\
%  	A; \Sigma; \Gamma \vdash S <: U}
%  {A; \Sigma; \Gamma \vdash \texttt{type} \ L : S \ldots  U \ \textbf{wf}}
%  \quad (\textsc {WF-Type})
%\end{mathpar}
%\hfill \fbox{$A; \Sigma \vdash \Gamma \  \textbf{wf}$}
%\begin{mathpar}
%\inferrule
%  {\forall x \in dom(\Gamma), A; \Sigma; \Gamma \vdash \Gamma(x) \ \textbf{wf}}
%  {\Sigma \vdash \Gamma \  \textbf{wf}}
%  \quad (\textsc {WF-Environment})
%\end{mathpar}
%\hfill \fbox{$\Sigma \  \textbf{wf}$}
%\begin{mathpar}
%\inferrule
%  {\forall l \in dom(\Sigma), \varnothing; \Sigma; \varnothing \vdash \Sigma(l) \ \textbf{wf}}
%  {\Sigma \  \textbf{wf}}
%  \quad (\textsc {WF-Store-Context})
%\end{mathpar}
%\begin{mathpar}
%\inferrule
%  {\forall l \in dom(\mu), \varnothing; \Sigma; \varnothing \vdash \mu(l) : \Sigma(l)}
%  {\mu : \Sigma}
%  \quad (\textsc {WF-Store})
%\end{mathpar}
%\caption{Well-Formedness}
%\label{f:wf}
%\end{figure}
%
%\begin{figure}[h]
%\hfill \fbox{$\Sigma; \Gamma \vdash T \  \textbf{wfe}$}
%\begin{mathpar}
%\inferrule
%  {\Sigma; \Gamma \vdash T \ \textbf{wf} \\
%  	\Sigma; \Gamma \vdash T \prec \overline{\sigma}}
%  {\Sigma; \Gamma \vdash T \ \textbf{wfe}}
%  \quad (\textsc {WFE})
%\end{mathpar}
%\caption{Well-Formed and Expanding Types}
%\label{f:wfe}
%\end{figure}

\begin{figure}[h]
\hfill \fbox{$\Sigma; \Gamma \vdash \sigma^z \in \tau$}
\begin{mathpar}
\inferrule
	{\sigma \in \overline{\sigma}}
	{\Sigma; \Gamma \vdash \sigma^z \in \{z \Rightarrow \overline{\sigma}\}}
	\quad (\textsc {MT-Struct})
	\and
\inferrule
	{\Sigma; \Gamma \vdash p \ni \text{\texttt{type}} \ L : \tau_1 \ldots \tau_2 \\
	 \Sigma; \Gamma \vdash \sigma^z \in \tau_2}
	{\Sigma; \Gamma \vdash [z \unlhd \tau_2/z]\sigma \in p.L}
	\quad (\textsc {MT-Sel})
	\and
\inferrule
	{\Sigma; \Gamma \vdash \sigma_1^z \in \tau_1 \\
	 \Sigma; \Gamma \vdash \sigma_2^z \not\in \tau_2}
	{\Sigma; \Gamma \vdash \sigma_1^z \in \tau_1 \wedge \tau_2}
	\quad (\textsc {MT-Int-L})
	\and
\inferrule
	{\Sigma; \Gamma \vdash \sigma_1^z \not\in \tau_1 \\
	 \Sigma; \Gamma \vdash \sigma_2^z \in \tau_2}
	{\Sigma; \Gamma \vdash \sigma_2^z \in \tau_1 \wedge \tau_2}
	\quad (\textsc {MT-Int-L})
	\and
\inferrule
	{\Sigma; \Gamma \vdash \sigma_1^z \in \tau_1 \\
	 \Sigma; \Gamma \vdash \sigma_2^z \in \tau_2}
	{\Sigma; \Gamma \vdash \sigma_1^z \wedge \sigma_2^z \in \tau_1 \wedge \tau_2}
	\quad (\textsc {MT-Int-LR})
	\and
\inferrule
	{\Sigma; \Gamma \vdash \sigma_1^z \in \tau_1 \\
	 \Sigma; \Gamma \vdash \sigma_2^z \in \tau_2}
	{\Sigma; \Gamma \vdash \sigma_1^z \vee \sigma_2^z \in \tau_1 \vee \tau_2 }
	\quad (\textsc {MT-Union})
\end{mathpar}
\caption{Type Membership}
\label{f:exp}
\end{figure}
\begin{figure}[h]
\hfill \fbox{$\Sigma; \Gamma \vdash t \ni \sigma$}
\begin{mathpar}
\inferrule
  {\Sigma; \Gamma \vdash p : \tau \\
  	\Sigma; \Gamma \vdash \sigma^z \in \tau}
  {\Sigma; \Gamma \vdash p \ni [p/z]\sigma}
  \quad (\textsc {M-Path})
	\and
\inferrule
  {\Sigma; \Gamma \vdash t : \tau \\
  	\Sigma; \Gamma \vdash \sigma^z \in \tau \\
  	z \notin \sigma}
  {\Sigma; \Gamma \vdash t \ni \sigma}
  \quad (\textsc {M-Exp})
\end{mathpar}
\caption{Term Membership}
\label{f:mem}
\end{figure}

%\subsubsection{Typing}

\begin{figure}[h]
\hfill \fbox{$\Sigma; \Gamma \vdash e:T$}
\begin{mathpar}
\inferrule
  {x \in dom(\Gamma)}
  {\Sigma; \Gamma \vdash x : \Gamma(x)}
  \quad (\textsc {T-Var})
	\and
\inferrule
  {	l \in dom(\Sigma)}
  {\Sigma; \Gamma \vdash l : \Sigma(l)}
  \quad (\textsc {T-Loc})
	\and
\inferrule
  {	\Sigma; \Gamma, x : \tau_1 \vdash t : \tau_2}
  {\Sigma; \Gamma \vdash (\lambda x:\tau_1.t) : \tau_1 \rightarrow \tau_2^x}
  \quad (\textsc {T-Lambda})
	\and
\inferrule
  {	\Sigma; \Gamma \vdash t : (x:\tau_1) \rightarrow \tau_2 \\
   x \not\in \tau_2 \\
   \Sigma; \Gamma \vdash t' : \tau_1}
  {\Sigma; \Gamma \vdash t\ t': \tau_2}
  \quad (\textsc {T-App})
	\and
\inferrule
  {	\Sigma; \Gamma \vdash t : (x:\tau_1) \rightarrow \tau_2 \\
   \Sigma; \Gamma \vdash p : \tau_1}
  {\Sigma; \Gamma \vdash t\ p: [p/x]\tau_2}
  \quad (\textsc {T-App-Free})
	\and
\inferrule
  {\Sigma; \Gamma, z : \{z \Rightarrow \overline{\sigma}\} 
  \vdash \overline{d} : \overline{\sigma}}
  {\Sigma; \Gamma \vdash \texttt{new} \ \{z \Rightarrow \overline{d}\} : 
  \{z \Rightarrow \overline{\sigma}\}}
  \quad (\textsc {T-New})
	\and
\inferrule
  {\Sigma; \Gamma \vdash t \ni \texttt{def} \ m:\tau}
  {\Sigma; \Gamma \vdash t.m : \tau}
  \quad (\textsc {T-Meth})
	\and
\inferrule
  {\Sigma; \Gamma \vdash t : \tau' \\
   \Sigma; \Gamma \vdash \tau' <: \tau}
  {\Sigma; \Gamma \vdash t \unlhd \tau : \tau}
  \quad (\textsc {T-Cast})
\end{mathpar}
\caption{Term Typing}
\label{f:t_typ}
\end{figure}
\begin{figure}[h]
\hfill \fbox{$\Sigma; \Gamma \vdash d:\sigma$}
\begin{mathpar}
\inferrule
  {\Sigma; \Gamma \vdash t : \tau}
  {\Sigma; \Gamma \vdash \texttt{def} \ m: \tau = t : \texttt{def} \ m:\tau}
  \quad (\textsc {T-Decl-Def})
	\and
\inferrule
  {}
  {\Sigma; \Gamma \vdash \text{\texttt{type}} \ L = \tau : \text{\texttt{type}} \ L : \tau \ldots \tau}
  \quad (\textsc {T-Decl-Type})
\end{mathpar}
\caption{Declaration Typing}
\label{f:d_typ}
\end{figure}

%\begin{figure}[h]
%\hfill \fbox{$\mu; \Sigma \vdash v \leadsto l$}
%\begin{mathpar}
%\inferrule
%  {}
%  {\mu; \Sigma \vdash l \leadsto l }
%  \quad (\textsc {P-Loc})
%	\and
%\inferrule
%  {\mu; \Sigma \vdash v \leadsto v'}
%  {\mu; \Sigma \vdash v \unlhd T \leadsto v' \unlhd T}
%  \quad (\textsc {L-Type})
%	\and
%\inferrule
%  {\mu; \Sigma \vdash v \leadsto v' \\
%   \mu; \Sigma \vdash v' \leadsto_{f} v_f}
%  {\mu; \Sigma \vdash v.f \leadsto v_f}
%  \quad (\textsc {L-Path})
%\end{mathpar}
%\hfill \fbox{$\mu; \Sigma \vdash d_v \leadsto d$}
%\begin{mathpar}
%\inferrule
%  {\mu; \Sigma \vdash v \leadsto v'}
%  {\mu; \Sigma \vdash \texttt{val} \ f : T = v \leadsto \texttt{def} \ m : S \rightarrow T}
%  \quad (\textsc {L-Val})
%	\and
%\inferrule
%  {}
%  {\mu; \Sigma \vdash \texttt{def} \ m : S(x:T) = e \leadsto \texttt{def} \ m(x:S) = e : T}
%  \quad (\textsc {L-Def})
%	\and
%\inferrule
%  {}
%  {\mu; \Sigma \vdash \texttt{type} \ L : S \ldots  U \leadsto \texttt{type} \ L : S \ldots  U}
%  \quad (\textsc {L-Type})
%\end{mathpar}
%\caption{Path Leads-to Judgement}
%\label{f:path}
%\end{figure}
%
%\begin{figure}[h]
%\hfill \fbox{$\mu; \Sigma \vdash v \leadsto_{f} v$}
%\begin{mathpar}
%\inferrule
%  {}
%  {\mu; \Sigma \vdash l \leadsto l }
%  \quad (\textsc {P-Loc})
%	\and
%\inferrule
%  {\mu; \Sigma \vdash v \leadsto v'}
%  {\mu; \Sigma \vdash v \unlhd T \leadsto v' \unlhd T}
%  \quad (\textsc {L-Type})
%	\and
%\inferrule
%  {\mu; \Sigma \vdash v \leadsto v' \\
%   \mu; \Sigma \vdash v' \leadsto_{f} v_f}
%  {\mu; \Sigma \vdash v.f \leadsto v_f}
%  \quad (\textsc {L-Path})
%\end{mathpar}
%\caption{Path Leads-to Judgement}
%\label{f:path}
%\end{figure}
%
% 
%
%\begin{figure}[h]
%\hfill \fbox{$\mu; \Sigma \vdash v_1 \leadsto v_2$}
%\begin{mathpar}
%\inferrule
%  {}
%  {\mu; \Sigma \vdash l \leadsto l}
%  \quad (\textsc {L-Loc})
%	\and
%\inferrule
%  {\mu; \Sigma \vdash {v_1}^{p'_z} \leadsto v_2}
%  {\mu; \Sigma \vdash v_1^{p'_z} \unlhd \{z \Rightarrow \overline{\sigma}\}_{p_z} \leadsto v_2 \unlhd \{z \Rightarrow [p_z/z]\overline{\sigma}\}}
%  \quad (\textsc {L-Type})
%	\and
%\inferrule
%  {\varnothing; \Sigma; \varnothing \vdash v_1 : p'.L \\
%  	\varnothing; \Sigma; \varnothing \vdash p \ni \texttt{type} \ L : S \ldots  U \\
%  	p \not\equiv p' \\
%  	\mu; \Sigma \vdash v_1 \unlhd S \unlhd U \leadsto v_2}
%  {\mu; \Sigma \vdash v_1 \unlhd p.L \leadsto v_2}
%  \quad (\textsc {L-Type-Select-Lower})
%	\and
%\inferrule
%  {\varnothing; \Sigma; \varnothing \vdash v_1 : p'.L \\
%  	\varnothing; \Sigma; \varnothing \vdash p \ni \texttt{type} \ L : S \ldots  U \\
%  	p \equiv p' \\
%  	\mu; \Sigma \vdash v_1 \unlhd U \leadsto v_2}
%  {\mu; \Sigma \vdash v_1 \unlhd p.L \leadsto v_2}
%  \quad (\textsc {L-Type-Select-Upper})
%%	\and
%%\inferrule
%%  {\varnothing; \Sigma; \varnothing \vdash v_1 : p'.L \\
%%   p' \equiv p \\
%%  	\varnothing; \Sigma; \varnothing \vdash p \ni \texttt{type} \ L : S \ldots  U \\
%%  	\mu; \Sigma \vdash v_1 \unlhd U \leadsto v_2}
%%  {\mu; \Sigma \vdash v_1 \unlhd p.L \leadsto v_2}
%%  \quad (\textsc {L-Type-Select-Refl})
%\end{mathpar}
%\caption{Leadsto Judgement}
%\label{f:leadsto}
%\end{figure}
%
%\begin{figure}[h]
%\hfill \fbox{$\mu; \Sigma \vdash v_1 \leadsto_{f} v_2$}
%\begin{mathpar}
%\inferrule
%  {\mu(l) = \{z \Rightarrow \ldots , \texttt{val} \ f : T = v, \ldots \}}
%  {\mu; \Sigma \vdash l \leadsto_{f} [l/z]v \unlhd T}
%  \quad (\textsc {L\textsubscript{$f$}-Loc})
%	\and
%\inferrule
%  {\mu; \Sigma \vdash v_1 \leadsto_{f} v_2 \\
%  \texttt{val} \ f:T \in \overline{\sigma}}
%  {\mu; \Sigma \vdash v_1 \unlhd \{z \Rightarrow \overline{\sigma}\} \leadsto_{f} v_2 \unlhd [v_1 \unlhd \{z \Rightarrow \overline{\sigma}\} / z]T}
%  \quad (\textsc {L\textsubscript{$f$}-Type})
%	\and
%\inferrule
%  {\mu; \Sigma \vdash v_1 \unlhd p.L \leadsto v_2 \\
%   \mu; \Sigma \vdash v_2 \leadsto_{f} v_3}
%  {\mu; \Sigma \vdash v_1 \unlhd p.L \leadsto_{f} v_3}
%  \quad (\textsc {L\textsubscript{$f$}-Type-Select})
%%	\and
%%\inferrule
%%  {\varnothing; \Sigma; \varnothing \vdash v_1 : T \\
%%  	\varnothing; \Sigma; \varnothing \vdash p \ni \texttt{type} \ L : S \ldots  U \\
%%  	\varnothing; \Sigma; \varnothing \not\vdash T <: U \\
%%  	\mu; \Sigma \vdash v_1 \unlhd S \unlhd U \leadsto v_2}
%%  {\mu; \Sigma \vdash v_1 \unlhd p.L \leadsto v_2}
%%  \quad (\textsc {L\textsubscript{$f$}-Type-Select-Lower})
%%	\and
%%\inferrule
%%  {\varnothing; \Sigma; \varnothing \vdash v_1 : T \\
%%  	\varnothing; \Sigma; \varnothing \vdash p \ni \texttt{type} \ L : S \ldots  U \\
%%  	\varnothing; \Sigma; \varnothing \vdash T <: U \\
%%  	\mu; \Sigma \vdash v_1 \unlhd U \leadsto v_2}
%%  {\mu; \Sigma \vdash v_1 \unlhd p.L \leadsto v_2}
%%  \quad (\textsc {L\textsubscript{$f$}-Type-Select-Upper})
%%	\and
%%\inferrule
%%  {\mu; \Sigma \vdash v_1 \leadsto_{f_1} v_2 \\
%%   v_2 \neq v_1.f_1 \\
%%  	\mu; \Sigma \vdash v_2 \leadsto_{f_2} v_3}
%%  {\mu; \Sigma \vdash v_1.f_1 \leadsto_{f_2} v3}
%%  \quad (\textsc {L\textsubscript{$f$}-Field})
%%	\and
%%\inferrule
%%  {\mu; \Sigma \vdash v_1 \leadsto_{f_1} v_1.f_1}
%%  {\mu; \Sigma \vdash v_1.f_1 \leadsto_{f_2} v_1.f_1.f_2}
%%  \quad (\textsc {L\textsubscript{$f$}-Field-Stop})
%\end{mathpar}
%\caption{Field Leadsto Judgement}
%\label{f:field_leadsto}
%\end{figure}
%
%\begin{figure}[h]
%\hfill \fbox{$\mu; \Sigma \vdash v_1 \leadsto_{m(v_2)} e$}
%\begin{mathpar}
%\inferrule
%  {\mu(l) = \{z \Rightarrow \ldots , \texttt{def} \ m (x : S) = e : T, \ldots \}}
%  {\mu; \Sigma \vdash l \leadsto_{m(v_2)} [v_2 \unlhd S/x, l/z]e \unlhd T}
%  \quad (\textsc {L\textsubscript{$m$}-Loc})
%	\and
%\inferrule
%  {\mu; \Sigma \vdash v_1 \leadsto_{m(v_2 \unlhd S)} e \\
%  \texttt{def} \ m : S \rightarrow T \in \overline{\sigma}}
%  {\mu; \Sigma \vdash v_1 \unlhd \{z \Rightarrow \overline{\sigma}\} \leadsto_{m(v_2)} e \unlhd T}
%  \quad (\textsc {L\textsubscript{$m$}-Type})
%	\and
%\inferrule
%  {\mu; \Sigma \vdash v_1 \unlhd p.L \leadsto v_2 \\
%   \mu; \Sigma \vdash v_2 \leadsto_{m(v_2)} v_3}
%  {\mu; \Sigma \vdash v_1 \unlhd p.L \leadsto_{m} v_3}
%  \quad (\textsc {L\textsubscript{$m$}-Type-Select})
%%	\and
%%\inferrule
%%  {\mu; \Sigma \vdash v_1 \leadsto_{f} v_2 \\
%%  	\mu; \Sigma \vdash v_2 \leadsto_{m} v_3}
%%  {\mu; \Sigma \vdash v_1.f \leadsto_{m} v3}
%%  \quad (\textsc {L\textsubscript{$m$}-Field})
%\end{mathpar}
%\caption{Method Leadsto Judgement}
%\label{f:meth_leadsto}
%\end{figure}

\begin{figure}[h]
\hfill \fbox{$\Sigma \vdash v \leadsto v'$}
\begin{mathpar}
\inferrule
	{}
	{\Sigma \vdash l \leadsto l}
	\quad{\textsc{L-Loc}}
	\and
\inferrule
	{\Sigma \vdash v \leadsto v'}
	{\Sigma \vdash v \unlhd \{z \Rightarrow \overline{\sigma}\} \leadsto v' \unlhd \{z \Rightarrow \overline{\sigma}\}}
	\quad{\textsc{L-Struct}}
	\and
\inferrule
	{\Sigma; \emptyset \vdash v : \tau \\
	 \tau \not\equiv p.L \\
	 \Sigma; \emptyset \vdash p \ni \text{\texttt{type}}\ L: \tau_1 \ldots \tau_2 \\
	 \Sigma \vdash v \unlhd \tau_1 \unlhd \tau_2 \leadsto v'}
	{\Sigma \vdash v \unlhd p.L \leadsto v'}
	\quad{\textsc{L-Select}}
	\and
\inferrule
	{\Sigma; \emptyset \vdash v : \tau \\
	 \tau \equiv p.L \\
	 \Sigma; \emptyset \vdash p \ni \text{\texttt{type}}\ L: \tau_1 \ldots \tau_2 \\
	 \Sigma \vdash v \unlhd \tau_2 \leadsto v'}
	{\Sigma \vdash v \unlhd p.L \leadsto v'}
	\quad{\textsc{L-Select-Path-Equiv}}
\end{mathpar}
\caption{Leadsto}
\label{f:leadsto}
\end{figure}

\begin{figure}[h]
\hfill \fbox{$\mu \vdash v \leadsto_m t$}
\begin{mathpar}
\inferrule
	{\mu(l) = \{z \Rightarrow \ldots, \text{\texttt{def}}\ m : \tau = t, \ldots\}}
	{\mu \vdash l \leadsto_m [l/z]t}
	\quad{\textsc{ML-Loc}}
	\and
\inferrule
	{\mu \vdash v \leadsto_m t \\
	 \text{\texttt{def}}\ m : \tau \in \overline{\sigma}}
	{\mu \vdash v \unlhd \{z \Rightarrow \overline{\sigma}\} \leadsto_m t \unlhd [v \unlhd \{z \Rightarrow \overline{\sigma}\}/z]\tau}
	\quad{\textsc{ML-Struct}}
\end{mathpar}
\caption{Method Leadsto}
\label{f:mleadsto}
\end{figure}

\begin{figure}[h]
\hfill \fbox{$v \leadsto_\lambda t^x$}
\begin{mathpar}
\inferrule
	{}
	{\lambda x:\tau.t \leadsto_\lambda ([x \unlhd \tau/x]t)^x}
	\quad{\lambda\textsc{L-Loc}}
	\and
\inferrule
	{v \leadsto_\lambda t^x}
	{v \unlhd \tau_1 \rightarrow \tau_2^x \leadsto_\lambda ([x \unlhd \tau_1/x]t \unlhd \tau_2)^x}
	\quad{\lambda\textsc{L-Struct}}
\end{mathpar}
\caption{Lambda Leadsto}
\label{f:lamleadsto}
\end{figure}

\begin{figure}[h]
\hfill \fbox{$\mu \ | \ t \ \rightarrow \mu' \ | \ t'$}
\begin{mathpar}
\inferrule
  {l \notin dom(\mu) \\
  	\mu' = \mu, l \mapsto \{\texttt{z} \Rightarrow \overline{d}\}}
  {\mu \ | \ \texttt{new} \ \{\texttt{z} \Rightarrow \overline{d}\} \ \rightarrow \mu' \ | \ l}
  \quad (\textsc {R-New})
  \and
\inferrule
  {\mu : \Sigma \\
   \Sigma \vdash v \leadsto v' \\ 
  	\mu \vdash v' \leadsto_m t}
  {\mu \ | \ v.m \rightarrow \mu \ | t}
  \quad (\textsc {R-Meth})
  \and
\inferrule
  {v_1 \leadsto_\lambda t^x}
  {\mu \ | \ v_1\ v_2 \rightarrow \mu \ | \ [v_2/x]t}
  \quad (\textsc {R-App})
  \and
\inferrule
  {	\mu \ | \ t \ \rightarrow \ \mu' \ | \ t'}
  {\mu \ | \ E[t] \ \rightarrow \mu' \ | \ E[t']}
  \quad (\textsc {R-Context})
\end{mathpar}
\caption{Term Reduction}
\label{f:red}
\end{figure}

\bibliographystyle{plain}
\bibliography{bib}

\end{document} 
