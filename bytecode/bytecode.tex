\documentclass{article}

\usepackage{listings}
\usepackage{proof}
\usepackage{amssymb}
\usepackage[margin=.5in]{geometry}
%\usepackage{amsmath}
\usepackage{mathpartir}
\usepackage{mathrsfs}

\lstdefinestyle{custom_lang}{
  xleftmargin=\parindent,
  showstringspaces=false,
  basicstyle=\ttfamily,
  keywordstyle=\bfseries
}

\lstset{emph={%
    var, def, type, new, self%
    },emphstyle={\bfseries \tt}%
}

%\newcommand{\keyw}[1]{\texttt{\textbf{#1}}}
\newcommand{\keywadj}[1]{\mathtt{#1}}
\newcommand{\keyw}[1]{\keywadj{#1}~}

\begin{document}





\section{Bytecode Abstract Syntax}


\[
\begin{array}{lll}
\begin{array}{lllr}
b   & ::= & v ~P ~\overline{i} ~\overline{M}        & bytecode file \\
&&\\
v   & ::= & \textit{magic major.minor}              & \textit{magic+version number} \\
&&\\

P   & ::= & \textit{fully qualified path}           & \textit{path to module} \\
&&\\

i   & ::= & import ~\mu ~\textit{URI} : \tau ~as ~x & \textit{module import} \\
&&\\

\mu & ::= & \keyw{[metadata]} ~\keyw{[type]}        & \\
&&\\


M & ::= & \keyw{module} P: \tau = e & \textit{top level modules} \\
  & |   & \keyw{type} P = T         & \\
&&\\

e & ::= & x                                                              & expressions \\
  & |   & \keywadj{new}~\tau~\{x \Rightarrow \overline{d}\}              &\\
  & |   & e.m(\overline{e})                                              &\\
  & |   & e.f                                                            &\\
  & |   & e.f = e                                                        &\\
  & |   & \keyw{let} x=e~\keyw{in}~e                                     &\\
  & |   & \mathscr{L}                                                    &\\
  & |   & e.\keyw{match} \overline{x:p.L \Rightarrow e} ~[\keyw{else} e] &\\
  & |   & \overline{e}                                                   &\\
&&\\
\mathscr{L} & ::= & string & literals \\
& | & integer &\\
&&\\
\end{array}
& ~~~~~~
&
\begin{array}{lllr}
d & ::= & \keyw{val} f : \tau = e                    & declarations \\
  & |   & \keyw{var} f : \tau = e                    &\\
  & |   & \keyw{def} m(\overline{x:\tau}) : \tau = e &\\
  & |   & \keyw{type} L = T                          &\\
&&\\
T & ::= & c                                & \textit{type desc.}\\
  & |   & \keyw{extag} c                   &\\
  & |   & \keyw{datatag} \overline{p.L} ~c &\\
&&\\
c & ::= & \tau                    & \textit{case desc.} \\
  & |   & \keyw{extends} p.L~\tau &\\
&&\\
\tau & ::= & \tau~\{\texttt{x} \Rightarrow \overline{\sigma}\}_{s} & type \\
     & |   & p.L                                                   &\\
     & |   & \top                                                  &\\
     & |   & \bot                                                  &\\
     & |   & ?                                                     &\\
&&\\
p & ::= & x   & paths \\
  & |   & p.f &\\
&&\\
s & ::= & \keyw{stateful} | ~\keyw{pure} \\
&&\\
\sigma & ::= & \texttt{val} \; f:\tau                           & \textit{decl type}\\
       & |   & \texttt{var} \; f:\tau                           &\\
       & |   & \texttt{def} \; m:\Pi \overline{x{:}\tau} . \tau &\\
       & |   & \texttt{type} \; L = T ~[m]                      &\\
       & |   & \texttt{type}_{s} \; L ~[m]                      &\\
&&\\
m & ::= & \keyw{metadata} ~e & metadata \\
&&\\
\end{array}
\end{array}
\]

Notation: overbar means a list of elements, as in Java



\end{document}
