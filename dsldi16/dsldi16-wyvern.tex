%-----------------------------------------------------------------------------
%
%               Template for sigplanconf LaTeX Class
%
% Name:         sigplanconf-template.tex
%
% Purpose:      A template for sigplanconf.cls, which is a LaTeX 2e class
%               file for SIGPLAN conference proceedings.
%
% Guide:        Refer to "Author's Guide to the ACM SIGPLAN Class,"
%               sigplanconf-guide.pdf
%
% Author:       Paul C. Anagnostopoulos
%               Windfall Software
%               978 371-2316
%               paul@windfall.com
%
% Created:      15 February 2005
%
%-----------------------------------------------------------------------------


\documentclass[preprint]{sigplanconf}

% The following \documentclass options may be useful:

% preprint      Remove this option only once the paper is in final form.
% 10pt          To set in 10-point type instead of 9-point.
% 11pt          To set in 11-point type instead of 9-point.
% numbers       To obtain numeric citation style instead of author/year.

\usepackage{amsmath}
\usepackage{graphicx}
\usepackage{listings}
\usepackage{caption}
\usepackage{float}
\usepackage{url}
\usepackage{listings}

\newcommand{\cL}{{\cal L}}
\setcitestyle{numbers}
\setcitestyle{square}
\lstset{basicstyle=\ttfamily,
	  numberstyle=\tiny,
	  numbersep=5pt,
	  xrightmargin=0.2\textwidth}

\begin{document}

\special{papersize=8.5in,11in}
\setlength{\pdfpageheight}{\paperheight}
\setlength{\pdfpagewidth}{\paperwidth}

\conferenceinfo{CONF 'yy}{Month d--d, 20yy, City, ST, Country}
\copyrightyear{20yy}
\copyrightdata{978-1-nnnn-nnnn-n/yy/mm}
\copyrightdoi{nnnnnnn.nnnnnnn}

% Uncomment the publication rights you want to use.
%\publicationrights{transferred}
%\publicationrights{licensed}     % this is the default
%\publicationrights{author-pays}

\titlebanner{banner above paper title}        % These are ignored unless
\preprintfooter{short description of paper}   % 'preprint' option specified.

\title{Naturally Embedded DSLs}
% \subtitle{Subtitle Text, if any}

\authorinfo{Jonathan Aldrich}
           {Carnegie Mellon University}
           {aldrich@cs.cmu.edu}

\authorinfo{OTHERS}
           {Your U}
           {email@my.edu}
		   
%\authorinfo{Cyrus Omar}
%           {Carnegie Mellon University}
%           {comar@cs.cmu.edu}
		   
%\authorinfo{Darya Kurilova}
%           {Carnegie Mellon University}
%           {darya@cs.cmu.edu}
		   
%\authorinfo{Alex Potanin}
%           {Victoria University of Wellington}
%           {Alex.Potanin@ecs.vuw.ac.nz}
		   
\maketitle

\begin{abstract}
Domain-specific languages can be embedded in a variety of ways within a
host language.  The choice of embedding approach entails significant
tradeoffs in the usability of the embedded DSL.  We argue embedding DSLs
\textit{naturally} within the host language results in the best
experience for end users of the DSL.  A \textit{naturally embedded DSL}
is one that uses natural syntax, static semantics, and dynamic semantics
for the DSL, all of which may differ from the host language.
Furthermore, it must be possible to use DSLs together
naturally---meaning that different DSLs cannot conflict, and the
programmer can easily tell which code is written in which language.
\end{abstract}

\category{CR-number}{subcategory}{third-level}

% general terms are not compulsory anymore,
% you may leave them out
% \terms
%

\keywords
naturally embedded domain specific languages

\section{Usability Challenges in Embedded DSLs}

Domain-specific languages (DSLs) can provide a number of advantages in
software development. Any rich-enough software system that works with multiple domains can take advantage of DSL's to capture each domain interaction in the most understandable and effective manner. Allowing the design decisions behind each DSL to be independent while providing a safe hosting environment in the overall language would enable us to avoid language usability compromises. Developers can express a program in the most
natural way for the domain they are working in.  Conversely, the DSL
compiler can take advantage of domain knowledge to catch programming
errors and produce more optimized code, compared to a general-purpose
language and compiler.

Domain-specific languages often describe part of a solution to a
larger problem.  If other parts of the problem are best expressed in
different DSLs, or in a general-purpose host language, then it is
important to semantically integrate the DSLs and general-purpose
language together.  A natural approach is to embed the DSL within the
host language.  Embedded DSLs can be defined in multiple ways.  In a
library-based embedding approach, the DSL is defined as a library in
the host language, so that expressions using that library make up a
kind of DSL.  In a macro-based embedding approach, macros are defined
for DSL constructs, and the compiler expands uses of the macros into
host language expressions, which are then typechecked and executed
as usual.

Both of these approaches have usability problems.  The host language
or its macro system may not support the most appropriate syntax for
the DSL, in which case DSL programs must be expressed awkwardly.
If the programmer makes a semantic mistake in the DSL, the mistake
will be caught after translation into the host language (if it is
caught at all), meaning that the error message may be difficult for
the programmer to understand.

A third approach is extensible languages, in which a language
can be extended through a library\footnote{or a compiler plug-in,
although this makes the DSL dependent on the particular compiler
used rather than on a language standard that can be implemented
by multiple compilers} with new syntax and semantics.  The
extensible language approach can address both of the problems above,
but can introduce new usability problems.  Depending on the
design of the extensible language, it may be difficult to know
what code is being expressed in the host language vs. in an
extension.  Furthermore, different DSLs may conflict when the
programmer tries to use them together, forcing the programmer
to disambiguate the uses somehow.

Note that we focus on usability for programmers who use the host
language and its embedded DSLs.  Making it easy to define embedded
DSLs is a different usability problem.  Solving both problems are
worthwhile, but we prioritize the usability for programmers who
use DSLs because a DSL is used many more times than it is defined,
and because programmers who define DSLs are likely to have more
expertise (and thus more ability to cope with complexity) than
programmers who use them.

\section{Naturally Embedded DSLs}

We argue that DSLs should be embedded \textit{naturally} within the
host language.  A system supporting \textit{naturally embedded DSLs}
has the following characteristics:

\begin{itemize}

\item The ability to define \textbf{new syntax} that expresses the ideas
in a domain in the most natural way.  The host language should place as
few as possible restrictions on the syntax of embedded DSLs.

%cite Hanenberg?

\item The ability to define \textbf{new semantics} that can be understood
in terms of the target domain, rather than as a mode of use of the host
language.  Note that the DSL may still be implemented by translation into
the host language, or by an interpreter written in the host language; what
is important is that this translation or interpretation is not visible to
the end user.

\item The ability to define \textbf{error messages} (both at compile time
and at run time), \textbf{introspection} and \textbf{debugging} in terms
of the target domain, rather than in terms of the host language.

\item Support for the programmer to easily \textbf{identify} what language
each expression is written in, and what host-language type each DSL
expression has.

\item A \textbf{modularity guarantee} that separately-defined DSLs can
be used together without conflicts.

\end{itemize} 


\begin{figure}[t]
\begin{lstlisting}
def mostOptimalTrade() : ShareTransaction
  new
    pick best 42-shot strategy
      from {APPL, XRO, MSFT}
    apply strategy for next 10 day range
    only keep the indexed shares

def storedTrade(t : ShareTransaction) : SQLQuery
  new
    CREATE in "TRADES" WHERE (t.name, t.buy, t.sell)
\end{lstlisting}
\caption{A simple example mixing a share trading and database DSL's in Wyvern.}
\label{f:dsl}
\end{figure}


\section{Extended example: Wyvern}

As part of our talk, we will describe the Wyvern programming language
as an example of the naturally embedded domain-specific language
approach~\cite{omar:2014:tsls} that was inspired by earlier work by Omar et al.~\cite{ACC_VLHCC}
Wyvern is a modularly extensible language in which language
extensions can be defined as libraries.

Wyvern allows DSLs to be defined with their own \textbf{syntax}, can be
an arbitrary sequence of characters that is distinguished from the host
language using indentation.  Delimiters such as \{ and \} can also be
used, at the cost of requiring the DSL to balance any occurrences of
these within its syntax.  Extensions define their own \textbf{semantics},
including custom error messages.\footnote{support for customizable
error messages is still being implemented as of this writing, but
hopefully will be complete for a demo at DSLDI}  Our approach of
using indentation or reserved delimiters distinguishes the DSL from
the host language, and we use expected types or macro names so that
developers can easily \textbf{identify} which DSL is being used in each
subexpression.  The delimiting and identification mechanisms also work
together to ensure that DSLs \textbf{cannot conflict} with one another.

Our talk will include a brief demonstration of Wyvern that illustrates
these features. Figure~\ref{f:dsl} demonstrates how a banking app can mix a specific trading language with a database manipulation language within a common language of Wyvern.

\section{Conclusion}

Related approaches to our work include tools such as \textit{ProteaJ}~\cite{Ichikawa:2014:CUO:2584469.2577092} that uses \textit{protean operators} that resolve potential syntactic conflicts by looking at expected types. Which we consider the core requirement towards natural DSL's as opposed to using keywords (like XJ~\cite{DBLP:conf/scam/ClarkSW08}) or indentation (like SRFI~\cite{srfi-49}).

We hope that we can convince the audience that our approach to type-based domain specific languages in Wyvern is the most \textit{natural} and usable way of implementing DSLs.


\bibliographystyle{plain}
\bibliography{alex,extra,biblio}

\end{document}