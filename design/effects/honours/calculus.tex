\chapter{Effect Calculi}

\section{$\opercalc$: Operation Calculus}

The operation calculus $\opercalc$ is an extension of $\stlc$ with primitive capabilities (resources) and their operations. Effects are identified with operation-calls. A program's runtime effects are those operations which it calls during execution. The static rules approximate the runtime effects of an expression, forming a simple effect system. These rules are very simple, but formalising and studying them will introduce new notations and a new concept of effect-soundness on which we shall build the $\epscalc$.

\subsection{Definition of $\opercalc$}

\begin{figure}[h]
\vspace{-5pt}

\[
\begin{array}{lll}

\begin{array}{lllr}

e & ::= & ~ & exprs: \\
	& | & x & variable \\
	& | & v & value \\
	& | & e ~ e & application \\
	& | & e.\pi & operation \\
	&&\\

v & ::= & ~ & values: \\
	& | & r & resource~literal \\
	& | & \lambda x: \tau.e & abstraction \\
	&&\\

\end{array}

& ~~~~~~~~&

\begin{array}{lllr}

\varepsilon & ::= & ~ & effects: \\
	& | & \{ \overline{r.\pi} \} \\
	&&\\

\tau & ::= & ~ & types: \\
		& | & \{ \bar r \} \\
		& | & \tau \rightarrow_{\varepsilon} \tau \\ 
		&&\\

\Gamma & ::= & ~ & type~ctx: \\
				& | & \varnothing \\
				& | & \Gamma, x: \tau \\
				&&\\
\end{array}

\end{array}
\]

\vspace{-7pt}
\caption{Grammar for $\opercalc$.}
\label{This is the label.}
\end{figure}

The base types of $\opercalc$ are sets of resources, denoted by $\{ \bar r \}$. Resources are drawn from a fixed set $R$ of variables, and model those initial capabilities passed in from the system environment. They cannot be created at runtime. When a resource type is ascribed to a program, as in the judgement $\Gamma \vdash e: \{ \bar r \}$, it means that if $e$ terminates it will result in a resource literal $r \in \bar r$. A resource literal $r$ is a variable ranging over the elements of $R$.

An operation is a special action that can be invoked on a resource-typed expression. For example, we might invoke the $\kwa{open}$ operation on a $\kwa{File}$ resource. Operations are drawn from a fixed set $\Pi$ of variables and cannot be created at runtime.

An effect is an operation performed on a resource. Formally, they are members of $R \times \Pi$, but for readability we write $\kwa{File.write}$ over $\kwa{(File, write)}$. A set of effects is denoted by $\varepsilon$. Effects and operations notationally look the same, but should be distinguished: an effect is some action upon a resource which may happen during runtime; an operation-call is an expression representing the actual invocation of an effect at runtime.

In practical applications, operations take arguments. For example, when writing to a file, we want to specify \textit{what} is being written to the file, e.g. $\kwa{File.write(``my msg'')}$. Because we are only concerned with the use and propagation of effects, and not the semantics of particular effects, we make the simplifying assumption that all operations are null-ary.

The only type constructor is $\rightarrow_{\varepsilon}$, where $\varepsilon$ is a concrete set of effects. $\tau_1 \rightarrow_{\varepsilon} \tau_2$ is the type of a function which takes inputs of type $\tau_1$, produces outputs of type $\tau_2$, and incurs no more effects than those contained in $\varepsilon$. For example, the type of a function which sends a message over a socket and returns a success flag could be $\kwa{Str} \rightarrow_{\kwa{Socket.write}} \Bool$. From this signature we can tell this function will not open or close the socket, because the annotation on the arrow does not have those effects. A valid implementation of this function might not write to the $\kwa{Socket}$, because $\{ \kwa{Socket.write} \}$ is an upper-bound on the effects which can happen. Every function type in $\opercalc$ must be annotated with an upper-bound in this way.



\begin{figure}[h]


\noindent
\fbox{$\Gamma \vdash e: \tau~\kw{with} \varepsilon$}

\[
\begin{array}{c}

\infer[\textsc{($\varepsilon$-Var)}]
	{ \Gamma, x:\tau \vdash x: \tau~\kw{with} \varnothing }
	{}
	
	~~~
	
\infer[\textsc{($\varepsilon$-Resource)}]
 	{ \Gamma, r: \{ r \} \vdash r : \{ r \}~\kw{with} \varnothing }
 	{} \\[3ex]
 	
 	~~~
	\infer[\textsc{($\varepsilon$-Abs)}]
	{ \Gamma \vdash \lambda x:\tau_2 . e : \tau_2 \rightarrow_{\varepsilon_3} \tau_3~\kw{with} \varnothing }
	{ \Gamma, x: \tau_2 \vdash e: \tau_3~\kw{with} \varepsilon_3 }
	
	~~~
	
\infer[\textsc{($\varepsilon$-App)}]
	{ \Gamma \vdash e_1~e_2 : \tau_3~\kw{with} \varepsilon_1 \cup \varepsilon_2 \cup \varepsilon  }
	{ \Gamma \vdash e_1: \tau_2 \rightarrow_{\varepsilon} \tau_3~\kw{with} \varepsilon_1 & \Gamma \vdash e_2: \tau_2~\kw{with} \varepsilon_2 } \\[3ex]
	
\infer[\textsc{($\varepsilon$-OperCall)}]
	{ \Gamma \vdash e.\pi: \kw{Unit} \kw{with} \{ \bar r.\pi \} }
	{ \Gamma \vdash e: \{ \bar r \} & \forall r \in \bar r \mid r: \{ r \} \in \Gamma & \pi \in \Pi } \\[3ex]

\infer[\textsc{($\varepsilon$-Subsume)}]
	{ \Gamma \vdash e: \tau' ~\kw{with} \varepsilon'}
	{ \Gamma \vdash e: \tau ~\kw{with} \varepsilon & \tau <: \tau' & \varepsilon \subseteq \varepsilon'}
\end{array}
\]


\noindent
\fbox{$\Gamma \vdash e: \tau~\kw{with} \varepsilon$}

\[
\begin{array}{c}

\infer[\textsc{(S-Arrow)}]
	{ \tau_1 \rightarrow_{\varepsilon} \tau_2 <: \tau_1' \rightarrow_{\varepsilon'} \tau_2' }
	{ \tau_1' <: \tau_1 & \tau_2 <: \tau_2' & \varepsilon \subseteq \varepsilon' }
~~~~~~
\infer[\textsc{(S-Resource)}]
	{ \{ \bar r_1 \} <: \{ \bar r_2 \} }
	{ r \in r_1 \implies r \in r_2 }

\end{array}
\]


\vspace{-7pt}
\caption{Type-with-effect judgements in $\opercalc$.}
\label{This is the label.}
\end{figure}

The static rules for $\opercalc$ are summarised in Figure 3.2. The typing judgement $\Gamma \vdash e: \tau~\kw{with} \varepsilon$ means that successive reductions on $e$ will yield terms of type $\tau$, and collectively incur no more than those effects in $\varepsilon$. This $\varepsilon$ is a conservative approximation to the runtime effects of executing $e$, so it may contain effects which don't happen at runtime.

The rules for variables and values are: \textsc{$\varepsilon$-Var}, \textsc{$\varepsilon$-Resource}, and \textsc{$\varepsilon$-Abs}. These are identical to the rules in $\stlc$, except they approximate the set of effects as $\varnothing$. Although a fucntion and a resource literal both encapsulate capabilities, something must be done to them (apply the function, operate on the resource) to incur a runtime effect.

The effects of a lambda application are: the effects of evaluating its subexpressions, and the effects incurred by executing the body of the lambda to which the left-hand side evaluates. Those last effects are obtained from the label on the lambda's arrow-type in the first premise.

The effects of an operation call are: the effects of evaluating the subexpression, and the single effect incurred when the subexpression is reduced to a resource literal $r$, and operation $\pi$ is invoked on it. It is not always possible to know statically which exact resource literal the subexpression reduces to (if it halts at all). For example, the program $\kwa{(if~System.randomBool~then~File~else~Socket).close}$ may either reduce to $\kwa{File.close}$ or $\kwa{Socket.close}$. In such cases, the safe approximation is to type the conditional as $\{ \kwa{File, Socket} \}$. \textsc{$\varepsilon$-OperCall} would then approximate the runtime effects of the operation call as  $\{ \kwa{File.close, Socket.close} \}$. Being able to type an expression as a (non-singleton) set of resources requires an extra rule: \textsc{S-Resource}. This says that a subset of resources is also a subtype. 

Because we're not really interested in what exactly an operation call does, every operation is valid for every resource. This can give bizarre programs --- $\kwa{Sensor.readTemp}$ seems like a sensible operation call, but what about $\kwa{File.readTemp}$? Because we are not interested in the specific behaviours of operations, we permit any operation on any resource to simplify the static rules.

The other subtyping rule is \textsc{S-Arrow}, a modification of the rule from $\stlc$. In addition to this rule being contravariant in the input and covariant in the output, it is also covariant in the effects. Justified in terms of the Liskov substitution principle, any possible effect which might be incurred by the subtype should be expected by the supertype, otherwise substitution of a supertype for a subtype would allow for the possibility of new effects not possible under the original.\\

\begin{figure}[h]

\bm{$\kwa{substitution :: e \times v \times v \rightarrow e}$}

\begin{itemize}
	\setlength\itemsep{-0.7em}
	\item[] $[v/y]x = v$, if $x = y$
	\item[] $[v/y]x = x$, if $x \neq y$
	\item[] $[v/y](\lambda x: \tau. e) = \lambda x: \tau.[v/y]e$, if $y \neq x$ and $y$ does not occur free in $e$
	\item[] $[v/y](e_1~e_2) = ([v/y]e_1)([v/y]e_2)$
	\item[] $[v/y](e_1.\pi) = ([v/y]e_1).\pi$
\end{itemize}

\vspace{-7pt}
\caption{Substitution function.}
\label{This is the label.}
\end{figure}

Figure 3.2. shows the updated definition of $\kwa{substitution}$. In $\opercalc$, a variable may only be substituted for a value, making the function a partial one. This restriction is imposed because, if a variable can be replaced with an arbitrary expression, then we might also be introducing arbitrary effects --- a situation which violates the preservation of effects under reduction. \\

\begin{figure}[h]

\noindent
\fbox{$e \longrightarrow e~|~\varepsilon$}

\[
\begin{array}{c}

\infer[\textsc{(E-App1)}]
	{e_1 e_2 \longrightarrow e_1'~ e_2~|~\varepsilon}
	{e_1 \longrightarrow e_1'~|~\varepsilon}
	~~~~~~
\infer[\textsc{(E-App2)}]
	{v_1 ~ e_2 \longrightarrow v_1 ~ e_2'~|~\varepsilon} 
	{e_2 \longrightarrow e_2'~|~\varepsilon}
~~~~~~
\infer[\textsc{(E-App3)}]
	{ (\lambda x: \hat \tau.\hat e) \hat v_2 \longrightarrow [\hat v_2/x]\hat e~|~\varnothing }
	{}\\[4ex]
	
\infer[\textsc{(E-OperCall1)}]
	{\hat e.\pi \longrightarrow \hat e'.\pi~|~\varepsilon }
	{\hat e \rightarrow \hat e'~|~\varepsilon}
		
	~~~~~~
	
\infer[\textsc{(E-OperCall2)}]
	{r.\pi \longrightarrow \kwa{unit}~|~\{ r.\pi \} }
	{r \in R & \pi \in \Pi}
	 \\[4ex]
	 
\end{array}
\]


\vspace{-7pt}
\caption{Single-step reductions.}
\label{This is the label.}
\end{figure}

Single-step reduction is now a relation on an expression $e$ and a pair $e \times \varepsilon$, representing the reduced expression and all effects incurred during the single-step of computation. \textsc{E-App1} and \textsc{E-App2} incur whatever is the effect of reducing their subexpressions. \textsc{E-App3} incurs no effects.

The new single-step rules are \textsc{E-OperCall1} and \textsc{E-OperCall2}. The former reduces the receiver of an operation-call, and the latter performs an operation on a resource literal. \textsc{E-OperCall1} incurs whatever is the effect of reducing the subexpression. \textsc{E-OperCall2}, which reduces the operation-call $r.\pi$, incurs the effect $r.\pi$.

Operation calls reduce to $\unit$ (which is a derived form; see Encodings). $\unit$ is a value representing the absence of information (because it is the only value of its type). Because we are not interested in the semantics of effects, we choose $\unit$ as the result of reducing an operation-call. \\

\begin{figure}[h]

\noindent
\fbox{$\hat e \longrightarrow^{*} \hat e~|~\varepsilon$}

\[
\begin{array}{c}

\infer[\textsc{(E-MultiStep1)}]
	{\hat e \rightarrow^{*} \hat e~|~\varnothing}
	{}
~~~
\infer[\textsc{(E-MultiStep2)}]
	{\hat e \rightarrow^{*} \hat e'~|~\varepsilon}
	{\hat e \rightarrow \hat e'~|~\varepsilon} \\[3ex]
	
\infer[\textsc{(E-MultiStep3)}]
	{\hat e \rightarrow^{*} \hat e''~|~\varepsilon_1 \cup \varepsilon_2}
	{\hat e \rightarrow^{*} \hat e'~|~\varepsilon_1 & \hat e' \rightarrow^{*} \hat e''~|~\varepsilon_2}
\end{array}
\]

\vspace{-7pt}
\caption{Multi-step reductions in $\opercalc$.}
\label{This is the label.}
\end{figure}

A multi-step reduction consists of zero\footnote{We permit multi-step reductions of length zero to be consistent with Pierce, who defines multi-step reduction as a reflexive relation\cite[p. 39]{tapl}.} or more single-step reductions. The resulting effect-set is the union of all the single-steps taken.

\subsection{Soundness of $\opercalc$}

Our goal is to show $\opercalc$ is sound. This requires an appropriate notion of \textit{effect-soudnness}, as we need both the type and approximated effects of an expression to be preserved under reduction. Intuitively, a reduction $e_A \longrightarrow^* e_B~|~\varepsilon$ is sound if the type system's approximation of the effects of $e$ contains the actual runtime effects $\varepsilon$, because then it has statically accounted for every possible runtime effect. Below is a definition, phrased in terms of single-step reduction.


\begin{theorem}[Soundness]
If $\hat \Gamma \vdash \hat e_A: \hat \tau_A~\kw{with} \varepsilon_A$ and $\hat e_A$ is not a value, then $e_A \longrightarrow e_B~|~\varepsilon$, where $\hat \Gamma \vdash e_B: \hat \tau_B~\kw{with} \varepsilon_B$ and $\hat \tau_B <: \hat \tau_A$ and $\varepsilon_B \cup \varepsilon \subseteq \varepsilon_A$.
\end{theorem}

This definition is slightly stronger, because it relates the approximation after reduction to the approximation before reduction, by saying it can only get more precise. This is analogous to the type-safe component of the definition, which states that the types of successive terms under reduction can only get more precise.

Our road to proving soundness takes the standard approach of showing that progress and preservation hold of $\opercalc$. Progress follows immediately by observing some properties of the typing rules.

\begin{lemma}[Canonical Forms]
The following are true:
\begin{itemize}
	\setlength\itemsep{-0.7em}
	\item If $\hat \Gamma \vdash \hat v: \hat \tau~\kw{with} \varepsilon$ then $\varepsilon = \varnothing$.
	\item If $\hat \Gamma \vdash \hat v: \{ \bar r \}$ then $\hat v = r$ for some $r \in R$ and $\{ \bar r \} = \{ r \}$.
\end{itemize}
\end{lemma}

\begin{theorem}[Progress]
If $\hat \Gamma \vdash \hat e: \hat \tau~\kw{with} \varepsilon$ and $\hat e$ is not a value, then $\hat e \longrightarrow \hat e'~|~\varepsilon$.
\end{theorem}

\begin{proof} By induction on $\hat \Gamma \vdash \hat e: \hat \tau~\kw{with} \varepsilon$, for $\hat e$ not a value. If the rule is \textsc{$\varepsilon$-Subsumption} it follows by inductive hypothesis. If $\hat e$ has a reducible subexpression then reduce it. Otherwise use one of \textsc{$\varepsilon$-App3} or \textsc{$\varepsilon$-OperCall2}.
\end{proof}

To show preservation holds we need to know that type-and-effect safety is preserved by the substitution in the rule \textsc{E-App3}. The formulation of the substitution lemam is largely the same as it is in $\stlc$, except the expression being substituted must be a value. This strengthening of the lemma is not a problem as the dynamic rules employ a strict evaluation strategy, so any expressions are reduced to values before they are substituted.

\begin{lemma}[Substitution]
If $\Gamma, x: \tau' \vdash e: \tau~\kw{with} \varepsilon$ and $\Gamma \vdash v: \tau'~\kw{with} \varnothing$ then $\Gamma \vdash [v/x]e: \tau~\kw{with} \varepsilon$.
\end{lemma}

\begin{proof} By induction on $\Gamma, x: \tau' \vdash e: \tau~\kw{with} \varepsilon$.
\end{proof}

With this lemma, we are ready to prove the preservation theorem.

\begin{theorem}[Preservation]
If $\Gamma \vdash e_A: \tau_A~\kw{with} \varepsilon_A$ and $e_A \longrightarrow e_B~|~\varepsilon$, then $\tau_B <: \tau_A$ and $\varepsilon_B \cup \varepsilon \subseteq \varepsilon_A$.
\end{theorem}

\begin{proof} By induction on $\Gamma \vdash e_A: \tau_A~\kw{with} \varepsilon_A$, and then on $e_A \longrightarrow e_B~|~\varepsilon$. Since $e_A$ can be reduced, we need only consider those rules which apply to non-values and non-variables. \\

\textit{Case:} \textsc{$\varepsilon$-App} Then $e_A = e_1 ~ e_2$ and $e_1: \tau_2 \rightarrow_{\varepsilon} \tau_3~\kw{with} \varepsilon_1$ and $\Gamma \vdash e_2: \tau_2~\kw{with} \varepsilon_2$. If the reduction rule used was \textsc{E-App1} or \textsc{E-App2}, then the result follows by applying the inductive hypothesis to $e_1$ and $e_2$ respectively.

Otherwise the rule used was \textsc{E-App3}. Then $(\lambda x: \tau_2. e) v_2 \longrightarrow [ v_2/x] e~|~\varnothing$. By inversion on the typing rule for $\lambda x:  \tau_2. e$ we know $\Gamma, x:  \tau_2 \vdash  e:  \tau_3~\kw{with} \varepsilon_3$. By canonical forms, $\varepsilon_2 = \varnothing$ because $ e_2 =  v_2$ is a value. Then by the substitution lemma, $ \Gamma \vdash [ v_2/x] e :  \tau_3~\kw{with} \varepsilon_3$. By canonical forms, $\varepsilon_1 = \varepsilon_2 = \varnothing = \varepsilon_C$. Therefore $\varepsilon_A = \varepsilon_3 = \varepsilon_B \cup \varepsilon_C$.\\

\textit{Case:} \textsc{$\varepsilon$-OperCall}. Then $e_A = e_1.\pi$ and $ \Gamma \vdash e_1 : \{ \bar r \}~\kw{with} \varepsilon_1$. If the reduction rule used was \textsc{E-OperCall1} then the result follows by applying the inductive hypothesis to $ e_1$.

Otherwise the reduction rule used was \textsc{E-OperCall2} and $v_1.\pi \longrightarrow \kwa{unit}~|~\{ r.\pi \}$. By canonical forms, $ \Gamma \vdash v_1: \kwa{unit}~\kw{with} \{ r.\pi \}$. Also, $ \Gamma \vdash \kwa{unit}: \kwa{Unit}~\kw{with} \varnothing$. Then $\tau_B = \tau_A$. Also, $\varepsilon_C \cup \varepsilon_B = \{ r.\pi \} = \varepsilon_A$.\\

\end{proof}

Our single-step soundness theorem now holds immediately by joining the progress and preservation theorems into one.

\begin{theorem}[Soundness]
If $ \Gamma \vdash  e_A:  \tau_A~\kw{with} \varepsilon_A$ and $ e_A$ is not a value, then $e_A \longrightarrow e_B~|~\varepsilon$, where $ \Gamma \vdash e_B:  \tau_B~\kw{with} \varepsilon_B$ and $ \tau_B <:  \tau_A$ and $\varepsilon_B \cup \varepsilon \subseteq \varepsilon_A$.
\end{theorem}
\begin{proof}
If $ e_A$ is not a value then the reduction exists by the progress theorem. The rest follows by the preservation theorem.
\end{proof}

Knowing that single-step reductions are sound, the soundness of multi-step reductions follows by inductively applying single-step soundness on the length of a multi-step reduction.

\begin{theorem}[Multi-step Soundness]
If $ \Gamma \vdash  e_A:  \tau_A~\kw{with} \varepsilon_A$ and $e_A \longrightarrow^{*} e_B~|~\varepsilon$, where $\Gamma \vdash e_B: \tau_B~\kw{with} \varepsilon_B$ and $ \tau_B <: \tau_A$ and $\varepsilon_B \cup \varepsilon \subseteq \varepsilon_A$.
\end{theorem}

\begin{proof} By induction on the length of the multi-step reduction. If the length is 0 then $e_A = e_B$ and the result holds vacuously. If the length is 1 the result holds by soundness of single-step reductions. if the length is $n+1$, then the first $n$-step reduction is sound by inductive hypothesis and the last step is sound by single-step soundness, so the entire $n+1$-step reduction is sound.
\end{proof}


























\section{$\epscalc$: Epsilon Calculus}


The effect calculus is based on the simply-typed lambda calculus $\lambda^{\rightarrow}$. There is one type constructor, $\rightarrow$. The base types are sets of resources, denoted by $\{ \bar r \}$. Although the calculus has no primitive notions of integers or booleans, we shall assume these may be encoded as they are in the usual way (e.g. as Church numerals) and make free use of them in examples as though they were standard, for the sake of readability.

\subsection{Definition of $\epscalc$}

\begin{figure}[h]
\vspace{-5pt}

\[
\begin{array}{lll}

\begin{array}{lllr}

e & ::= & ~ & exprs: \\
	& | & x & variable \\
	& | & v & value \\
	& | & e ~ e & application \\
	& | & e.\pi & operation \\
	&&\\

v & ::= & ~ & values: \\
	& | & r & resource~literal \\
	& | & \lambda x: \tau.e & abstraction \\
	&&\\
	
\hat e & ::= & ~ & labelled~exprs: \\
	& | & x \\
	& | & r \\
	& | & \lambda x: \hat \tau.\hat e \\
	& | & \hat e ~ \hat e \\
	& | & \hat e.\pi \\
	& | & \kwa{import}(\varepsilon)~x = \hat e~\kwa{in}~e & import \\
	&&\\

\hat v & ::= & ~ & labelled~values: \\
	& | & r \\
	& | & \lambda x: \hat \tau.\hat e \\
	&&\\

\end{array}

& ~~~~~~~~&

\begin{array}{lllr}

\varepsilon & ::= & ~ & effects: \\
	& | & \{ \overline{r.\pi} \} \\
	&&\\

\tau & ::= & ~ & types: \\
		& | & \{ \bar r \} \\
		& | & \tau \rightarrow \tau \\ 
		&&\\

\hat \tau & ::= & ~ & labelled ~types: \\
		& | & \{ \bar r \} \\
		& | & \hat \tau \rightarrow_{\varepsilon} \hat \tau \\
		&&\\

\Gamma & ::= & ~ & type~ctx: \\
				& | & \varnothing \\
				& | & \Gamma, x: \tau \\
				&&\\
				
\hat \Gamma & ::= & ~ & labelled~type~ctx:\\
				& | & \varnothing \\
				& | & \hat \Gamma, x: \hat \tau \\
				&&\\

\end{array}

\end{array}
\]

\vspace{-7pt}
\caption{Effect calculus.}
\label{This is the label.}
\end{figure}

Resources are drawn from a fixed set $R$ of variables, and model those initial capabilities passed in from the system environment. Resources cannot be created at runtime. When a resource type is ascribed to a program, as in the judgement $\Gamma \vdash e: \{ \bar r \}$, it means that if $e$ terminates it will result in a resource literal $r \in \bar r$.

A value $v$ is either a resource literal $r$ or a lambda abstraction $\lambda x: \tau.e$. The other forms of an expression are lambda application $e~e$, variable $x$, and operation $e.\pi$. An operation is an action invoked on a resource. For example, we might invoke the $\kwa{open}$ operation on a $\kwa{File}$ resource. Operations are drawn from a fixed-set $\Pi$ of variables. They cannot be created at runtime.

An effect is an operation performed on a resource. Formally, they are members of $R \times \Pi$, but for readability we write $\kwa{File.write}$ over $\kwa{(File, write)}$. A set of effects is denoted by $\varepsilon$. Effects and operations notationally look the same, but should be distinguished: an effect is some action upon a resource which may happen during runtime; an operation is the actual invocation of an effect at runtime.

In a practical language, operations should take arguments. For example, when writing to a file, we want to specify \textit{what} is being written to the file, ala $\kwa{File.write(``my msg'')}$. Because $\epscalc$ is only concerned with the use and propagation of effects, and not the semantics of particular effects, we make the simplifying assumption that all operations are null-ary.

Expressions may be labelled with the set of effects they might incur during execution. This is achieved by annotating all arrow types inside the expression. If a metavariable represents a labelled expression, it will be written with a hat; if it represents an unlabelled expression, it will have no hat. Compare $e$ and $\hat e$.

Labelling of an expression is \textit{deep}. That is, every subterm of a labelled term is also labelled. Unlabelled terms are also deeply unlabelled. The only exception is the $\kwa{import}$ expression, which is the only way to compose labelled and unlabelled code. $\kwa{import}$ nests unlabelled code inside labelled code, and selects those capabilities $\varepsilon$ over which the unlabelled code has authority. It is not possible to nest labelled code inside unlabelled code.

The distinction between labelled and unlabelled types and expressions requires us to have the notion of labelled and unlabelled contexts. Labelled contexts only bind variables to labelled types, whereas unlabelled contexts only bind variables to unlabelled types. There is no valid context which mixes labelled and unlabelled types.

A construct's labelled version is always denoted with a hat. \\

\begin{figure}[h]
\vspace{-5pt}

	
$\bm{\kwa{annot :: e \times \varepsilon \rightarrow \hat e}}$

\begin{itemize}
	\setlength\itemsep{-0.7em}
	\item[] $\annot{r}{\_} = r$
	\item[] $\annot{\lambda x: \tau_1 . e}{\varepsilon} = \lambda x: \annot{\tau_1}{\varepsilon} . \annot{e}{\varepsilon}$
	\item[] $\annot{e_1~e_2}{\varepsilon} = \kwa{annot}(e_1, \varepsilon)~\kwa{annot}(e_2, \varepsilon)$
	\item[] $\annot{e_1.\pi}{\varepsilon} = \annot{e_1}{\varepsilon}.\pi$
\end{itemize}
	
$\bm{\kwa{annot :: \tau \times \varepsilon \rightarrow \hat \tau}}$

\begin{itemize}
	\setlength\itemsep{-0.7em}
	\item[] $\annot{\{ \bar r \}}{\_} = \{ \bar r \}$
	\item[] $\annot{\tau \rightarrow \tau}{\varepsilon} = \tau \rightarrow_{\varepsilon} \tau$.	
\end{itemize}

$\bm{\kwa{annot :: \Gamma \times \varepsilon \rightarrow \hat \Gamma}}$

\begin{itemize}
	\setlength\itemsep{-0.7em}
	\item[] $\annot{\varnothing}{\_} = \varnothing$
	\item[] $\annot{\Gamma, x: \tau}{\varepsilon} = \annot{\Gamma}{\varepsilon}, x: \annot{\tau}{\varepsilon}$
\end{itemize}

$\bm{\kwa{erase :: \hat \tau \rightarrow \tau}}$
\begin{itemize}
	\setlength\itemsep{-0.7em}
	\item[] $\erase{\{ \bar r \}}$
	\item[] $\erase{\hat \tau_1 \rightarrow_{\varepsilon} \hat \tau_2} = \erase{\hat \tau_1} \rightarrow \erase{\hat \tau_2}$
\end{itemize}

$\bm{\kwa{erase :: \hat e \rightarrow e}}$
\begin{itemize}
	\setlength\itemsep{-0.7em}
	\item[] $\erase{r} = r$
	\item[] $\erase{\lambda x: \hat \tau_1 . \hat e} = \lambda x: \erase{\hat \tau_1} . \erase{\hat e}$
	\item[] $\erase{e_1~e_2} = \erase{e_1}~\erase{e_2}$
	\item[] $\erase{e_1.\pi} = \erase{e_1}.\pi$
\end{itemize}

\vspace{-7pt}
\caption{Annotation functions.}
\label{This is the label.}
\end{figure}


\noindent
Given a piece of unlabelled code $e$ and static effects $\varepsilon$ we can produce a labelled piece of code $\kwa{annot}(e, \varepsilon) = \hat e$ by annotating every function with $\varepsilon$. In the reverse direction, given some labelled code $\hat e$ we can produce an unlabelled piece of code $\kwa{erase}(\hat e) = e$ by removing the labels on functions. Full definitions for these functions on expressions, types, and contexts are given in Figure 3.2. Note that $\kwa{erase}$ is undefined on $\kwa{import}$ expressions. We won't ever need to erase import expressions, but it means the function is partial, so we need to be careful when we use it.

Annotation is not always safe. For instance, $\kwa{annot}(\lambda l: \kwa{Int} \rightarrow_{File.read} \kwa{Int} . ~l~1, \varnothing)$ would overwrite the $\kwa{File.read}$ effect permitted by $l$. $\kwa{annot}$ is used in one place, in the dynamic rules, and for that limited use we will have to prove its safety. \\




















\begin{figure}[h]

$\bm{\kwa{effects :: \hat \tau \rightarrow \varepsilon}}$ \begin{itemize}
	\setlength\itemsep{-0.7em}
	\item[] $\fx{\{ \bar r \}} = \{ r.\pi \mid r \in \bar r, \pi \in \Pi \}$
	\item[] $\fx{\hat \tau_1 \rightarrow_{\varepsilon} \hat \tau_2} = \hofx{\hat \tau_1} \cup \varepsilon \cup \fx{\hat \tau_2}$
\end{itemize}

$\bm{\kwa{ho \hyphen effects} :: \hat \tau \rightarrow \varepsilon}$ \begin{itemize}
	\setlength\itemsep{-0.7em}
	\item[] $\hofx{\{ \bar r \}} = \varnothing$
	\item[] $\hofx{\hat \tau_1 \rightarrow_{\varepsilon} \hat \tau_2} = \fx{\hat \tau_1} \cup \hofx{\hat \tau_2}$
\end{itemize}

\vspace{-7pt}
\caption{Effect functions.}
\label{This is the label.}
\end{figure}

\noindent
We may wish to know what effects are encapsulated by a piece of labelled code. This is achieved by two functions, $\kwa{effects}(\hat e)$ and $\kwa{ho \hyphen effects}(\hat e)$, which collectively compute the set of effects captured by $\hat e$. These are effects which may, directly or indirectly, be invoked by $\hat e$. The difference between the two functions is in who supplies the effect. $\kwa{effect}(\hat e)$ is the set of effects for which $\hat e$ has direct authority, while $\kwa{ho \hyphen effects}$ is the set of effects for which $\hat e$ has (strictly) transitive authority. These higher-order effects are always supplied by some external environment.

For example, take the function which, given a file, reads and returns its contents (which are perhaps encoded as an integer). Its signature would be $\kwa{f : \{ File \} \rightarrow_{File.read} Int}$. The $\kwa{\fx{f} = \{File.read\} \cup \fx{Int}}$, because any client using $\kwa{f}$ will directly invoke the $\kwa{File.read}$ operation and may use any resource encapsulated by the $\kwa{Int}$ type. The $\kwa{\hofx{f} = \{ File.\pi \mid \pi \in \Pi \} }$, because to use $\kwa{f}$ it must be supplied with a $\kwa{File}$ literal from some outside source. Therefore, every possible effect on $\kwa{File}$ is a higher-order effect.\\


\begin{figure}[h]

\bm{$\kwa{substitution :: \hat e \times \hat v \times \hat v \rightarrow \hat e}$}

\begin{itemize}
	\setlength\itemsep{-0.7em}
	\item[] $[\hat v/y]x = \hat v$, if $x = y$
	\item[] $[\hat v/y]x = x$, if $x \neq y$
	\item[] $[\hat v/y](\lambda x: \hat \tau. \hat e) = \lambda x: \hat \tau.[\hat v/y]\hat e$, if $y \neq x$ and $y$ does not occur free in $\hat e$
	\item[] $[\hat v/y](\hat e_1~\hat e_2) = ([\hat v/y]\hat e_1)([\hat v/y]\hat e_2)$
	\item[] $[\hat v/y](\hat e_1.\pi) = ([\hat v/y]e_1).\pi$
	\item[] $[\hat v/y](\import{\varepsilon}{x}{\hat e}{e}) = \import{\varepsilon}{x}{[\hat v/y]\hat e}{e}$
\end{itemize}

\vspace{-7pt}
\caption{Substitution function.}
\label{This is the label.}
\end{figure}

\noindent
The substitution function $\kwa{substitution}(\hat e, \hat v, x)$ replaces all free occurrences of $x$ with $\hat v$ in $\hat e$. The short-hand is $[\hat v/x]\hat e$. When performing multiple substitutions we use the notation $[\hat v_1/x_1, \hat v_2/x_2]\hat e$ as shorthand for $[\hat v_2/x_2]([\hat v_1/x_1]\hat e)$. Note how the order of the variables has been flipped; the substitutions occur as they are written, left-to-right.

Note that substitution is partial, because it is only defined when a free-variable is being replaced with a value. This is important for proving preservation, because if we replace variables with arbitrary expressions, then we might also be introducing arbitrary effects.

To avoid accidental variable capture we adopt the convention of $\alpha$-conversion, whereby we freely and implicitly interchange expressions which are equivalent up to the naming of bound variables \cite[p. 71]{tapl}. This elides some tedious bookkeeping. Consequently, we shall assume variables are (re-)named in this way to avoid accidental capture.



\begin{figure}[h]

\fbox{$\Gamma \vdash e: \tau$}

\[
\begin{array}{c}


\infer[\textsc{(T-Var)}]
	{\Gamma, x: \tau \vdash x: \tau}
	{}
~~~~~~
\infer[\textsc{(T-Resource)}]
	{\Gamma, r: \{ r \} \vdash r : \{ r \}}
	{}

~~~~~~
\infer[\textsc{(T-Abs)}]
	{\Gamma \vdash \lambda x: \tau_1.e : \tau_1 \rightarrow \tau_2}
	{\Gamma, x: \tau_1 \vdash e: \tau_2}\\[4ex]
	
\infer[\textsc{(T-App)}]
	{\Gamma \vdash e_1~e_2: \tau_3}
	{\Gamma \vdash e_1: \tau_2 \rightarrow \tau_3 & \Gamma \vdash e_2: \tau_2}
~~~~~~
\infer[\textsc{(T-OperCall)}]
	{\Gamma \vdash e.\pi: \kwa{Unit}}
	{\Gamma \vdash e: \{ \bar r \} & \forall r \in \bar r \mid r \in R & \pi \in \Pi}

\end{array}
\]

\vspace{-7pt}
\caption{Typing judgements in the epsilon calculus.}
\label{This is the label.}
\end{figure}

\noindent

\noindent
The first sort of static judgement ascribes a type to a piece of unlabelled code. \textsc{T-Var}, \textsc{T-App}, and \textsc{T-OperCall} are the same as they are in $\lambda^{\rightarrow}$. \textsc{T-Resource} is the same as \textsc{T-Var}, but for variables representing primitive capabilities. \textsc{T-OperCall} is the rule for typing an operation call $e_1.\pi$. Such an expression is well-typed if $e_1$ types to some valid resource, and $\pi$ is a known operation.


\begin{figure}[h]

\noindent
$\fbox{$\kwa{safe(\hat \tau, \varepsilon)}$}$

\[
\begin{array}{c}

\infer[\textsc{(Safe-Resource)}]
	{\kwa{safe}(\{ \bar r \}, \varepsilon)}
	{}
~~~~~
\infer[\textsc{(Safe-Unit)}]
	{\kwa{safe}(\kwa{Unit}, \varepsilon)}
	{} \\[3ex]

\infer[\textsc{(Safe-Arrow)}]
	{\kwa{safe}(\hat \tau_1 \rightarrow_{\varepsilon'} \hat \tau_2, \varepsilon)}
	{\varepsilon \subseteq \varepsilon' & \kwa{ho \hyphen safe}(\hat \tau_1, \varepsilon) & \kwa{safe}(\hat \tau_2, \varepsilon)} \\[3ex]

\end{array}
\]

\noindent
$\fbox{$\kwa{ho \hyphen safe(\hat \tau, \varepsilon)}$}$

\[
\begin{array}{c}

\infer[\textsc{(HOSafe-Resource)}]
	{ \kwa{ho \hyphen safe}( \{ \bar r \}, \varepsilon)} 
	{}
	~~~~~~
\infer[\textsc{(HOSafe-Unit)}]
	{ \kwa{ho \hyphen safe}( \kwa{Unit}, \varepsilon)} 
	{}\\[3ex]

\infer[\textsc{(HOSafe-Arrow)}]
	{ \kwa{ho \hyphen safe}( \hat \tau_1 \rightarrow_{\varepsilon'} \hat \tau_2, \varepsilon ) }
	{ \kwa{safe}(\hat \tau_1, \varepsilon)  & \kwa{ho \hyphen safe}(\hat \tau_2, \varepsilon) }\\[3ex]

\end{array}
\]

\vspace{-7pt}
\caption{Safety judgements in the epsilon calculus.}
\label{This is the label.}
\end{figure}

\noindent 
Before presenting the type-with-effect rules for labelled expressions, we first define a few safety predicates. Intuitively, the type $\hat \tau$ is $\kwa{safe}$ for $\varepsilon$ if it has declared every (non higher-order) effect $r.\pi \in \varepsilon$ in its signature. $\hat \tau$ is $\kwa{ho \hyphen safe}$ for $\varepsilon$ if $\hat \tau$ has declared every higher-order effect $r.\pi \in \varepsilon$ in its signature. One way to think about these predicates is as a contract between caller and callee. If the caller supplies a set of capabilities $\varepsilon$ to a piece of code typing to $\hat \tau$, it would violate the restriction on \textit{ambient authority} if a capability was supplied that $\hat \tau$ had not explicitly asked for. Therefore, $\safe{\hat \tau}{\varepsilon}$ holds when the (non higher-order) effects selected by $\hat \tau$ include $\varepsilon$. $\hosafe{\hat \tau}{\varepsilon}$ holds when the higher-order effects selected by $\hat  \tau$ include $\varepsilon$.

Because the implementation of $\hat \tau$ might internally propagate capabilities, the definitions of safety and higher-order safety need to be transitive. \textbf{Give an example of why this is so.}



\begin{figure}[h]


\noindent
\fbox{$\hat \Gamma \vdash \hat e: \hat \tau~\kw{with} \varepsilon$}

\[
\begin{array}{c}

\infer[\textsc{($\varepsilon$-Var)}]
	{ \hat \Gamma, x:\tau \vdash x: \tau~\kw{with} \varnothing }
	{}
	
	~~~
	
\infer[\textsc{($\varepsilon$-Resource)}]
 	{ \hat \Gamma, r: \{ r \} \vdash r : \{ r \}~\kw{with} \varnothing }
 	{} \\[3ex]
 	
 	~~~
	\infer[\textsc{($\varepsilon$-Abs)}]
	{ \hat \Gamma \vdash \lambda x:\tau_2 . \hat e : \hat \tau_2 \rightarrow_{\varepsilon_3} \hat \tau_3~\kw{with} \varnothing }
	{ \hat \Gamma, x: \hat \tau_2 \vdash \hat e: \hat \tau_3~\kw{with} \varepsilon_3 }
	
	~~~
	
\infer[\textsc{($\varepsilon$-App)}]
	{ \hat \Gamma \vdash \hat e_1 \hat e_2 : \hat \tau_3~\kw{with} \varepsilon_1 \cup \varepsilon_2 \cup \varepsilon  }
	{ \hat \Gamma \vdash \hat e_1: \hat \tau_2 \rightarrow_{\varepsilon} \hat \tau_3~\kw{with} \varepsilon_1 & \hat \Gamma \vdash \hat e_2: \hat \tau_2~\kw{with} \varepsilon_2 } \\[3ex]
	
\infer[\textsc{($\varepsilon$-OperCall)}]
	{ \hat \Gamma \vdash \hat e.\pi: \kw{Unit} \kw{with} \{ \bar r.\pi \} }
	{ \hat \Gamma \vdash \hat e: \{ \bar r \} & \forall r \in \bar r \mid r: \{ r \} \in \Gamma & \pi \in \Pi } \\[3ex]

\infer[\textsc{($\varepsilon$-Subsume)}]
	{ \hat \Gamma \vdash e: \tau' ~\kw{with} \varepsilon'}
	{ \hat \Gamma \vdash e: \tau ~\kw{with} \varepsilon & \tau <: \tau' & \varepsilon \subseteq \varepsilon'}  \\[3ex]

\infer[\textsc{($\varepsilon$-Import)}]
	{ \hat \Gamma \vdash \kwa{import}(\varepsilon)~x = \hat e~\kw{in} e: \kwa{annot}(\tau, \varepsilon)~\kw{with} \varepsilon \cup \varepsilon_1 }
{{\def\arraystretch{1.4}
  \begin{array}{c}
\hat \Gamma \vdash \hat e: \hat \tau~\kw{with} \varepsilon_1
~~~~~~
\varepsilon = \kwa{effects}(\hat \tau) \\
\kwa{ho \hyphen safe}(\hat \tau, \varepsilon) ~~~~~~ x: \kwa{erase}(\hat \tau) \vdash e: \tau
  \end{array}}} 
 
\end{array}
\]


\vspace{-7pt}
\caption{Type-with-effect judgements.}
\label{This is the label.}
\end{figure}

\noindent
$\epscalc$ has a new kind of judgement. $\Gamma \vdash \hat e: \hat \tau~\kw{with} \varepsilon$ can be read as saying that $\hat e$, if it halts, will produce a value of type $\hat \tau$ and incur at most the set of effects $\varepsilon$. This judgement gives a conservative approximation as to what will happen; some of the effects in the typing judgement may not actually happen at runtime.

The simplest rules are those which type values as having no effect. Although a function and a resource literal can both capture capabilities, you must do something with them (apply the function, operate on the resource) to incur a runtime effect.

The effects of a lambda application are: the effects of evaluating its subexpressions, and the effects incurred by executing the body of the lambda to which the left-hand side evaluates. Those last effects are pulled from the label on the lambda's arrow-type.

The effects of an operation call are: the effects of evaluating the subexpression, and the single effect incurred when the subexpression is reduced to a resource literal $r$, and operation $\pi$ is invoked on it. It is not always possible to know statically which exact resource literal the subexpression reduces to (if it halts at all). Figure 3.8. shows such an example. The safe approximation is to say that the operation call $\hat e.\pi$ incurs $\pi$ on every possible resource to which $\hat e$ might evaluate. In the case of Figure 3.8., this would be $\kwa{\{ File.write, Socket.write \}}$.

Because we're not really interested in what exactly an operation call does, every operation is valid for every resource. This can give bizarre programs --- $\kwa{Sensor.readTemp}$ seems like a sensible operation call, but what about $\kwa{File.readTemp}$? --- however, an adequate treatment is outside of the scope of $\epscalc$, so we gloss over such programs.

\textbf{It actually might be possible to figure out the exact literal if the system's not Turing complete, since the simply-typed lambda calculus is strongly normalising (and this is basically that, with a few extras), so be careful about this claim} \\

\begin{figure}[h]

\begin{lstlisting}
def getResource(b: Bool): { File, Socket } with $\varnothing$ =
    if b then File else Socket
 
val boolVal: Bool = System.randomBool
getResource(boolVal).write

\end{lstlisting}
\vspace{-7pt}
\caption{We cannot statically determine which branch will execute, so the safe approximation for $\kwa{getResource(boolVal).write}$ is $\{ \kwa{File.write, Socket.write} \}$.}
\label{This is the label.}
\end{figure}

\noindent
The most interesting rule is $\varepsilon$-Import. This rule is set up to ensure the interaction between labelled and unlabelled code is capability-safe. We type $e$ with $x: \kwa{erase}(\hat \tau)$. This eliminates ambient authority, because the only free variables in $e$ will be those selected by the interface $\hat \tau$.

For our rule to be capability-safe, we need to ensure that any higher-order function in scope is expecting the set of capabilities in $\hat \tau$. If not, we could exercise ambient authority by passing that higher-order function a capability from $\hat \tau$ which it hadn't selected. This is the purpose of $\hosafe{\hat \tau}{\varepsilon}$: all higher-order functions in scope need to be expecting any capability they might be passed.

In the conclusion of the rule we annotate the unlabelled code's effects as $\kwa{effects}(\hat \tau)$. Because this is the full set of capabilities over which $e$ has access, and because this set is higher-order safe, we shall see this annotation is sound.



\begin{figure}[h]
\noindent
$\fbox{$\hat \tau <: \hat \tau$}$

\[
\begin{array}{c}

\infer[(\textsc{S-Effects})]
	{\hat \tau_1 \rightarrow_{\varepsilon} \hat \tau_2 <: \hat \tau_1' \rightarrow_{\varepsilon'} \hat \tau_2'}
	{\varepsilon \subseteq \varepsilon' & \hat \tau_2 <: \hat \tau_2' & \hat \tau_1' <: \hat \tau_1}
~~~
\infer[(\textsc{S-Resources})]
	{ \{ \bar r_1 \} <: \{ \bar r_2 \} }
	{ r \in r_1 \implies r \in r_2 }

\end{array}
\]
\vspace{-7pt}
\caption{Subtyping judgements in the epsilon calculus.}
\label{This is the label.}
\end{figure}

\noindent
In addition to the usual subtyping rules from $\lambda^{\rightarrow}$ between $\tau$ terms, we introduce two more for $\hat \tau$ terms. 

The rule for functions is contravariant in the input-type and covariant in the output-type (as in $\lambda^{\rightarrow}$), and requires the effects of the super-type to be an upper-bound of the effects of the sub-type. We can think of this in terms of Liskov's substitution principle: if the subtype incurred an effect the supertype hadn't declared, it would violate the supertype's interface.

The rule for resources says that a superset of resources is a subtype.













\begin{figure}[h]

\noindent
\fbox{$\hat e \longrightarrow \hat e~|~\varepsilon$}

\[
\begin{array}{c}

\infer[\textsc{(E-App1)}]
	{\hat e_1 \hat e_2 \longrightarrow \hat e_1' \hat e_2~|~\varepsilon}
	{\hat e_1 \longrightarrow \hat e_1'~|~\varepsilon}
	~~~~~~
\infer[\textsc{(E-App2)}]
	{\hat v_1 \hat e_2 \longrightarrow \hat v_1 \hat e_2'~|~\varepsilon} 
	{\hat e_2 \longrightarrow \hat e_2'~|~\varepsilon}
~~~~~~
\infer[\textsc{(E-App3)}]
	{ (\lambda x: \hat \tau.\hat e) \hat v_2 \longrightarrow [\hat v_2/x]\hat e~|~\varnothing }
	{}\\[4ex]
	
\infer[\textsc{(E-OperCall1)}]
	{\hat e.\pi \longrightarrow \hat e'.\pi~|~\varepsilon }
	{\hat e \rightarrow \hat e'~|~\varepsilon}
		
	~~~~~~
	
\infer[\textsc{(E-OperCall2)}]
	{r.\pi \longrightarrow \kwa{unit}~|~\{ r.\pi \} }
	{r \in R & \pi \in \Pi}
	 \\[4ex]
	 
\infer[\textsc{(E-Import1)}]
	{\kwa{import}(\varepsilon)~x = \hat e~\kw{in} e \longrightarrow \kwa{import}(\varepsilon)~x = \hat e'~\kw{in} e~|~\varepsilon'}
	{\hat e \longrightarrow \hat e'~|~\varepsilon'}\\[4ex]

\infer[\textsc{(E-Import2)}]
	{\kwa{import}(\varepsilon)~x = \hat v~\kw{in} e \longrightarrow [\hat v/x]\kwa{annot}(e, \varepsilon)~|~\varnothing}
	{}

\end{array}
\]


\vspace{-7pt}
\caption{Single-step reductions.}
\label{This is the label.}
\end{figure}

\noindent
A single-step reduction takes an expression to a pair consisting of an expression and a set of runtime effects. The rules $\textsc{E-App1}$, $\textsc{E-App2}$, $\textsc{E-OperCall1}$, $\textsc{E-Import1}$ all reduce a single subexpression.

$\textsc{E-App3}$ is the standard $\lambda^{\rightarrow}$ rule for applying a value to a lambda, by performing substitution on the lambda body.

$\textsc{E-OperCall2}$ performs an operation on a resource literal. In this case it reduces to $\kwa{unit}$ (which is a derived form in our calculus; see 3.4. Encodings). This choice reflects the fact that $\epscalc$ doesn't model the potentially varied return types of functions. 

$\textsc{E-Import2}$ performs module resolution. The (unlabelled) body of code is annotated with the set of effects captured by the interface, and then the value being imported is substituted into the body of code.


\begin{figure}[h]

\noindent
\fbox{$\hat e \longrightarrow^{*} \hat e~|~\varepsilon$}

\[
\begin{array}{c}

\infer[\textsc{(E-MultiStep1)}]
	{\hat e \rightarrow^{*} \hat e~|~\varnothing}
	{}
~~~
\infer[\textsc{(E-MultiStep2)}]
	{\hat e \rightarrow^{*} \hat e'~|~\varepsilon}
	{\hat e \rightarrow \hat e'~|~\varepsilon} \\[3ex]
	
\infer[\textsc{(E-MultiStep3)}]
	{\hat e \rightarrow^{*} \hat e''~|~\varepsilon_1 \cup \varepsilon_2}
	{\hat e \rightarrow^{*} \hat e'~|~\varepsilon_1 & \hat e' \rightarrow^{*} \hat e''~|~\varepsilon_2}
\end{array}
\]

\vspace{-7pt}
\caption{Multi-step reductions.}
\label{This is the label.}
\end{figure}

\noindent
A multi-step reduction consists of zero\footnote{We permit multi-step reductions of length zero to be consistent with Pierce, who defines multi-step reduction as a reflexive relation\cite[p. 39]{tapl}.} or more single-step reductions. The resulting effect-set is the union of all the single-steps taken.

\subsection{Soundness of $\epscalc$}

\noindent
Our goal is to show $\epscalc$ is sound. This requires an appropriate notion of \textit{effect-soundness}.

\begin{theorem}[Soundness]
If $\hat \Gamma \vdash \hat e_A: \hat \tau_A~\kw{with} \varepsilon_A$ and $\hat e_A$ is not a value, then $e_A \longrightarrow e_B~|~\varepsilon$, where $\hat \Gamma \vdash e_B: \hat \tau_B~\kw{with} \varepsilon_B$ and $\hat \tau_B <: \hat \tau_A$ and $\varepsilon_B \cup \varepsilon \subseteq \varepsilon_A$.
\end{theorem}

\noindent
This definition of soundness is the same as in $\lambda^{\rightarrow}$ but for an extra conclusion: $\varepsilon_B \cup \varepsilon \subseteq \varepsilon_A$. Intuitively, $\varepsilon_A$ is the approximation of what runtime effects the reduction of $\hat e_A$ will incur, $\varepsilon$ is the actual set of effects $\hat e_A$ incurred (at most a singleton because we are working with single-step reduction), and $\varepsilon_B$ is the approximation of what runtime effects the reduction of $\hat e_B$ will incur. Evidently we want $\varepsilon \subseteq \varepsilon_A$; an approximation which accounts for every runtime effect is a sound one. We also want $\varepsilon_B \subseteq \varepsilon_A$, so successive approximations only get better.

The soundness proof takes the standard approach of showing that progress and preservation hold of the calculus. This can be done immediately by observing some properties that follow immediately from the typing rules.

\begin{lemma}[Canonical Forms]
The following are true:
\begin{itemize}
	\setlength\itemsep{-0.7em}
	\item If $\hat \Gamma \vdash \hat v: \hat \tau~\kw{with} \varepsilon$ then $\varepsilon = \varnothing$.
	\item If $\hat \Gamma \vdash \hat v: \{ \bar r \}$ then $\hat v = r$ for some $r \in R$ and $\{ \bar r \} = \{ r \}$.
\end{itemize}
\end{lemma}


\begin{theorem}[Progress]
If $\hat \Gamma \vdash \hat e: \hat \tau~\kw{with} \varepsilon$ and $\hat e$ is not a value, then $\hat e \longrightarrow \hat e'~|~\varepsilon$.
\end{theorem}

\begin{proof} By induction on $\hat \Gamma \vdash \hat e: \hat \tau~\kw{with} \varepsilon$, for $\hat e$ not a value. If the rule is \textsc{$\varepsilon$-Subsumption} it follows by inductive hypothesis. If $\hat e$ has a reducible subexpression then reduce it. Otherwise use one of \textsc{$\varepsilon$-App3}, \textsc{$\varepsilon$-OperCall2}, or \textsc{$\varepsilon$-Import2}.
\end{proof}

\noindent
To prove preservation, we need to know types and effects are preserved under substitution. The substitution lemma gives us this result. It says that if $x$ is bound to a type, and a value $\hat v$ of that type is substituted into $\hat e$, then the type and effect of $\hat e$ remain unchanged. Key to this property is that $\hat v$ is a value, so by canonical forms it cannot introduce effects that weren't already in $\hat e$. Beyond this observation, the proof is routine.

\begin{lemma}[Substitution]
If $\hat \Gamma, x: \hat \tau' \vdash e: \hat \tau~\kw{with} \varepsilon$ and $\hat \Gamma \vdash \hat v: \hat \tau'~\kw{with} \varnothing$ then $\hat \Gamma \vdash [\hat v/x]e: \hat \tau~\kw{with} \varepsilon$.
\end{lemma}

\begin{proof} By induction on $\hat \Gamma, x: \hat \tau' \vdash e: \hat \tau~\kw{with} \varepsilon$.
\end{proof}

\noindent
The tricky case in preservation is when an $\kwa{import}$ expression is resolved. To show the reduction $\import{\varepsilon}{x}{\hat v}{e} \longrightarrow [\hat v/x]\annot{e}{\varepsilon}~|~\varnothing$ preserves soundness requires a few things. First, if $\hat \Gamma \vdash \import{\varepsilon}{x}{\hat v}{e}: \hat \tau_A~\kw{with} \varepsilon_A$, then we need to be able to type the reduced expression in the same context: $\hat \Gamma \vdash [\hat v/x]\annot{e}{\varepsilon}: \hat \tau_B~\kw{with} \varepsilon_B$. To be effect-sound, we need $\varepsilon_B \subseteq \varepsilon_A$. To be type-sound, we need $\hat \tau_B <: \hat \tau_A$. This motivates the next lemma, which relates a typing judgement of $e$ to a typing judgement of $\annot{e}{\varepsilon}$.

\begin{lemma}[Annotation]
If the following are true:

\begin{itemize}
	\setlength\itemsep{-0.7em}
	\item $\hat \Gamma \vdash \hat v : \hat \tau~\kw{with} \varnothing$
	\item $\Gamma, y: \kwa{erase}(\hat \tau) \vdash e: \tau$
	\item $\varepsilon = \kwa{effects}(\hat \tau)$
	\item $\kwa{ho \hyphen safe}(\hat \tau, \varepsilon)$
\end{itemize}

\noindent
Then $\hat \Gamma, \kwa{annot}(\Gamma, \varepsilon), y: \hat \tau \vdash \kwa{annot}(e, \varepsilon) : \kwa{annot}(\tau, \varepsilon)~\kw{with} \varepsilon \cup \kwa{effects}(\kwa{annot}(\Gamma, \varepsilon))$.
\end{lemma}

\begin{proof}
By induction on $\Gamma, y: \kwa{erase}(\hat \tau) \vdash e: \tau$.
\end{proof}

\noindent
The exact formulation of the Annotation lemma is very specific to the premises of \textsc{$\varepsilon$-Import2}, but generalised slightly to accommodate a proof by induction. The generalisation is to allow $e$ to be typed in any context $\Gamma$ with a binding for $y$. We can think of $\Gamma$ as encapsulating the ambient authority exercised by $e$. At the top-level of any program, we will always have $\Gamma = \varnothing$, because the typing judgement \textsc{$\varepsilon$-Import} always types $\kwa{import}$ expressions with just the authority being selected. However, inductively-speaking, there may be ambient capabilities. Consider $(\lambda x: \{ \kwa{File} \}.~\kwa{x.write})~\kwa{File}$. From the perspcetive of $\kwa{x.write}$, $\kwa{File}$ is an ambient capability, and so if we were to inductively apply the Annotation lemma, at this point, $\kwa{File} \in \Gamma$. However, because the code encapsulating $\kwa{x.write}$ selects $\kwa{File}$ by binding it to $x$ in the function declaration, this code is capability-safe.

Proving the annotation lemma requires an additional pair of lemmas, to relate $\hat \tau$ and $\kwa{annot}(\kwa{erase}(\hat \tau), \varepsilon)$.

\begin{lemma}
If $\kwa{effects}(\hat \tau) \subseteq \varepsilon$ and $\kwa{ho \hyphen safe}(\hat \tau, \varepsilon)$ then $\hat \tau <: \kwa{annot}(\kwa{erase}(\hat \tau), \varepsilon)$.
\end{lemma}

\begin{lemma}
If $\kwa{ho \hyphen effects}(\hat \tau) \subseteq \varepsilon$ and $\safe{\hat \tau}{\varepsilon}$ then $\kwa{annot(erase}(\hat \tau), \varepsilon) <: \hat \tau$.
\end{lemma}

\begin{proof}
By simultaneous induction on $\kwa{ho \hyphen safe}$ and $\kwa{safe}$.
\end{proof}

\noindent
There is a close relation between these lemmas and the subtyping rule for functions. In a subtyping relation between functions, the input type is contravariant. Therefore, if $\hat \tau = \hat \tau_1 \rightarrow_{\varepsilon'} \tau_2$ and we have $\hat \tau <: \kwa{annot}(\tau, \varepsilon)$, then we need to know $\kwa{annot}(\tau_1) <: \hat \tau_1$. This is why there are two lemmas, one for each direction.

Armed with the annotation lemma, we are now ready to prove the preservation theorem.

\begin{theorem}[Preservation]
If $\hat \Gamma \vdash \hat e_A: \hat \tau_A~\kw{with} \varepsilon_A$ and $\hat e_A \longrightarrow \hat e_B~|~\varepsilon$, then $\hat \tau_B <: \hat \tau_A$ and $\varepsilon_B \cup \varepsilon \subseteq \varepsilon_A$.
\end{theorem}

\begin{proof} By induction on $\hat \Gamma \vdash \hat e_A: \hat \tau_A~\kw{with} \varepsilon_A$, and then on $\hat e_A \longrightarrow \hat e_B~|~\varepsilon$. \\

\textit{Case:} \textsc{$\varepsilon$-App} Then $e_A = \hat e_1 \hat e_2$ and $\hat e_1: \hat \tau_2 \rightarrow_{\varepsilon} \hat \tau_3~\kw{with} \varepsilon_1$ and $\hat \Gamma \vdash \hat e_2: \hat \tau_2~\kw{with} \varepsilon_2$. If the reduction rule used was \textsc{E-App1} or \textsc{E-App2}, then the result follows by applying the inductive hypothesis to $\hat e_1$ and $\hat e_2$ respectively.

Otherwise the rule used was \textsc{E-App3}. Then $(\lambda x: \hat \tau_2.\hat e)\hat v_2 \longrightarrow [\hat v_2/x]\hat e~|~\varnothing$. By inversion on the typing rule for $\lambda x: \hat \tau_2.\hat e$ we know $\Gamma, x: \hat \tau_2 \vdash \hat e: \hat \tau_3~\kw{with} \varepsilon_3$. By canonical forms, $\varepsilon_2 = \varnothing$ because $\hat e_2 = \hat v_2$ is a value. Then by the substitution lemma, $\hat \Gamma \vdash [\hat v_2/x]\hat e : \hat \tau_3~\kw{with} \varepsilon_3$. By canonical forms, $\varepsilon_1 = \varepsilon_2 = \varnothing = \varepsilon_C$. Therefore $\varepsilon_A = \varepsilon_3 = \varepsilon_B \cup \varepsilon_C$.\\

\textit{Case:} \textsc{$\varepsilon$-OperCall}. Then $e_A = e_1.\pi$ and $\hat \Gamma \vdash e_1 : \{ \bar r \}~\kw{with} \varepsilon_1$. If the reduction rule used was \textsc{E-OperCall1} then the result follows by applying the inductive hypothesis to $\hat e_1$.

Otherwise the reduction rule used was \textsc{E-OperCall2} and $v_1.\pi \longrightarrow \kwa{unit}~|~\{ r.\pi \}$. By canonical forms, $\hat \Gamma \vdash v_1: \kwa{unit}~\kw{with} \{ r.\pi \}$. Also, $\hat \Gamma \vdash \kwa{unit}: \kwa{Unit}~\kw{with} \varnothing$. Then $\tau_B = \tau_A$. Also, $\varepsilon_C \cup \varepsilon_B = \{ r.\pi \} = \varepsilon_A$.\\

\textit{Case:} \textsc{$\varepsilon$-Import}. Then $e_A = \import{\varepsilon}{x}{\hat e}{e}$. If the reduction rule used was \textsc{E-Import1} then the result follows by applying the inductive hypothesis to $\hat e$.

Otherwise $\hat e$ is a value and the reduction used was \textsc{E-Import2}. The following are true:
\begin{enumerate}
	\setlength\itemsep{-0.7em}
	\item $e_A = \kwa{import}(\varepsilon)~x = \hat v~\kw{in} e$
	\item $\hat \Gamma \vdash e_A: \kwa{annot}(\tau, \varepsilon)~\kw{with} \varepsilon \cup \varepsilon_1$
	\item $\kwa{import}(\varepsilon)~x = \hat v~\kw{in} e \longrightarrow [\hat v/x]\kwa{annot}(e, \varepsilon)~|~\varnothing$
	\item $\hat \Gamma \vdash \hat v: \hat \tau~\kw{with} \varnothing$
	\item $\varepsilon = \kwa{effects}(\hat \tau)$
	\item $\kwa{ho \hyphen safe}(\hat \tau, \varepsilon)$
	\item $x: \kwa{erase}(\hat \tau) \vdash e: \tau$
\end{enumerate}

\noindent
Apply the annotation lemma with $\Gamma = \varnothing$ to get $\hat \Gamma, x: \hat \tau \vdash \kwa{annot}(e, \varepsilon): \kwa{annot}(\tau, \varepsilon)~\kw{with} \varepsilon$. From assumption (4) we know $\hat \Gamma \vdash \hat v: \hat \tau~\kw{with} \varnothing$, and so the substitution lemma may be applied, giving $\hat \Gamma \vdash [\hat v/x]\kwa{annot}(e, \varepsilon): \kwa{annot}(\tau, \varepsilon)~\kw{with} \varepsilon$. By canonical forms, $\varepsilon_1 = \varepsilon_C = \varnothing$. Then $\varepsilon_B = \varepsilon = \varepsilon_A \cup \varepsilon_C$. By examination, $\tau_A = \tau_B = \kwa{annot}(\tau, \varepsilon)$.
\end{proof}

\noindent
Our statement of soundness combines the progress and preservation theorems into one.

\begin{theorem}[Soundness]
If $\hat \Gamma \vdash \hat e_A: \hat \tau_A~\kw{with} \varepsilon_A$ and $\hat e_A$ is not a value, then $e_A \longrightarrow e_B~|~\varepsilon$, where $\hat \Gamma \vdash e_B: \hat \tau_B~\kw{with} \varepsilon_B$ and $\hat \tau_B <: \hat \tau_A$ and $\varepsilon_B \cup \varepsilon \subseteq \varepsilon_A$.
\end{theorem}
\begin{proof}
If $\hat e_A$ is not a value then the reduction exists by the progress theorem. The rest follows by the preservation theorem.
\end{proof}

Knowing that single-step reductions are sound, multi-step reductions can straight-forwardly be be shown to also be sound. This is done by inductively applying single-step soundness to the length of the multi-step reduction.

\begin{theorem}[Multi-step Soundness]
If $\hat \Gamma \vdash \hat e_A: \hat \tau_A~\kw{with} \varepsilon_A$ and $e_A \longrightarrow^{*} e_B~|~\varepsilon$, where $\hat \Gamma \vdash e_B: \hat \tau_B~\kw{with} \varepsilon_B$ and $\hat \tau_B <: \hat \tau_A$ and $\varepsilon_B \cup \varepsilon \subseteq \varepsilon_A$.
\end{theorem}

\begin{proof} By induction on the length of the multi-step reduction. If the length is 0 then $e_A = e_B$ and the result holds vacuously. If the length is 1 the result holds by soundness of single-step reductions. if the length is $n+1$, then the first $n$-step reduction is sound by inductive hypothesis and the last step is sound by single-step soundness, so the entire $n+1$-step reduction is sound.
\end{proof}






