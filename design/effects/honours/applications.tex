
\chapter{Applications}

In this chapter we show how $\epscalc$ can be used in practice, and show how its rules can enable effect reasoning in existing capability-safe languages. This will take the form of writing a program in a high-level, capability-safe language, translating it to an equivalent $\epscalc$ program, and demonstrating how the rules of $\epscalc$ enable reasoning about the use of effects.

In this section the high-level programs will be written in a version of Wyvern. Wyvern is a pure, object-oriented, capability-safe language. It has a first-class module system, in which modules and objects are treated uniformly. Although $\epscalc$ does not have objects, the example Wyvern programs can be expressed using functions. This does not mean the examples given aren't demonstrating useful and realistic situations --- we simply do not need the added expressiveness given by self-referential objects.

In section 4.1. we discuss how the translation from Wyvern to $\epscalc$ will work, and what simplifying assumptions are made in our examples. This also serves as a gentle introduction to Wyvern's syntax. A variety of scenarios are then explored in 4.2. to show how the rules of $\epscalc$ can help developers in practice.

\section{Translations and Encodings}

Our aim is to develop some notation to help us translate Wyvern programs into $\epscalc$. Our approach will be to encode these additional rules and forms into the base language of $\epscalc$; essentially, to give common patterns and forms a short-hand, so they can be easily named and recalled. This is called \textit{sugaring}. When these derived forms are collapsed into their underlying representation, it is called \textit{desugaring}. We are going to introduce several rules to show a Wyvern program might be considered syntactic sugar for an $\epscalc$ program, and translate examples by desugaring according to our rules.

\subsection{Unit}

$\kwa{Unit}$ is a type inhabited by exactly one value. It conveys the absence of information; in $\epscalc$ an operation call on a resource literal reduces to $\unit$ for this reason. We define $\unit \defn \lambda x: \varnothing. x$. The $\unit$ literal is the same in both annotated and naked code. In annotated code, it has the type $\Unit \defn \varnothing \rightarrow_{\varnothing} \varnothing$, while in naked code it has the type $\Unit \defn \varnothing \rightarrow \varnothing$. While these are technically two seperate types, we will not distinguish between the annotated and naked versions, simply referring to them both as $\Unit$.

Note that $\unit$ is a value, and because $\varnothing$ is uninhabited (there is no empty resource literal), $\unit$ cannot be applied to anything. Furthermore, $\vdash \unit: \Unit~\kw{with} \varnothing$ by \textsc{$\varepsilon$-Abs}, and $\vdash \unit: \Unit$ by \textsc{T-Abs}. This leads to the derived rules in \ref{fig:unit_rules}.

\begin{figure}[h]


\fbox{$\Gamma \vdash e: \tau$} \\
\fbox{$\hat \Gamma \vdash \hat e: \hat \tau~\kw{with} \varepsilon$}


\[
\begin{array}{c}

\infer[\textsc{(T-Unit)}]
	{\Gamma \vdash \unit : \Unit}
	{} ~~~~

\infer[(\textsc{$\varepsilon$-Unit})]
	{\hat \Gamma \vdash \unit : \Unit~\kw{with} \varnothing}
	{}

\end{array}
\]

\caption{Derived $\kwa{Unit}$ rules.}
\label{fig:unit_rules}
\end{figure}

Since $\unit$ represents the absence of information, we also use it as the type when a function either takes no argument, or returns nothing. \ref{fig:unit_sugaring} shows the definition of a Wyvern function which takes no argument and returns nothing, and its corresponding representation in $\epscalc$.

\begin{figure}[h]

\begin{lstlisting}
def method():Unit
   unit
\end{lstlisting}

\begin{lstlisting}
$\lambda$x:Unit. unit
\end{lstlisting}

\caption{Desugaring of functions which take no arguments or return nothing.}
\label{fig:unit_sugaring}
\end{figure}

\subsection{Let}

\noindent
The expression $\letxpr{x}{\hat e_1}{\hat e_2}$ first binds the value $\hat e_1$ to the name $x$ and then evaluates $\hat e_2$. We can generalise by allowing $\hat e_1$ to be a non-value, in which case it must first be reduced to a value. If $\Gamma \vdash \hat e_1: \hat \tau_1$, then $\letxpr{x}{\hat e_1}{\hat e_2} \defn (\lambda x: \hat \tau_1 . \hat e_2) \hat e_1$. Note that if $\hat e_1$ is a non-value, we can reduce the $\kwa{let}$ by \textsc{E-App2}. If $\hat e_1$ is a value, we may apply \textsc{E-App3}, which binds $\hat e_1$ to $x$ in $\hat e_2$. This is fundamentally a lambda application, so it can be typed using \textsc{$\varepsilon$-App} (or \textsc{T-App}, if the terms involved are unlabelled). The new rules in \ref{fig:let_rules} capture these derivations.

\begin{figure}[h]

\fbox{$\Gamma \vdash e: \tau$} \\
\fbox{$\hat \Gamma \vdash \hat e: \hat \tau~\kw{with} \varepsilon$} \\
\fbox{$\hat e \rightarrow \hat e ~|~ \varepsilon$}

\[
\begin{array}{c}

	~~~
	
	\infer[\textsc{($\varepsilon$-Let)}]
	{\Gamma \vdash \letxpr{x}{e_1}{e_2}: \tau_2}
	{\Gamma \vdash e_1: \tau_1 & \Gamma, x: \tau_1 \vdash e_2: \tau_2} \\[2ex]

\infer[\textsc{($\varepsilon$-Let)}]
	{\hat \Gamma \vdash \letxpr{x}{\hat e_1}{\hat e_2} : \hat \tau_2~\kw{with} \varepsilon_1 \cup \varepsilon_2}
	{\hat \Gamma \vdash \hat e_1 : \hat \tau_1~\kw{with} \varepsilon_1 & \hat \Gamma, x: \hat \tau_1 \vdash \hat e_2: \hat \tau_2~\kw{with} \varepsilon_2} \\[2ex]
	
\infer[\textsc{(E-Let1)}]
	{\letxpr{x}{\hat e_1}{\hat e_2} \longrightarrow \letxpr{x}{\hat e_1'}{\hat e_2}~|~\varepsilon_1}
	{\hat e_1 \longrightarrow \hat e_1'~|~\varepsilon_1} \\[2ex]
	
\infer[\textsc{(E-Let2)}]
	{\letxpr{x}{\hat v}{\hat e} \longrightarrow [\hat v/x]\hat e~|~\varnothing}
	{} 

\end{array}
\]

\caption{Derived $\kwa{let}$ rules.}
\label{fig:let_rules}
\end{figure}

$\kwa{let}$ expressions can be used to sequence computations. Intuitively, the $\kwa{let}$ expression simply names the results of the intemediate steps and then ignores them in its body. When we ignore the result of a computation we shall bind it to $\_$ instead of a real name, to suggest the result isn't important and prevent the naming of unused variables. \ref{fig:let_rules} shows how this is done.

\begin{figure}[h]

\begin{lstlisting}
def method(f: {File}):Unit with {File.open, File.write, File.close}
   f.open
   f.write(``hello, world!'')
   f.close
\end{lstlisting}

\begin{lstlisting}
$\lambda$f: {File}.
   let _ = f.open in
   let _ = f.write in
   f.close
\end{lstlisting}

\caption{Desugaring of a sequence of computations.}
\label{fig:let_rules}
\end{figure}

\subsection{Modules and Objects}

Wyvern's modules are first-class and desugar into objects; invoking a method inside a module is no different from invoking an object's method. There are two kinds of modules: pure and resourceful. For our purposes, a pure module is one with no (transitive) authority over any resources, while a resource module has (transitive) authority over some resource. A pure module may still be given a capability, for example by requesting it in a function signature, but it may not possess or capture the capability for longer than the duration of the method call. \ref{fig:wyv_modules} shows an example of two modules, one pure and one resourceful, each declared in a seperate file. Note how pure modules are declared with the $\kwa{module}$ keyword, while resource modules are declared with the $\kwa{resource~module}$ keywords.

\begin{figure}[h]

\begin{lstlisting}
module PureMod

def tick(f: {File}):Unit
   f.append

\end{lstlisting}

\begin{lstlisting}
resource module ResourceMod
require File

def tick():Unit with {File.append}
   File.append
\end{lstlisting}

\caption{Definition of two modules, one pure and the other resourceful.}
\label{fig:wyv_modules}
\end{figure}

Wyvern is capability-safe, so resource modules must be instantiated with the capabilities they require. In \ref{fig:wyv_modules}, $\kwa{ResourceMod}$ requests the use of a $\kwa{File}$ capability, which must be supplied to it from someone already possessing it. Modules are behaving like objects in this way, because they require explicit instantiation. \ref{fig:wyv_module_instantiation} demonstrates how the two modules above would be instantiated and used.

To prevent infinite regress the $\kwa{File}$ must, at some point, be introduced into the program. This happens in a special main module. When the program begins execution, the $\kwa{File}$ capability is passed into the program from the system environment. All these initial capabilities are modelled in $\epscalc$ as resource literals. They are then propagated by the top-level entry point.

\begin{figure}[h]

\begin{lstlisting}
require File
instantiate PureMod
instantiate ResourceMod(File)

def main():Unit
   PureMod.tick(File)
   ResourceMod.tick()
\end{lstlisting}

\caption{Definition of two modules, one pure and the other resourceful.}
\label{fig:wyv_module_instantiation}
\end{figure}

Before explaining our translation of Wyvern programs into $\epscalc$, we must explain several simplifications made in all of our examples which enable our particular desugaring.

Objects are only ever used in the form of modules. Modules only ever contain functions and other modules, and have no mutable fields. The examples contain no recursion or self-reference, including a module invoking its own functions. Modules will not reference each other cyclically. Lastly, modules only contain one function definition. Despite these simplifications, the chosen examples will highlight the essential aspects of $\epscalc$.

Because modules do not exercise self-reference and only contain one function definition, they will be modelled as functions in $\epscalc$. Applying this function will be equivalent to applying the single function definition in the module.

A collection of modules is desugared into $\epscalc$ as follows. First, a sequence of let-bindings are used to name constructor functions which, when given the capabilities requested by a module, will return an instance of the module. If the module does not require any capabilities then it will take $\Unit$ as its argument. The constructor function for $\kwa{M}$ is called $\kwa{MakeM}$. A function is then defined which represents the $\kwa{main}$ function, which is the entry point into the program. This $\kwa{main}$ function will instantiate all the modules by invoking the constructor functions, and then execute the body of code in main. Finally, the main function is invoked with the primitive capabilities it needs.

To demonstrate this process, \ref{fig:wyv_tutorial_desugaring} shows how the examples above desugar. Lines 1-3 define the constructor for $\kwa{PureMod}$; since $\kwa{PureMod}$ requires no capabilities, the constructor takes $\Unit$ as an argument on line 2. Lines 6-8 define the constructor for $\kwa{ResourceMod}$; it requires a $\kwa{File}$ capability, so the constructor takes $\kwa{\{File\}}$ as its input type on line 7. The entry point to the program is defiend on lines 11-15, which invokes the constructors and then runs the body of the $\kwa{main}$ method. Lastly, line 17 starts everything off by invoking $\kwa{Main}$ with the initial set of capabilities, which in this case is just $\kwa{File}$.

\begin{figure}[h]

\begin{lstlisting}
let MakePureMod =
   $\lambda$x:Unit.
      $\lambda$f:{File}. f.append
in

let MakeResourceMod =
   $\lambda$f:{File}.
      $\lambda$x:Unit. f.append
in

let MakeMain =
   $\lambda$f:{File}.
      $\lambda$x: Unit.
         let PureMod = (MakePureMod unit) in
         let ResourceMod = (MakeResourceMod f) in
         let _ = (PureMod f) in (ResourceMod unit) in

(MakeMain File) unit
\end{lstlisting}

\caption{Desugaring of $\kwa{PureMod}$ and $\kwa{ResourceMod}$ into $\epscalc$.}
\label{fig:wyv_tutorial_desugaring}
\end{figure}




\section{Examples}

In this section we present several scenarios where a developer may be forced to reason about the use of effects, and show how the capability-based reasoning of effects can assist them. In some scenarios, a program exhibits a certain nefarious behaviour, in which case capability-based reasoning can automatically detect this behaviour and reject it. Other scenarios are more qualitative; perhaps a developer must make a design choice and none of the alternatives \textit{prima facie} stand out. In such cases, capability-based reasoning might supply them with useful information, enabling tehm to make more informed design choices. We also hope to convince the reader that the rules of $\epscalc$ have practical worth, and could be used to enrich existing capability-safe languages.

The format of each section is as follows. A program is introduced which exhibits some bad behaviour or demonstrates a particular story about software development. The language used is \textit{Wyvern}; a pure, object-oriented, capability-safe language with first-class modules-as-objects. We show how the Wyvern program can be written as a corresponding $\epscalc$ program and sketch a derivation showing how the rules of $\epscalc$ and a sketch a derivation showing how the rules of $\epscalc$ would solve the relevant problem.

We take some shortcuts with the translation of Wyvern into $\epscalc$. Our ``objects'' are really records of functions; the difference between the two is self-reference. The particular examples chosen do not require self-reference, so no important properties are lost by treating Wyvern objects as records.

\subsection{Unannotated Client}

In Figure \ref{fig:eg1} an annotated $\kwa{Logger}$ module provides its client the ability to append to a particular $\kwa{File}$ resource. $\kwa{File}$ is a primitive capability passed into the program when it begins execution, perhaps from the system environment or a virtual machine. The $\kwa{Logger}$ module presents a controlled subset of operations on the $\kwa{File}$ viz. $\kwa{File.append}$. The program consists of an unannotated client which instantiates the $\kwa{Logger}$ module with the capability it selects ($\kwa{File}$) and then attempts to log.

If the client code is executed, what effects will it have? The answer is not immediately clear from the client's source-code, but a capability-based argument goes as follows: because the client code can typecheck needing only $\kwa{Logger}$, then whatever effects presented by $\kwa{Logger}$ are an upper-bound on the effects of the client.

\begin{figure}[h]

\begin{lstlisting}
resource module Logger
require File

def log(): Unit with File.append =
    File.append(``message logged'')
\end{lstlisting}

\begin{lstlisting}
module Client
require Logger

def run(): Unit =
   Logger.log()
\end{lstlisting}

\begin{lstlisting}
resource module Main
require File
instantiate Logger(File)
instantiate Client(Logger)

Client.run()
\end{lstlisting}

\caption{A $\kwa{logger}$ client doesn't need to add effect labels; these can be inferred.}
\label{fig:eg1}
\end{figure}

The desugaring first creates two functions, $\kwa{MakeLogger}$ and $\kwa{MakeClient}$, which instantiate the $\kwa{Logger}$ and $\kwa{Client}$ modules; the client code is treated as an implicit module. Lines 1-4 define a function which, given a $\kwa{File}$, returns a record containing a single $\kwa{log}$ function. Lines 6-8 define a function which, given a $\kwa{Logger}$, returns the unannotated client code, wrapped inside an $\kwa{import}$ expression selecting its needed authority. Lines 10-14 are the meat of the program; this function, when given a $\kwa{File}$ capability, creates the modules and then runs the client code. Program execution begins on line $16$, where $\kwa{Main}$ is given its initial set of capabilities --- which, in this case, is just $\kwa{File}$.

\begin{figure}[h]

\begin{lstlisting}
let MakeLogger =
   ($\lambda$f: File.
      $\lambda$x: Unit. f.append) in
          
let MakeClient =
   ($\lambda$logger: Logger.
      import(File.append) logger = logger in
         $\lambda$x: Unit. logger unit) in
          
let MakeMain =
   ($\lambda$f: File.
      $\lambda$x: Unit.
         let LoggerModule = MakeLogger f in
         let ClientModule = MakeClient LoggerModule in
         ClientModule unit) in

(MakeMain File) unit
\end{lstlisting}

\caption{Desugared version of Figure \ref{fig:eg1}.}
\label{fig:eg1_desugared}
\end{figure}

The interesting part  is on lines 7-8, where the unannotated code selects $\kwa{File.append}$ as its authority. This is exactly the effects of the logger, i.e. $\kwa{effects}(\Unit \rightarrow_{\kwa{File.append}} \Unit) = \{ \kwa{File.append} \}$. The code also satisfies the higher-order safety predicates, and the body of the $\kwa{import}$ expression typechecks in the empty context. Therefore, the unannotated code typechecks by \textsc{$\varepsilon$-Import}.

In such a small example the client could simply inspect the source code of $\kwa{Logger}$ to determine what effects it might have. Several situations can make this impossible or tedious. First, the manual approach loses efficiency when the system involves many modules of large size across code-ownership boundaries; capability-based reasoning tells you automatically. Second, the source code of $\kwa{Logger}$ might be obfuscated or unavailable, and the only useful information is that given by its signature. Lastly, the client may not care about effects in this situation; the program may be a quick-and-dirty throwaway, in which case it is nice that the capability-based reasoning still accepts the client code without annotations..

\subsection{API Violation}

Figure \ref{fig:eg2} inverts the roles of the last scenario: now, the annotated $\kwa{Client}$ wants to use the unannotated $\kwa{Logger}$. The $\kwa{Logger}$ module captures the $\kwa{File}$ capability, and exposes a single function $\kwa{log}$ with the $\kwa{File.append}$ effect. However, the $\kwa{Client}$ has a function $\kwa{run}$ which executes $\kwa{Logger.log}$, incurring the effect of $\kwa{File.append}$, but declares its set of effects as $\varnothing$. The implementation and the signature of $\kwa{Client.run}$ are inconsistent --- does the type system recognise this?

\begin{figure}[h]

\begin{lstlisting}
resource module Logger
require File

def log(): Unit =
    File.append(``message logged'')
\end{lstlisting}

\begin{lstlisting}
resource module Client
require Logger

def run(): Unit with $\varnothing$ =
   Logger.log()
\end{lstlisting}

\begin{lstlisting}
resource module Main
require File
instantiate Logger(File)
instantiate Client(Logger)

Client.run()
\end{lstlisting}

\caption{The unlabelled code in $\kwa{Logger}$ exercises authority exceeding that selected by $\kwa{Client}$.}
\label{fig:eg2}
\end{figure}

A desugaring is given in Figure \ref{fig:eg2_desugared}. Lines 1-3 define the function which instantiates the $\kwa{Logger}$ module. Lines 5-8 define the function which instantiates the $\kwa{Client}$ module. Lines 10-15 define the function which instantiates the $\kwa{Main}$ module. Line 17 initiates the program, supplying $\kwa{File}$ to the $\kwa{Main}$ module and invoking its main method. On lines 3-4, the unannotated code is modelled using an $\kwa{import}$ expression which selects $\varnothing$ as its authority. So far this coheres to the expectations of $\kwa{Client}$. However, \textsc{$\varepsilon$-Import} cannot be applied because the name being bound, $f$, has the type $\{ \File \}$, and $\fx{\{ \File \}} = \{ \kwa{File}.* \}$, which is inconsistent with the declared effects $\varnothing$.

\begin{figure}[h]

\begin{lstlisting}
let MakeLogger =
   ($\lambda$f: File.
      import($\varnothing$) f = f in
         $\lambda$x: Unit. f.append) in

let MakeClient =
	($\lambda$logger: Logger.
	   $\lambda$x: Unit. logger unit) in

let MakeMain =
   ($\lambda$f: File.
      let LoggerModule = MakeLogger f in
      let ClientModule = MakeClient LoggerModule in
      ClientModule unit) in

(MakeMain File) unit
\end{lstlisting}

\caption{Desugared version of Figure \ref{fig:eg2}.}
\label{fig:eg2_desugared}
\end{figure}

The only way for this to typecheck would be to annotate $\kwa{Client.run}$ as having every effect on $\File$. This demonstrates how the effect-system of $\epscalc$ approximates unlabelled code: it simply considers it as having every effect which could be incurred on those resources in scope, which here is $\kwa{File}.*$.

\subsection{API Violation}

Figure \ref{fig:eg3} is a variation of the last example, but now $\kwa{Logger.log}$ is passed the $\kwa{File}$ capability, rather than possessing it. $\kwa{Logger.log}$ still incurs $\kwa{File.append}$ inside unannotated code, which causes the implementation of $\kwa{Client.run}$ to violate its signature.

\begin{figure}[h]

\begin{lstlisting}
module Logger

def log(f: {File}): Unit
    f.append(``message logged'')
\end{lstlisting}

\begin{lstlisting}
module Client
instantiate Logger(File)

def run(f: {File}): Unit with $\varnothing$
   Logger.log(File)
   
\end{lstlisting}

A desugared version is given in Figure \ref{fig:eg3}, which is largely the same as in the previous example, except 

\begin{lstlisting}
resource module Main
require File
instantiate Client

Client.run(File)
\end{lstlisting}

\caption{The unlabelled code in $\kwa{Logger}$ exercises authority exceeding that selected by $\kwa{Client}$.}
\label{fig:eg3}
\end{figure}




\begin{figure}[h]

\begin{lstlisting}
let MakeClient =
	($\lambda$x: Unit.
	   let MakeLogger =
	      ($\lambda$x: Unit.
	         import($\varnothing$) x=x in
	            $\lambda$f: {File}. f.append) in
      let LoggerModule = MakeLogger unit in
      $\lambda$f: {File}. LoggerModule f) in
	
let MakeMain =
   ($\lambda$f: {File}.
      $\lambda$x: Unit.
         let ClientModule = MakeClient unit in
         ClientModule f) in

(MakeMain File) unit
\end{lstlisting}

\caption{Desugared version of \ref{fig:eg3}.}
\label{fig:eg3_desugared}
\end{figure}


\subsection{API Violation}


\begin{figure}[h]

\begin{lstlisting}
resource module Logger
require File

def log(): Unit with {File.append, File.write} =
    File.append(``message logged'')
    File.write(``message written'')
\end{lstlisting}

\begin{lstlisting}
module Client

def run(l: Logger): Unit with {File.append} =
    l.log()
\end{lstlisting}

\begin{lstlisting}
resource module Main
require File
instantiate Logger(File)

Client.run(Logger)
\end{lstlisting}

\caption{This won't type because of a mismatch between the effects of $\kwa{Client}$ and the effects of $\kwa{Logger}$.}
\label{fig:eg4}
\end{figure}


\begin{figure}[h]

\begin{lstlisting}
let MakeLogger =
   ($\lambda$f: File.
      $\lambda$x: Unit. let _ = f.append in f.write) in
           
let MakeClient =
   ($\lambda$x: Unit.
      $\lambda$logger: Logger. logger unit) in
                  
let MakeMain =
   ($\lambda$f: File.
      $\lambda$x: Unit.
         let LoggerModule = MakeLogger f in
         let ClientModule = MakeClient unit in
         ClientModule.run LoggerModule) in

(MakeMain File) unit
\end{lstlisting}

\caption{Desugared version of Figure \ref{fig:eg4}.}
\label{fig:eg4_desugared}
\end{figure}


\subsection{Hidden Authority}

\begin{figure}[h]

\begin{lstlisting}
module Malicious

def stealData(f: {File}):Unit with {File.read} =
   f.read
\end{lstlisting}

\begin{lstlisting}
module Plugin
instantiate Malicious

def run(f: {File}): Unit with $\varnothing$ =
   Malicious.stealData(f)
\end{lstlisting}

\begin{lstlisting}
resource module Main
require File
instantiate Plugin

Plugin.run(File)
\end{lstlisting}

\caption{The $\kwa{Main}$ module transitively invokes a $\kwa{File.read}$ effect, violating its selected authority.}
\label{fig:eg5}
\end{figure}

\begin{figure}[h]

\begin{lstlisting}
let MakePlugin =
   ($\lambda$x: Unit.
      let MakeMalicious =
         ($\lambda$x: Unit. $\lambda$f: {File}. f.read) in
      let MaliciousModule = (MakeMalicious unit) in
      $\lambda$f: {File}. MaliciousModule f) in
      
let MakeMain =
   ($\lambda$f: File.
      $\lambda$x: Unit.
         let PluginModule = MakePlugin unit in
         PluginModule.run f) in

(MakeMain File) unit
\end{lstlisting}

\caption{Desugared version of Figure \ref{fig:eg5}.}
\label{fig:eg5_desugared}
\end{figure}

\subsection{Hidden Authority 2} 

\begin{figure}[h]

\begin{lstlisting}
module Malicious

def stealData(f: {File}):Unit =
   f.read
\end{lstlisting}

\begin{lstlisting}
module Plugin
instantiate Malicious

def run(f: {File}): Unit with $\varnothing$ =
   Malicious.stealData(f)
\end{lstlisting}

\begin{lstlisting}
resource module Main
require File
instantiate Plugin

Plugin.run(File)
\end{lstlisting}

\caption{The transitive invocation of $\kwa{File.read}$ now happens inside unannotated code, but the type system will still reject this program.}
\label{fig:eg6}
\end{figure}

\begin{figure}[h]

\begin{lstlisting}
let MakePlugin =
   ($\lambda$x: Unit.
      let MakeMalicious =
         ($\lambda$x: Unit.
            import($\varnothing$) x=x in
               $\lambda$f: {File}. f.read) in
      let MaliciousModule = (MakeMalicious unit) in
      $\lambda$f: {File}. MaliciousModule f) in
      
let MakeMain =
   ($\lambda$f: File.
      $\lambda$x: Unit.
         let PluginModule = MakePlugin unit in
         PluginModule.run f) in

(MakeMain File) unit
\end{lstlisting}

\caption{Desugared version of Figure \ref{fig:eg6}.}
\label{fig:eg6_desugared}
\end{figure}

\subsection{Hidden Authority 2} 

\begin{figure}[h]

\begin{lstlisting}
module Malicious

def log(f: Unit $\rightarrow$ Unit):Unit
   f()
\end{lstlisting}

\begin{lstlisting}
module Plugin
instantiate Malicious

def run(f: {File}): Unit with $\varnothing$
   Malicious.log($\lambda$x:Unit. f.read)
\end{lstlisting}

\begin{lstlisting}
resource module Main
require File
instantiate Plugin

Plugin.run(File)
\end{lstlisting}

\caption{The transitive invocation of $\kwa{File.read}$ happens when the unannotated code executes the function given to it.}
\label{This is the label.}
\end{figure}


\subsection{Resource Leak}

\begin{figure}[h]

\begin{lstlisting}
module Malicious

def log(f: Unit $\rightarrow$ File):Unit
   f().read
\end{lstlisting}

\begin{lstlisting}
module Plugin
instantiate Malicious

def run(f: {File}): Unit with $\varnothing$
   Malicious.log($\lambda$x:Unit. f)
\end{lstlisting}

\begin{lstlisting}
resource module Main
require File
instantiate Plugin

Plugin.run(File)
\end{lstlisting}

\caption{A resource leak allows $\kwa{Malicious}$ to gain access to the $\kwa{File}$ capability directly.}
\label{This is the label.}
\end{figure}
