\documentclass[12pt, a4paper, twoside, openright]{book}

\usepackage{vuwthesis} % sets up some local things, mostly the front page

\usepackage{palatino} % sets palatino as the default font

\usepackage{url} % for typesetting urls

\usepackage{bm}
\usepackage{listings}
\usepackage{proof}
\usepackage{amssymb}
\usepackage[margin=.9in]{geometry}
\usepackage{amsmath}
\usepackage{amsthm}
\usepackage[english]{babel}
\usepackage[utf8]{inputenc}
\usepackage{enumitem}
\usepackage{filecontents}
\usepackage{calc}
\usepackage[linewidth=0.5pt]{mdframed}
\usepackage{changepage}
\usepackage{ebproof}

\usepackage{thmtools}
\usepackage{thm-restate}

\usepackage{color}
\definecolor{grey}{gray}{0.92}

\lstset{
tabsize=3,
basicstyle=\ttfamily\small, commentstyle=\itshape\rmfamily, 
backgroundcolor=\color{grey},
numbers=left,
numberstyle=\tiny,
language=java,
moredelim=[il][\sffamily]{?},
mathescape=true,
showspaces=false,
showstringspaces=false,
columns=fullflexible,
escapeinside={(@}{@)}, morekeywords=[1]{def, if, then, else, with, val}}
\lstloadlanguages{Java,VBScript,XML,HTML}



% Allows numbered theorems, lemmata, etc.
\newtheorem{theorem}{Theorem}
\newtheorem{lemma}{Lemma}
\newtheorem{definition}{Definition}

\newcommand\defn{\mathrel{\overset{\makebox[0pt]{\mbox{\normalfont\tiny\sffamily def}}}{=}}}

\newcommand{\letxpr}[3]{\kw{let} #1 = #2~\kw{in} #3}	
\newcommand{\cond}[3]{\kw{if} #1~ \kw{then} #2~\kw{else} #3}
\newcommand{\fix}[1]{\kw{fix} #1}

\newcommand{\unit}{\kwa{unit}}
\newcommand{\Unit}{\kwa{Unit}}
\newcommand{\Int}{\kwa{Int}}
\newcommand{\Bool}{\kwa{Bool}}
\newcommand{\Real}{\kwa{Real}}
\newcommand{\Nat}{\kwa{Nat}}
\newcommand{\true}{\kwa{true}}
\newcommand{\false}{\kwa{false}}
\newcommand{\fst}{\kwa{fst}}
\newcommand{\snd}{\kwa{snd}}
\newcommand{\File}{\kwa{File}}
\newcommand{\hyphen}{\hbox{-}}
\newcommand{\kwa}[1]{\mathtt{#1}}
\newcommand{\kw}[1]{\mathtt{#1}~}
\newcommand{\reftype}[2]{\kwa{ref}(#1, #2)}

% Expressions.
\newcommand{\import}[4]{\kwa{import}(#1)~#2 = #3~\kw{in} #4}

\newcommand{\refnew}[4]{\kwa{new}_{#1} #2 = #3~\kw{in} #4}

% Functions.
\newcommand{\annot}[2]{\kwa{annot}(#1, #2)}
\newcommand{\erase}[1]{\kwa{erase}(#1)}
\newcommand{\fx}[1]{\kwa{effects}(#1)}
\newcommand{\hofx}[1]{\kwa{ho \hyphen effects}(#1)}

% Predicates.
\newcommand{\safe}[2]{\kwa{safe}(#1, #2)}
\newcommand{\hosafe}[2]{\kwa{ho \hyphen safe}(#1, #2)}

% Calculi

\newcommand{\stlc}{\lambda^{\rightarrow}}

\newcommand{\opercalc}{\kwa{OC}}

\newcommand{\epscalc}{\kwa{CC}}

\newcommand{\calc}
{\kwa{EBL}}

\newcommand{\fxtute}{\kwa{SEA}}



%\renewcommand{\baselinestretch}{1.00}


\begin{document}

\frontmatter
% Book style knows about front matter
% Report style doesn't so you need to set roman numbering etc yourself :-(

%%%%%%%%%%%%%%%%%%%%%%%%%%%%%%%%%%%%%%%%%%%%%%%%%%%%%%%

\title{Capability-Flavoured Effects}
\author{Aaron Craig}

\subject{Computer Science}
\abstract{
Privilege separation and least authority are principles for developing safe software, but existing languages offer insufficient techniques for allowing developers and architects to make informed design choices enforcing them. Languages adhering to the object-capability model impose constraints on the ways in which privileges are used and exchanged, giving rise to a form of lightweight effect-system. This effect-system allows architects and developers to make more informed choices about whether code from untrusted sources should be used. This paper develops an extension of the simply-typed lambda calculus to illustrate the ideas and proves it sound. }


\otherdegree{Bachelor of Science with Honours}



%%%%%%%%%%%%%%%%%%%%%%%%%%%%%%%%%%%%%%%%%%%%%%%%%%%%%%%




\maketitle

\chapter*{Acknowledgments}\label{C:ack} 

I would like to give thanks to the following people.

\begin{itemize}

	\item Alex Potanin, Jonathan Aldrich, and Lindsay Groves --- for being wonderful supervisors and giving their endless wisdom and support.
	
	\item Darya Kurilova and Julian Mackay --- for feedback and suggestions on the formalisms.
	
	\item Ewan Moshi --- for his friendship and support over the years.
	
	\item My family --- especially my parents, Mary and John, and my sisters, Amber and Rachel.

\end{itemize}


\tableofcontents


%%%%%%%%%%%%%%%%%%%%%%%%%%%%%%%%%%%%%%%%%%%%%%%%%%%%%%%

% book style knows about mainmatter
% if you are using report style you will have to rest page numbering etc.
\mainmatter

%%%%%%%%%%%%%%%%%%%%%%%%%%%%%%%%%%%%%%%%%%%%%%%%%%%%%%%

% individual chapters included here

\chapter{Introduction}\label{C:intro}

Good software is distinguished from bad software by design qualities such as security, maintainability, and performance. We are interested in how the design of a programming language and its type system can make it easier to write secure software.

There are different situations where we may not trust code. One example is in any development environment adhering to ideas of \textit{code ownership}, wherein developers may be responsible for particular components in the system \cite{bird_ownership}. When a developer writes code to interact with a component outside their domain of responsibility, they can make false assumptions or interact with the other component incorrectly, potentially introducing security bugs. Another setting involves applications which allow third-party plug-ins, in which case third-party code could be written by anyone, including the untrustworthy. One kind of application which does this is the web mash-up, which brings together several existing, disparate services into one system. In both cases we want the entire system to function securely, despite the existence of untrustworthy components.

It is difficult to determine if a piece of code is trustworthy, but a range of techniques might be used. One approach is to \textit{sandbox} the untrusted code inside a virtual environment. If anything goes wrong, damage is theoretically limited to the virtual environment, but in practice, this approach has many vulnerabilities \cite{coker15, maass16, watson07, schreuders13}. On the other hand, verification techniques allow for a robust analysis of the behaviour of code, but are heavyweight and require the developers using them to have a deep understanding of the techniques being employed \cite{kneuper97}. Furthermore, verification requires one to supply a complete specification of the system, which may itself be an undefined or evolving artifact during the development process. Lightweight analyses, such as type systems, are easy for the developer to use, but existing languages lack adequate controls for detecting and isolating untrustworthy components \cite{chen07, ter-louw08}. A qualitative approach might instead be employed, where software is developed according to best-practice guidelines. One such guideline is the \textit{principle of least authority}: that software components should only have access to the information and resources necessary for their purpose \cite{saltzer74}. For example, a logger module, which needs only to append to a file, should not have arbitrary read-write access. Another is \textit{privilege separation}, where the division of a program into components is informed by what resources are needed and how they are to be propagated \cite{saltzer75}. This report is interested in the class of lightweight analyses, and in particular how type systems could be used to reject unsafe programs or put developers in a more informed position to make qualitative assessments about their code.

One approach to privilege separation is the capability model. A \textit{capability} is an unforgeable token granting its bearer permission to perform some operation \cite{dennis66}. For example, a system resource like a file or socket can only be used through a capability granting operations on it. Capabilities also encapsulate the source of \textit{effects}, which describe intensional details about the way in which a program executes \cite{nielson99}. For example, a logger might $\kwa{append}$ to a $\kwa{File}$, and so executing its code would incur the $\kwa{File.append}$ effect. In the capability model, this would require the logger to possess a capability granting it the ability to append to files.

Although the idea of a capability is an old one in the access literature, there has been recent interest in the application of the idea to programming language design. Miller has identified the ways in which capabilities should proliferate to encourage \textit{robust composition} --- a set of ideas summarised as ``only connectivity begets connectivity'' \cite{miller06}. In this paradigm, actors in a program are explicitly parametrised by what capabilities they use. This enables one to reason about what privileges a component might exercise by examining its interface. Building on these ideas, Maffeis et. al. formalised the notion of a \textit{capability-safe} language, showing a subset of Caja (a JavaScript implementation) is capability-safe \cite{maffeis10}.

Effect systems were introduced by Lucassen and Gifford for the purposes of optimising pure code \cite{lucassen88}. They have also been applied to problems such as determining which functions might be invoked in a program \cite{tang94}, or determining which regions in memory may be accessed or updated \cite{talpin94}. Knowing what effects a piece of code might incur allows a developer to determine if code is trustworthy before executing it. This can be qualitatively assessed by comparing the static approximation of its effects to its expected least authority --- a ``logger'' implementation which writes to a $\kwa{Socket}$ is not to be trusted!

Despite these benefits, effect systems have seen little use in mainstream programming languages. Rytz et. al. believe verbosity is the main reason \cite{rytz2012}. Successive works have focussed on reducing the developer overhead through techniques such as effect-inference, but the benefit of capabilities for enabling effect-inference has not received much attention. Because capabilities encapsulate the source of effects, and because capability-safety impose constraints on how they propagate through a system, the task of determining what effects might be incurred by a piece of code is simplified. This is the key contribution of this report: the idea that capability-safety facilitates a low-cost effect system with minimal user overhead. 

We begin this report by discussing preliminary concepts involving the formal definition of programming languages, effect systems, and Miller's capability model. Chapter 3 introduces the Operation Calculus $\opercalc$, a typed, effect-annotated lambda calculus with a simple notion of capabilities and effect. Dropping the requirement that all code in a program must be effect-annotated, we develop the Capability Calculus $\epscalc$, which permits the nesting of unannotated code inside annotated code in a controlled, capability-safe manner with a new $\kwa{import}$ construct. A safe inference about the unannotated code can be made at these junctions. In chapter 4 we demonstrate how $\epscalc$ can model practical examples, finishing with a summary and comparison of some of the existing work in this area.
\chapter{Background}\label{C:background}

In this chapter we cover the necessary concepts and existing work informing this report. First we detail how a programming language and its type system are defined, and how to prove the type system is correct. For this purpose, we present a toy language called $\calc$. We then summarise a variant of the simply-typed lambda calculus $\stlc$. $\stlc$ is an historically important model of computation which serves as a basis for many programming languages, including the capability calculus $\epscalc$. $\epscalc$ is also a capability-based language with an effect system. To understand what this means we cover some existing work on effect systems and discuss Miller's capability model.

\section{Formally Defining a Programming Language}

A programming language can be defined by giving three sets of rules: a grammar, which defines syntactically legal terms; dynamic rules, which give the meaning of a program by how it is executed; and static rules, which determine whether programs meet certain well-behavedness properties. When a language has been defined we want to know its static rules are mathematically correct with respect to the dynamic rules.

Alongside the explanation of these concepts we develop Expression-based Language ($\calc$), a simple, typed language of arithmetic and boolean expressions. It is a language invented in this report for demonstrative purposes. Like every language we cover, it is expression-based, meaning that programs are evaluated to yield a value. Although $\calc$ is not very interesting, it will illustrate our general approach.

\subsection{Grammar}

The grammar of a language specifies what strings are syntactically legal. A syntactically legal string is called a \textit{term}. It is specified by giving the different categories of terms and the forms which instantiate those categories. The conventions for specifying a grammar are based on standard Backur-Naur form \cite{bnf}. Figure \ref{fig:ebl_grammar} shows a simple grammar describing integer literals and arithmetic expressions on them. In each rule, the metavariables range over the terms of the category for which they are named.

A $\calc$ program is an expression $e$, consisting of variable definitions, constants, and the application of boolean and arithmetic operators. A valid expression is either a variable, a constant (such as $3$, $0$, $\true$, or $\false$), the application of an operator $+$ or $\lor$ to two subexpressions, or a binding for a variable in a piece of code ($\kwa{let}$ expression). The following are $\calc$ terms: $x$, $y$, $3$, $3+2$, $\false \lor \true$, $3 \lor \false$, $\true + \false$, $\letxpr{x}{3}{x+1}$.

Although the grammar hs no brackets, a string like $3 + (x + 2)$ should be seen as a short-hand for the corresponding abstract syntax tree (AST), whose structure is given by the rules of the grammar. For some strings the AST is ambiguous, as in $3 + x + 2$, which might be parsed as $3 + (x + 2)$ or as $(3 + x) + 2$. How we parse and disambiguate strings is not relevant to us, so throughout the report we only ever consider strings which unambiguously correspond to terms in the grammar.\\

\begin{figure}[h]

\[
\begin{array}{c}

\begin{array}{lllr}

e & ::= & ~ & exprs: \\
	& | & x & variable \\
	& | & e + e & addition \\
	& | & e \lor e & disjunction \\
	& | & \letxpr{x}{e}{e} & let~expr. \\
	&&\\
	
v & ::= & ~ & values: \\
	& | & l & \Nat~constant \\
	& | & b & \Bool~constant \\
	&&\\

\end{array}

\end{array}
\]

\vspace{-12pt}
\caption{Grammar for $\calc$ expressions.}
\label{fig:ebl_grammar}
\end{figure}


\subsection{Dynamic Rules}

The dynamic rules of a language specify the meaning of terms. There are different approaches, but the one we use is called \textit{small-step semantics}, where the meaning of a program is given by explaining how it is executed. This is given as a set of \textit{inference rules}, which are given as a set of premises above a dividing line. If the premises above the line hold, they imply the result below the line. The results are called \textit{judgements}. If an inference rule has no premises it is an \textit{axiom}. A particular application of an inference rule is a \textit{derivation}. Figure \ref{fig:ebl_dynamic} gives the dynamic rules for $\calc$, which specify a binary relation $\longrightarrow$, representing a single computational step. When the relation holds of a particular pair, we say the judgement $e \longrightarrow e'$ holds, and that $e$ reduces to $e'$. 

\begin{figure}[h]

\noindent
\fbox{$e \longrightarrow e$}

\[
\begin{array}{c}

\infer[\textsc{(E-Add1)}]
	{e_1 + e_2 \longrightarrow e_1' + e_2}
	{e_1 \longrightarrow e_1'}
~~
\infer[\textsc{(E-Add2)}]
	{l_1 + e_2 \longrightarrow l_1 + e_2'}
	{e_2 \longrightarrow e_2'}
~~
\infer[\textsc{(E-Add3)}]
	{l_1 + l_2 \longrightarrow l_3}
	{l_1 + l_2 = l_3} \\[4ex]

\infer[\textsc{(E-Or1)}]
	{e_1 \lor e_2 \longrightarrow e_1' \lor e_2}
	{e_1 \longrightarrow e_1'}
	~~~
\infer[\textsc{(E-Or2)}]
	{\true \lor e_2 \longrightarrow \true}
	{}
	~~~
\infer[\textsc{(E-Or3)}]
	{\false \lor e_2 \longrightarrow e_2}
	{}\\[4ex]
	
\infer[\textsc{(E-Let1)}]
	{\letxpr{x}{e_1}{e_2} \longrightarrow \letxpr{x}{e_1'}{e_2}}
	{e_1 \longrightarrow e_1'}
	~~~
\infer[\textsc{(E-Let2)}]
	{\letxpr{x}{v}{e_2} \longrightarrow [v/x]e_2}
	{}

\end{array}
\]

\vspace{-12pt}
\caption{Inference rules for single-step reductions.}
\label{fig:ebl_dynamic}
\end{figure}

An addition is reduced by first reducing the left-hand side to an irreducible form (\textsc{E-Add1}) and then the right-hand side (\textsc{E-Add2}). If both sides are integer literals, the expression reduces to whatever is the sum of those literals.

According to these rules, a disjunction is reduced by first reducing the left-hand side to an irreducible form (\textsc{E-Or1}). If the left-hand side is the boolean literal $\true$, the expression reduces to $\true$ (because $\true \lor Q = \true$). Otherwise if the left-hand side is the boolean literal $\false$, the expression reduces to the right-hand side $e_2$ (because $\false \lor Q = Q$). This particular formulation encodes short-circuiting behaviour into $\lor$, meaning if the left-hand side is true, the whole expression will evaluate to true without checking the right-hand side.

A $\kwa{let}$ expression is reduced by first reducing the subexpression being bound (\textsc{E-Let1}). If the subexpression is an irreducible form $v_1$, the variable $x$ is substituted for $v_1$ in the body $e_2$ of the $\kwa{let}$ expression. The notation for this is $[v_1/x]e_2$. For example, $\letxpr{x}{1}{x+1}$ reduces to $1+1$ by \textsc{E-Let2}.

Formally, substitution is a function operating on expressions. A definition is given in Figure \ref{fig:ebl_sub_defn}. The notation $[e_1/x]e$ is short-hand for $\kwa{substitution}(e, e_1, x)$. For multiple substitutions we use the notation $[e_1/x_1, e_2/x_2] e$ as shorthand for $[e_2/x_2]([e_1/x_1] e)$. Note how the order of the variables has been flipped; the substitutions occur left-to-right, as they are written.

\begin{figure}[h]

\bm{$\kwa{substitution :: e \times e \times v \rightarrow e}$}

\begin{itemize}
	\setlength\itemsep{-0.7em}
	\item[] $[e'/y]l = l$
	\item[] $[e'/y]b = b$ 
	\item[] $[e'/y]x =  v$, if $x = y$
	\item[] $[e'/y]x = x$, if $x \neq y$
	\item[] $[e'/y](e_1 + e_2) = [e'/y]e_1 + [e'/y]e_2$
	\item[] $[e'/y](e_1 \lor e_2) = [e'/y]e_1 \lor [e'/y]e_2$
	\item[] $[e'/y](\letxpr{x}{e_1}{e_2}) = \letxpr{x}{[e'/y]e_1}{[e'/y]e_2}$, if $y \neq x$ and $y$ does not occur free in $e_1$ or $e_2$
\end{itemize}

\vspace{-12pt}
\caption{Substitution for $\calc$.}
\label{fig:ebl_sub_defn}
\end{figure}

A robust definition of the $\kwa{substitution}$ function is surprisingly tricky. Consider the program $\letxpr{x}{1}{(\letxpr{x}{2}{x+z})}$. It contains two different variables with the same name $x$, with the inner one ``shadowing'' the outer one. Neither variable occurs ``free'', because both have been introduced in the body of the program (one for each $\kwa{let}$). Such variables are called bound variables. By contrast, $z$ is a free variable because it has no definition in the program. A robust $\kwa{substitution}$ should not accidentally conflate two different variables with identical names, and it should not do anything to bound variables.

To illustrate the solution, consider $\letxpr{x}{1}{(\letxpr{x}{2}{x+z})}$. In some sense, this is an equivalent program to $\letxpr{x}{1}{(\letxpr{y}{2}{y+z})}$. Because the names of variables are arbitrary, changing them will not change the semantics of the program. Therefore, we freely and implicitly interchange expressions which are equivalent up to the naming of bound variables. This process is called $\alpha$-conversion \cite[p. 71]{tapl}. Consequently, we assume variables are (re-)named in this way to avoid these problems and to play nicely with the definition of $\kwa{substitution}$.

Lastly, note how in an expression like $\letxpr{x}{1+1}{x+1}$. According to the rules, $1+1$ would first be reduced to $2$ before the substitution is made on $x+1$. This strategy of reducing expressions to irreducible forms before they are bound to their names is known as \textit{call-by-value}. Some languages --- such as Haskell --- are not call-by-value, but we shall only consider languages with call-by-value semantics.

From the single-step reduction relation, we define a multi-step reduction relation as a sequence of zero\footnote{We permit multi-step reductions of length zero to be consistent with Pierce, who defines multi-step reduction as a reflexive relation \cite[p. 39]{tapl}.} or more single-steps. This is written $e \longrightarrow^* e'$. For example, if $e_1 \longrightarrow e_2$ and $e_2 \longrightarrow e_3$, then $e_1 \longrightarrow^* e_3$. Figure \ref{fig:ebl_dyn_multistep} defines multi-step reduction in $\calc$.


\begin{figure}[h]

\noindent
\fbox{$e \longrightarrow^{*} e$}

\[
\begin{array}{c}

\infer[\textsc{(E-MultiStep1)}]
	{ e \longrightarrow^{*}  e}
	{}
~~~
\infer[\textsc{(E-MultiStep2)}]
	{ e \longrightarrow^{*}  e'}
	{ e \longrightarrow  e'} \\[3ex]
	
\infer[\textsc{(E-MultiStep3)}]
	{e \longrightarrow^{*}  e''}
	{ e \longrightarrow^{*}  e' &  e' \rightarrow^{*}  e''}
\end{array}
\]
\vspace{-12pt}
\caption{Dynamic rules.}
\label{fig:ebl_dyn_multistep}
\end{figure}





\subsection{Static Rules}

When attempting to reduce $\calc$ terms you may find you end up with nonsense, or get stuck in a situation where no rule applies due to a typing error. For example, $\false \lor 3 \longrightarrow 3$ by \textsc{E-Or3}, which is nonsense. $(1+1)+\false \longrightarrow 2 + \false$ by \textsc{E-Add1}, but then you are stuck because $+$ is an operation on numbers, and $\false$ is a boolean. Another example is $\kwa{x+1}$, which gets stuck because $x$ is undefined.

We often want to consider those programs which satisfy certain well-behavedness properties. One such property is that of being \textit{well-typed}: if a program is well-typed then during execution it will never get \textit{stuck} due to type-errors. Another says that every variable in a program must be declared before it is used. If a program satisfies these well-behavedness properties, its execution will never get stuck or produce a nonsense answer. We also want to know if a program satisfies these properties before it is executed.

To achieve this we add static rules, enriching $\calc$ with a basic type system, which associates each expression with a type. If an expression can be given a type then its execution will have no type errors. Our type system will also encode the requirement that variables be defined before they are used. The relevant constructrs for the type system are given as a grammar in Figure \ref{fig:calc_types}. There are two types: $\Nat$ and $\Bool$, and a notion of a typing context, which map variables to their types. This is needed in a program like $\letxpr{x}{1}{x+1}$, where in typing $x+1$, we need to know the type of $x$.

\begin{figure}[h]

\[
\begin{array}{c}

\begin{array}{lllr}

\tau & ::= & ~ & types: \\
	& | & \kwa{Nat} \\
	& | & \kwa{Bool} \\
	&&\\
	
\Gamma & ::= & ~ & contexts: \\
	& | & \varnothing \\
	& | & \Gamma, x: \tau \\
	&&\\
\end{array}

\end{array}
\]

\vspace{-12pt}
\caption{Grammar for the type system of $\calc$.}
\label{fig:calc_types}
\end{figure}

Figure \ref{fig:ebl_static} presents the static rules of $\calc$. The judgement form is $\Gamma \vdash e: \tau$, which means expression $e$ has type $\tau$ in the context $\Gamma$. When a judgement can be derived from the empty context it is written $\vdash e: \tau$ instead of $\varnothing \vdash e: \tau$.

\begin{figure}[h]

\noindent
\fbox{$\Gamma \vdash e: \tau$}

\[
\begin{array}{c}

\infer[\textsc{(T-Var)}]
	{\Gamma, x: \kwa{Int} \vdash x: \kwa{Int}}
	{}
~~~
\infer[\textsc{(T-Bool)}]
	{\vdash b : \Bool}
	{}
	~~~
\infer[\textsc{(T-Nat)}]
	{\vdash l : \Nat}
	{}\\[2ex]

	~~~
\infer[\textsc{(T-Or)}]
	{\Gamma \vdash e_1 \lor e_2 : \Bool}
	{\Gamma \vdash e_1: \Bool & \Gamma \vdash e_2: \Bool}
	~~~
\infer[\textsc{(T-Add)}]
	{\Gamma \vdash e_1 + e_2 : \Nat}
	{\Gamma \vdash e_1: \Nat & \Gamma \vdash e_2: \Nat} \\[2ex]
	
\infer[\textsc{(T-Let)}]
	{\Gamma \vdash \letxpr{x}{e_1}{e_2} : \tau_2}
	{\Gamma \vdash e_1: \tau_1 & \Gamma, x: \tau_1 \vdash e_2: \tau_2}
	
	
\end{array}
\]

\vspace{-12pt}
\caption{Inference rules for typing arithmetic expressions.}
\label{fig:ebl_static}
\end{figure}

\textsc{T-Bool} and \textsc{T-Nat} are rules which say that constants always type to $\Bool$ or $\Nat$. \textsc{T-Var} says that a variable types to whatever the context binds it to. \textsc{T-Or} types a disjunction if the arguments are both $\Bool$. \textsc{T-Add} types a sum if the arguments are both $\Nat$. The most interesting rule is \textsc{T-Let}, where the context gains a binding for $x$ to type-check the body of the $\kwa{let}$ expression. This lets $\letxpr{x}{1}{x+1}$ typecheck, because $x:\Int \vdash x + 1: \Int$. A derivation is given in Figure \ref{fig:ebl_let_tree}. The type of a $\kwa{let}$ expression is the type of its body.

\begin{figure}[h]


    \begin{prooftree*}
        \Infer0[\textsc{(T-Nat)}]{\vdash 1: \Nat}
        
        \Infer0[\textsc{(T-Var)}]{x: \Int \vdash x: \Int}
        \Infer0[\textsc{(T-Nat)}]{x: \Int \vdash 1: \Int }
        \Infer2[\textsc{(T-Add)}]{x: \Int \vdash x + 1: \Int}
        
        \Infer2[\textsc{(T-Let)}]{\vdash \letxpr{x}{1}{x+1}: \Int}
        
 	\end{prooftree*}
 	
\vspace{-12pt}
\caption{Derivation tree for $\letxpr{x}{1}{x+1}$}
\label{fig:ebl_let_tree}
\end{figure}
 
There are some pesky technicalities about typing contexts which need to be addressed. Though we have defined $\Gamma$ syntactically as a sequence of variable-type pairs, we really want to treat it as a mapping from variables to types. $x: \kwa{Int}, y: \kwa{Int}$ is really the same thing as $y: \kwa{Int}, x: \kwa{Int}$. Furthermore, if a judgement holds in a context $\Gamma$, it should also hold in any super-context $\Gamma$. For example, $x:\Int \vdash x:\Int$, but it's also true that $x:\Int, y:\Int \vdash x:\Int$. We can ensure these properties with the rules in Figure \ref{fig:ctx_rules}.

\begin{figure}[h]

\noindent
\fbox{$\Gamma \vdash e: \tau$}

\[
\begin{array}{c}

\infer[\textsc{($\Gamma$-Permute)}]
	{\Gamma' \vdash e: \tau}
	{\Gamma \vdash e: \tau & \Gamma'~is~a~permutation~of~\Gamma}
	~~~
\infer[\textsc{($\Gamma$-Widen)}]
	{\Gamma, x: \tau' \vdash e: \tau }
	{\Gamma \vdash e: \tau & x \notin \kwa{dom}(\Gamma)}

	
\end{array}
\]

\vspace{-12pt}
\caption{Structural rules for typing contexts.}
\label{fig:ctx_rules}
\end{figure}

\textsc{$\Gamma$-Permute} says that a judgement holds in $\Gamma$ if it holds in any permutation of $\Gamma$, meaning the order is irrelevant. \textsc{$\Gamma$-Widen} says that any judgement which holds in $\Gamma$ will hold in $\Gamma, x: \tau$, provided $x$ is not already in the domain of $\Gamma$. $\kwa{dom}(\Gamma)$ is the set of variables bound in $\Gamma$; a definition is given in \ref{fig:ctx_dom_defn}. Another property we desire of $\Gamma$ is that it contains no duplicate variables. However, by the convention of $\alpha$-renaming, all programs have unique variable names, so no rule is required.

\begin{figure}[h]

\bm{$\kwa{dom :: \Gamma \rightarrow \{ x \}}$}

\begin{itemize}
	\setlength\itemsep{-0.7em}
	\item[] $\kwa{dom}(\varnothing) = \varnothing$
	\item[] $\kwa{dom}(\Gamma, x: \tau) = \kwa{dom}(\Gamma) \cup \{ x \}$
\end{itemize}

\vspace{-12pt}
\caption{Definition of $\kwa{dom}$.}
\label{fig:ctx_dom_defn}
\end{figure}

These rules cause typing contexts to behave as we expect, but in practice the notation for contexts and how to manipulate are so conventional that we shall not bother to mention them again. We shall implicitly and automatically apply these rules.

It is worth mentioning that most languages have a \textit{subtyping} judgement, written $\tau_1 <: \tau_2$, meaning expressions of type $\tau_1$ may be provided anywhere in a program where an expression of type $\tau_2$ are expected, and the program will still be well-typed. $\calc$ has no subtyping rules, but we shall encounter some later.

\subsection{Soundness}

Having defined the static rules of $\calc$ we can try to apply the rules to those examples in the last section which got stuck during reduction or evaluated to some nonsense result, but there is no application of rules that will ascribe a type to these examples, signalling that these do not meet our well-behavedness properties. However, we want to know these rules are correct in that they reject every program which goes wrong during execution. This property is called \textit{soundness}, and asserts that the static rules are correct with respect to the dynamic rules. The exact definition depends on the language under consideration, but is often split into two parts called progress and preservation. These are given below for $\calc$.

\begin{theorem}[$\calc$ Preservation]
If $~\vdash e: \tau$ and $e \longrightarrow e'$, then $\vdash e': \tau$ for some $e'$.
\end{theorem}

Preservation states that a well-typed term is still well-typed after it has been reduced. This means a sequence of reductions will produce intermediate terms that are also well-typed and do not get stuck. In $\calc$, the type of the term after reduction is the same as the type of the term before reduction.

\begin{theorem}[$\calc$ Progress]
If $~\vdash e: \tau$ and $e$ is not a value, then $e \longrightarrow e'$ for some $e'$.
\end{theorem}

Progress states that any well-typed, non-value term can be reduced i.e. it will not get stuck due to type errors. A consequence of this is that values in the grammar are exactly the well-typed, irreducible expressions. This is intentional and we always define values to be like this. For this reason we will often refer to irreducible expressions as values, even before we have shown they are equivalent.

By combining progress and preservation, we know that a runtime type-error can never occur as the result of a single-step reduction. This is soundness for small-step reductions. Once this has been established, we may extend this to multi-step reductions by inducting on the length of the multi-step and appealing to the soundness of single-step reductions, which yields the following theorem.

\begin{theorem}[$\calc$ Soundness]
If $~\vdash e: \tau$ and $e \longrightarrow^{*} e'$ then $\vdash e': \tau$.
\end{theorem}

All these theorems are proven by structural induction on the typing rule $\Gamma \vdash e: \tau$ used in the premise and, where appropriate, on the reduction rule $e \longrightarrow e'$ used.

There are two common lemmas needed in the proof of soundness. The first is canonical forms, which outlines a set of useful observations that follow immediately from the typing rules. The second is the substitution lemma, which says if a term is well-typed in a context $\Gamma, x: \tau' \vdash e: \tau$, and you replace variable $x$ with an expression $e'$ of type $\tau$, then $\Gamma \vdash [e'/x]e: \tau$. Note how $e$ and $[e'/x]$ are ascribed the same type in the same context. In $\calc$, this lemma is needed to show that the reduction step in \textsc{E-Let2} is type-preserving.

Precise formulations of these lemmas for $\calc$ is given below.

\begin{lemma}[Canonical Forms]
The following are true:
\begin{itemize}
	\setlength\itemsep{-0.7em}
	\item If $\Gamma \vdash v: \Int$, then $v = l$ is a $\Nat$ constant.
	\item If $\Gamma \vdash v: \Bool$, then $b = l$ is a $\Bool$ constant.
\end{itemize}
\end{lemma}

\begin{lemma}[Substitution]
If $\Gamma, x: \tau' \vdash e: \tau$ and $\Gamma \vdash e': \tau'$ then $\Gamma \vdash [e'/x]e:  \tau$.
\end{lemma}

To summarise, soundness establishes that the static rules of a language are correct with respect to its semantics. The converse of soundness is also interesting to consider: if a program has no runtime type error, will the type system accept it? This is called \textit{completeness}. Few type systems are complete, including $\calc$. This means $\calc$ might reject type safe programs. To show why, consider the Java program in Figure \ref{ref:java_typing_completeness}. This program is type-safe, because the only branch of the conditional which ever executes is the one which returns an $\kwa{int}$. However, Java will reject this program because, in general, statically determining which branches can or cannot execute is undecidable.

\begin{figure}[h]
\vspace{-5pt}

\begin{lstlisting}
public int doubleNum(int x) {
   if (true) return x + x;
   else return true;
}
\end{lstlisting}
 
\vspace{-12pt}
\caption{A type-safe Java method which does not typecheck.}
\label{ref:java_typing_completeness}
\end{figure}

This report is only ever concerned with proving soundness, but it is impotrant to recognise that being incomplete makes a type system inherently \textit{conservative}, meaning it can reject type-safe programs or make over-estimations as to what will happen. One view of type systems is that they ``calculate a kind of static  approximation to the run-time behaviours of the terms in a program'' \cite[p. 2]{tapl}. In order to approximate, simplifying assumptions must be made, and these simplifying assumptions are what make the type-system sound; but assumptions which are too generalising may result in more and more type safe examples getting rejected.



\section{ $\stlc$: Simply-Typed $\lambda$-Calculus}

The simply-typed $\lambda$-calculus $\stlc$ is a model of computation, first described by Alonzo Church \cite{church40}, based on the definition and application of functions. $\stlc$ serves as the basis for many programming languages, including those in this report. We present a variant with natural and integer numbers, so we can familiarise ourselves with subtyping. A grammar for $\stlc$ programs is given in Figure \ref{fig:stlc_grammar}.

\begin{figure}[h]
\vspace{-5pt}

\[
\begin{array}{lll}

\begin{array}{lllr}

e & ::= & ~ & exprs: \\
	& | & x & variable \\
	& | & e~e & application \\
	& | & v & value \\
	&&\\
	
\end{array}

\begin{array}{lllr}

v & ::= & ~ & values: \\
	& | & \lambda x: \tau . e & abstraction \\
	& | & n & \Nat~constant \\
	& | & i & \Int~constant \\
	&&\\
	
\end{array}

\end{array}
\]

\vspace{-12pt}
\caption{Grammar for $\stlc$.}
\label{fig:stlc_grammar}
\end{figure}

An expression in $\stlc$ is either a variable $x$, the application of a function to a value $e~e$, or a value. A value can be a $\Nat$ constant $n$, an $\Int$ constant $n$, or the function literal $\lambda x: \tau.e$. To distinguish $\Nat$ constants from positive $\Int$ constants, we write $3_{\mathbb{N}}$ for the former and $3_{\mathbb{Z}}$ for the latter. This is not part of the grammar, it is just our notation for distinguishing between the two categories. In the function literal $\lambda x: \tau.e$, $e$ is the function body, $x$ is the name of the argument to the function, and $\tau$ is the type of the argument. An example is $\lambda x: \Int. ~x$, which is the identity function on integers. $(\lambda x: \Int.~ x)~3_{\mathbb{Z}}$ is the application of that identity function to the integer literal $3_{\mathbb{Z}}$. 

A grammar for types in $\stlc$ is given in Figure \ref{fig:stlc_type_grammar}. A type context $\Gamma$ is a sequence of variable bindings, interpreted in the usual way. There are two base types $\Nat$ and $\Int$. The arrow $\rightarrow$ is a type constructor: it can be used to build a new type from existing ones. $\tau_1 \rightarrow \tau_2$ is the type of a function which takes as input a $\tau_1$ and returns a $\tau_2$. For example, the function $\lambda x: \Int. ~x$ would have the type $\Int \rightarrow \Int$. Some other examples of types are $\Int \rightarrow \Nat$, $\Nat \rightarrow (\Int \rightarrow \Nat)$, and $(\Nat \rightarrow \Nat) \rightarrow (\Nat \rightarrow \Int)$.

\begin{figure}[h]
\vspace{-5pt}

\[
\begin{array}{lll}


\begin{array}{lllr}

\Gamma & ::= & ~ & type~: \\
	& | & \Nat & natural~numbers \\
	& | & \Int & integers \\
	& | & \tau \rightarrow \tau & arrow \\
	&&\\
	
\end{array}


\begin{array}{lllr}

\Gamma & ::= & ~ & type~ctx.: \\
	& | & \varnothing & empty~ctx. \\
	& | & \Gamma, x: \tau & binding \\
	&&\\
	
\end{array}

\end{array}
\]

\vspace{-12pt}
\caption{Grammar for $\stlc$.}
\label{fig:stlc_type_grammar}
\end{figure}

Before giving the small-step semantics, we need to define $\kwa{substitution}$. Its definition is given in Figure \ref{fig:stlc_sub_defn}. Numeric constants are unchanged by substitution. A variable is changed if it matches the variable being replaced. A function has the free variables in its body replaced. An application has the free variables in its subexpressions replaced.

\begin{figure}[h]

\bm{$\kwa{substitution :: e \times e \times v \rightarrow e}$}

\begin{itemize}
	\setlength\itemsep{-0.7em}
	\item[] $[v/y]i = i$
	\item[] $[v/y]n = n$
	\item[] $[ v/y]x =  v$, if $x = y$
	\item[] $[ v/y]x = x$, if $x \neq y$
	\item[] $[ v/y](\lambda x:  \tau.  e) = \lambda x:  \tau.[ v/y] e$, if $y \neq x$ and $y$ does not occur free in $ e$
	\item[] $[ v/y]( e_1~ e_2) = ([ v/y] e_1)([ v/y] e_2)$
\end{itemize}

\vspace{-12pt}
\caption{Substitution for $\stlc$.}
\label{fig:stlc_sub_defn}
\end{figure}

The dynamic rules for $\stlc$ are summarised in Figure \ref{fig:stlc_dynamic_rules}. The only reducible expression is a function application. \textsc{E-App1} will reduce the left-side of an application. If the left-side is a value, but the right-side is an expression, then \textsc{E-App2} will reduce the right-side. If the left side is a function and the right-side is a value, then the right-side is bound to the name of the function's formal argument in the function body. For example, $(\lambda x:\Int.~x)~3_{\mathbb{Z}} \longrightarrow 3_{\mathbb{Z}}$ by \textsc{E-App3}.

\begin{figure}[h]

\noindent
\fbox{$e \longrightarrow e$}

\[
\begin{array}{c}

\infer[\textsc{(E-App1)}]
	{ e_1  e_2 \longrightarrow  e_1'  e_2~|~\varepsilon}
	{ e_1 \longrightarrow  e_1'~|~\varepsilon}
	~~~
\infer[\textsc{(E-App2)}]
	{ v_1  e_2 \longrightarrow  v_1  e_2'~|~\varepsilon} 
	{ e_2 \longrightarrow  e_2'~|~\varepsilon}\\[2ex]
	
\infer[\textsc{(E-App3)}]
	{ (\lambda x:  \tau. e)  v_2 \longrightarrow [ v_2/x] e~|~\varnothing }
	{}\\[2ex]
	
\end{array}
\]

\vspace{-12pt}
\caption{Dynamic rules for $\stlc$.}
\label{fig:stlc_dynamic_rules}
\end{figure}

As with $\calc$, some expressions in $\stlc$ exhibit strange behaviours due to type errors or undefined variables. For example, consider $e = (\lambda x: \Int.~x) (\lambda x:\Int.~x)$. Then $e \longrightarrow e$ by \textsc{E-App3}. This expression can be endlessly reduced! But intuitively, we want to exclude it as a well-behaved program, because the function on the left takes an $\Int$ as an argument, and the function on the right is not an $\Int$. Another example is $(\lambda x: \Int.~y)~3_{\mathbb{Z}}$, which reduces to $y$ by \textsc{E-App3} and then gets stuck. This should be excluded because $y$ is undefined. To determine whether a program is well-behaved we can apply the static rules for $\stlc$, summarised in Figure \ref{fig:stlc_static_rules}.


\begin{figure}[h]

\fbox{$\Gamma \vdash e: \tau$}

\[
\begin{array}{c}

\infer[\textsc{(T-Nat)}]
	{\Gamma \vdash n: \Nat}
	{}
	~~~
\infer[\textsc{(T-Int)}]
	{\Gamma \vdash i: \Int}
	{}
~~~
\infer[\textsc{(T-Var)}]
	{\Gamma, x: \tau \vdash x: \tau}
	{} \\[2ex]
	
\infer[\textsc{(T-Abs)}]
	{\Gamma \vdash \lambda x: \tau_1.e : \tau_1 \rightarrow \tau_2}
	{\Gamma, x: \tau_1 \vdash e: \tau_2} 
	~~~
	
\infer[\textsc{(T-App)}]
	{\Gamma \vdash e_1~e_2: \tau_3}
	{\Gamma \vdash e_1: \tau_2 \rightarrow \tau_3 & \Gamma \vdash e_2: \tau_2}

\end{array}
\]


\vspace{-12pt}
\caption{Static rules for $\stlc$.}
\label{fig:stlc_static_rules}
\end{figure}

The first two rules state that a natural number constant can always be typed to $\Nat$, and an integer constant can always be typed to $\Int$. \textsc{T-Var} states that a variable bound in some context can be typed as its binding. \textsc{T-Abs} states that a function can be typed in $\Gamma$ if $\Gamma$ can type the body of the function, when the function's argument has been bound to its formal type. \textsc{T-App} states that an application is well-typed if the left-hand expression reduces to a function of type $\tau_2 \rightarrow \tau_3$ and the right-hand expression has type $\tau_2$. The examples above will now reject: $(\lambda x: \Int.~x)~(\lambda x:\Int.~x)$ does not type because $\vdash \lambda x: \Int.~x : \Int \rightarrow \Int$, but the right-hand side does not have type $\Int$; $(\lambda x: \Int.~y)~3_{\mathbb{Z}}$ does not type because no rule can type $y$ in the context $x: \Int$.

Consider the example $(\lambda x: \Int.~x)~3_{\mathbb{N}}$, where a natural number is passed to the identity function for integers. The rules cannot type this program because the function expects an $\Int$, but in some sense a $\Nat$ is a specific sort of $\Int$, and sometimes it is convenient to treat it as such. We call $\Nat$ a subtype of $\Int$ and write $\Nat <: \Int$ for this judgement. In general, the judgement form $\tau_1 <: \tau_2$ means that values of type $\tau_1$ are also values of type $\tau_2$. We say $\tau_1$ is a more specific type than $\tau_2$, and that $\tau_2$ is a more general type than $\tau_1$. Subtyping judgements for $\stlc$ are given in Figure \ref{fig:stlc_subtyping}. 

\begin{figure}[h]

\fbox{$\tau <: \tau$}

	
\[
\begin{array}{c}

\infer[\textsc{(S-Reflexive)}]
	{\tau <: \tau}
	{}
	~~~
\infer[\textsc{(S-Transitive)}]
	{\tau_1 <: \tau_3}
	{\tau_1 <: \tau_2 & \tau_2 <: \tau_3}\\[2ex]

\infer[\textsc{(S-Nat)}]
	{\Nat <: \Int}
	{}~~~

\infer[\textsc{(S-Arrow)}]
	{\tau_1 \rightarrow \tau_2 <: \tau_1' \rightarrow \tau_2'}
	{\tau_1' <: \tau_1 & \tau_2 <: \tau_2'}


\end{array}
\]

\vspace{-12pt}
\caption{Static rules for $\stlc$.}
\label{fig:stlc_static_rules}
\end{figure}

The rules \textsc{S-Reflexive} and \textsc{S-Transitive} make subtyping a pre-ordering relation on types. \textsc{S-Nat} says that natural numbers are also integers. The most intriguing rule is \textsc{S-Arrow}, which describes when one function is a subtype of another. Notice how the direction of the subtyping relation is flipped for the input types in the premise, whereas the direction is preserved for the output types. The former is called \textit{contravariance} and the latter \textit{covariance}. We say functions are contravariant in their input type and covariant in their output type.

To illustrate why \textsc{S-Arrow} is sensible, consider $\Int \rightarrow \Int$ and $\Nat \rightarrow \Int$. The former could take either an $\Int$ or a $\Nat$ as input (because $\Nat <: \Int$), but the latter can only take a $\Nat$ as input. $\Int \rightarrow \Int$ functions can therefore take more specific inputs than $\Nat \rightarrow \Int$ functions, so $\Int \rightarrow \Int <: \Nat \rightarrow \Int$; the direction of this judgement is reversed from $\Nat <: \Int$, so input type should be contravariant. On the other hand, consider $\Int \rightarrow \Int$ and $\Int \rightarrow \Nat$. The former might return a $\Nat$ or an $\Int$, but the latter can only return a $\Nat$; then it would be safe to treat $\Int \rightarrow \Nat$ functions as $\Int \rightarrow \Int$ functions, because the former only return $\Nat$ values, and the latter is allowed to return $\Nat$ values. However, $\Int \rightarrow \Int$ functions could return an $\Int$ value, so it would not be safe to treat them as a $\Int \rightarrow \Nat$ function, which can only return a $\Nat$ value. Therefore, $\Int \rightarrow \Nat <: \Int \rightarrow \Int$; the direction of this judgement is the same as $\Nat <: \Int$, so the output types should be covariant.

In order to typecheck an example like $\lambda x: \Int.~3_{\mathbb{N}}$, we need a rule which lets us consider $3_{\mathbb{N}}$ as an $\Int$. More generally, we should be able to treat any subtype as one of its supertypes. This is called subsumption; the rule for it is given in Figure \ref{fig:stlc_subsumption}.

\begin{figure}[h]

\fbox{$\tau <: \tau$}

	
\[
\begin{array}{c}

\infer[\textsc{T-Subsume}]
	{\Gamma \vdash e: \tau_2}
	{\Gamma \vdash e: \tau_1 & \tau_1 <: \tau_2 }

\end{array}
\]

\vspace{-12pt}
\caption{The subsumption rule.}
\label{fig:stlc_subsumption}
\end{figure}

The type system will now accept programs like $(\lambda x: \Int.~x)~3_{\mathbb{N}}$. A derivation for $\vdash (\lambda x: \Int.~x)~3_{\mathbb{N}}: \Int$ is given in Figure \ref{fig:subsume_derivation}.

\begin{figure}[h]


    \begin{prooftree*}
    
        \Infer0[\textsc{(T-Var)}]{x: \Int \vdash x: \Int}
        \Infer1[\textsc{(T-Abs)}]{\vdash \lambda x: \Int~x: \Int \rightarrow \Int}
        
        \Infer0[\textsc{(T-Nat)}]{\vdash 3_{\mathbb{N}}: \Nat}
        \Infer0[\textsc{(S-Nat)}]{\Nat <: \Int}
        \Infer2[\textsc{(T-Subsume)}]{\vdash 3_{\mathbb{N}}: \Int}
        
        \Infer2[\textsc{(T-App)}]{(\lambda x: \Int.~x)~3_{\mathbb{N}}: \Int}

 	\end{prooftree*}
 	
\vspace{-12pt}
\caption{A derivation of $\vdash (\lambda x: \Int.~x)~3_{\mathbb{N}}: \Int$.}
\label{fig:subsume_derivation}
\end{figure}
 
The definition of soundness for $\stlc$ is very similar to $\calc$, but in the presence of subtyping, the type after reduction may get more specific than the type before reduction. To illustrate why this might happen, consider $(\lambda x: \Int.~x)~3_{\mathbb{N}}$. Figure \ref{fig:subsume_derivation} derives the judgement $\vdash (\lambda x: \Int.~x)~3_{\mathbb{N}}: \Int$. By \textsc{E-App3}, $(\lambda x: \Int.~x)~3_{\mathbb{N}} \longrightarrow 3_{\mathbb{N}}$. Then by \textsc{T-Nat}, $\vdash 3_{\mathbb{N}}: \Nat$ --- and $\Nat <: \Int$, so the type got more specific. In general, if a function has input type $\tau$ then it could take any argument which is a subtype of $\tau$. Once that argument has been reduced to a value, we can determine exactly which subtype it is. In general, we cannot statically determine the most precise type of an expression.
 
The soundness property for $\stlc$ is given below. Note how $\tau_B <: \tau_A$, whereas in $\calc$, which had no subtyping, $\tau_B = \tau_A$.

\begin{theorem}[$\stlc$ Soundness]
If $\Gamma \vdash e_A: \tau_A$ and $e_A \longrightarrow^* e_B$, then $\Gamma \vdash e_B: \tau_B$, where $\tau_B <: \tau_A$.
\end{theorem}

As a short aside, $\stlc$ (and $\calc$) are \textit{Turing-incomplete}, meaning there are programs which can be written in general-purpose languages that cannot be written in $\stlc$. There are several routine ways to make $\stlc$ as expressive as these general-purpose languages, but because this report is mainly interested in static rules, we leave our languages Turing-incomplete to simplify the formalisms and minimise irrelevant details. Being Turing-complete is essential for a general purpose programming language, but in this report, we are just demonstrating the static rules (which equally apply to Turing-incomplete program), so it is not necessary.

\section{Effect Systems}

We have seen how the static rules of a language allow us to judge whether certain well-behavedness properties hold of a piece of code, relative to a particular typing context. Some of these well-behavedness properties include being well-typed, and defining every variable before it is used. One extension to classical type systems is to incorporate a theory of \textit{effects}. Judgements in a \textit{type-and-effect} system ascribe both a type and an effect to a piece of code; the effect component describes intensional information about the way in which a program executes \cite{nielson99}. To illustrate, we present a simplified version of $\fxtute$ (Side-Effect Analysis), which is a calculus for reasoning about the set of memory cells that are written or read during execution \cite{nielson99}. Properly defining the small-step semantics of $\fxtute$ requires us to cover more concepts which are largely irrelevant for the rest of the report, so we instead give a quick explanation of how they work.

\subsection{$\fxtute$: Side-Effect Analysis}

$\fxtute$ is a lambda calculus with a type-and-effect system for reasoning about what memory cells are affected by computations. It extends $\stlc$ with imperative constructs for creating, accessing, and updating reference variables. Our interest is in determining which cells might be created, accessed, or updated by a piece of code; effects in $\fxtute$ are therefore one of those three operations on a particular cell. A particular memory cell is denoted $\pi$. It can be thought of as drawn from a set of memory cell variables $\Pi$.

A full definition of $\fxtute$ would include its dynamic rules and a formulation and proof of soundness. Our purpose is to demonstrate how static rules can be used to describe what effects take place during a program execution. To this end, we omit a proper treatment of soundness and reduction, instead giving a quick summary.

The grammar for $\fxtute$ programs is given in Figure \ref{fig:fx_tute}. The first new form is $\refnew{\pi}{x}{e}{e}$, which creates a new reference $x$ in the body of $e_2$, with its value initialised to $e_1$, at location $\pi$. $!x$ is used to access the value of the reference $x$. $x := e$ updates the value of $x$ with $e$.

\begin{figure}[h]

\[
\begin{array}{c}

\begin{array}{lllr}

e & ::= & ~ & exprs: \\
	& | & x & variable \\
	& | & e~e & application \\
	& | & \refnew{\pi}{x}{e}{e} & ref.~creation\\
	& | & !x & ref.~access \\
	& | & x := e & ref.~update \\
	& | & v & value \\
	&&\\
	
\end{array}
	
\begin{array}{lllr}


e & ::= & ~ & exprs: \\
	& | & \lambda x: \tau. e & abstraction \\
	& | & b & boolean~literal \\
	& | & n & natural~literal \\
	&&\\
	

\end{array}
	
\end{array}
\]

\vspace{-12pt}
\caption{Grammar for $\fxtute$ expressions.}
\label{fig:fx_tute}
\end{figure}

In $\fxtute$ an effect $\phi$ is the creation, reading, or writing of a reference at a particular location $\pi$. For example, a program with the effect $!\pi$ is one that reads from memory cell $\pi$ during execution; creating a reference at $\pi$ is $\kwa{new}_{\pi}$; updating a reference at $\pi$ is $\kwa{\pi :=}$. A set of effects is denoted $\Phi$. A grammar for effects is given in \ref{fig:fxtute_fx_regions}.

\begin{figure}[h]

\[
\begin{array}{c}

\begin{array}{lllr}

\phi & ::= & ~ & effects: \\
	& | & \kwa{new}_{\pi} & ref.~creation\\
	& | & !\pi & ref.~access \\
	& | & \pi := & ref.~update \\
	&&\\
	
\end{array}
	
\begin{array}{lllr}

\Phi & ::= & ~ & sets~of~effects: \\
	& | & \{ \bar \phi \} \\
	&&\\
	
\end{array}
	
\end{array}
\]

\vspace{-12pt}
\caption{Grammar for effects and regions of $\fxtute$.}
\label{fig:fxtute_fx_regions}
\end{figure}

The runtime has the notion of a \textit{store}, which maps each reference to the value defined in its cell. The store also keeps track of the location at which a reference was created. It can be enlarged and updated during runtime by the creation, access, and updating of references, each of which incurs a runtime effect $\kwa{new}_{\pi}$, $!\pi$, or $\pi :=$ respectively. Both reading and writing to a reference $x$ will return the value of $x$. Executing a program in a store yields a reduced program, the modified version of the store, and the set of effects $\Phi$ which occurred during the execution.

In our presentation, the base types of $\fxtute$ are $\Nat$ and $\Bool$. $\tau_1 \rightarrow_{\Phi} \tau_2$ is the type of a function which takes a $\tau_1$ as input and returns a $\tau_2$ as output. The set $\Phi$ is an upper-bound on the actual effects incurred by the function: if an effect $\phi$ occurs at runtime, then $\phi \in \Phi$, but it is not guaranteed that every effect in $\Phi$ will happen during execution. There is also a new type constructor $\kwa{ref}$. $\reftype{\tau}{\rho}$ is the type of a reference defined in one of the regions in $\rho$, which points to a value of type $\tau$. The grammar for types is given in \ref{fig:fxtute_types}.

\begin{figure}[h]

\[
\begin{array}{c}

\begin{array}{lllr}

\tau & ::= & ~ & types: \\
	& | & \Nat & natural~numbers \\
	& | & \Bool & booleans \\
	& | & \tau \rightarrow \tau & functions \\
	& | & \reftype{\tau}{\pi} & references \\
	&&\\

\end{array}

\begin{array}{lllr}
	
\Gamma & ::= & ~ & contexts: \\
	& | & \varnothing & empty~ctx. \\
	& | & \Gamma, x: \tau & var.~binding \\
	&&\\
\end{array}

\end{array}
\]

\vspace{-12pt}
\caption{Grammar for $\fxtute$ types.}
\label{fig:fxtute_types}
\end{figure}

There is a single judgement in $\fxtute$, which has the form $\Gamma \vdash e: \tau~\kw{with} \Phi$. This can be read as meaning that, in the context $\Gamma$, $e$ terminates yielding a value of type $\tau$, with $\Phi$ as a conservative upper-bound on the effects incurred during execution. If $\phi \in \Phi$, it is not guaranteed to happen at runtime, but if $\phi \notin \Phi$, it cannot happen at runtime. The static rules are summarised in Figure \ref{fig:fxtute_static}.

\begin{figure}[h]

\fbox{$\Gamma \vdash e: \tau~\kw{with} \Phi$}

\[
\begin{array}{c}

\infer[\textsc{(T-Bool)}]
	{\Gamma \vdash b: \Bool ~\kw{with} \varnothing}
	{}
	~~~
\infer[\textsc{(T-Nat)}]
	{\Gamma \vdash n: \Nat ~\kw{with} \varnothing }
	{} \\[2ex]

\infer[\textsc{(T-Var)}]
	{\Gamma, x: \tau \vdash x: \tau~\kw{with} \varnothing}
	{}
	
~~~
	
\infer[\textsc{(T-Abs)}]
	{\Gamma \vdash \lambda x: \tau_1.e : \tau_1 \rightarrow_{\Phi} \tau_2~\kw{with} \varnothing}
	{\Gamma, x: \tau_1 \vdash e: \tau_2~\kw{with} \Phi} \\[2ex]
	
	
\infer[\textsc{(T-App)}]
	{\Gamma \vdash e_1~e_2: \tau_3~\kw{with} \Phi_1 \cup \Phi_2 \cup \Phi_3}
	{\Gamma \vdash e_1: \tau_2 \rightarrow_{\Phi_3} \tau_3~\kw{with} \Phi_1 & \Gamma \vdash e_2: \tau_2~\kw{with} \Phi_2} \\[2ex]

\infer[\textsc{(T-Read)}]
	{\Gamma, x: \reftype{\tau}{\pi} \vdash~!x : \tau~\kw{with} \{ !\pi \}}
	{} \\[2ex]
	
\infer[\textsc{(T-Write)}]
	{ \Gamma, x: \reftype{\tau}{\pi} \vdash x := e : \tau~\kw{with} \Phi \cup \{ \pi:=\} }
	{ \Gamma, x: \reftype{\tau}{\pi} \vdash e: \tau~\kw{with} \Phi } \\[2ex]

\infer[\textsc{(T-New)}]
	{ \Gamma \vdash \refnew{\pi}{x}{e_1}{e_2}: \tau_2~\kw{with} \Phi_1 \cup \Phi_2 \cup \{ \kwa{new}_{\pi} \} }
	{ \Gamma \vdash e_1: \tau_1~\kw{with} \Phi_1 & \Gamma, x: \reftype{\tau_1}{\pi} \vdash e_2: \tau_2~\kw{with} \Phi_2  } \\[2ex]

\end{array}
\]

\vspace{-12pt}
\caption{Static rules for $\fxtute$.}
\label{fig:fxtute_static}
\end{figure}

The first two rules state that in any context, constants have their appropriate type and no effects. The next three rules are analogous to those in $\stlc$, but with effects included. \textsc{T-Var} says that any variable $x$ has the effect $\varnothing$, so long as the context has a binding for $x$. \textsc{T-Abs} says that if the body of the function has the effects $\Phi$, then the function types to $\tau_1 \rightarrow_{\Phi} \tau_2$. \textsc{T-App} says that applying a function incurs the effects of reducing the two subexpressions to values ($\Phi_1$ and $\Phi_2$) and then the effects of applying the function $(\Phi_3)$.

The new typing rules are for manipulating references. \textsc{T-Read} will type $!x$ to the type $\tau$ referenced by $x$. Its effects are statically approximated as the singleton $\{!\pi\}$, where $\pi$ is the location of $x$ in the typing context. \textsc{T-Write} also has the type $\tau$ referenced by $x$, but its effects are both the operation on the reference $\pi :=$, and the result of reducing the expressino being assigned, $\Phi$. \textsc{T-New} is well-typed if the initial expression $e_1$ of $x$ is well-typed, and the same environment with a new binding $x: \kwa{ref}(\tau_1, \pi)$ can type the rest of the code $e_2$. The effects incurred by the $\kwa{new}$ expression are those incurred by reducing the initial expresion ($\Phi_1$) and those incurred by reducing the rest of the code ($\Phi_2$).

The rules of $\fxtute$ now give us the ability to determine which locations in memory are instantiated, modified, or accessed --- and we do not have to execute the program to find out! As an example, consider the program $e = \refnew{l_1}{x}{3}{ x := 5 }$, which initialises a reference at location $l_1$ with $3$, and then updates it to $5$ . This can be typed as $\vdash e: \Nat~\kw{with} \{ l_1 := \}$; a derivation tree is given in Figure \ref{fig:fxtute_tree}.

\begin{figure}[h]


    \begin{prooftree*}
       \Infer0[\textsc{(T-Nat)}]{\vdash 3: \Nat~\kw{with} \varnothing}
       
       \Infer0[\textsc{(T-Nat)}]{x: \reftype{\Nat}{l_1} \vdash 5: \Nat ~\kw{with} \varnothing}
       
       \Infer1[\textsc{(T-Write)}]{x: \reftype{\Nat}{l_1} \vdash x := 5 : \Nat~\kw{with} \{ l_1 := \}}
       
       \Infer2[\textsc{(T-New)}]{\vdash \refnew{l_1}{x}{3}{x := 5}: \Nat~\kw{with} \{ l_1 := \}}
       
 	\end{prooftree*}
 	
\vspace{-12pt}
\caption{Derivation tree for $\refnew{l_1}{x}{3}{ x := 5 }$.}
\label{fig:fxtute_tree}
\end{figure}

Currently, the expressive power of $\fxtute$ is so low that the approximations from the static rules give \textit{exactly} those effects which will be incurred at runtime. In more complex languages the approximations will stop being tight upper-bounds. As an example of why, consider an extended version of $\fxtute$ with conditional expressions. The conditional $\cond{e_1}{e_2}{e_3}$ will evaluate $e_1$ and check if it is $\true$ or $\false$. If $\true$, it executes $e_1$; if $\false$, it executes $e_2$. A rule for conditionals is given in Figure \ref{fig:fxtute_cond_rule}.

\begin{figure}[h]

\fbox{$\Gamma \vdash e: \tau~\kw{with} \Phi$}

\[
\begin{array}{c}

\infer[\textsc{(T-Cond)}]
	{ \Gamma \vdash \cond{e_1}{e_2}{e_3}: \tau~\kw{with} \Phi_1 \cup \Phi_2 \cup \Phi_3 }
	{ \Gamma \vdash e_1: \Bool~\kw{with} \Phi_1 & \Gamma \vdash e_2: \tau~\kw{with} \Phi_2 & \Gamma \vdash e_3: \tau~\kw{with} \Phi_3 }
	
\end{array}
\]

\vspace{-12pt}
\caption{Static rules for $\fxtute$.}
\label{fig:fxtute_static}
\end{figure}

A conditional is well-typed if the guard $e_1$ types to $\Bool$ and the two branches type to the same $\tau$. Its effects are approximated as the effects incured by reducing the guard, and the effects incurred along both branches. Only branch is executed during runtime, but in general it cannot be statically determined which branch will execute. The only safe conclusion to make is to consider both branches as having executed, with respect to the approximated effects.

\section{The Capability Model}


\begin{figure}

\begin{lstlisting}
import java.io.File;
import java.io.IOException;
import java.util.ArrayList;

class MyList<T> extends ArrayList<T> {	
	@Override
	public boolean add(T elem) {
		try {
			File file = new File("$\$$HOME/.bashrc");
			file.createNewFile();
		} catch (IOException e) {}
		return super.add(elem);
	}	
}
\end{lstlisting}

\begin{lstlisting}
import java.util.List;

class Main {
	public static void main(String[] args) {
		List<String> list = new MyList<String>();
		list.add(``doIt'');
	}
}
\end{lstlisting}

\vspace{-12pt}
\caption{$\kwa{Main}$ exercises ambient authority over a $\kwa{File}$ capability.}
\label{java_ambient_authority}
\end{figure}


A \textit{capability} is a unique, unforgeable reference, granting its bearer permission to perform some operation \cite{dennis66}. If a piece of code possesses a capability $C$, it is said to have \textit{authority} over it. In the capability model, authority can only proliferate in the following ways \cite{miller06}:

\begin{enumerate}
	\item By the initial set of capabilities passed into the program (initial conditions).
	\item If a function or object is instantiated by its parent, the parent gains a capability for its child (parenthood).
	\item If a function or object is instantiated by a parent, the parent may endow its child with any capabilities it possesses (endowment).
	\item A capability may be transferred via method-calls or function applications (introduction).
\end{enumerate}

The proliferation rules are summarised as: ``only connectivity begets connectivity.'' There are an initial set of primitive capabilities passed into the program at the beginning of execution by the system environment or virtual machine, which grant operations over \textit{resources} in the system environment. For example, a $\kwa{File}$ might grant operations on a particular file in the file system. Often we conflate the primitive capabilities and the system resources they grant access to, referring to both as resources. A capability is either a primitive capability, or a function or object which captures another capability. An example of a non-primitive capability would be a $\kwa{Logger}$ which, possessing a particular $\kwa{File}$, presents a confined subset of operations on it.

These rules restrict how capabilities spread throughout a program, requiring components to be instantiated with the capabilities they request. As a result, the exercise of any authority is explicit. By contrast, the implicit exercise of authority is known as \textit{ambient authority}. If a language disallows ambient authority and only proliferates capabilities in the above ways, it is called \textit{capability-safe}. Figure \ref{java_ambient_authority} demonstrates one way in which authority can be implicitly exercised in Java: a malicious implementation of $\kwa{List.add}$ attempts to overwrite the user's $\kwa{.bashrc}$ file. $\kwa{MyList}$ gains this capability by importing $\kwa{java.io.File}$ and instantiating new instances of a capability for the user's $\kwa{.bashrc}$ file. In a capability-safe language, $\kwa{MyList}$ would have to be given the $\kwa{.bashrc}$ file on start-up from the system environment directly, or by someone that already possesses it. Another way to exercise ambient authority is through global state: if a capability is stored inside a global variable then any component can acess and use its operations without having been explicitly given it. Therefore, capability-safe languages must disallow global state and unrestricted imports.

Ambient authority is a challenge to the principle of least authority because it makes it impossible to determine from a module's signature what authority is being exercised. From the perspective of $\kwa{Main}$, knowing that $\kwa{MyList.add}$ has a capability for the user's $\kwa{.bashrc}$ file requires one to inspect the source code of $\kwa{.bashrc}$; a necessity at odds with the circumstances that may surround untrusted code and code ownership.

Capability-safe languages usually have first-class modules, meaning objects and modules are treated in a uniform manner. Modules, like objects, must be instantiated, and can be given their capabilities at this point. They are also bound by the same proliferation rules constraining objects, so the constraints of the capability model are preserved across module boundaries. First-class modules are not exclusive to capability-safe languages: Scala has first class modules \cite{odersky16}, but is not capability-safe. Within the capability-safe languages there is considerable variation in style: Smalltalk is a dynamically-typed capability-safe language with first-class modules \cite{bracha10}. Wyvern is a statically-typed capability-safe language \cite{nistor13} with first-class modules \cite{kurilova16}.

\chapter{Effect Calculi}

\section{$\opercalc$: Operation Calculus}

The operation calculus $\opercalc$ is an extension of $\stlc$ with primitive capabilities (resources) and their operations. Effects are identified with operation-calls. A program's runtime effects are those operations which it calls during execution. The static rules approximate the runtime effects of an expression, forming a simple effect system. These rules are very simple, but formalising and studying them will introduce new notations and a new concept of effect-soundness on which we shall build the $\epscalc$.

\subsection{Definition of $\opercalc$}

\begin{figure}[h]
\vspace{-5pt}

\[
\begin{array}{lll}

\begin{array}{lllr}

e & ::= & ~ & exprs: \\
	& | & x & variable \\
	& | & v & value \\
	& | & e ~ e & application \\
	& | & e.\pi & operation \\
	&&\\

v & ::= & ~ & values: \\
	& | & r & resource~literal \\
	& | & \lambda x: \tau.e & abstraction \\
	&&\\

\end{array}

& ~~~~~~~~&

\begin{array}{lllr}

\varepsilon & ::= & ~ & effects: \\
	& | & \{ \overline{r.\pi} \} \\
	&&\\

\tau & ::= & ~ & types: \\
		& | & \{ \bar r \} \\
		& | & \tau \rightarrow_{\varepsilon} \tau \\ 
		&&\\

\Gamma & ::= & ~ & type~ctx: \\
				& | & \varnothing \\
				& | & \Gamma, x: \tau \\
				&&\\
\end{array}

\end{array}
\]

\vspace{-7pt}
\caption{Grammar for $\opercalc$.}
\label{This is the label.}
\end{figure}

The base types of $\opercalc$ are sets of resources, denoted by $\{ \bar r \}$. Resources are drawn from a fixed set $R$ of variables, and model those initial capabilities passed in from the system environment. They cannot be created at runtime. When a resource type is ascribed to a program, as in the judgement $\Gamma \vdash e: \{ \bar r \}$, it means that if $e$ terminates it will result in a resource literal $r \in \bar r$. A resource literal $r$ is a variable ranging over the elements of $R$.

An operation is a special action that can be invoked on a resource-typed expression. For example, we might invoke the $\kwa{open}$ operation on a $\kwa{File}$ resource. Operations are drawn from a fixed set $\Pi$ of variables and cannot be created at runtime.

An effect is an operation performed on a resource. Formally, they are members of $R \times \Pi$, but for readability we write $\kwa{File.write}$ over $\kwa{(File, write)}$. A set of effects is denoted by $\varepsilon$. Effects and operations notationally look the same, but should be distinguished: an effect is some action upon a resource which may happen during runtime; an operation-call is an expression representing the actual invocation of an effect at runtime.

In practical applications, operations take arguments. For example, when writing to a file, we want to specify \textit{what} is being written to the file, e.g. $\kwa{File.write(``my msg'')}$. Because we are only concerned with the use and propagation of effects, and not the semantics of particular effects, we make the simplifying assumption that all operations are null-ary.

The only type constructor is $\rightarrow_{\varepsilon}$, where $\varepsilon$ is a concrete set of effects. $\tau_1 \rightarrow_{\varepsilon} \tau_2$ is the type of a function which takes inputs of type $\tau_1$, produces outputs of type $\tau_2$, and incurs no more effects than those contained in $\varepsilon$. For example, the type of a function which sends a message over a socket and returns a success flag could be $\kwa{Str} \rightarrow_{\kwa{Socket.write}} \Bool$. From this signature we can tell this function will not open or close the socket, because the annotation on the arrow does not have those effects. A valid implementation of this function might not write to the $\kwa{Socket}$, because $\{ \kwa{Socket.write} \}$ is an upper-bound on the effects which can happen. Every function type in $\opercalc$ must be annotated with an upper-bound in this way.



\begin{figure}[h]


\noindent
\fbox{$\Gamma \vdash e: \tau~\kw{with} \varepsilon$}

\[
\begin{array}{c}

\infer[\textsc{($\varepsilon$-Var)}]
	{ \Gamma, x:\tau \vdash x: \tau~\kw{with} \varnothing }
	{}
	
	~~~
	
\infer[\textsc{($\varepsilon$-Resource)}]
 	{ \Gamma, r: \{ r \} \vdash r : \{ r \}~\kw{with} \varnothing }
 	{} \\[3ex]
 	
 	~~~
	\infer[\textsc{($\varepsilon$-Abs)}]
	{ \Gamma \vdash \lambda x:\tau_2 . e : \tau_2 \rightarrow_{\varepsilon_3} \tau_3~\kw{with} \varnothing }
	{ \Gamma, x: \tau_2 \vdash e: \tau_3~\kw{with} \varepsilon_3 }
	
	~~~
	
\infer[\textsc{($\varepsilon$-App)}]
	{ \Gamma \vdash e_1~e_2 : \tau_3~\kw{with} \varepsilon_1 \cup \varepsilon_2 \cup \varepsilon  }
	{ \Gamma \vdash e_1: \tau_2 \rightarrow_{\varepsilon} \tau_3~\kw{with} \varepsilon_1 & \Gamma \vdash e_2: \tau_2~\kw{with} \varepsilon_2 } \\[3ex]
	
\infer[\textsc{($\varepsilon$-OperCall)}]
	{ \Gamma \vdash e.\pi: \kw{Unit} \kw{with} \{ \bar r.\pi \} }
	{ \Gamma \vdash e: \{ \bar r \} & \forall r \in \bar r \mid r: \{ r \} \in \Gamma & \pi \in \Pi } \\[3ex]

\infer[\textsc{($\varepsilon$-Subsume)}]
	{ \Gamma \vdash e: \tau' ~\kw{with} \varepsilon'}
	{ \Gamma \vdash e: \tau ~\kw{with} \varepsilon & \tau <: \tau' & \varepsilon \subseteq \varepsilon'}
\end{array}
\]


\noindent
\fbox{$\Gamma \vdash e: \tau~\kw{with} \varepsilon$}

\[
\begin{array}{c}

\infer[\textsc{(S-Arrow)}]
	{ \tau_1 \rightarrow_{\varepsilon} \tau_2 <: \tau_1' \rightarrow_{\varepsilon'} \tau_2' }
	{ \tau_1' <: \tau_1 & \tau_2 <: \tau_2' & \varepsilon \subseteq \varepsilon' }
~~~~~~
\infer[\textsc{(S-Resource)}]
	{ \{ \bar r_1 \} <: \{ \bar r_2 \} }
	{ r \in r_1 \implies r \in r_2 }

\end{array}
\]


\vspace{-7pt}
\caption{Type-with-effect judgements in $\opercalc$.}
\label{This is the label.}
\end{figure}

The static rules for $\opercalc$ are summarised in Figure 3.2. The typing judgement $\Gamma \vdash e: \tau~\kw{with} \varepsilon$ means that successive reductions on $e$ will yield terms of type $\tau$, and collectively incur no more than those effects in $\varepsilon$. This $\varepsilon$ is a conservative approximation to the runtime effects of executing $e$, so it may contain effects which don't happen at runtime.

The rules for variables and values are: \textsc{$\varepsilon$-Var}, \textsc{$\varepsilon$-Resource}, and \textsc{$\varepsilon$-Abs}. These are identical to the rules in $\stlc$, except they approximate the set of effects as $\varnothing$. Although a fucntion and a resource literal both encapsulate capabilities, something must be done to them (apply the function, operate on the resource) to incur a runtime effect.

The effects of a lambda application are: the effects of evaluating its subexpressions, and the effects incurred by executing the body of the lambda to which the left-hand side evaluates. Those last effects are obtained from the label on the lambda's arrow-type in the first premise.

The effects of an operation call are: the effects of evaluating the subexpression, and the single effect incurred when the subexpression is reduced to a resource literal $r$, and operation $\pi$ is invoked on it. It is not always possible to know statically which exact resource literal the subexpression reduces to (if it halts at all). For example, the program $\kwa{(if~System.randomBool~then~File~else~Socket).close}$ may either reduce to $\kwa{File.close}$ or $\kwa{Socket.close}$. In such cases, the safe approximation is to type the conditional as $\{ \kwa{File, Socket} \}$. \textsc{$\varepsilon$-OperCall} would then approximate the runtime effects of the operation call as  $\{ \kwa{File.close, Socket.close} \}$. Being able to type an expression as a (non-singleton) set of resources requires an extra rule: \textsc{S-Resource}. This says that a subset of resources is also a subtype. 

Because we're not really interested in what exactly an operation call does, every operation is valid for every resource. This can give bizarre programs --- $\kwa{Sensor.readTemp}$ seems like a sensible operation call, but what about $\kwa{File.readTemp}$? Because we are not interested in the specific behaviours of operations, we permit any operation on any resource to simplify the static rules.

The other subtyping rule is \textsc{S-Arrow}, a modification of the rule from $\stlc$. In addition to this rule being contravariant in the input and covariant in the output, it is also covariant in the effects. Justified in terms of the Liskov substitution principle, any possible effect which might be incurred by the subtype should be expected by the supertype, otherwise substitution of a supertype for a subtype would allow for the possibility of new effects not possible under the original.\\

\begin{figure}[h]

\bm{$\kwa{substitution :: e \times v \times v \rightarrow e}$}

\begin{itemize}
	\setlength\itemsep{-0.7em}
	\item[] $[v/y]x = v$, if $x = y$
	\item[] $[v/y]x = x$, if $x \neq y$
	\item[] $[v/y](\lambda x: \tau. e) = \lambda x: \tau.[v/y]e$, if $y \neq x$ and $y$ does not occur free in $e$
	\item[] $[v/y](e_1~e_2) = ([v/y]e_1)([v/y]e_2)$
	\item[] $[v/y](e_1.\pi) = ([v/y]e_1).\pi$
\end{itemize}

\vspace{-7pt}
\caption{Substitution function.}
\label{This is the label.}
\end{figure}

Figure 3.2. shows the updated definition of $\kwa{substitution}$. In $\opercalc$, a variable may only be substituted for a value, making the function a partial one. This restriction is imposed because, if a variable can be replaced with an arbitrary expression, then we might also be introducing arbitrary effects --- a situation which violates the preservation of effects under reduction. \\

\begin{figure}[h]

\noindent
\fbox{$e \longrightarrow e~|~\varepsilon$}

\[
\begin{array}{c}

\infer[\textsc{(E-App1)}]
	{e_1 e_2 \longrightarrow e_1'~ e_2~|~\varepsilon}
	{e_1 \longrightarrow e_1'~|~\varepsilon}
	~~~~~~
\infer[\textsc{(E-App2)}]
	{v_1 ~ e_2 \longrightarrow v_1 ~ e_2'~|~\varepsilon} 
	{e_2 \longrightarrow e_2'~|~\varepsilon}
~~~~~~
\infer[\textsc{(E-App3)}]
	{ (\lambda x: \hat \tau.\hat e) \hat v_2 \longrightarrow [\hat v_2/x]\hat e~|~\varnothing }
	{}\\[4ex]
	
\infer[\textsc{(E-OperCall1)}]
	{\hat e.\pi \longrightarrow \hat e'.\pi~|~\varepsilon }
	{\hat e \rightarrow \hat e'~|~\varepsilon}
		
	~~~~~~
	
\infer[\textsc{(E-OperCall2)}]
	{r.\pi \longrightarrow \kwa{unit}~|~\{ r.\pi \} }
	{r \in R & \pi \in \Pi}
	 \\[4ex]
	 
\end{array}
\]


\vspace{-7pt}
\caption{Single-step reductions.}
\label{This is the label.}
\end{figure}

Single-step reduction is now a relation on an expression $e$ and a pair $e \times \varepsilon$, representing the reduced expression and all effects incurred during the single-step of computation. \textsc{E-App1} and \textsc{E-App2} incur whatever is the effect of reducing their subexpressions. \textsc{E-App3} incurs no effects.

The new single-step rules are \textsc{E-OperCall1} and \textsc{E-OperCall2}. The former reduces the receiver of an operation-call, and the latter performs an operation on a resource literal. \textsc{E-OperCall1} incurs whatever is the effect of reducing the subexpression. \textsc{E-OperCall2}, which reduces the operation-call $r.\pi$, incurs the effect $r.\pi$.

Operation calls reduce to $\unit$ (which is a derived form; see Encodings). $\unit$ is a value representing the absence of information (because it is the only value of its type). Because we are not interested in the semantics of effects, we choose $\unit$ as the result of reducing an operation-call. \\

\begin{figure}[h]

\noindent
\fbox{$\hat e \longrightarrow^{*} \hat e~|~\varepsilon$}

\[
\begin{array}{c}

\infer[\textsc{(E-MultiStep1)}]
	{\hat e \rightarrow^{*} \hat e~|~\varnothing}
	{}
~~~
\infer[\textsc{(E-MultiStep2)}]
	{\hat e \rightarrow^{*} \hat e'~|~\varepsilon}
	{\hat e \rightarrow \hat e'~|~\varepsilon} \\[3ex]
	
\infer[\textsc{(E-MultiStep3)}]
	{\hat e \rightarrow^{*} \hat e''~|~\varepsilon_1 \cup \varepsilon_2}
	{\hat e \rightarrow^{*} \hat e'~|~\varepsilon_1 & \hat e' \rightarrow^{*} \hat e''~|~\varepsilon_2}
\end{array}
\]

\vspace{-7pt}
\caption{Multi-step reductions in $\opercalc$.}
\label{This is the label.}
\end{figure}

A multi-step reduction consists of zero\footnote{We permit multi-step reductions of length zero to be consistent with Pierce, who defines multi-step reduction as a reflexive relation\cite[p. 39]{tapl}.} or more single-step reductions. The resulting effect-set is the union of all the single-steps taken.

\subsection{Soundness of $\opercalc$}

Our goal is to show $\opercalc$ is sound. This requires an appropriate notion of \textit{effect-soudnness}, as we need both the type and approximated effects of an expression to be preserved under reduction. Intuitively, a reduction $e_A \longrightarrow^* e_B~|~\varepsilon$ is sound if the type system's approximation of the effects of $e$ contains the actual runtime effects $\varepsilon$, because then it has statically accounted for every possible runtime effect. Below is a definition, phrased in terms of single-step reduction.


\begin{theorem}[Soundness]
If $\hat \Gamma \vdash \hat e_A: \hat \tau_A~\kw{with} \varepsilon_A$ and $\hat e_A$ is not a value, then $e_A \longrightarrow e_B~|~\varepsilon$, where $\hat \Gamma \vdash e_B: \hat \tau_B~\kw{with} \varepsilon_B$ and $\hat \tau_B <: \hat \tau_A$ and $\varepsilon_B \cup \varepsilon \subseteq \varepsilon_A$.
\end{theorem}

This definition is slightly stronger, because it relates the approximation after reduction to the approximation before reduction, by saying it can only get more precise. This is analogous to the type-safe component of the definition, which states that the types of successive terms under reduction can only get more precise.

Our road to proving soundness takes the standard approach of showing that progress and preservation hold of $\opercalc$. Progress follows immediately by observing some properties of the typing rules.

\begin{lemma}[Canonical Forms]
The following are true:
\begin{itemize}
	\setlength\itemsep{-0.7em}
	\item If $\hat \Gamma \vdash \hat v: \hat \tau~\kw{with} \varepsilon$ then $\varepsilon = \varnothing$.
	\item If $\hat \Gamma \vdash \hat v: \{ \bar r \}$ then $\hat v = r$ for some $r \in R$ and $\{ \bar r \} = \{ r \}$.
\end{itemize}
\end{lemma}

\begin{theorem}[Progress]
If $\hat \Gamma \vdash \hat e: \hat \tau~\kw{with} \varepsilon$ and $\hat e$ is not a value, then $\hat e \longrightarrow \hat e'~|~\varepsilon$.
\end{theorem}

\begin{proof} By induction on $\hat \Gamma \vdash \hat e: \hat \tau~\kw{with} \varepsilon$, for $\hat e$ not a value. If the rule is \textsc{$\varepsilon$-Subsumption} it follows by inductive hypothesis. If $\hat e$ has a reducible subexpression then reduce it. Otherwise use one of \textsc{$\varepsilon$-App3} or \textsc{$\varepsilon$-OperCall2}.
\end{proof}

To show preservation holds we need to know that type-and-effect safety is preserved by the substitution in the rule \textsc{E-App3}. The formulation of the substitution lemam is largely the same as it is in $\stlc$, except the expression being substituted must be a value. This strengthening of the lemma is not a problem as the dynamic rules employ a strict evaluation strategy, so any expressions are reduced to values before they are substituted.

\begin{lemma}[Substitution]
If $\Gamma, x: \tau' \vdash e: \tau~\kw{with} \varepsilon$ and $\Gamma \vdash v: \tau'~\kw{with} \varnothing$ then $\Gamma \vdash [v/x]e: \tau~\kw{with} \varepsilon$.
\end{lemma}

\begin{proof} By induction on $\Gamma, x: \tau' \vdash e: \tau~\kw{with} \varepsilon$.
\end{proof}

With this lemma, we are ready to prove the preservation theorem.

\begin{theorem}[Preservation]
If $\Gamma \vdash e_A: \tau_A~\kw{with} \varepsilon_A$ and $e_A \longrightarrow e_B~|~\varepsilon$, then $\tau_B <: \tau_A$ and $\varepsilon_B \cup \varepsilon \subseteq \varepsilon_A$.
\end{theorem}

\begin{proof} By induction on $\Gamma \vdash e_A: \tau_A~\kw{with} \varepsilon_A$, and then on $e_A \longrightarrow e_B~|~\varepsilon$. Since $e_A$ can be reduced, we need only consider those rules which apply to non-values and non-variables. \\

\textit{Case:} \textsc{$\varepsilon$-App} Then $e_A = e_1 ~ e_2$ and $e_1: \tau_2 \rightarrow_{\varepsilon} \tau_3~\kw{with} \varepsilon_1$ and $\Gamma \vdash e_2: \tau_2~\kw{with} \varepsilon_2$. If the reduction rule used was \textsc{E-App1} or \textsc{E-App2}, then the result follows by applying the inductive hypothesis to $e_1$ and $e_2$ respectively.

Otherwise the rule used was \textsc{E-App3}. Then $(\lambda x: \tau_2. e) v_2 \longrightarrow [ v_2/x] e~|~\varnothing$. By inversion on the typing rule for $\lambda x:  \tau_2. e$ we know $\Gamma, x:  \tau_2 \vdash  e:  \tau_3~\kw{with} \varepsilon_3$. By canonical forms, $\varepsilon_2 = \varnothing$ because $ e_2 =  v_2$ is a value. Then by the substitution lemma, $ \Gamma \vdash [ v_2/x] e :  \tau_3~\kw{with} \varepsilon_3$. By canonical forms, $\varepsilon_1 = \varepsilon_2 = \varnothing = \varepsilon_C$. Therefore $\varepsilon_A = \varepsilon_3 = \varepsilon_B \cup \varepsilon_C$.\\

\textit{Case:} \textsc{$\varepsilon$-OperCall}. Then $e_A = e_1.\pi$ and $ \Gamma \vdash e_1 : \{ \bar r \}~\kw{with} \varepsilon_1$. If the reduction rule used was \textsc{E-OperCall1} then the result follows by applying the inductive hypothesis to $ e_1$.

Otherwise the reduction rule used was \textsc{E-OperCall2} and $v_1.\pi \longrightarrow \kwa{unit}~|~\{ r.\pi \}$. By canonical forms, $ \Gamma \vdash v_1: \kwa{unit}~\kw{with} \{ r.\pi \}$. Also, $ \Gamma \vdash \kwa{unit}: \kwa{Unit}~\kw{with} \varnothing$. Then $\tau_B = \tau_A$. Also, $\varepsilon_C \cup \varepsilon_B = \{ r.\pi \} = \varepsilon_A$.\\

\end{proof}

Our single-step soundness theorem now holds immediately by joining the progress and preservation theorems into one.

\begin{theorem}[Soundness]
If $ \Gamma \vdash  e_A:  \tau_A~\kw{with} \varepsilon_A$ and $ e_A$ is not a value, then $e_A \longrightarrow e_B~|~\varepsilon$, where $ \Gamma \vdash e_B:  \tau_B~\kw{with} \varepsilon_B$ and $ \tau_B <:  \tau_A$ and $\varepsilon_B \cup \varepsilon \subseteq \varepsilon_A$.
\end{theorem}
\begin{proof}
If $ e_A$ is not a value then the reduction exists by the progress theorem. The rest follows by the preservation theorem.
\end{proof}

Knowing that single-step reductions are sound, the soundness of multi-step reductions follows by inductively applying single-step soundness on the length of a multi-step reduction.

\begin{theorem}[Multi-step Soundness]
If $ \Gamma \vdash  e_A:  \tau_A~\kw{with} \varepsilon_A$ and $e_A \longrightarrow^{*} e_B~|~\varepsilon$, where $\Gamma \vdash e_B: \tau_B~\kw{with} \varepsilon_B$ and $ \tau_B <: \tau_A$ and $\varepsilon_B \cup \varepsilon \subseteq \varepsilon_A$.
\end{theorem}

\begin{proof} By induction on the length of the multi-step reduction. If the length is 0 then $e_A = e_B$ and the result holds vacuously. If the length is 1 the result holds by soundness of single-step reductions. if the length is $n+1$, then the first $n$-step reduction is sound by inductive hypothesis and the last step is sound by single-step soundness, so the entire $n+1$-step reduction is sound.
\end{proof}


























\section{$\epscalc$: Epsilon Calculus}


The effect calculus is based on the simply-typed lambda calculus $\lambda^{\rightarrow}$. There is one type constructor, $\rightarrow$. The base types are sets of resources, denoted by $\{ \bar r \}$. Although the calculus has no primitive notions of integers or booleans, we shall assume these may be encoded as they are in the usual way (e.g. as Church numerals) and make free use of them in examples as though they were standard, for the sake of readability.

\subsection{Definition of $\epscalc$}

\begin{figure}[h]
\vspace{-5pt}

\[
\begin{array}{lll}

\begin{array}{lllr}

e & ::= & ~ & exprs: \\
	& | & x & variable \\
	& | & v & value \\
	& | & e ~ e & application \\
	& | & e.\pi & operation \\
	&&\\

v & ::= & ~ & values: \\
	& | & r & resource~literal \\
	& | & \lambda x: \tau.e & abstraction \\
	&&\\
	
\hat e & ::= & ~ & labelled~exprs: \\
	& | & x \\
	& | & r \\
	& | & \lambda x: \hat \tau.\hat e \\
	& | & \hat e ~ \hat e \\
	& | & \hat e.\pi \\
	& | & \kwa{import}(\varepsilon)~x = \hat e~\kwa{in}~e & import \\
	&&\\

\hat v & ::= & ~ & labelled~values: \\
	& | & r \\
	& | & \lambda x: \hat \tau.\hat e \\
	&&\\

\end{array}

& ~~~~~~~~&

\begin{array}{lllr}

\varepsilon & ::= & ~ & effects: \\
	& | & \{ \overline{r.\pi} \} \\
	&&\\

\tau & ::= & ~ & types: \\
		& | & \{ \bar r \} \\
		& | & \tau \rightarrow \tau \\ 
		&&\\

\hat \tau & ::= & ~ & labelled ~types: \\
		& | & \{ \bar r \} \\
		& | & \hat \tau \rightarrow_{\varepsilon} \hat \tau \\
		&&\\

\Gamma & ::= & ~ & type~ctx: \\
				& | & \varnothing \\
				& | & \Gamma, x: \tau \\
				&&\\
				
\hat \Gamma & ::= & ~ & labelled~type~ctx:\\
				& | & \varnothing \\
				& | & \hat \Gamma, x: \hat \tau \\
				&&\\

\end{array}

\end{array}
\]

\vspace{-7pt}
\caption{Effect calculus.}
\label{This is the label.}
\end{figure}

Resources are drawn from a fixed set $R$ of variables, and model those initial capabilities passed in from the system environment. Resources cannot be created at runtime. When a resource type is ascribed to a program, as in the judgement $\Gamma \vdash e: \{ \bar r \}$, it means that if $e$ terminates it will result in a resource literal $r \in \bar r$.

A value $v$ is either a resource literal $r$ or a lambda abstraction $\lambda x: \tau.e$. The other forms of an expression are lambda application $e~e$, variable $x$, and operation $e.\pi$. An operation is an action invoked on a resource. For example, we might invoke the $\kwa{open}$ operation on a $\kwa{File}$ resource. Operations are drawn from a fixed-set $\Pi$ of variables. They cannot be created at runtime.

An effect is an operation performed on a resource. Formally, they are members of $R \times \Pi$, but for readability we write $\kwa{File.write}$ over $\kwa{(File, write)}$. A set of effects is denoted by $\varepsilon$. Effects and operations notationally look the same, but should be distinguished: an effect is some action upon a resource which may happen during runtime; an operation is the actual invocation of an effect at runtime.

In a practical language, operations should take arguments. For example, when writing to a file, we want to specify \textit{what} is being written to the file, ala $\kwa{File.write(``my msg'')}$. Because $\epscalc$ is only concerned with the use and propagation of effects, and not the semantics of particular effects, we make the simplifying assumption that all operations are null-ary.

Expressions may be labelled with the set of effects they might incur during execution. This is achieved by annotating all arrow types inside the expression. If a metavariable represents a labelled expression, it will be written with a hat; if it represents an unlabelled expression, it will have no hat. Compare $e$ and $\hat e$.

Labelling of an expression is \textit{deep}. That is, every subterm of a labelled term is also labelled. Unlabelled terms are also deeply unlabelled. The only exception is the $\kwa{import}$ expression, which is the only way to compose labelled and unlabelled code. $\kwa{import}$ nests unlabelled code inside labelled code, and selects those capabilities $\varepsilon$ over which the unlabelled code has authority. It is not possible to nest labelled code inside unlabelled code.

The distinction between labelled and unlabelled types and expressions requires us to have the notion of labelled and unlabelled contexts. Labelled contexts only bind variables to labelled types, whereas unlabelled contexts only bind variables to unlabelled types. There is no valid context which mixes labelled and unlabelled types.

A construct's labelled version is always denoted with a hat. \\

\begin{figure}[h]
\vspace{-5pt}

	
$\bm{\kwa{annot :: e \times \varepsilon \rightarrow \hat e}}$

\begin{itemize}
	\setlength\itemsep{-0.7em}
	\item[] $\annot{r}{\_} = r$
	\item[] $\annot{\lambda x: \tau_1 . e}{\varepsilon} = \lambda x: \annot{\tau_1}{\varepsilon} . \annot{e}{\varepsilon}$
	\item[] $\annot{e_1~e_2}{\varepsilon} = \kwa{annot}(e_1, \varepsilon)~\kwa{annot}(e_2, \varepsilon)$
	\item[] $\annot{e_1.\pi}{\varepsilon} = \annot{e_1}{\varepsilon}.\pi$
\end{itemize}
	
$\bm{\kwa{annot :: \tau \times \varepsilon \rightarrow \hat \tau}}$

\begin{itemize}
	\setlength\itemsep{-0.7em}
	\item[] $\annot{\{ \bar r \}}{\_} = \{ \bar r \}$
	\item[] $\annot{\tau \rightarrow \tau}{\varepsilon} = \tau \rightarrow_{\varepsilon} \tau$.	
\end{itemize}

$\bm{\kwa{annot :: \Gamma \times \varepsilon \rightarrow \hat \Gamma}}$

\begin{itemize}
	\setlength\itemsep{-0.7em}
	\item[] $\annot{\varnothing}{\_} = \varnothing$
	\item[] $\annot{\Gamma, x: \tau}{\varepsilon} = \annot{\Gamma}{\varepsilon}, x: \annot{\tau}{\varepsilon}$
\end{itemize}

$\bm{\kwa{erase :: \hat \tau \rightarrow \tau}}$
\begin{itemize}
	\setlength\itemsep{-0.7em}
	\item[] $\erase{\{ \bar r \}}$
	\item[] $\erase{\hat \tau_1 \rightarrow_{\varepsilon} \hat \tau_2} = \erase{\hat \tau_1} \rightarrow \erase{\hat \tau_2}$
\end{itemize}

$\bm{\kwa{erase :: \hat e \rightarrow e}}$
\begin{itemize}
	\setlength\itemsep{-0.7em}
	\item[] $\erase{r} = r$
	\item[] $\erase{\lambda x: \hat \tau_1 . \hat e} = \lambda x: \erase{\hat \tau_1} . \erase{\hat e}$
	\item[] $\erase{e_1~e_2} = \erase{e_1}~\erase{e_2}$
	\item[] $\erase{e_1.\pi} = \erase{e_1}.\pi$
\end{itemize}

\vspace{-7pt}
\caption{Annotation functions.}
\label{This is the label.}
\end{figure}


\noindent
Given a piece of unlabelled code $e$ and static effects $\varepsilon$ we can produce a labelled piece of code $\kwa{annot}(e, \varepsilon) = \hat e$ by annotating every function with $\varepsilon$. In the reverse direction, given some labelled code $\hat e$ we can produce an unlabelled piece of code $\kwa{erase}(\hat e) = e$ by removing the labels on functions. Full definitions for these functions on expressions, types, and contexts are given in Figure 3.2. Note that $\kwa{erase}$ is undefined on $\kwa{import}$ expressions. We won't ever need to erase import expressions, but it means the function is partial, so we need to be careful when we use it.

Annotation is not always safe. For instance, $\kwa{annot}(\lambda l: \kwa{Int} \rightarrow_{File.read} \kwa{Int} . ~l~1, \varnothing)$ would overwrite the $\kwa{File.read}$ effect permitted by $l$. $\kwa{annot}$ is used in one place, in the dynamic rules, and for that limited use we will have to prove its safety. \\




















\begin{figure}[h]

$\bm{\kwa{effects :: \hat \tau \rightarrow \varepsilon}}$ \begin{itemize}
	\setlength\itemsep{-0.7em}
	\item[] $\fx{\{ \bar r \}} = \{ r.\pi \mid r \in \bar r, \pi \in \Pi \}$
	\item[] $\fx{\hat \tau_1 \rightarrow_{\varepsilon} \hat \tau_2} = \hofx{\hat \tau_1} \cup \varepsilon \cup \fx{\hat \tau_2}$
\end{itemize}

$\bm{\kwa{ho \hyphen effects} :: \hat \tau \rightarrow \varepsilon}$ \begin{itemize}
	\setlength\itemsep{-0.7em}
	\item[] $\hofx{\{ \bar r \}} = \varnothing$
	\item[] $\hofx{\hat \tau_1 \rightarrow_{\varepsilon} \hat \tau_2} = \fx{\hat \tau_1} \cup \hofx{\hat \tau_2}$
\end{itemize}

\vspace{-7pt}
\caption{Effect functions.}
\label{This is the label.}
\end{figure}

\noindent
We may wish to know what effects are encapsulated by a piece of labelled code. This is achieved by two functions, $\kwa{effects}(\hat e)$ and $\kwa{ho \hyphen effects}(\hat e)$, which collectively compute the set of effects captured by $\hat e$. These are effects which may, directly or indirectly, be invoked by $\hat e$. The difference between the two functions is in who supplies the effect. $\kwa{effect}(\hat e)$ is the set of effects for which $\hat e$ has direct authority, while $\kwa{ho \hyphen effects}$ is the set of effects for which $\hat e$ has (strictly) transitive authority. These higher-order effects are always supplied by some external environment.

For example, take the function which, given a file, reads and returns its contents (which are perhaps encoded as an integer). Its signature would be $\kwa{f : \{ File \} \rightarrow_{File.read} Int}$. The $\kwa{\fx{f} = \{File.read\} \cup \fx{Int}}$, because any client using $\kwa{f}$ will directly invoke the $\kwa{File.read}$ operation and may use any resource encapsulated by the $\kwa{Int}$ type. The $\kwa{\hofx{f} = \{ File.\pi \mid \pi \in \Pi \} }$, because to use $\kwa{f}$ it must be supplied with a $\kwa{File}$ literal from some outside source. Therefore, every possible effect on $\kwa{File}$ is a higher-order effect.\\


\begin{figure}[h]

\bm{$\kwa{substitution :: \hat e \times \hat v \times \hat v \rightarrow \hat e}$}

\begin{itemize}
	\setlength\itemsep{-0.7em}
	\item[] $[\hat v/y]x = \hat v$, if $x = y$
	\item[] $[\hat v/y]x = x$, if $x \neq y$
	\item[] $[\hat v/y](\lambda x: \hat \tau. \hat e) = \lambda x: \hat \tau.[\hat v/y]\hat e$, if $y \neq x$ and $y$ does not occur free in $\hat e$
	\item[] $[\hat v/y](\hat e_1~\hat e_2) = ([\hat v/y]\hat e_1)([\hat v/y]\hat e_2)$
	\item[] $[\hat v/y](\hat e_1.\pi) = ([\hat v/y]e_1).\pi$
	\item[] $[\hat v/y](\import{\varepsilon}{x}{\hat e}{e}) = \import{\varepsilon}{x}{[\hat v/y]\hat e}{e}$
\end{itemize}

\vspace{-7pt}
\caption{Substitution function.}
\label{This is the label.}
\end{figure}

\noindent
The substitution function $\kwa{substitution}(\hat e, \hat v, x)$ replaces all free occurrences of $x$ with $\hat v$ in $\hat e$. The short-hand is $[\hat v/x]\hat e$. When performing multiple substitutions we use the notation $[\hat v_1/x_1, \hat v_2/x_2]\hat e$ as shorthand for $[\hat v_2/x_2]([\hat v_1/x_1]\hat e)$. Note how the order of the variables has been flipped; the substitutions occur as they are written, left-to-right.

Note that substitution is partial, because it is only defined when a free-variable is being replaced with a value. This is important for proving preservation, because if we replace variables with arbitrary expressions, then we might also be introducing arbitrary effects.

To avoid accidental variable capture we adopt the convention of $\alpha$-conversion, whereby we freely and implicitly interchange expressions which are equivalent up to the naming of bound variables \cite[p. 71]{tapl}. This elides some tedious bookkeeping. Consequently, we shall assume variables are (re-)named in this way to avoid accidental capture.



\begin{figure}[h]

\fbox{$\Gamma \vdash e: \tau$}

\[
\begin{array}{c}


\infer[\textsc{(T-Var)}]
	{\Gamma, x: \tau \vdash x: \tau}
	{}
~~~~~~
\infer[\textsc{(T-Resource)}]
	{\Gamma, r: \{ r \} \vdash r : \{ r \}}
	{}

~~~~~~
\infer[\textsc{(T-Abs)}]
	{\Gamma \vdash \lambda x: \tau_1.e : \tau_1 \rightarrow \tau_2}
	{\Gamma, x: \tau_1 \vdash e: \tau_2}\\[4ex]
	
\infer[\textsc{(T-App)}]
	{\Gamma \vdash e_1~e_2: \tau_3}
	{\Gamma \vdash e_1: \tau_2 \rightarrow \tau_3 & \Gamma \vdash e_2: \tau_2}
~~~~~~
\infer[\textsc{(T-OperCall)}]
	{\Gamma \vdash e.\pi: \kwa{Unit}}
	{\Gamma \vdash e: \{ \bar r \} & \forall r \in \bar r \mid r \in R & \pi \in \Pi}

\end{array}
\]

\vspace{-7pt}
\caption{Typing judgements in the epsilon calculus.}
\label{This is the label.}
\end{figure}

\noindent

\noindent
The first sort of static judgement ascribes a type to a piece of unlabelled code. \textsc{T-Var}, \textsc{T-App}, and \textsc{T-OperCall} are the same as they are in $\lambda^{\rightarrow}$. \textsc{T-Resource} is the same as \textsc{T-Var}, but for variables representing primitive capabilities. \textsc{T-OperCall} is the rule for typing an operation call $e_1.\pi$. Such an expression is well-typed if $e_1$ types to some valid resource, and $\pi$ is a known operation.


\begin{figure}[h]

\noindent
$\fbox{$\kwa{safe(\hat \tau, \varepsilon)}$}$

\[
\begin{array}{c}

\infer[\textsc{(Safe-Resource)}]
	{\kwa{safe}(\{ \bar r \}, \varepsilon)}
	{}
~~~~~
\infer[\textsc{(Safe-Unit)}]
	{\kwa{safe}(\kwa{Unit}, \varepsilon)}
	{} \\[3ex]

\infer[\textsc{(Safe-Arrow)}]
	{\kwa{safe}(\hat \tau_1 \rightarrow_{\varepsilon'} \hat \tau_2, \varepsilon)}
	{\varepsilon \subseteq \varepsilon' & \kwa{ho \hyphen safe}(\hat \tau_1, \varepsilon) & \kwa{safe}(\hat \tau_2, \varepsilon)} \\[3ex]

\end{array}
\]

\noindent
$\fbox{$\kwa{ho \hyphen safe(\hat \tau, \varepsilon)}$}$

\[
\begin{array}{c}

\infer[\textsc{(HOSafe-Resource)}]
	{ \kwa{ho \hyphen safe}( \{ \bar r \}, \varepsilon)} 
	{}
	~~~~~~
\infer[\textsc{(HOSafe-Unit)}]
	{ \kwa{ho \hyphen safe}( \kwa{Unit}, \varepsilon)} 
	{}\\[3ex]

\infer[\textsc{(HOSafe-Arrow)}]
	{ \kwa{ho \hyphen safe}( \hat \tau_1 \rightarrow_{\varepsilon'} \hat \tau_2, \varepsilon ) }
	{ \kwa{safe}(\hat \tau_1, \varepsilon)  & \kwa{ho \hyphen safe}(\hat \tau_2, \varepsilon) }\\[3ex]

\end{array}
\]

\vspace{-7pt}
\caption{Safety judgements in the epsilon calculus.}
\label{This is the label.}
\end{figure}

\noindent 
Before presenting the type-with-effect rules for labelled expressions, we first define a few safety predicates. Intuitively, the type $\hat \tau$ is $\kwa{safe}$ for $\varepsilon$ if it has declared every (non higher-order) effect $r.\pi \in \varepsilon$ in its signature. $\hat \tau$ is $\kwa{ho \hyphen safe}$ for $\varepsilon$ if $\hat \tau$ has declared every higher-order effect $r.\pi \in \varepsilon$ in its signature. One way to think about these predicates is as a contract between caller and callee. If the caller supplies a set of capabilities $\varepsilon$ to a piece of code typing to $\hat \tau$, it would violate the restriction on \textit{ambient authority} if a capability was supplied that $\hat \tau$ had not explicitly asked for. Therefore, $\safe{\hat \tau}{\varepsilon}$ holds when the (non higher-order) effects selected by $\hat \tau$ include $\varepsilon$. $\hosafe{\hat \tau}{\varepsilon}$ holds when the higher-order effects selected by $\hat  \tau$ include $\varepsilon$.

Because the implementation of $\hat \tau$ might internally propagate capabilities, the definitions of safety and higher-order safety need to be transitive. \textbf{Give an example of why this is so.}



\begin{figure}[h]


\noindent
\fbox{$\hat \Gamma \vdash \hat e: \hat \tau~\kw{with} \varepsilon$}

\[
\begin{array}{c}

\infer[\textsc{($\varepsilon$-Var)}]
	{ \hat \Gamma, x:\tau \vdash x: \tau~\kw{with} \varnothing }
	{}
	
	~~~
	
\infer[\textsc{($\varepsilon$-Resource)}]
 	{ \hat \Gamma, r: \{ r \} \vdash r : \{ r \}~\kw{with} \varnothing }
 	{} \\[3ex]
 	
 	~~~
	\infer[\textsc{($\varepsilon$-Abs)}]
	{ \hat \Gamma \vdash \lambda x:\tau_2 . \hat e : \hat \tau_2 \rightarrow_{\varepsilon_3} \hat \tau_3~\kw{with} \varnothing }
	{ \hat \Gamma, x: \hat \tau_2 \vdash \hat e: \hat \tau_3~\kw{with} \varepsilon_3 }
	
	~~~
	
\infer[\textsc{($\varepsilon$-App)}]
	{ \hat \Gamma \vdash \hat e_1 \hat e_2 : \hat \tau_3~\kw{with} \varepsilon_1 \cup \varepsilon_2 \cup \varepsilon  }
	{ \hat \Gamma \vdash \hat e_1: \hat \tau_2 \rightarrow_{\varepsilon} \hat \tau_3~\kw{with} \varepsilon_1 & \hat \Gamma \vdash \hat e_2: \hat \tau_2~\kw{with} \varepsilon_2 } \\[3ex]
	
\infer[\textsc{($\varepsilon$-OperCall)}]
	{ \hat \Gamma \vdash \hat e.\pi: \kw{Unit} \kw{with} \{ \bar r.\pi \} }
	{ \hat \Gamma \vdash \hat e: \{ \bar r \} & \forall r \in \bar r \mid r: \{ r \} \in \Gamma & \pi \in \Pi } \\[3ex]

\infer[\textsc{($\varepsilon$-Subsume)}]
	{ \hat \Gamma \vdash e: \tau' ~\kw{with} \varepsilon'}
	{ \hat \Gamma \vdash e: \tau ~\kw{with} \varepsilon & \tau <: \tau' & \varepsilon \subseteq \varepsilon'}  \\[3ex]

\infer[\textsc{($\varepsilon$-Import)}]
	{ \hat \Gamma \vdash \kwa{import}(\varepsilon)~x = \hat e~\kw{in} e: \kwa{annot}(\tau, \varepsilon)~\kw{with} \varepsilon \cup \varepsilon_1 }
{{\def\arraystretch{1.4}
  \begin{array}{c}
\hat \Gamma \vdash \hat e: \hat \tau~\kw{with} \varepsilon_1
~~~~~~
\varepsilon = \kwa{effects}(\hat \tau) \\
\kwa{ho \hyphen safe}(\hat \tau, \varepsilon) ~~~~~~ x: \kwa{erase}(\hat \tau) \vdash e: \tau
  \end{array}}} 
 
\end{array}
\]


\vspace{-7pt}
\caption{Type-with-effect judgements.}
\label{This is the label.}
\end{figure}

\noindent
$\epscalc$ has a new kind of judgement. $\Gamma \vdash \hat e: \hat \tau~\kw{with} \varepsilon$ can be read as saying that $\hat e$, if it halts, will produce a value of type $\hat \tau$ and incur at most the set of effects $\varepsilon$. This judgement gives a conservative approximation as to what will happen; some of the effects in the typing judgement may not actually happen at runtime.

The simplest rules are those which type values as having no effect. Although a function and a resource literal can both capture capabilities, you must do something with them (apply the function, operate on the resource) to incur a runtime effect.

The effects of a lambda application are: the effects of evaluating its subexpressions, and the effects incurred by executing the body of the lambda to which the left-hand side evaluates. Those last effects are pulled from the label on the lambda's arrow-type.

The effects of an operation call are: the effects of evaluating the subexpression, and the single effect incurred when the subexpression is reduced to a resource literal $r$, and operation $\pi$ is invoked on it. It is not always possible to know statically which exact resource literal the subexpression reduces to (if it halts at all). Figure 3.8. shows such an example. The safe approximation is to say that the operation call $\hat e.\pi$ incurs $\pi$ on every possible resource to which $\hat e$ might evaluate. In the case of Figure 3.8., this would be $\kwa{\{ File.write, Socket.write \}}$.

Because we're not really interested in what exactly an operation call does, every operation is valid for every resource. This can give bizarre programs --- $\kwa{Sensor.readTemp}$ seems like a sensible operation call, but what about $\kwa{File.readTemp}$? --- however, an adequate treatment is outside of the scope of $\epscalc$, so we gloss over such programs.

\textbf{It actually might be possible to figure out the exact literal if the system's not Turing complete, since the simply-typed lambda calculus is strongly normalising (and this is basically that, with a few extras), so be careful about this claim} \\

\begin{figure}[h]

\begin{lstlisting}
def getResource(b: Bool): { File, Socket } with $\varnothing$ =
    if b then File else Socket
 
val boolVal: Bool = System.randomBool
getResource(boolVal).write

\end{lstlisting}
\vspace{-7pt}
\caption{We cannot statically determine which branch will execute, so the safe approximation for $\kwa{getResource(boolVal).write}$ is $\{ \kwa{File.write, Socket.write} \}$.}
\label{This is the label.}
\end{figure}

\noindent
The most interesting rule is $\varepsilon$-Import. This rule is set up to ensure the interaction between labelled and unlabelled code is capability-safe. We type $e$ with $x: \kwa{erase}(\hat \tau)$. This eliminates ambient authority, because the only free variables in $e$ will be those selected by the interface $\hat \tau$.

For our rule to be capability-safe, we need to ensure that any higher-order function in scope is expecting the set of capabilities in $\hat \tau$. If not, we could exercise ambient authority by passing that higher-order function a capability from $\hat \tau$ which it hadn't selected. This is the purpose of $\hosafe{\hat \tau}{\varepsilon}$: all higher-order functions in scope need to be expecting any capability they might be passed.

In the conclusion of the rule we annotate the unlabelled code's effects as $\kwa{effects}(\hat \tau)$. Because this is the full set of capabilities over which $e$ has access, and because this set is higher-order safe, we shall see this annotation is sound.



\begin{figure}[h]
\noindent
$\fbox{$\hat \tau <: \hat \tau$}$

\[
\begin{array}{c}

\infer[(\textsc{S-Effects})]
	{\hat \tau_1 \rightarrow_{\varepsilon} \hat \tau_2 <: \hat \tau_1' \rightarrow_{\varepsilon'} \hat \tau_2'}
	{\varepsilon \subseteq \varepsilon' & \hat \tau_2 <: \hat \tau_2' & \hat \tau_1' <: \hat \tau_1}
~~~
\infer[(\textsc{S-Resources})]
	{ \{ \bar r_1 \} <: \{ \bar r_2 \} }
	{ r \in r_1 \implies r \in r_2 }

\end{array}
\]
\vspace{-7pt}
\caption{Subtyping judgements in the epsilon calculus.}
\label{This is the label.}
\end{figure}

\noindent
In addition to the usual subtyping rules from $\lambda^{\rightarrow}$ between $\tau$ terms, we introduce two more for $\hat \tau$ terms. 

The rule for functions is contravariant in the input-type and covariant in the output-type (as in $\lambda^{\rightarrow}$), and requires the effects of the super-type to be an upper-bound of the effects of the sub-type. We can think of this in terms of Liskov's substitution principle: if the subtype incurred an effect the supertype hadn't declared, it would violate the supertype's interface.

The rule for resources says that a superset of resources is a subtype.













\begin{figure}[h]

\noindent
\fbox{$\hat e \longrightarrow \hat e~|~\varepsilon$}

\[
\begin{array}{c}

\infer[\textsc{(E-App1)}]
	{\hat e_1 \hat e_2 \longrightarrow \hat e_1' \hat e_2~|~\varepsilon}
	{\hat e_1 \longrightarrow \hat e_1'~|~\varepsilon}
	~~~~~~
\infer[\textsc{(E-App2)}]
	{\hat v_1 \hat e_2 \longrightarrow \hat v_1 \hat e_2'~|~\varepsilon} 
	{\hat e_2 \longrightarrow \hat e_2'~|~\varepsilon}
~~~~~~
\infer[\textsc{(E-App3)}]
	{ (\lambda x: \hat \tau.\hat e) \hat v_2 \longrightarrow [\hat v_2/x]\hat e~|~\varnothing }
	{}\\[4ex]
	
\infer[\textsc{(E-OperCall1)}]
	{\hat e.\pi \longrightarrow \hat e'.\pi~|~\varepsilon }
	{\hat e \rightarrow \hat e'~|~\varepsilon}
		
	~~~~~~
	
\infer[\textsc{(E-OperCall2)}]
	{r.\pi \longrightarrow \kwa{unit}~|~\{ r.\pi \} }
	{r \in R & \pi \in \Pi}
	 \\[4ex]
	 
\infer[\textsc{(E-Import1)}]
	{\kwa{import}(\varepsilon)~x = \hat e~\kw{in} e \longrightarrow \kwa{import}(\varepsilon)~x = \hat e'~\kw{in} e~|~\varepsilon'}
	{\hat e \longrightarrow \hat e'~|~\varepsilon'}\\[4ex]

\infer[\textsc{(E-Import2)}]
	{\kwa{import}(\varepsilon)~x = \hat v~\kw{in} e \longrightarrow [\hat v/x]\kwa{annot}(e, \varepsilon)~|~\varnothing}
	{}

\end{array}
\]


\vspace{-7pt}
\caption{Single-step reductions.}
\label{This is the label.}
\end{figure}

\noindent
A single-step reduction takes an expression to a pair consisting of an expression and a set of runtime effects. The rules $\textsc{E-App1}$, $\textsc{E-App2}$, $\textsc{E-OperCall1}$, $\textsc{E-Import1}$ all reduce a single subexpression.

$\textsc{E-App3}$ is the standard $\lambda^{\rightarrow}$ rule for applying a value to a lambda, by performing substitution on the lambda body.

$\textsc{E-OperCall2}$ performs an operation on a resource literal. In this case it reduces to $\kwa{unit}$ (which is a derived form in our calculus; see 3.4. Encodings). This choice reflects the fact that $\epscalc$ doesn't model the potentially varied return types of functions. 

$\textsc{E-Import2}$ performs module resolution. The (unlabelled) body of code is annotated with the set of effects captured by the interface, and then the value being imported is substituted into the body of code.


\begin{figure}[h]

\noindent
\fbox{$\hat e \longrightarrow^{*} \hat e~|~\varepsilon$}

\[
\begin{array}{c}

\infer[\textsc{(E-MultiStep1)}]
	{\hat e \rightarrow^{*} \hat e~|~\varnothing}
	{}
~~~
\infer[\textsc{(E-MultiStep2)}]
	{\hat e \rightarrow^{*} \hat e'~|~\varepsilon}
	{\hat e \rightarrow \hat e'~|~\varepsilon} \\[3ex]
	
\infer[\textsc{(E-MultiStep3)}]
	{\hat e \rightarrow^{*} \hat e''~|~\varepsilon_1 \cup \varepsilon_2}
	{\hat e \rightarrow^{*} \hat e'~|~\varepsilon_1 & \hat e' \rightarrow^{*} \hat e''~|~\varepsilon_2}
\end{array}
\]

\vspace{-7pt}
\caption{Multi-step reductions.}
\label{This is the label.}
\end{figure}

\noindent
A multi-step reduction consists of zero\footnote{We permit multi-step reductions of length zero to be consistent with Pierce, who defines multi-step reduction as a reflexive relation\cite[p. 39]{tapl}.} or more single-step reductions. The resulting effect-set is the union of all the single-steps taken.

\subsection{Soundness of $\epscalc$}

\noindent
Our goal is to show $\epscalc$ is sound. This requires an appropriate notion of \textit{effect-soundness}.

\begin{theorem}[Soundness]
If $\hat \Gamma \vdash \hat e_A: \hat \tau_A~\kw{with} \varepsilon_A$ and $\hat e_A$ is not a value, then $e_A \longrightarrow e_B~|~\varepsilon$, where $\hat \Gamma \vdash e_B: \hat \tau_B~\kw{with} \varepsilon_B$ and $\hat \tau_B <: \hat \tau_A$ and $\varepsilon_B \cup \varepsilon \subseteq \varepsilon_A$.
\end{theorem}

\noindent
This definition of soundness is the same as in $\lambda^{\rightarrow}$ but for an extra conclusion: $\varepsilon_B \cup \varepsilon \subseteq \varepsilon_A$. Intuitively, $\varepsilon_A$ is the approximation of what runtime effects the reduction of $\hat e_A$ will incur, $\varepsilon$ is the actual set of effects $\hat e_A$ incurred (at most a singleton because we are working with single-step reduction), and $\varepsilon_B$ is the approximation of what runtime effects the reduction of $\hat e_B$ will incur. Evidently we want $\varepsilon \subseteq \varepsilon_A$; an approximation which accounts for every runtime effect is a sound one. We also want $\varepsilon_B \subseteq \varepsilon_A$, so successive approximations only get better.

The soundness proof takes the standard approach of showing that progress and preservation hold of the calculus. This can be done immediately by observing some properties that follow immediately from the typing rules.

\begin{lemma}[Canonical Forms]
The following are true:
\begin{itemize}
	\setlength\itemsep{-0.7em}
	\item If $\hat \Gamma \vdash \hat v: \hat \tau~\kw{with} \varepsilon$ then $\varepsilon = \varnothing$.
	\item If $\hat \Gamma \vdash \hat v: \{ \bar r \}$ then $\hat v = r$ for some $r \in R$ and $\{ \bar r \} = \{ r \}$.
\end{itemize}
\end{lemma}


\begin{theorem}[Progress]
If $\hat \Gamma \vdash \hat e: \hat \tau~\kw{with} \varepsilon$ and $\hat e$ is not a value, then $\hat e \longrightarrow \hat e'~|~\varepsilon$.
\end{theorem}

\begin{proof} By induction on $\hat \Gamma \vdash \hat e: \hat \tau~\kw{with} \varepsilon$, for $\hat e$ not a value. If the rule is \textsc{$\varepsilon$-Subsumption} it follows by inductive hypothesis. If $\hat e$ has a reducible subexpression then reduce it. Otherwise use one of \textsc{$\varepsilon$-App3}, \textsc{$\varepsilon$-OperCall2}, or \textsc{$\varepsilon$-Import2}.
\end{proof}

\noindent
To prove preservation, we need to know types and effects are preserved under substitution. The substitution lemma gives us this result. It says that if $x$ is bound to a type, and a value $\hat v$ of that type is substituted into $\hat e$, then the type and effect of $\hat e$ remain unchanged. Key to this property is that $\hat v$ is a value, so by canonical forms it cannot introduce effects that weren't already in $\hat e$. Beyond this observation, the proof is routine.

\begin{lemma}[Substitution]
If $\hat \Gamma, x: \hat \tau' \vdash e: \hat \tau~\kw{with} \varepsilon$ and $\hat \Gamma \vdash \hat v: \hat \tau'~\kw{with} \varnothing$ then $\hat \Gamma \vdash [\hat v/x]e: \hat \tau~\kw{with} \varepsilon$.
\end{lemma}

\begin{proof} By induction on $\hat \Gamma, x: \hat \tau' \vdash e: \hat \tau~\kw{with} \varepsilon$.
\end{proof}

\noindent
The tricky case in preservation is when an $\kwa{import}$ expression is resolved. To show the reduction $\import{\varepsilon}{x}{\hat v}{e} \longrightarrow [\hat v/x]\annot{e}{\varepsilon}~|~\varnothing$ preserves soundness requires a few things. First, if $\hat \Gamma \vdash \import{\varepsilon}{x}{\hat v}{e}: \hat \tau_A~\kw{with} \varepsilon_A$, then we need to be able to type the reduced expression in the same context: $\hat \Gamma \vdash [\hat v/x]\annot{e}{\varepsilon}: \hat \tau_B~\kw{with} \varepsilon_B$. To be effect-sound, we need $\varepsilon_B \subseteq \varepsilon_A$. To be type-sound, we need $\hat \tau_B <: \hat \tau_A$. This motivates the next lemma, which relates a typing judgement of $e$ to a typing judgement of $\annot{e}{\varepsilon}$.

\begin{lemma}[Annotation]
If the following are true:

\begin{itemize}
	\setlength\itemsep{-0.7em}
	\item $\hat \Gamma \vdash \hat v : \hat \tau~\kw{with} \varnothing$
	\item $\Gamma, y: \kwa{erase}(\hat \tau) \vdash e: \tau$
	\item $\varepsilon = \kwa{effects}(\hat \tau)$
	\item $\kwa{ho \hyphen safe}(\hat \tau, \varepsilon)$
\end{itemize}

\noindent
Then $\hat \Gamma, \kwa{annot}(\Gamma, \varepsilon), y: \hat \tau \vdash \kwa{annot}(e, \varepsilon) : \kwa{annot}(\tau, \varepsilon)~\kw{with} \varepsilon \cup \kwa{effects}(\kwa{annot}(\Gamma, \varepsilon))$.
\end{lemma}

\begin{proof}
By induction on $\Gamma, y: \kwa{erase}(\hat \tau) \vdash e: \tau$.
\end{proof}

\noindent
The exact formulation of the Annotation lemma is very specific to the premises of \textsc{$\varepsilon$-Import2}, but generalised slightly to accommodate a proof by induction. The generalisation is to allow $e$ to be typed in any context $\Gamma$ with a binding for $y$. We can think of $\Gamma$ as encapsulating the ambient authority exercised by $e$. At the top-level of any program, we will always have $\Gamma = \varnothing$, because the typing judgement \textsc{$\varepsilon$-Import} always types $\kwa{import}$ expressions with just the authority being selected. However, inductively-speaking, there may be ambient capabilities. Consider $(\lambda x: \{ \kwa{File} \}.~\kwa{x.write})~\kwa{File}$. From the perspcetive of $\kwa{x.write}$, $\kwa{File}$ is an ambient capability, and so if we were to inductively apply the Annotation lemma, at this point, $\kwa{File} \in \Gamma$. However, because the code encapsulating $\kwa{x.write}$ selects $\kwa{File}$ by binding it to $x$ in the function declaration, this code is capability-safe.

Proving the annotation lemma requires an additional pair of lemmas, to relate $\hat \tau$ and $\kwa{annot}(\kwa{erase}(\hat \tau), \varepsilon)$.

\begin{lemma}
If $\kwa{effects}(\hat \tau) \subseteq \varepsilon$ and $\kwa{ho \hyphen safe}(\hat \tau, \varepsilon)$ then $\hat \tau <: \kwa{annot}(\kwa{erase}(\hat \tau), \varepsilon)$.
\end{lemma}

\begin{lemma}
If $\kwa{ho \hyphen effects}(\hat \tau) \subseteq \varepsilon$ and $\safe{\hat \tau}{\varepsilon}$ then $\kwa{annot(erase}(\hat \tau), \varepsilon) <: \hat \tau$.
\end{lemma}

\begin{proof}
By simultaneous induction on $\kwa{ho \hyphen safe}$ and $\kwa{safe}$.
\end{proof}

\noindent
There is a close relation between these lemmas and the subtyping rule for functions. In a subtyping relation between functions, the input type is contravariant. Therefore, if $\hat \tau = \hat \tau_1 \rightarrow_{\varepsilon'} \tau_2$ and we have $\hat \tau <: \kwa{annot}(\tau, \varepsilon)$, then we need to know $\kwa{annot}(\tau_1) <: \hat \tau_1$. This is why there are two lemmas, one for each direction.

Armed with the annotation lemma, we are now ready to prove the preservation theorem.

\begin{theorem}[Preservation]
If $\hat \Gamma \vdash \hat e_A: \hat \tau_A~\kw{with} \varepsilon_A$ and $\hat e_A \longrightarrow \hat e_B~|~\varepsilon$, then $\hat \tau_B <: \hat \tau_A$ and $\varepsilon_B \cup \varepsilon \subseteq \varepsilon_A$.
\end{theorem}

\begin{proof} By induction on $\hat \Gamma \vdash \hat e_A: \hat \tau_A~\kw{with} \varepsilon_A$, and then on $\hat e_A \longrightarrow \hat e_B~|~\varepsilon$. \\

\textit{Case:} \textsc{$\varepsilon$-App} Then $e_A = \hat e_1 \hat e_2$ and $\hat e_1: \hat \tau_2 \rightarrow_{\varepsilon} \hat \tau_3~\kw{with} \varepsilon_1$ and $\hat \Gamma \vdash \hat e_2: \hat \tau_2~\kw{with} \varepsilon_2$. If the reduction rule used was \textsc{E-App1} or \textsc{E-App2}, then the result follows by applying the inductive hypothesis to $\hat e_1$ and $\hat e_2$ respectively.

Otherwise the rule used was \textsc{E-App3}. Then $(\lambda x: \hat \tau_2.\hat e)\hat v_2 \longrightarrow [\hat v_2/x]\hat e~|~\varnothing$. By inversion on the typing rule for $\lambda x: \hat \tau_2.\hat e$ we know $\Gamma, x: \hat \tau_2 \vdash \hat e: \hat \tau_3~\kw{with} \varepsilon_3$. By canonical forms, $\varepsilon_2 = \varnothing$ because $\hat e_2 = \hat v_2$ is a value. Then by the substitution lemma, $\hat \Gamma \vdash [\hat v_2/x]\hat e : \hat \tau_3~\kw{with} \varepsilon_3$. By canonical forms, $\varepsilon_1 = \varepsilon_2 = \varnothing = \varepsilon_C$. Therefore $\varepsilon_A = \varepsilon_3 = \varepsilon_B \cup \varepsilon_C$.\\

\textit{Case:} \textsc{$\varepsilon$-OperCall}. Then $e_A = e_1.\pi$ and $\hat \Gamma \vdash e_1 : \{ \bar r \}~\kw{with} \varepsilon_1$. If the reduction rule used was \textsc{E-OperCall1} then the result follows by applying the inductive hypothesis to $\hat e_1$.

Otherwise the reduction rule used was \textsc{E-OperCall2} and $v_1.\pi \longrightarrow \kwa{unit}~|~\{ r.\pi \}$. By canonical forms, $\hat \Gamma \vdash v_1: \kwa{unit}~\kw{with} \{ r.\pi \}$. Also, $\hat \Gamma \vdash \kwa{unit}: \kwa{Unit}~\kw{with} \varnothing$. Then $\tau_B = \tau_A$. Also, $\varepsilon_C \cup \varepsilon_B = \{ r.\pi \} = \varepsilon_A$.\\

\textit{Case:} \textsc{$\varepsilon$-Import}. Then $e_A = \import{\varepsilon}{x}{\hat e}{e}$. If the reduction rule used was \textsc{E-Import1} then the result follows by applying the inductive hypothesis to $\hat e$.

Otherwise $\hat e$ is a value and the reduction used was \textsc{E-Import2}. The following are true:
\begin{enumerate}
	\setlength\itemsep{-0.7em}
	\item $e_A = \kwa{import}(\varepsilon)~x = \hat v~\kw{in} e$
	\item $\hat \Gamma \vdash e_A: \kwa{annot}(\tau, \varepsilon)~\kw{with} \varepsilon \cup \varepsilon_1$
	\item $\kwa{import}(\varepsilon)~x = \hat v~\kw{in} e \longrightarrow [\hat v/x]\kwa{annot}(e, \varepsilon)~|~\varnothing$
	\item $\hat \Gamma \vdash \hat v: \hat \tau~\kw{with} \varnothing$
	\item $\varepsilon = \kwa{effects}(\hat \tau)$
	\item $\kwa{ho \hyphen safe}(\hat \tau, \varepsilon)$
	\item $x: \kwa{erase}(\hat \tau) \vdash e: \tau$
\end{enumerate}

\noindent
Apply the annotation lemma with $\Gamma = \varnothing$ to get $\hat \Gamma, x: \hat \tau \vdash \kwa{annot}(e, \varepsilon): \kwa{annot}(\tau, \varepsilon)~\kw{with} \varepsilon$. From assumption (4) we know $\hat \Gamma \vdash \hat v: \hat \tau~\kw{with} \varnothing$, and so the substitution lemma may be applied, giving $\hat \Gamma \vdash [\hat v/x]\kwa{annot}(e, \varepsilon): \kwa{annot}(\tau, \varepsilon)~\kw{with} \varepsilon$. By canonical forms, $\varepsilon_1 = \varepsilon_C = \varnothing$. Then $\varepsilon_B = \varepsilon = \varepsilon_A \cup \varepsilon_C$. By examination, $\tau_A = \tau_B = \kwa{annot}(\tau, \varepsilon)$.
\end{proof}

\noindent
Our statement of soundness combines the progress and preservation theorems into one.

\begin{theorem}[Soundness]
If $\hat \Gamma \vdash \hat e_A: \hat \tau_A~\kw{with} \varepsilon_A$ and $\hat e_A$ is not a value, then $e_A \longrightarrow e_B~|~\varepsilon$, where $\hat \Gamma \vdash e_B: \hat \tau_B~\kw{with} \varepsilon_B$ and $\hat \tau_B <: \hat \tau_A$ and $\varepsilon_B \cup \varepsilon \subseteq \varepsilon_A$.
\end{theorem}
\begin{proof}
If $\hat e_A$ is not a value then the reduction exists by the progress theorem. The rest follows by the preservation theorem.
\end{proof}

Knowing that single-step reductions are sound, multi-step reductions can straight-forwardly be be shown to also be sound. This is done by inductively applying single-step soundness to the length of the multi-step reduction.

\begin{theorem}[Multi-step Soundness]
If $\hat \Gamma \vdash \hat e_A: \hat \tau_A~\kw{with} \varepsilon_A$ and $e_A \longrightarrow^{*} e_B~|~\varepsilon$, where $\hat \Gamma \vdash e_B: \hat \tau_B~\kw{with} \varepsilon_B$ and $\hat \tau_B <: \hat \tau_A$ and $\varepsilon_B \cup \varepsilon \subseteq \varepsilon_A$.
\end{theorem}

\begin{proof} By induction on the length of the multi-step reduction. If the length is 0 then $e_A = e_B$ and the result holds vacuously. If the length is 1 the result holds by soundness of single-step reductions. if the length is $n+1$, then the first $n$-step reduction is sound by inductive hypothesis and the last step is sound by single-step soundness, so the entire $n+1$-step reduction is sound.
\end{proof}








\chapter{Applications}

In this chapter we show how $\epscalc$ can be used in practice, and show how its rules can enable effect reasoning in existing capability-safe languages. This will take the form of writing a program in a high-level, capability-safe language, translating it to an equivalent $\epscalc$ program, and demonstrating how the rules of $\epscalc$ enable reasoning about the use of effects.

In this section the high-level programs will be written in a version of Wyvern. Wyvern is a pure, object-oriented, capability-safe language. It has a first-class module system, in which modules and objects are treated uniformly. Although $\epscalc$ does not have objects, the example Wyvern programs can be expressed using functions. This does not mean the examples given aren't demonstrating useful and realistic situations --- we simply do not need the added expressiveness given by self-referential objects.

In section 4.1. we discuss how the translation from Wyvern to $\epscalc$ will work, and what simplifying assumptions are made in our examples. This also serves as a gentle introduction to Wyvern's syntax. A variety of scenarios are then explored in 4.2. to show how the rules of $\epscalc$ can help developers in practice.

\section{Translations and Encodings}

Our aim is to develop some notation to help us translate Wyvern programs into $\epscalc$. Our approach will be to encode these additional rules and forms into the base language of $\epscalc$; essentially, to give common patterns and forms a short-hand, so they can be easily named and recalled. This is called \textit{sugaring}. When these derived forms are collapsed into their underlying representation, it is called \textit{desugaring}. We are going to introduce several rules to show a Wyvern program might be considered syntactic sugar for an $\epscalc$ program, and translate examples by desugaring according to our rules.

\subsection{Unit}

$\kwa{Unit}$ is a type inhabited by exactly one value. It conveys the absence of information; in $\epscalc$ an operation call on a resource literal reduces to $\unit$ for this reason. We define $\unit \defn \lambda x: \varnothing. x$. The $\unit$ literal is the same in both annotated and naked code. In annotated code, it has the type $\Unit \defn \varnothing \rightarrow_{\varnothing} \varnothing$, while in naked code it has the type $\Unit \defn \varnothing \rightarrow \varnothing$. While these are technically two seperate types, we will not distinguish between the annotated and naked versions, simply referring to them both as $\Unit$.

Note that $\unit$ is a value, and because $\varnothing$ is uninhabited (there is no empty resource literal), $\unit$ cannot be applied to anything. Furthermore, $\vdash \unit: \Unit~\kw{with} \varnothing$ by \textsc{$\varepsilon$-Abs}, and $\vdash \unit: \Unit$ by \textsc{T-Abs}. This leads to the derived rules in \ref{fig:unit_rules}.

\begin{figure}[h]


\fbox{$\Gamma \vdash e: \tau$} \\
\fbox{$\hat \Gamma \vdash \hat e: \hat \tau~\kw{with} \varepsilon$}


\[
\begin{array}{c}

\infer[\textsc{(T-Unit)}]
	{\Gamma \vdash \unit : \Unit}
	{} ~~~~

\infer[(\textsc{$\varepsilon$-Unit})]
	{\hat \Gamma \vdash \unit : \Unit~\kw{with} \varnothing}
	{}

\end{array}
\]

\caption{Derived $\kwa{Unit}$ rules.}
\label{fig:unit_rules}
\end{figure}

Since $\unit$ represents the absence of information, we also use it as the type when a function either takes no argument, or returns nothing. \ref{fig:unit_sugaring} shows the definition of a Wyvern function which takes no argument and returns nothing, and its corresponding representation in $\epscalc$.

\begin{figure}[h]

\begin{lstlisting}
def method():Unit
   unit
\end{lstlisting}

\begin{lstlisting}
$\lambda$x:Unit. unit
\end{lstlisting}

\caption{Desugaring of functions which take no arguments or return nothing.}
\label{fig:unit_sugaring}
\end{figure}

\subsection{Let}

\noindent
The expression $\letxpr{x}{\hat e_1}{\hat e_2}$ first binds the value $\hat e_1$ to the name $x$ and then evaluates $\hat e_2$. We can generalise by allowing $\hat e_1$ to be a non-value, in which case it must first be reduced to a value. If $\Gamma \vdash \hat e_1: \hat \tau_1$, then $\letxpr{x}{\hat e_1}{\hat e_2} \defn (\lambda x: \hat \tau_1 . \hat e_2) \hat e_1$. Note that if $\hat e_1$ is a non-value, we can reduce the $\kwa{let}$ by \textsc{E-App2}. If $\hat e_1$ is a value, we may apply \textsc{E-App3}, which binds $\hat e_1$ to $x$ in $\hat e_2$. This is fundamentally a lambda application, so it can be typed using \textsc{$\varepsilon$-App} (or \textsc{T-App}, if the terms involved are unlabelled). The new rules in \ref{fig:let_rules} capture these derivations.

\begin{figure}[h]

\fbox{$\Gamma \vdash e: \tau$} \\
\fbox{$\hat \Gamma \vdash \hat e: \hat \tau~\kw{with} \varepsilon$} \\
\fbox{$\hat e \rightarrow \hat e ~|~ \varepsilon$}

\[
\begin{array}{c}

	~~~
	
	\infer[\textsc{($\varepsilon$-Let)}]
	{\Gamma \vdash \letxpr{x}{e_1}{e_2}: \tau_2}
	{\Gamma \vdash e_1: \tau_1 & \Gamma, x: \tau_1 \vdash e_2: \tau_2} \\[2ex]

\infer[\textsc{($\varepsilon$-Let)}]
	{\hat \Gamma \vdash \letxpr{x}{\hat e_1}{\hat e_2} : \hat \tau_2~\kw{with} \varepsilon_1 \cup \varepsilon_2}
	{\hat \Gamma \vdash \hat e_1 : \hat \tau_1~\kw{with} \varepsilon_1 & \hat \Gamma, x: \hat \tau_1 \vdash \hat e_2: \hat \tau_2~\kw{with} \varepsilon_2} \\[2ex]
	
\infer[\textsc{(E-Let1)}]
	{\letxpr{x}{\hat e_1}{\hat e_2} \longrightarrow \letxpr{x}{\hat e_1'}{\hat e_2}~|~\varepsilon_1}
	{\hat e_1 \longrightarrow \hat e_1'~|~\varepsilon_1} \\[2ex]
	
\infer[\textsc{(E-Let2)}]
	{\letxpr{x}{\hat v}{\hat e} \longrightarrow [\hat v/x]\hat e~|~\varnothing}
	{} 

\end{array}
\]

\caption{Derived $\kwa{let}$ rules.}
\label{fig:let_rules}
\end{figure}

$\kwa{let}$ expressions can be used to sequence computations. Intuitively, the $\kwa{let}$ expression simply names the results of the intemediate steps and then ignores them in its body. When we ignore the result of a computation we shall bind it to $\_$ instead of a real name, to suggest the result isn't important and prevent the naming of unused variables. \ref{fig:let_rules} shows how this is done.

\begin{figure}[h]

\begin{lstlisting}
def method(f: {File}):Unit with {File.open, File.write, File.close}
   f.open
   f.write(``hello, world!'')
   f.close
\end{lstlisting}

\begin{lstlisting}
$\lambda$f: {File}.
   let _ = f.open in
   let _ = f.write in
   f.close
\end{lstlisting}

\caption{Desugaring of a sequence of computations.}
\label{fig:let_rules}
\end{figure}

\subsection{Modules and Objects}

Wyvern's modules are first-class and desugar into objects; invoking a method inside a module is no different from invoking an object's method. There are two kinds of modules: pure and resourceful. For our purposes, a pure module is one with no (transitive) authority over any resources, while a resource module has (transitive) authority over some resource. A pure module may still be given a capability, for example by requesting it in a function signature, but it may not possess or capture the capability for longer than the duration of the method call. \ref{fig:wyv_modules} shows an example of two modules, one pure and one resourceful, each declared in a seperate file. Note how pure modules are declared with the $\kwa{module}$ keyword, while resource modules are declared with the $\kwa{resource~module}$ keywords.

\begin{figure}[h]

\begin{lstlisting}
module PureMod

def tick(f: {File}):Unit
   f.append

\end{lstlisting}

\begin{lstlisting}
resource module ResourceMod
require File

def tick():Unit with {File.append}
   File.append
\end{lstlisting}

\caption{Definition of two modules, one pure and the other resourceful.}
\label{fig:wyv_modules}
\end{figure}

Wyvern is capability-safe, so resource modules must be instantiated with the capabilities they require. In \ref{fig:wyv_modules}, $\kwa{ResourceMod}$ requests the use of a $\kwa{File}$ capability, which must be supplied to it from someone already possessing it. Modules are behaving like objects in this way, because they require explicit instantiation. \ref{fig:wyv_module_instantiation} demonstrates how the two modules above would be instantiated and used.

To prevent infinite regress the $\kwa{File}$ must, at some point, be introduced into the program. This happens in a special main module. When the program begins execution, the $\kwa{File}$ capability is passed into the program from the system environment. All these initial capabilities are modelled in $\epscalc$ as resource literals. They are then propagated by the top-level entry point.

\begin{figure}[h]

\begin{lstlisting}
require File
instantiate PureMod
instantiate ResourceMod(File)

def main():Unit
   PureMod.tick(File)
   ResourceMod.tick()
\end{lstlisting}

\caption{Definition of two modules, one pure and the other resourceful.}
\label{fig:wyv_module_instantiation}
\end{figure}

Before explaining our translation of Wyvern programs into $\epscalc$, we must explain several simplifications made in all of our examples which enable our particular desugaring.

Objects are only ever used in the form of modules. Modules only ever contain functions and other modules, and have no mutable fields. The examples contain no recursion or self-reference, including a module invoking its own functions. Modules will not reference each other cyclically. Lastly, modules only contain one function definition. Despite these simplifications, the chosen examples will highlight the essential aspects of $\epscalc$.

Because modules do not exercise self-reference and only contain one function definition, they will be modelled as functions in $\epscalc$. Applying this function will be equivalent to applying the single function definition in the module.

A collection of modules is desugared into $\epscalc$ as follows. First, a sequence of let-bindings are used to name constructor functions which, when given the capabilities requested by a module, will return an instance of the module. If the module does not require any capabilities then it will take $\Unit$ as its argument. The constructor function for $\kwa{M}$ is called $\kwa{MakeM}$. A function is then defined which represents the $\kwa{main}$ function, which is the entry point into the program. This $\kwa{main}$ function will instantiate all the modules by invoking the constructor functions, and then execute the body of code in main. Finally, the main function is invoked with the primitive capabilities it needs.

To demonstrate this process, \ref{fig:wyv_tutorial_desugaring} shows how the examples above desugar. Lines 1-3 define the constructor for $\kwa{PureMod}$; since $\kwa{PureMod}$ requires no capabilities, the constructor takes $\Unit$ as an argument on line 2. Lines 6-8 define the constructor for $\kwa{ResourceMod}$; it requires a $\kwa{File}$ capability, so the constructor takes $\kwa{\{File\}}$ as its input type on line 7. The entry point to the program is defiend on lines 11-15, which invokes the constructors and then runs the body of the $\kwa{main}$ method. Lastly, line 17 starts everything off by invoking $\kwa{Main}$ with the initial set of capabilities, which in this case is just $\kwa{File}$.

\begin{figure}[h]

\begin{lstlisting}
let MakePureMod =
   $\lambda$x:Unit.
      $\lambda$f:{File}. f.append
in

let MakeResourceMod =
   $\lambda$f:{File}.
      $\lambda$x:Unit. f.append
in

let MakeMain =
   $\lambda$f:{File}.
      $\lambda$x: Unit.
         let PureMod = (MakePureMod unit) in
         let ResourceMod = (MakeResourceMod f) in
         let _ = (PureMod f) in (ResourceMod unit) in

(MakeMain File) unit
\end{lstlisting}

\caption{Desugaring of $\kwa{PureMod}$ and $\kwa{ResourceMod}$ into $\epscalc$.}
\label{fig:wyv_tutorial_desugaring}
\end{figure}




\section{Examples}

In this section we present several scenarios where a developer may be forced to reason about the use of effects, and show how the capability-based reasoning of effects can assist them. In some scenarios, a program exhibits a certain nefarious behaviour, in which case capability-based reasoning can automatically detect this behaviour and reject it. Other scenarios are more qualitative; perhaps a developer must make a design choice and none of the alternatives \textit{prima facie} stand out. In such cases, capability-based reasoning might supply them with useful information, enabling tehm to make more informed design choices. We also hope to convince the reader that the rules of $\epscalc$ have practical worth, and could be used to enrich existing capability-safe languages.

The format of each section is as follows. A program is introduced which exhibits some bad behaviour or demonstrates a particular story about software development. The language used is \textit{Wyvern}; a pure, object-oriented, capability-safe language with first-class modules-as-objects. We show how the Wyvern program can be written as a corresponding $\epscalc$ program and sketch a derivation showing how the rules of $\epscalc$ and a sketch a derivation showing how the rules of $\epscalc$ would solve the relevant problem.

We take some shortcuts with the translation of Wyvern into $\epscalc$. Our ``objects'' are really records of functions; the difference between the two is self-reference. The particular examples chosen do not require self-reference, so no important properties are lost by treating Wyvern objects as records.

\subsection{Unannotated Client}

In Figure \ref{fig:eg1} an annotated $\kwa{Logger}$ module provides its client the ability to append to a particular $\kwa{File}$ resource. $\kwa{File}$ is a primitive capability passed into the program when it begins execution, perhaps from the system environment or a virtual machine. The $\kwa{Logger}$ module presents a controlled subset of operations on the $\kwa{File}$ viz. $\kwa{File.append}$. The program consists of an unannotated client which instantiates the $\kwa{Logger}$ module with the capability it selects ($\kwa{File}$) and then attempts to log.

If the client code is executed, what effects will it have? The answer is not immediately clear from the client's source-code, but a capability-based argument goes as follows: because the client code can typecheck needing only $\kwa{Logger}$, then whatever effects presented by $\kwa{Logger}$ are an upper-bound on the effects of the client.

\begin{figure}[h]

\begin{lstlisting}
resource module Logger
require File

def log(): Unit with File.append =
    File.append(``message logged'')
\end{lstlisting}

\begin{lstlisting}
module Client
require Logger

def run(): Unit =
   Logger.log()
\end{lstlisting}

\begin{lstlisting}
resource module Main
require File
instantiate Logger(File)
instantiate Client(Logger)

Client.run()
\end{lstlisting}

\caption{A $\kwa{logger}$ client doesn't need to add effect labels; these can be inferred.}
\label{fig:eg1}
\end{figure}

The desugaring first creates two functions, $\kwa{MakeLogger}$ and $\kwa{MakeClient}$, which instantiate the $\kwa{Logger}$ and $\kwa{Client}$ modules; the client code is treated as an implicit module. Lines 1-4 define a function which, given a $\kwa{File}$, returns a record containing a single $\kwa{log}$ function. Lines 6-8 define a function which, given a $\kwa{Logger}$, returns the unannotated client code, wrapped inside an $\kwa{import}$ expression selecting its needed authority. Lines 10-14 are the meat of the program; this function, when given a $\kwa{File}$ capability, creates the modules and then runs the client code. Program execution begins on line $16$, where $\kwa{Main}$ is given its initial set of capabilities --- which, in this case, is just $\kwa{File}$.

\begin{figure}[h]

\begin{lstlisting}
let MakeLogger =
   ($\lambda$f: File.
      $\lambda$x: Unit. f.append) in
          
let MakeClient =
   ($\lambda$logger: Logger.
      import(File.append) logger = logger in
         $\lambda$x: Unit. logger unit) in
          
let MakeMain =
   ($\lambda$f: File.
      $\lambda$x: Unit.
         let LoggerModule = MakeLogger f in
         let ClientModule = MakeClient LoggerModule in
         ClientModule unit) in

(MakeMain File) unit
\end{lstlisting}

\caption{Desugared version of Figure \ref{fig:eg1}.}
\label{fig:eg1_desugared}
\end{figure}

The interesting part  is on lines 7-8, where the unannotated code selects $\kwa{File.append}$ as its authority. This is exactly the effects of the logger, i.e. $\kwa{effects}(\Unit \rightarrow_{\kwa{File.append}} \Unit) = \{ \kwa{File.append} \}$. The code also satisfies the higher-order safety predicates, and the body of the $\kwa{import}$ expression typechecks in the empty context. Therefore, the unannotated code typechecks by \textsc{$\varepsilon$-Import}.

In such a small example the client could simply inspect the source code of $\kwa{Logger}$ to determine what effects it might have. Several situations can make this impossible or tedious. First, the manual approach loses efficiency when the system involves many modules of large size across code-ownership boundaries; capability-based reasoning tells you automatically. Second, the source code of $\kwa{Logger}$ might be obfuscated or unavailable, and the only useful information is that given by its signature. Lastly, the client may not care about effects in this situation; the program may be a quick-and-dirty throwaway, in which case it is nice that the capability-based reasoning still accepts the client code without annotations..

\subsection{API Violation}

Figure \ref{fig:eg2} inverts the roles of the last scenario: now, the annotated $\kwa{Client}$ wants to use the unannotated $\kwa{Logger}$. The $\kwa{Logger}$ module captures the $\kwa{File}$ capability, and exposes a single function $\kwa{log}$ with the $\kwa{File.append}$ effect. However, the $\kwa{Client}$ has a function $\kwa{run}$ which executes $\kwa{Logger.log}$, incurring the effect of $\kwa{File.append}$, but declares its set of effects as $\varnothing$. The implementation and the signature of $\kwa{Client.run}$ are inconsistent --- does the type system recognise this?

\begin{figure}[h]

\begin{lstlisting}
resource module Logger
require File

def log(): Unit =
    File.append(``message logged'')
\end{lstlisting}

\begin{lstlisting}
resource module Client
require Logger

def run(): Unit with $\varnothing$ =
   Logger.log()
\end{lstlisting}

\begin{lstlisting}
resource module Main
require File
instantiate Logger(File)
instantiate Client(Logger)

Client.run()
\end{lstlisting}

\caption{The unlabelled code in $\kwa{Logger}$ exercises authority exceeding that selected by $\kwa{Client}$.}
\label{fig:eg2}
\end{figure}

A desugaring is given in Figure \ref{fig:eg2_desugared}. Lines 1-3 define the function which instantiates the $\kwa{Logger}$ module. Lines 5-8 define the function which instantiates the $\kwa{Client}$ module. Lines 10-15 define the function which instantiates the $\kwa{Main}$ module. Line 17 initiates the program, supplying $\kwa{File}$ to the $\kwa{Main}$ module and invoking its main method. On lines 3-4, the unannotated code is modelled using an $\kwa{import}$ expression which selects $\varnothing$ as its authority. So far this coheres to the expectations of $\kwa{Client}$. However, \textsc{$\varepsilon$-Import} cannot be applied because the name being bound, $f$, has the type $\{ \File \}$, and $\fx{\{ \File \}} = \{ \kwa{File}.* \}$, which is inconsistent with the declared effects $\varnothing$.

\begin{figure}[h]

\begin{lstlisting}
let MakeLogger =
   ($\lambda$f: File.
      import($\varnothing$) f = f in
         $\lambda$x: Unit. f.append) in

let MakeClient =
	($\lambda$logger: Logger.
	   $\lambda$x: Unit. logger unit) in

let MakeMain =
   ($\lambda$f: File.
      let LoggerModule = MakeLogger f in
      let ClientModule = MakeClient LoggerModule in
      ClientModule unit) in

(MakeMain File) unit
\end{lstlisting}

\caption{Desugared version of Figure \ref{fig:eg2}.}
\label{fig:eg2_desugared}
\end{figure}

The only way for this to typecheck would be to annotate $\kwa{Client.run}$ as having every effect on $\File$. This demonstrates how the effect-system of $\epscalc$ approximates unlabelled code: it simply considers it as having every effect which could be incurred on those resources in scope, which here is $\kwa{File}.*$.

\subsection{API Violation}

Figure \ref{fig:eg3} is a variation of the last example, but now $\kwa{Logger.log}$ is passed the $\kwa{File}$ capability, rather than possessing it. $\kwa{Logger.log}$ still incurs $\kwa{File.append}$ inside unannotated code, which causes the implementation of $\kwa{Client.run}$ to violate its signature.

\begin{figure}[h]

\begin{lstlisting}
module Logger

def log(f: {File}): Unit
    f.append(``message logged'')
\end{lstlisting}

\begin{lstlisting}
module Client
instantiate Logger(File)

def run(f: {File}): Unit with $\varnothing$
   Logger.log(File)
   
\end{lstlisting}

A desugared version is given in Figure \ref{fig:eg3}, which is largely the same as in the previous example, except 

\begin{lstlisting}
resource module Main
require File
instantiate Client

Client.run(File)
\end{lstlisting}

\caption{The unlabelled code in $\kwa{Logger}$ exercises authority exceeding that selected by $\kwa{Client}$.}
\label{fig:eg3}
\end{figure}




\begin{figure}[h]

\begin{lstlisting}
let MakeClient =
	($\lambda$x: Unit.
	   let MakeLogger =
	      ($\lambda$x: Unit.
	         import($\varnothing$) x=x in
	            $\lambda$f: {File}. f.append) in
      let LoggerModule = MakeLogger unit in
      $\lambda$f: {File}. LoggerModule f) in
	
let MakeMain =
   ($\lambda$f: {File}.
      $\lambda$x: Unit.
         let ClientModule = MakeClient unit in
         ClientModule f) in

(MakeMain File) unit
\end{lstlisting}

\caption{Desugared version of \ref{fig:eg3}.}
\label{fig:eg3_desugared}
\end{figure}


\subsection{API Violation}


\begin{figure}[h]

\begin{lstlisting}
resource module Logger
require File

def log(): Unit with {File.append, File.write} =
    File.append(``message logged'')
    File.write(``message written'')
\end{lstlisting}

\begin{lstlisting}
module Client

def run(l: Logger): Unit with {File.append} =
    l.log()
\end{lstlisting}

\begin{lstlisting}
resource module Main
require File
instantiate Logger(File)

Client.run(Logger)
\end{lstlisting}

\caption{This won't type because of a mismatch between the effects of $\kwa{Client}$ and the effects of $\kwa{Logger}$.}
\label{fig:eg4}
\end{figure}


\begin{figure}[h]

\begin{lstlisting}
let MakeLogger =
   ($\lambda$f: File.
      $\lambda$x: Unit. let _ = f.append in f.write) in
           
let MakeClient =
   ($\lambda$x: Unit.
      $\lambda$logger: Logger. logger unit) in
                  
let MakeMain =
   ($\lambda$f: File.
      $\lambda$x: Unit.
         let LoggerModule = MakeLogger f in
         let ClientModule = MakeClient unit in
         ClientModule.run LoggerModule) in

(MakeMain File) unit
\end{lstlisting}

\caption{Desugared version of Figure \ref{fig:eg4}.}
\label{fig:eg4_desugared}
\end{figure}


\subsection{Hidden Authority}

\begin{figure}[h]

\begin{lstlisting}
module Malicious

def stealData(f: {File}):Unit with {File.read} =
   f.read
\end{lstlisting}

\begin{lstlisting}
module Plugin
instantiate Malicious

def run(f: {File}): Unit with $\varnothing$ =
   Malicious.stealData(f)
\end{lstlisting}

\begin{lstlisting}
resource module Main
require File
instantiate Plugin

Plugin.run(File)
\end{lstlisting}

\caption{The $\kwa{Main}$ module transitively invokes a $\kwa{File.read}$ effect, violating its selected authority.}
\label{fig:eg5}
\end{figure}

\begin{figure}[h]

\begin{lstlisting}
let MakePlugin =
   ($\lambda$x: Unit.
      let MakeMalicious =
         ($\lambda$x: Unit. $\lambda$f: {File}. f.read) in
      let MaliciousModule = (MakeMalicious unit) in
      $\lambda$f: {File}. MaliciousModule f) in
      
let MakeMain =
   ($\lambda$f: File.
      $\lambda$x: Unit.
         let PluginModule = MakePlugin unit in
         PluginModule.run f) in

(MakeMain File) unit
\end{lstlisting}

\caption{Desugared version of Figure \ref{fig:eg5}.}
\label{fig:eg5_desugared}
\end{figure}

\subsection{Hidden Authority 2} 

\begin{figure}[h]

\begin{lstlisting}
module Malicious

def stealData(f: {File}):Unit =
   f.read
\end{lstlisting}

\begin{lstlisting}
module Plugin
instantiate Malicious

def run(f: {File}): Unit with $\varnothing$ =
   Malicious.stealData(f)
\end{lstlisting}

\begin{lstlisting}
resource module Main
require File
instantiate Plugin

Plugin.run(File)
\end{lstlisting}

\caption{The transitive invocation of $\kwa{File.read}$ now happens inside unannotated code, but the type system will still reject this program.}
\label{fig:eg6}
\end{figure}

\begin{figure}[h]

\begin{lstlisting}
let MakePlugin =
   ($\lambda$x: Unit.
      let MakeMalicious =
         ($\lambda$x: Unit.
            import($\varnothing$) x=x in
               $\lambda$f: {File}. f.read) in
      let MaliciousModule = (MakeMalicious unit) in
      $\lambda$f: {File}. MaliciousModule f) in
      
let MakeMain =
   ($\lambda$f: File.
      $\lambda$x: Unit.
         let PluginModule = MakePlugin unit in
         PluginModule.run f) in

(MakeMain File) unit
\end{lstlisting}

\caption{Desugared version of Figure \ref{fig:eg6}.}
\label{fig:eg6_desugared}
\end{figure}

\subsection{Hidden Authority 2} 

\begin{figure}[h]

\begin{lstlisting}
module Malicious

def log(f: Unit $\rightarrow$ Unit):Unit
   f()
\end{lstlisting}

\begin{lstlisting}
module Plugin
instantiate Malicious

def run(f: {File}): Unit with $\varnothing$
   Malicious.log($\lambda$x:Unit. f.read)
\end{lstlisting}

\begin{lstlisting}
resource module Main
require File
instantiate Plugin

Plugin.run(File)
\end{lstlisting}

\caption{The transitive invocation of $\kwa{File.read}$ happens when the unannotated code executes the function given to it.}
\label{This is the label.}
\end{figure}


\subsection{Resource Leak}

\begin{figure}[h]

\begin{lstlisting}
module Malicious

def log(f: Unit $\rightarrow$ File):Unit
   f().read
\end{lstlisting}

\begin{lstlisting}
module Plugin
instantiate Malicious

def run(f: {File}): Unit with $\varnothing$
   Malicious.log($\lambda$x:Unit. f)
\end{lstlisting}

\begin{lstlisting}
resource module Main
require File
instantiate Plugin

Plugin.run(File)
\end{lstlisting}

\caption{A resource leak allows $\kwa{Malicious}$ to gain access to the $\kwa{File}$ capability directly.}
\label{This is the label.}
\end{figure}


\chapter{Conclusions}

In this report we have looked at $\opercalc$, which extends $\stlc$ with a notion of primitive capabilities and their effects. $\kwa{OC}$ programs are fully annotated with their effects; relaxing this requirement, we obtained $\epscalc$, which allows unannotated code to be nested inside annotated code with a new $\kwa{construct}$. The capability-safe design of $\epscalc$ allows us to make a safe inference about the effects of the unannotated code by inspecting what capabilities are passed to it by its annotated surroundings. The result is a sound, lightweight effect-system. Such an approach allows code to be incrementally annotated, giving developers a balance between safety and convenience and alleviating the verbosity that has discouraged the adoption of effect systems \cite{rytz2012}.

\section{Related Work}

Capabilities were introduced by Dennis and Van Horn as a way to control which processes in an operating system had permission to access certain parts of memory \cite{dennis66}. An \textit{access control list} would declare what permissions a program may exercise. These early ideas are considerably different to the object capability model introduced by Mark Miller \cite{miller06}, which imposes constraint on how permissions can proliferate. Maffeis et. al. formalised the notion of a capability-safe language and showed that a subset of Caja (a Javascript implementation) is capability-safe \cite{maffeis10}. Miller's model has also has been applied to more heavyweight formal systems: Drossopoulou et. al. combined Hoare logic with capabilities to determine whether components of a system can be trusted \cite{drossopoulou07}. Other capability-safe languages include Wyvern \cite{nistor13} and Newspeak \cite{bracha10}.

The original effect system by Lucassen and Gifford was used to determine if two expressions could safely run in parallel \cite{lucassen88}. Subsequent applications include determining what functions a program might invoke \cite{tang94} and what regions in memory might be accessed or updated during execution \cite{talpin94}. In these systems, ``effects'' are performed upon ``regions''; in ours, ``operations'' are performed upon ``resources''. An important difference in $\epscalc$ is the distinction between annotated and unannotated code: only the former will type-and-effect-check. This approach allows for an effect discipline to be incrementally imposed on an otherwise effect-unconscious system.

Fengyun Liu has also combined capability-safety and effect systems, with applications to purity analysis in Scala \cite{liu16}. If a function is known to be pure then optimisations such as inlining and parallelisation can be made. Liu's work is motivated by achieving such optimisations for Scala compilers. It distinguishes between free and stoic functions: free functions may exercise ambient authority whereas stoic functions may not. Stoic functions are therefore capability-safe pockets whose purity can be determined by examining what capabilities are passed into them. Liu's System F-Impure does not track effects, whereas $\epscalc$, by distinguishing between regular effects and higher-order effects, gives more fine-grained detail about what a piece of code will do when executed.

The systems by Lucassen and Liu have effect polymorphism, whereas $\epscalc$ does not. Another capability-safe effect-system is the one by Devriese et. al., who use effect polymorphism and possible world semantics to guarantee behavioural invariants on data structures \cite{devriese16}. Our approach is not as expressive, based only on a topological analysis of how capabilities can be passed around the program, but the formalism is much more lightweight.

\section{Conclusions and Future Work}

In this report we have explored, through $\epscalc$, how capability-safety enables a lightweight, incremental effect system that can help developers reason about the safety of components in a program.

We have seen that approximating some unannotated code by every effect it exercises can lead to overapproximations. Recall the example from Subsection 4.2.4.: $\hat e = \import{\kwa{File.*}}{x}{\unit}{\lambda f: \kwa{ \{ \File \}}.~\kwa{f.write}}$. The unannotated code returns a function which, when given a $\kwa{File}$, will write to it. The approximation for this code is $\kwa{File.*}$, but defining a function literal incurs no effects. Perhaps by a refinement of the rules for higher-order effects, the rules of $\epscalc$ might be able to do better.

The current theory contains no effect polymorphism. This would allow the type of a function to be parameterised by a set of effects. For an example of such a function, consider $\kwa{map}$: given a function $f$ and a list $l$, map applies $f$ to every element of $l$ to produce a new list $l'$. The effects of $\kwa{map}$ are dependent on the effects of $f$. The only way to define $\kwa{map}$ in $\epscalc$ would be to conservatively approximate it as having every effect, in which case all precision has been lost. A polymorphic effect system which considers the type of $\kwa{map}$ as being parameterised by a set of effects could give more meaningful approximations.

Lastly, the ideas in this paper might be extended and developed to the point where they can be used in real-world situations. Implementing these ideas in an existing, general-purpose language would do much towards that end.

%Many believe in the real and practical value of the object capability model, but we do not fully understand its formal benefits. 







\chapter{Appendix}

\begin{lemma}[Canonical Forms]
The following are true:
\begin{itemize}
	\setlength\itemsep{-0.7em}
	\item If $\hat \Gamma \vdash \hat v: \hat \tau~\kw{with} \varepsilon$ then $\varepsilon = \varnothing$.
	\item If $\hat \Gamma \vdash \hat v: \{ \bar r \}$ then $\hat v = r$ for some $r \in R$ and $\{ \bar r \} = \{ r \}$.
\end{itemize}
\end{lemma}


\begin{theorem}[Progress]
If $\hat \Gamma \vdash \hat e: \hat \tau~\kw{with} \varepsilon$ and $\hat e$ is not a value, then $\hat e \longrightarrow \hat e'~|~\varepsilon$.
\end{theorem}

\begin{proof} By induction on $\hat \Gamma \vdash \hat e: \hat \tau~\kw{with} \varepsilon$, for $\hat e$ not a value.

Case: \textsc{$\varepsilon$-App}. Then $\hat e = \hat e_1~\hat e_2$. If $\hat e_1$ is a non-value, then $\hat e_1~\hat e_2 \longrightarrow \hat e_1'~\hat e_2$ by \textsc{E-App1}. If $\hat e_1 = \hat v_1$ is a value and $\hat e_2$ is a non-value, then $\hat e_1~\hat e_2 \longrightarrow \hat v_1~\hat e_2'$ by \textsc{E-App2}. Otherwise $\hat e_1$ and $\hat e_2$ are both values. By inversion, $\hat e_1 = \lambda x: \hat \tau . \hat e$, so $(\lambda x: \hat \tau. \hat e) \hat v_2 \longrightarrow [\hat v_2/x]~|~\varnothing$ by \textsc{E-App3}.

Case: \textsc{$\varepsilon$-Oper}. Then $\hat e = \hat e_1.\pi$. If $\hat e_1$ is a non-value, then $\hat e_1.\pi \longrightarrow \hat e_1'.\pi~|~\varepsilon_1$ by \textsc{E-OperCall1}. Otherwise $\hat e_1 = \hat v_1$ is a value. By canonical forms, $\hat v_1 = r$ and $\hat \Gamma \vdash v_1: \{ r \}~\kw{with} \varnothing$. Then $r.\pi \longrightarrow \kwa{unit}~|~\{ r.\pi \}$ by \textsc{E-OperCall2}.

Case: \textsc{$\varepsilon$-Subsume}. Then $\hat \Gamma \vdash \hat e: \hat \tau'~\kw{with} \varepsilon'$. By inversion, $\hat \Gamma \vdash \hat e: \tau~\kw{with} \varepsilon$, where $\tau' <: \tau$ and $\varepsilon' \subseteq \varepsilon$. These are subderivations, so the result holds by inductive assumption.

Case: \textsc{$\varepsilon$-Module}. Then $\hat e = \import{\varepsilon}{x}{\hat e'}{e}$. If $\hat e'$ is a non-value then $\import{\varepsilon}{x}{\hat e'}{e} \longrightarrow \import{\varepsilon}{x}{\hat e''}{e}~|~\varepsilon'$ by \textsc{E-Module1}. Otherwise $\hat e' = \hat v$ is a value. Then $\import{\varepsilon}{x}{\hat v}{e} \longrightarrow [\hat v/x]\kwa{annot}(e, \varepsilon)~|~\varnothing$ by \textsc{E-Module2}.
\end{proof}



\begin{lemma}[Substitution]
If $\hat \Gamma, x: \hat \tau' \vdash e: \hat \tau~\kw{with} \varepsilon$ and $\hat \Gamma \vdash \hat v: \hat \tau'~\kw{with} \varnothing$ then $\hat \Gamma \vdash [\hat v/x]e: \hat \tau~\kw{with} \varepsilon$.
\end{lemma}

\begin{proof} By induction on $\hat \Gamma, x: \hat \tau' \vdash e: \hat \tau~\kw{with} \varepsilon$. \\

\textit{Case}: \textsc{$\varepsilon$-Var}. Then $\hat e = y$ and either $y = x$ or $y \neq x$. If $y \neq x$. Then $[\hat v/x]y = y$ and $\hat \Gamma \vdash y: \hat \tau~\kw{with} \varnothing$. Therefore $\hat \Gamma \vdash [\hat v/x]y: \hat \tau~\kw{with} \varnothing$. Otherwise $y = x$. By inversion on \textsc{$\varepsilon$-Var}, the typing judgement from the theorem assumption is $\hat \Gamma, x: \hat \tau' \vdash x: \hat \tau'~\kw{with} \varnothing$. Since $[\hat v/x]y = \hat v$, and by assumption $\hat \Gamma \vdash \hat v: \hat \tau'~\kw{with} \varnothing$, then $\hat \Gamma \vdash [\hat v/x]x: \hat \tau'~\kw{with} \varnothing$. \\

\textit{Case}: \textsc{$\varepsilon$-Resource}. Because $\hat e = r$ is a resource literal then $\hat \Gamma \vdash r: \hat \tau~\kw{with} \varnothing$ by canonical forms. By definition $[\hat v/x]r = r$, so $\hat \Gamma \vdash [\hat v/x]r: \hat \tau~\kw{with} \varnothing$. \\

\textit{Case:} \textsc{$\varepsilon$-App} By inversion we know $\hat \Gamma, x: \hat \tau' \vdash \hat e_1:\hat \tau_2 \rightarrow_{\varepsilon_3} \hat \tau_3~\kw{with} \varepsilon_A$ and $\hat \Gamma, x: \hat \tau' \vdash \hat e_2: \hat \tau_2~\kw{with} \varepsilon_B$, where $\varepsilon = \varepsilon_A \cup \varepsilon_B \cup \varepsilon_3$ and $\hat \tau = \hat \tau_3$. By inductive assumption, $\hat \Gamma \vdash [\hat v/x]\hat e_1: \hat \tau_2 \rightarrow_{\varepsilon_3} \hat \tau_3~\kw{with} \varepsilon_A$ and $\hat \Gamma \vdash [\hat v/x]\hat e_2: \hat \tau_2~\kw{with} \varepsilon_B$. By \textsc{$\varepsilon$-App} we have $\hat \Gamma \vdash ([\hat v/x]\hat e_1) ([\hat v/x]\hat e_2) : \hat \tau_3~\kw{with} \varepsilon_A \cup \varepsilon_B \cup \varepsilon_3$. By simplifying and applying the definition of $\kwa{substitution}$, this is the same as $\hat \Gamma \vdash [\hat v/x](\hat e_1 \hat e_2): \hat \tau~\kw{with} \varepsilon$. \\

\textit{Case:} \textsc{$\varepsilon$-OperCall} By inversion we know $\hat \Gamma, x: \hat \tau' \vdash \hat e_1: \{ \bar r \}~\kw{with} \varepsilon_1$, where $\varepsilon = \varepsilon_1 \cup \{ r.\pi \mid r.\pi \in \bar r \times \Pi \}$ and $\hat \tau = \{ \bar r \}$. By applying the inductive assumption, $\hat \Gamma \vdash [\hat v/x]\hat e_1 : \{ \bar r \} ~\kw{with} \varepsilon_1$. Then by \textsc{$\varepsilon$-OperCall}, $\hat \Gamma \vdash ([\hat v/x]\hat e_1).\pi: \{ \bar r \}~\kw{with} \varepsilon_1 \cup \{ r.\pi \mid r.\pi \in \bar r \times \Pi \}$. By simplifying and applying the definition of $\kwa{substitution}$, this is the same as $\hat \Gamma \vdash [\hat v/x](\hat e_1.\pi): \hat \tau~\kw{with} \varepsilon$.\\

\textit{Case:} \textsc{$\varepsilon$-Subsume} By inversion we know $\hat \Gamma, x: \hat \tau' \vdash \hat e: \hat \tau_2~\kw{with} \varepsilon_2$, where $\hat \tau_2 <: \hat \tau$ and $\varepsilon_2 \subseteq \varepsilon$. By inductive hypothesis, $\hat \Gamma \vdash [\hat v/x]\hat e: \hat \tau_2~\kw{with} \varepsilon_2$. Then by \textsc{$\varepsilon$-Subsume} we get $\hat \Gamma \vdash [\hat v/x]\hat e: \hat \tau~\kw{with} \varepsilon$. \\

\textit{Case:} \textsc{$\varepsilon$-Module} Then $\hat \Gamma, x: \hat \tau' \vdash \import : \kwa{annot}(\tau, \varepsilon)~\kw{with} \varepsilon \cup \varepsilon_1$. By inversion we know $\hat \Gamma, x: \hat \tau' \vdash \hat e: \hat \tau_1~\kw{with} \varepsilon_1$. By inductive assumption, $\hat \Gamma \vdash [\hat v/x]\hat e: \hat \tau_1~\kw{with} \varepsilon_1$. Then by \textsc{$\varepsilon$-Module} we have $\hat \Gamma \vdash \import: \kwa{annot}(\tau, \varepsilon)~\kw{with} \varepsilon \cup \varepsilon_1$.
\end{proof}






%\bibliographystyle{iee2etr}
\bibliographystyle{acm}
\bibliography{myrefs}


\end{document}
