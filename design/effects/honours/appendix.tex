\begin{appendix}

\chapter{$\opercalc$ Proofs}

\begin{lemma}[Canonical Forms]
The following are true:
\begin{itemize}
	\setlength\itemsep{-0.7em}
	\item If $ \Gamma \vdash v: \tau~\kw{with} \varepsilon$ then $\varepsilon = \varnothing$.
	\item If $ \Gamma \vdash v: \{ \bar r \}$ then $ v = r$ for some $r \in R$ and $\{ \bar r \} = \{ r \}$.
\end{itemize}
\end{lemma}

\hrulefill


\begin{theorem}[Progress]
If $ \Gamma \vdash  e:  \tau~\kw{with} \varepsilon$ and $ e$ is not a value, then $ e \longrightarrow  e'~|~\varepsilon$.
\end{theorem}

\begin{proof} By induction on $ \Gamma \vdash  e:  \tau~\kw{with} \varepsilon$, for $ e$ not a value.\\

Case: \textsc{$\varepsilon$-App}. Then $ e =  e_1~ e_2$. If $ e_1$ is a non-value, then $ e_1~ e_2 \longrightarrow  e_1'~ e_2$ by \textsc{E-App1}. If $ e_1 =  v_1$ is a value and $ e_2$ is a non-value, then $ e_1~ e_2 \longrightarrow  v_1~ e_2'$ by \textsc{E-App2}. Otherwise $ e_1$ and $ e_2$ are both values. By inversion, $ e_1 = \lambda x:  \tau .  e$, so $(\lambda x:  \tau.  e)  v_2 \longrightarrow [ v_2/x]~|~\varnothing$ by \textsc{E-App3}.\\

Case: \textsc{$\varepsilon$-Oper}. Then $ e =  e_1.\pi$. If $ e_1$ is a non-value, then $ e_1.\pi \longrightarrow  e_1'.\pi~|~\varepsilon_1$ by \textsc{E-OperCall1}. Otherwise $ e_1 =  v_1$ is a value. By canonical forms, $ v_1 = r$ and $ \Gamma \vdash v_1: \{ r \}~\kw{with} \varnothing$. Then $r.\pi \longrightarrow \kwa{unit}~|~\{ r.\pi \}$ by \textsc{E-OperCall2}.\\

Case: \textsc{$\varepsilon$-Subsume}. Then $ \Gamma \vdash  e:  \tau'~\kw{with} \varepsilon'$. By inversion, $ \Gamma \vdash  e: \tau~\kw{with} \varepsilon$, where $\tau' <: \tau$ and $\varepsilon' \subseteq \varepsilon$. These are subderivations, so the result holds by inductive assumption.\\
\end{proof}

\hrulefill

\begin{lemma}[Substitution]
If $\Gamma, x: \tau' \vdash e: \tau~\kw{with} \varepsilon$ and $ \Gamma \vdash  v: \tau'~\kw{with} \varnothing$ then $ \Gamma \vdash [ v/x]e:  \tau~\kw{with} \varepsilon$.
\end{lemma}

\begin{proof} By induction on $ \Gamma, x:  \tau' \vdash e:  \tau~\kw{with} \varepsilon$. \\

\textit{Case}: \textsc{$\varepsilon$-Var}. Then $ e = y$ and either $y = x$ or $y \neq x$. If $y \neq x$. Then $[ v/x]y = y$ and $ \Gamma \vdash y:  \tau~\kw{with} \varnothing$. Therefore $ \Gamma \vdash [ v/x]y:  \tau~\kw{with} \varnothing$. Otherwise $y = x$. By inversion on \textsc{$\varepsilon$-Var}, the typing judgement from the theorem assumption is $ \Gamma, x:  \tau' \vdash x:  \tau'~\kw{with} \varnothing$. Since $[ v/x]y =  v$, and by assumption $ \Gamma \vdash  v:  \tau'~\kw{with} \varnothing$, then $ \Gamma \vdash [ v/x]x:  \tau'~\kw{with} \varnothing$. \\

\textit{Case}: \textsc{$\varepsilon$-Resource}. Because $ e = r$ is a resource literal then $ \Gamma \vdash r:  \tau~\kw{with} \varnothing$ by canonical forms. By definition $[ v/x]r = r$, so $ \Gamma \vdash [ v/x]r:  \tau~\kw{with} \varnothing$. \\

\textit{Case:} \textsc{$\varepsilon$-App} By inversion we know $ \Gamma, x:  \tau' \vdash  e_1: \tau_2 \rightarrow_{\varepsilon_3}  \tau_3~\kw{with} \varepsilon_A$ and $ \Gamma, x:  \tau' \vdash  e_2:  \tau_2~\kw{with} \varepsilon_B$, where $\varepsilon = \varepsilon_A \cup \varepsilon_B \cup \varepsilon_3$ and $ \tau =  \tau_3$. By inductive assumption, $ \Gamma \vdash [ v/x] e_1:  \tau_2 \rightarrow_{\varepsilon_3}  \tau_3~\kw{with} \varepsilon_A$ and $ \Gamma \vdash [ v/x] e_2:  \tau_2~\kw{with} \varepsilon_B$. By \textsc{$\varepsilon$-App} we have $ \Gamma \vdash ([ v/x] e_1) ([ v/x] e_2) :  \tau_3~\kw{with} \varepsilon_A \cup \varepsilon_B \cup \varepsilon_3$. By simplifying and applying the definition of $\kwa{substitution}$, this is the same as $ \Gamma \vdash [ v/x]( e_1  e_2):  \tau~\kw{with} \varepsilon$. \\

\textit{Case:} \textsc{$\varepsilon$-OperCall} By inversion we know $ \Gamma, x:  \tau' \vdash  e_1: \{ \bar r \}~\kw{with} \varepsilon_1$, where $\varepsilon = \varepsilon_1 \cup \{ r.\pi \mid r.\pi \in \bar r \times \Pi \}$ and $ \tau = \{ \bar r \}$. By applying the inductive assumption, $ \Gamma \vdash [ v/x] e_1 : \{ \bar r \} ~\kw{with} \varepsilon_1$. Then by \textsc{$\varepsilon$-OperCall}, $ \Gamma \vdash ([ v/x] e_1).\pi: \{ \bar r \}~\kw{with} \varepsilon_1 \cup \{ r.\pi \mid r.\pi \in \bar r \times \Pi \}$. By simplifying and applying the definition of $\kwa{substitution}$, this is the same as $ \Gamma \vdash [ v/x]( e_1.\pi):  \tau~\kw{with} \varepsilon$.\\

\textit{Case:} \textsc{$\varepsilon$-Subsume} By inversion we know $ \Gamma, x:  \tau' \vdash  e:  \tau_2~\kw{with} \varepsilon_2$, where $ \tau_2 <:  \tau$ and $\varepsilon_2 \subseteq \varepsilon$. By inductive hypothesis, $ \Gamma \vdash [ v/x] e:  \tau_2~\kw{with} \varepsilon_2$. Then by \textsc{$\varepsilon$-Subsume} we get $ \Gamma \vdash [ v/x] e:  \tau~\kw{with} \varepsilon$. \\

\end{proof}


\hrulefill

\begin{theorem}[Preservation]
If $\hat \Gamma \vdash \hat e_A: \hat \tau_A~\kw{with} \varepsilon_A$ and $e_A \longrightarrow e_B~|~\varepsilon_C$, then $\hat \Gamma \vdash e_B: \tau_B~\kw{with} \varepsilon_B$, where $e_B <: e_A$ and $\varepsilon \cup \varepsilon_B \subseteq \varepsilon_A$.
\end{theorem}

\begin{proof}
By induction on $\hat \Gamma \vdash \hat e_A: \hat \tau_A~\kw{with} \varepsilon_A$, and then on $e_A \longrightarrow e_B~|~\varepsilon$. \\

\textit{Case:} \textsc{$\varepsilon$-Var}, \textsc{$\varepsilon$-Resource}, \textsc{$\varepsilon$-Unit}, \textsc{$\varepsilon$-Abs}. Then $e_A$ is a value and cannot be reduced, so the theorem holds vacuously. \\

\textit{Case:} \textsc{$\varepsilon$-App}. Then $e_A = \hat e_1~\hat e_2$ and $\hat e_1: \hat \tau_2 \rightarrow_{\varepsilon} \hat \tau_3~\kw{with} \varepsilon_1$ and $\hat \Gamma \vdash \hat e_2: \hat \tau_2~\kw{with} \varepsilon_2$.

\textbf{Subcase:} \textsc{E-App1}. Todo.

\textbf{Subcase:} \textsc{E-App2}. Todo.

\textbf{Subcase:} \textsc{E-App3}. Then $(\lambda x: \hat \tau_2.\hat e)\hat v_2 \longrightarrow [\hat v_2/x]\hat e~|~\varnothing$. By inversion on the typing rule for $\lambda x: \hat \tau_2.\hat e$ we know $\Gamma, x: \hat \tau_2 \vdash \hat e: \hat \tau_3~\kw{with} \varepsilon_3$. By canonical forms, $\varepsilon_2 = \varnothing$ because $\hat e_2 = \hat v_2$ is a value. Then by the substitution lemma, $\hat \Gamma \vdash [\hat v_2/x]\hat e : \hat \tau_3~\kw{with} \varepsilon_3$. By canonical forms, $\varepsilon_1 = \varepsilon_2 = \varnothing = \varepsilon_C$. Therefore $\varepsilon_A = \varepsilon_3 = \varepsilon_B \cup \varepsilon_C$. \\

\textit{Case:}  \textsc{$\varepsilon$-OperCall}. 

\textbf{Subcase:} \textsc{E-OperCall1}.

\textbf{Subcase:} Otherwise the reduction rule used was \textsc{E-OperCall2} and $v_1.\pi \longrightarrow \kwa{unit}~|~\{ r.\pi \}$. By canonical forms, $\hat \Gamma \vdash v_1: \kwa{unit}~\kw{with} \{ r.\pi \}$. Also, $\hat \Gamma \vdash \kwa{unit}: \kwa{Unit}~\kw{with} \varnothing$. Then $\tau_B = \tau_A$. Also, $\varepsilon_C \cup \varepsilon_B = \{ r.\pi \} = \varepsilon_A$.
\end{proof}

\hrulefill

\begin{theorem}[Soundness]
If $\hat \Gamma \vdash \hat e_A: \hat \tau_A~\kw{with} \varepsilon_A$ and $\hat e_A$ is not a value, then $e_A \longrightarrow e_B~|~\varepsilon$, where $\hat \Gamma \vdash e_B: \hat \tau_B~\kw{with} \varepsilon_B$ and $\hat \tau_B <: \hat \tau_A$ and $\varepsilon_B \cup \varepsilon \subseteq \varepsilon_A$.
\end{theorem}
\begin{proof}
If $\hat e_A$ is not a value then the reduction exists by the progress theorem. The rest follows by the preservation theorem.
\end{proof}

\hrulefill

\begin{theorem}[Multi-step Soundness]
If $\hat \Gamma \vdash \hat e_A: \hat \tau_A~\kw{with} \varepsilon_A$ and $e_A \longrightarrow^{*} e_B~|~\varepsilon$, where $\hat \Gamma \vdash e_B: \hat \tau_B~\kw{with} \varepsilon_B$ and $\hat \tau_B <: \hat \tau_A$ and $\varepsilon_B \cup \varepsilon \subseteq \varepsilon_A$.
\end{theorem}

\begin{proof} By induction on the length of the multi-step reduction.\\

\textit{Case:} Length $0$. Then $e_A = e_B$, and therefore $\tau_A = \tau_B$ and $\varepsilon = \varnothing$ and $\varepsilon_A = \varepsilon_B$.\\

\textit{Case:} Length $1$. Then the result follows by single-step soundness.\\

\textit{Case:} Length $n+1$. Then by inversion the multi-step can be split into a multi-step of length $n$, which is $\hat e_A \longrightarrow^{*} \hat e_C~|~\varepsilon'$ and a single-step of length $1$, which is $e_C \longrightarrow e_B~|~\varepsilon''$, where $\varepsilon = \varepsilon' \cup \varepsilon''$. By inductive assumption and preservation theorem, $\hat \Gamma \vdash \hat e_C: \hat \tau_C~\kw{with} \varepsilon_C$ and $\hat \Gamma \vdash \hat e_B: \hat \tau_B~\kw{with} \varepsilon_B$. By inductive assumption, $\hat \tau_C <: \hat \tau_A$ and $\hat \varepsilon_C \cup \varepsilon' \subseteq \varepsilon_A$. By single-step soundness, $\hat \tau_B <: \hat \tau_C$ and $\hat \varepsilon_B \cup \varepsilon'' \subseteq \varepsilon_C$. Then by transitivity, $\hat \tau_B <: \hat \tau$ and $\hat \varepsilon_B \cup \varepsilon' \cup \varepsilon'' = \varepsilon_B \cup \varepsilon \subseteq \varepsilon_A$.
\end{proof}















\chapter{$\epscalc$ Proofs}\label{appendix:LAS File}
\begin{lemma}[Canonical Forms]
The following are true:
\begin{itemize}
	\setlength\itemsep{-0.7em}
	\item If $\hat \Gamma \vdash \hat v: \hat \tau~\kw{with} \varepsilon$ then $\varepsilon = \varnothing$.
	\item If $\hat \Gamma \vdash \hat v: \{ \bar r \}$ then $\hat v = r$ for some $r \in R$ and $\{ \bar r \} = \{ r \}$.
\end{itemize}
\end{lemma}

\hrulefill

\begin{theorem}[Progress]
If $\hat \Gamma \vdash \hat e: \hat \tau~\kw{with} \varepsilon$ and $\hat e$ is not a value, then $\hat e \longrightarrow \hat e'~|~\varepsilon$.
\end{theorem}

\begin{proof} By induction on $\hat \Gamma \vdash \hat e: \hat \tau~\kw{with} \varepsilon$, for $\hat e$ not a value.\\

Case: \textsc{$\varepsilon$-App}. Then $\hat e = \hat e_1~\hat e_2$. If $\hat e_1$ is a non-value, then $\hat e_1~\hat e_2 \longrightarrow \hat e_1'~\hat e_2$ by \textsc{E-App1}. If $\hat e_1 = \hat v_1$ is a value and $\hat e_2$ is a non-value, then $\hat e_1~\hat e_2 \longrightarrow \hat v_1~\hat e_2'$ by \textsc{E-App2}. Otherwise $\hat e_1$ and $\hat e_2$ are both values. By inversion, $\hat e_1 = \lambda x: \hat \tau . \hat e$, so $(\lambda x: \hat \tau. \hat e) \hat v_2 \longrightarrow [\hat v_2/x]~|~\varnothing$ by \textsc{E-App3}.\\

Case: \textsc{$\varepsilon$-Oper}. Then $\hat e = \hat e_1.\pi$. If $\hat e_1$ is a non-value, then $\hat e_1.\pi \longrightarrow \hat e_1'.\pi~|~\varepsilon_1$ by \textsc{E-OperCall1}. Otherwise $\hat e_1 = \hat v_1$ is a value. By canonical forms, $\hat v_1 = r$ and $\hat \Gamma \vdash v_1: \{ r \}~\kw{with} \varnothing$. Then $r.\pi \longrightarrow \kwa{unit}~|~\{ r.\pi \}$ by \textsc{E-OperCall2}.\\

Case: \textsc{$\varepsilon$-Subsume}. Then $\hat \Gamma \vdash \hat e: \hat \tau'~\kw{with} \varepsilon'$. By inversion, $\hat \Gamma \vdash \hat e: \tau~\kw{with} \varepsilon$, where $\tau' <: \tau$ and $\varepsilon' \subseteq \varepsilon$. These are subderivations, so the result holds by inductive assumption.\\

Case: \textsc{$\varepsilon$-Module}. Then $\hat e = \import{\varepsilon}{x}{\hat e'}{e}$. If $\hat e'$ is a non-value then $\import{\varepsilon}{x}{\hat e'}{e} \longrightarrow \import{\varepsilon}{x}{\hat e''}{e}~|~\varepsilon'$ by \textsc{E-Module1}. Otherwise $\hat e' = \hat v$ is a value. Then $\import{\varepsilon}{x}{\hat v}{e} \longrightarrow [\hat v/x]\kwa{annot}(e, \varepsilon)~|~\varnothing$ by \textsc{E-Module2}.
\end{proof}


\hrulefill

\begin{lemma}[Substitution]
If $\hat \Gamma, x: \hat \tau' \vdash e: \hat \tau~\kw{with} \varepsilon$ and $\hat \Gamma \vdash \hat v: \hat \tau'~\kw{with} \varnothing$ then $\hat \Gamma \vdash [\hat v/x]e: \hat \tau~\kw{with} \varepsilon$.
\end{lemma}

\begin{proof} By induction on $\hat \Gamma, x: \hat \tau' \vdash e: \hat \tau~\kw{with} \varepsilon$. \\

\textit{Case}: \textsc{$\varepsilon$-Var}. Then $\hat e = y$ and either $y = x$ or $y \neq x$. If $y \neq x$. Then $[\hat v/x]y = y$ and $\hat \Gamma \vdash y: \hat \tau~\kw{with} \varnothing$. Therefore $\hat \Gamma \vdash [\hat v/x]y: \hat \tau~\kw{with} \varnothing$. Otherwise $y = x$. By inversion on \textsc{$\varepsilon$-Var}, the typing judgement from the theorem assumption is $\hat \Gamma, x: \hat \tau' \vdash x: \hat \tau'~\kw{with} \varnothing$. Since $[\hat v/x]y = \hat v$, and by assumption $\hat \Gamma \vdash \hat v: \hat \tau'~\kw{with} \varnothing$, then $\hat \Gamma \vdash [\hat v/x]x: \hat \tau'~\kw{with} \varnothing$. \\

\textit{Case}: \textsc{$\varepsilon$-Resource}. Because $\hat e = r$ is a resource literal then $\hat \Gamma \vdash r: \hat \tau~\kw{with} \varnothing$ by canonical forms. By definition $[\hat v/x]r = r$, so $\hat \Gamma \vdash [\hat v/x]r: \hat \tau~\kw{with} \varnothing$. \\

\textit{Case:} \textsc{$\varepsilon$-App} By inversion we know $\hat \Gamma, x: \hat \tau' \vdash \hat e_1:\hat \tau_2 \rightarrow_{\varepsilon_3} \hat \tau_3~\kw{with} \varepsilon_A$ and $\hat \Gamma, x: \hat \tau' \vdash \hat e_2: \hat \tau_2~\kw{with} \varepsilon_B$, where $\varepsilon = \varepsilon_A \cup \varepsilon_B \cup \varepsilon_3$ and $\hat \tau = \hat \tau_3$. By inductive assumption, $\hat \Gamma \vdash [\hat v/x]\hat e_1: \hat \tau_2 \rightarrow_{\varepsilon_3} \hat \tau_3~\kw{with} \varepsilon_A$ and $\hat \Gamma \vdash [\hat v/x]\hat e_2: \hat \tau_2~\kw{with} \varepsilon_B$. By \textsc{$\varepsilon$-App} we have $\hat \Gamma \vdash ([\hat v/x]\hat e_1) ([\hat v/x]\hat e_2) : \hat \tau_3~\kw{with} \varepsilon_A \cup \varepsilon_B \cup \varepsilon_3$. By simplifying and applying the definition of $\kwa{substitution}$, this is the same as $\hat \Gamma \vdash [\hat v/x](\hat e_1 \hat e_2): \hat \tau~\kw{with} \varepsilon$. \\

\textit{Case:} \textsc{$\varepsilon$-OperCall} By inversion we know $\hat \Gamma, x: \hat \tau' \vdash \hat e_1: \{ \bar r \}~\kw{with} \varepsilon_1$, where $\varepsilon = \varepsilon_1 \cup \{ r.\pi \mid r.\pi \in \bar r \times \Pi \}$ and $\hat \tau = \{ \bar r \}$. By applying the inductive assumption, $\hat \Gamma \vdash [\hat v/x]\hat e_1 : \{ \bar r \} ~\kw{with} \varepsilon_1$. Then by \textsc{$\varepsilon$-OperCall}, $\hat \Gamma \vdash ([\hat v/x]\hat e_1).\pi: \{ \bar r \}~\kw{with} \varepsilon_1 \cup \{ r.\pi \mid r.\pi \in \bar r \times \Pi \}$. By simplifying and applying the definition of $\kwa{substitution}$, this is the same as $\hat \Gamma \vdash [\hat v/x](\hat e_1.\pi): \hat \tau~\kw{with} \varepsilon$.\\

\textit{Case:} \textsc{$\varepsilon$-Subsume} By inversion we know $\hat \Gamma, x: \hat \tau' \vdash \hat e: \hat \tau_2~\kw{with} \varepsilon_2$, where $\hat \tau_2 <: \hat \tau$ and $\varepsilon_2 \subseteq \varepsilon$. By inductive hypothesis, $\hat \Gamma \vdash [\hat v/x]\hat e: \hat \tau_2~\kw{with} \varepsilon_2$. Then by \textsc{$\varepsilon$-Subsume} we get $\hat \Gamma \vdash [\hat v/x]\hat e: \hat \tau~\kw{with} \varepsilon$. \\

\textit{Case:} \textsc{$\varepsilon$-Module} Then $\hat \Gamma, x: \hat \tau' \vdash \import : \kwa{annot}(\tau, \varepsilon)~\kw{with} \varepsilon \cup \varepsilon_1$. By inversion we know $\hat \Gamma, x: \hat \tau' \vdash \hat e: \hat \tau_1~\kw{with} \varepsilon_1$. By inductive assumption, $\hat \Gamma \vdash [\hat v/x]\hat e: \hat \tau_1~\kw{with} \varepsilon_1$. Then by \textsc{$\varepsilon$-Module} we have $\hat \Gamma \vdash \import: \kwa{annot}(\tau, \varepsilon)~\kw{with} \varepsilon \cup \varepsilon_1$.
\end{proof}

\hrulefill

\begin{lemma}
If $\kwa{effects}(\hat \tau) \subseteq \varepsilon$ and $\kwa{ho \hyphen safe}(\hat \tau, \varepsilon)$ then $\hat \tau <: \kwa{annot}(\kwa{erase}(\hat \tau), \varepsilon)$.
\end{lemma}

\begin{lemma}
If $\kwa{ho \hyphen effects}(\hat \tau) \subseteq \varepsilon$ and $\safe{\hat \tau}{\varepsilon}$ then $\kwa{annot(erase}(\hat \tau), \varepsilon) <: \hat \tau$.
\end{lemma}

\begin{proof}
By simultaneous induction.\\

\textit{Case:} $\hat \tau = \{ \bar r \}$ Then $\hat \tau = \kwa{annot(erase}(\hat \tau), \varepsilon)$ and the results for both lemmas hold immediately. \\

\textit{Case: $\hat \tau = \hat \tau_1 \rightarrow_{\varepsilon'} \hat \tau_2$, $\fx{\hat \tau} \subseteq \varepsilon$, $\hosafe{\hat \tau}{\varepsilon}$} It is sufficient to show $\hat \tau_2 <: \kwa{annot(erase}(\hat \tau_2), \varepsilon)$ and $\kwa{annot(erase}(\hat \tau_1), \varepsilon) <: \hat \tau_1$, because the result will hold by \textsc{S-Effects}. To achieve this we shall inductively apply \textbf{lemma 2} to $\hat \tau_2$ and \textbf{lemma 3} to $\hat \tau_1$. 

From $\fx{\hat \tau} \subseteq \varepsilon$ we have $\hofx{\hat \tau_1} \cup \varepsilon' \cup \fx{\hat \tau_2} \subseteq \varepsilon$ and therefore $\fx{\hat \tau_2} \subseteq \varepsilon$. From $\hosafe{\hat \tau}{\varepsilon}$ we have $\hosafe{\hat \tau_2}{\varepsilon}$. Therefore we can apply \textbf{lemma 2} to $\hat \tau_2$.

From $\fx{\hat \tau} \subseteq \varepsilon$ we have $\hofx{\hat \tau_1} \cup \varepsilon' \cup \fx{\hat \tau_2} \subseteq \varepsilon$ and therefore $\hofx{\hat \tau_1} \subseteq \varepsilon$. From $\hosafe{\hat \tau}{\varepsilon}$ we have $\hosafe{\hat \tau_1}{\varepsilon}$. Therefore we can apply \textbf{lemma 3} to $\hat \tau_1$.\\

\textit{Case: $\hat \tau = \hat \tau_1 \rightarrow_{\varepsilon'} \hat \tau_2$, $\hofx{\hat \tau} \subseteq \varepsilon$, $\safe{\hat \tau}{\varepsilon}$ } It is sufficient to show $\kwa{annot(erase}(\hat \tau_2), \varepsilon) <: \hat \tau_2$ and $\hat \tau_1 <: \kwa{annot(erase}(\hat \tau_1), \varepsilon)$, because the result will hold by \textsc{S-Effects}. To achieve this we shall inductively apply \textbf{lemma 3} to $\hat \tau_2$ and \textbf{lemma 2} to $\hat \tau_1$.

From $\hofx{\hat \tau} \subseteq \varepsilon$ we have $\fx{\hat \tau_1} \cup \hofx{\hat \tau_2} \subseteq \varepsilon$ and therefore $\hofx{\hat \tau_2} \subseteq \varepsilon$. From $\safe{\hat \tau}{\varepsilon}$ we have $\safe{\hat \tau_2}{\varepsilon}$. Therefore we can apply \textbf{lemma 3} to $\hat \tau_2$.

From $\hofx{\hat \tau} \subseteq \varepsilon$ we have $\fx{\hat \tau_1} \cup \hofx{\hat \tau_2} \subseteq \varepsilon$ and therefore $\fx{\hat \tau_1} \subseteq \varepsilon$. From $\safe{\hat \tau}{\varepsilon}$ we have $\hosafe{\hat \tau_1}{\varepsilon}$. Therefore we can apply \textbf{lemma 2} to $\hat \tau_1$.

\end{proof}


\hrulefill

\begin{lemma}[Annotation]
If the following are true:

\begin{itemize}
	\setlength\itemsep{-0.7em}
	\item $\hat \Gamma \vdash \hat v : \hat \tau~\kw{with} \varnothing$
	\item $\Gamma, y: \kwa{erase}(\hat \tau) \vdash e: \tau$
	\item $\varepsilon = \kwa{effects}(\hat \tau)$
	\item $\kwa{ho \hyphen safe}(\hat \tau, \varepsilon)$
\end{itemize}

\noindent
Then $\hat \Gamma, \kwa{annot}(\Gamma, \varepsilon), y: \hat \tau \vdash \kwa{annot}(e, \varepsilon) : \kwa{annot}(\tau, \varepsilon)~\kw{with} \varepsilon \cup \kwa{effects}(\kwa{annot}(\Gamma, \varepsilon))$.
\end{lemma}

\begin{proof}
By induction on $\Gamma, y: \kwa{erase}(\hat \tau) \vdash e: \tau$.\\

\textit{Case: \textsc{T-Var}} Then $e=x$ and $\Gamma, y: \kwa{erase}(\hat \tau) \vdash x: \tau$. Either $x=y$ or $x \neq y$. \\

\textbf{Subcase 1: $x = y$}. Then by \textsc{$\varepsilon$-Var} we get $\hat \Gamma, \kwa{annot}(\Gamma, \varepsilon), y: \hat \tau \vdash x: \hat \tau~\kw{with} \varnothing$. First note that $\kwa{annot}(x, \varepsilon) = x$ in this case. Therefore $\Gamma, y: \kwa{erase}(\hat \tau) \vdash \kwa{annot}(\kwa{erase}(x), \varepsilon): \hat \tau~\kw{with} \varnothing$. We know by assumption that $\kwa{effects}(\hat \tau) = \varepsilon$ and $\kwa{ho \hyphen safe}(\hat \tau, \varepsilon)$. Applying \textbf{Lemma 2} we know $\hat \tau <: \kwa{annot}(\kwa{erase}(\hat \tau), \varepsilon)$. Lastly, by \textsc{$\varepsilon$-Subsume} we have $\Gamma, y: \kwa{erase}(\hat \tau) \vdash \kwa{annot}(\kwa{erase}(x), \varepsilon): \kwa{annot}(\kwa{erase}(x), \varepsilon)~\kw{with} \varepsilon~\cup~\kw{effects}(\kwa{annot}(\Gamma, \varepsilon))$. \\

\textbf{Subcase 2: $x \neq y$}. Then $x: \tau \in \Gamma$. Together with the definition $\kwa{annot}(x, \varepsilon) = x$, we know $x: \kwa{annot}(\tau, \varepsilon) \in \kwa{annot}(\Gamma, \varepsilon)$. By \textsc{$\varepsilon$-Var} we have $\hat \Gamma, \kwa{annot}(\Gamma, \varepsilon), y: \hat \tau \vdash \kwa{annot}(x, \varepsilon): \kwa{annot}(\tau, \varepsilon)~\kw{with} \varnothing$. Lastly, by \textsc{$\varepsilon$-Subsume} we have $\Gamma, y: \kwa{erase}(\hat \tau) \vdash \kwa{annot}(\kwa{erase}(x), \varepsilon): \kwa{annot}(\kwa{erase}(x), \varepsilon)~\kw{with} \varepsilon~\cup~\kw{effects}(\kwa{annot}(\Gamma, \varepsilon))$.\\

\textit{Case: \textsc{T-Resource}} Then $\Gamma, y: \kwa{erase}(\hat \tau) \vdash r : \{ r \}$. By definition, $\kwa{annot}(r, \varepsilon) = r$ and $\kwa{annot}(\{ r \}, \varepsilon$). By \textsc{$\varepsilon$-Resource}  $\hat \Gamma, \kwa{annot}(\Gamma, \varepsilon), y: \hat \tau \vdash r: \{ r \}~\kw{with} \varnothing$. By \textsc{$\varepsilon$-Subsume}, $\hat \Gamma, \kwa{annot}(\Gamma, \varepsilon), y: \hat \tau \vdash r: \{ r \}~\kw{with} \varepsilon \cup \kwa{effects}(\kwa{annot}(\Gamma, \varepsilon))$.\\

\textit{Case: \textsc{T-Abs}} Then $\Gamma, y: \kwa{erase}(\hat \tau) \vdash \lambda x: \tau_1.e_{body} : \tau_1 \rightarrow \tau_2$. By inversion, we get the sub-derivation $\Gamma, y: \kwa{erase}(\hat \tau), x:\tau_1  \vdash e_2: \tau_2$. By definition, $\kwa{annot}(e, \varepsilon) = \kwa{annot}(\lambda x: \tau_1.e_2, \varepsilon) = \lambda x: \kwa{annot}(\tau_1, \varepsilon).\kwa{annot}(e_2, \varepsilon)$ and $\kwa{annot}(\tau, \varepsilon) = \kwa{annot}(\tau_1 \rightarrow \tau_2, \varepsilon) = \kwa{annot}(\tau_1, \varepsilon) \rightarrow_{\varepsilon} \kwa{annot}(\tau_2, \varepsilon)$. 

To apply the inductive assumption to $e_2$ we use the unlabelled context $\Gamma, x: \tau_1$. The inductive assumption tells us $\hat \Gamma, \kwa{annot}(\Gamma, \varepsilon), y: \hat \tau, x: \kwa{annot}(\tau_1, \varepsilon) \vdash \kwa{annot}(e_2, \varepsilon): \kwa{annot}(\tau_2, \varepsilon)~\kw{with} \varepsilon \cup \kwa{effects}(\kwa{annot}(\Gamma, \varepsilon)) \cup \kwa{effects}(\kwa{annot}(\tau_1, \varepsilon))$. Call this last effect-set $\varepsilon'$. By \textsc{$\varepsilon$-Abs}, we get $\hat \Gamma, \kwa{annot}(\Gamma, \varepsilon), y: \hat \tau \vdash \lambda x: \kwa{annot}(\tau_1, \varepsilon) . \kwa{annot}(e_2, \varepsilon) : \kwa{annot}(\hat \tau_1) \rightarrow_{\varepsilon'} \kwa{annot}(\hat \tau_2)~\kw{with} \varnothing$. Then by \textsc{$\varepsilon$-Subsume}, we get $\hat \Gamma, \kwa{annot}(\Gamma, \varepsilon), y: \hat \tau \vdash \kwa{annot}(e, \varepsilon) : \kwa{annot}(\hat \tau_1) \rightarrow_{\varepsilon} \kwa{annot}(\hat \tau_2)~\kw{with} \varepsilon \cup \kwa{effects}(\kwa{annot}(\Gamma), \varepsilon) $.\\

\textit{Case: \textsc{T-App}} Then $\Gamma, y: \kwa{erase}(\hat \tau) \vdash e_1~e_2: \tau_3$, where $\Gamma, y:\kwa{erase}(\hat \tau) \vdash e_1: \tau_2 \rightarrow \tau_3$ and $\Gamma, y: \kwa{erase}(\hat \tau) \vdash e_2: \tau_2$. By applying the inductive assumption to $e_1$ and $e_2$, we get $\hat \Gamma, \kwa{annot}(\Gamma, \varepsilon), y: \hat \tau \vdash \kwa{annot}(e_1, \varepsilon): \kwa{annot}(\tau_1, \varepsilon)~\kw{with} \varepsilon$ and $\hat \Gamma, \kwa{annot}(\Gamma, \varepsilon), y: \hat \tau \vdash \kwa{annot}(e_2, \varepsilon): \kwa{annot}(\tau_2, \varepsilon)~\kw{with} \varepsilon$. Simplifying, $\hat \Gamma, \kwa{annot}(\Gamma, \varepsilon), y: \hat \tau \vdash \kwa{annot}(e_1, \varepsilon): \kwa{annot}(\tau_2, \varepsilon) \rightarrow_{\varepsilon} \kwa{annot}(\tau_3, \varepsilon)~\kw{with} \varepsilon$. Then by \textsc{$\varepsilon$-App}, we get $\hat \Gamma, \kwa{annot}(\Gamma, \varepsilon), y: \hat \tau \vdash \kwa{annot}(e_1~e_2, \varepsilon):  \kwa{annot}(\tau_3, \varepsilon)~\kw{with} \varepsilon$.
\\

\noindent
\textit{Case: \textsc{T-OperCall}} Then $\Gamma, y: \kwa{erase}(\hat \tau) \vdash e_1.\pi : \Unit$. By inversion we get the sub-derivation $\Gamma, y: \kwa{erase}(\hat \tau) \vdash e_1: \{ \bar r \}$. By definition, $\kwa{annot}(\{ \bar r \}, \varepsilon) = \{ \bar r \}$. By inductive assumption, $\hat \Gamma, \kwa{annot}(\Gamma, \varepsilon), y: \hat \tau \vdash e_1: \{ \bar r \} ~\kw{with} \varepsilon \cup \kwa{effects}(\kwa{annot}(\Gamma, \varepsilon))$. By \textsc{$\varepsilon$-OperCall}, $\hat \Gamma, \kwa{annot}(\Gamma, \varepsilon), y: \hat \tau \vdash e_1.\pi: \{ \bar r \} ~\kw{with} \varepsilon \cup \{ \bar r.\pi \}$. \\

It remains to show $\{ \bar r.\pi \} \subseteq \varepsilon$. We shall do this by considering where $r$ must have come from (which subcontext left of the turnstile).  \\

\textbf{Subcase 1.} $r = \hat \tau$. As $\varepsilon = \kwa{effects}(\hat \tau)$, then $r.\pi \in \kwa{effects}(\hat \tau)$. \\

\textbf{Subcase 2.} $r: \{ r \} \in \Gamma$. As $\kwa{annot}(r, \varepsilon) = r$, then $r.\pi \in \kwa{annot}(\Gamma, \varepsilon)$. \\

\textbf{Subcase 3.} $r: \{ r \} \in \hat \Gamma$. Then because $\Gamma, y: \kwa{erase}(\hat \tau) \vdash e_1: \{ \bar r \}$, then $r \in \Gamma$ or $r = \kwa{erase}(\hat \tau) = \hat \tau$ and one of the above subcases must also hold. \\

\end{proof}

\hrulefill

\begin{theorem}[Preservation]
If $\hat \Gamma \vdash \hat e_A: \hat \tau_A~\kw{with} \varepsilon_A$ and $e_A \longrightarrow e_B~|~\varepsilon_C$, then $\hat \Gamma \vdash e_B: \tau_B~\kw{with} \varepsilon_B$, where $e_B <: e_A$ and $\varepsilon \cup \varepsilon_B \subseteq \varepsilon_A$.
\end{theorem}

\begin{proof}
By induction on $\hat \Gamma \vdash \hat e_A: \hat \tau_A~\kw{with} \varepsilon_A$, and then on $e_A \longrightarrow e_B~|~\varepsilon$. \\

\textit{Case:} \textsc{$\varepsilon$-Var}, \textsc{$\varepsilon$-Resource}, \textsc{$\varepsilon$-Unit}, \textsc{$\varepsilon$-Abs}. Then $e_A$ is a value and cannot be reduced, so the theorem holds vacuously. \\

\textit{Case:} \textsc{$\varepsilon$-App}. Then $e_A = \hat e_1~\hat e_2$ and $\hat e_1: \hat \tau_2 \rightarrow_{\varepsilon} \hat \tau_3~\kw{with} \varepsilon_1$ and $\hat \Gamma \vdash \hat e_2: \hat \tau_2~\kw{with} \varepsilon_2$.

\textbf{Subcase:} \textsc{E-App1}. Todo.

\textbf{Subcase:} \textsc{E-App2}. Todo.

\textbf{Subcase:} \textsc{E-App3}. Then $(\lambda x: \hat \tau_2.\hat e)\hat v_2 \longrightarrow [\hat v_2/x]\hat e~|~\varnothing$. By inversion on the typing rule for $\lambda x: \hat \tau_2.\hat e$ we know $\Gamma, x: \hat \tau_2 \vdash \hat e: \hat \tau_3~\kw{with} \varepsilon_3$. By canonical forms, $\varepsilon_2 = \varnothing$ because $\hat e_2 = \hat v_2$ is a value. Then by the substitution lemma, $\hat \Gamma \vdash [\hat v_2/x]\hat e : \hat \tau_3~\kw{with} \varepsilon_3$. By canonical forms, $\varepsilon_1 = \varepsilon_2 = \varnothing = \varepsilon_C$. Therefore $\varepsilon_A = \varepsilon_3 = \varepsilon_B \cup \varepsilon_C$. \\

\textit{Case:}  \textsc{$\varepsilon$-OperCall}. 

\textbf{Subcase:} \textsc{E-OperCall1}.

\textbf{Subcase:} Otherwise the reduction rule used was \textsc{E-OperCall2} and $v_1.\pi \longrightarrow \kwa{unit}~|~\{ r.\pi \}$. By canonical forms, $\hat \Gamma \vdash v_1: \kwa{unit}~\kw{with} \{ r.\pi \}$. Also, $\hat \Gamma \vdash \kwa{unit}: \kwa{Unit}~\kw{with} \varnothing$. Then $\tau_B = \tau_A$. Also, $\varepsilon_C \cup \varepsilon_B = \{ r.\pi \} = \varepsilon_A$. \\

\textit{Case:} \textsc{$\varepsilon$-Module}
Then $e_A = \import{\varepsilon}{x}{\hat e}{e}$.

\textbf{Subcase:} \textsc{E-Module1} If the reduction rule used was \textsc{E-ModuleCall1} then the result follows by applying the inductive hypothesis to $\hat e$.

\textbf{Subcase:} \textsc{E-Module2} Otherwise $\hat e$ is a value and the reduction used was \textsc{E-ModuleCall2}. The following are true:
\begin{enumerate}
	\setlength\itemsep{-0.7em}
	\item $e_A = \kwa{import}(\varepsilon)~x = \hat v~\kw{in} e$
	\item $\hat \Gamma \vdash e_A: \kwa{annot}(\tau, \varepsilon)~\kw{with} \varepsilon \cup \varepsilon_1$
	\item $\kwa{import}(\varepsilon)~x = \hat v~\kw{in} e \longrightarrow [\hat v/x]\kwa{annot}(e, \varepsilon)~|~\varnothing$
	\item $\hat \Gamma \vdash \hat v: \hat \tau~\kw{with} \varnothing$
	\item $\varepsilon = \kwa{effects}(\hat \tau)$
	\item $\kwa{ho \hyphen safe}(\hat \tau, \varepsilon)$
	\item $x: \kwa{erase}(\hat \tau) \vdash e: \tau$
\end{enumerate}

Apply the annotation lemma with $\Gamma = \varnothing$ to get $\hat \Gamma, x: \hat \tau \vdash \kwa{annot}(e, \varepsilon): \kwa{annot}(\tau, \varepsilon)~\kw{with} \varepsilon$. From \textbf{4.} we have $\hat \Gamma \vdash \hat v: \hat \tau~\kw{with} \varnothing$, so we can apply the substitution lemma, giving $\hat \Gamma \vdash [\hat v/x]\kwa{annot}(e, \varepsilon): \kwa{annot}(\tau, \varepsilon)~\kw{with} \varepsilon$. By canonical forms, $\varepsilon_1 = \varepsilon_C = \varnothing$. Then $\varepsilon_B = \varepsilon = \varepsilon_A \cup \varepsilon_C$. By examination, $\tau_A = \tau_B = \kwa{annot}(\tau, \varepsilon)$.

\end{proof}

\hrulefill

\begin{theorem}[Soundness]
If $\hat \Gamma \vdash \hat e_A: \hat \tau_A~\kw{with} \varepsilon_A$ and $\hat e_A$ is not a value, then $e_A \longrightarrow e_B~|~\varepsilon$, where $\hat \Gamma \vdash e_B: \hat \tau_B~\kw{with} \varepsilon_B$ and $\hat \tau_B <: \hat \tau_A$ and $\varepsilon_B \cup \varepsilon \subseteq \varepsilon_A$.
\end{theorem}
\begin{proof}
If $\hat e_A$ is not a value then the reduction exists by the progress theorem. The rest follows by the preservation theorem.
\end{proof}

\hrulefill

\begin{theorem}[Multi-step Soundness]
If $\hat \Gamma \vdash \hat e_A: \hat \tau_A~\kw{with} \varepsilon_A$ and $e_A \longrightarrow^{*} e_B~|~\varepsilon$, where $\hat \Gamma \vdash e_B: \hat \tau_B~\kw{with} \varepsilon_B$ and $\hat \tau_B <: \hat \tau_A$ and $\varepsilon_B \cup \varepsilon \subseteq \varepsilon_A$.
\end{theorem}

\begin{proof} By induction on the length of the multi-step reduction.

\textit{Case:} Length $0$. Then $e_A = e_B$, and therefore $\tau_A = \tau_B$ and $\varepsilon = \varnothing$ and $\varepsilon_A = \varepsilon_B$.

\textit{Case:} Length $1$. Then the result follows by single-step soundness.

\textit{Case:} Length $n+1$. Then by inversion the multi-step can be split into a multi-step of length $n$, which is $\hat e_A \longrightarrow^{*} \hat e_C~|~\varepsilon'$ and a single-step of length $1$, which is $e_C \longrightarrow e_B~|~\varepsilon''$, where $\varepsilon = \varepsilon' \cup \varepsilon''$. By inductive assumption and preservation theorem, $\hat \Gamma \vdash \hat e_C: \hat \tau_C~\kw{with} \varepsilon_C$ and $\hat \Gamma \vdash \hat e_B: \hat \tau_B~\kw{with} \varepsilon_B$. By inductive assumption, $\hat \tau_C <: \hat \tau_A$ and $\hat \varepsilon_C \cup \varepsilon' \subseteq \varepsilon_A$. By single-step soundness, $\hat \tau_B <: \hat \tau_C$ and $\hat \varepsilon_B \cup \varepsilon'' \subseteq \varepsilon_C$. Then by transitivity, $\hat \tau_B <: \hat \tau$ and $\hat \varepsilon_B \cup \varepsilon' \cup \varepsilon'' = \varepsilon_B \cup \varepsilon \subseteq \varepsilon_A$.
\end{proof}

\end{appendix}
