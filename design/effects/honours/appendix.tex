\chapter{Appendix}

\begin{lemma}[Canonical Forms]
The following are true:
\begin{itemize}
	\setlength\itemsep{-0.7em}
	\item If $\hat \Gamma \vdash \hat v: \hat \tau~\kw{with} \varepsilon$ then $\varepsilon = \varnothing$.
	\item If $\hat \Gamma \vdash \hat v: \{ \bar r \}$ then $\hat v = r$ for some $r \in R$ and $\{ \bar r \} = \{ r \}$.
\end{itemize}
\end{lemma}


\begin{theorem}[Progress]
If $\hat \Gamma \vdash \hat e: \hat \tau~\kw{with} \varepsilon$ and $\hat e$ is not a value, then $\hat e \longrightarrow \hat e'~|~\varepsilon$.
\end{theorem}

\begin{proof} By induction on $\hat \Gamma \vdash \hat e: \hat \tau~\kw{with} \varepsilon$, for $\hat e$ not a value.

Case: \textsc{$\varepsilon$-App}. Then $\hat e = \hat e_1~\hat e_2$. If $\hat e_1$ is a non-value, then $\hat e_1~\hat e_2 \longrightarrow \hat e_1'~\hat e_2$ by \textsc{E-App1}. If $\hat e_1 = \hat v_1$ is a value and $\hat e_2$ is a non-value, then $\hat e_1~\hat e_2 \longrightarrow \hat v_1~\hat e_2'$ by \textsc{E-App2}. Otherwise $\hat e_1$ and $\hat e_2$ are both values. By inversion, $\hat e_1 = \lambda x: \hat \tau . \hat e$, so $(\lambda x: \hat \tau. \hat e) \hat v_2 \longrightarrow [\hat v_2/x]~|~\varnothing$ by \textsc{E-App3}.

Case: \textsc{$\varepsilon$-Oper}. Then $\hat e = \hat e_1.\pi$. If $\hat e_1$ is a non-value, then $\hat e_1.\pi \longrightarrow \hat e_1'.\pi~|~\varepsilon_1$ by \textsc{E-OperCall1}. Otherwise $\hat e_1 = \hat v_1$ is a value. By canonical forms, $\hat v_1 = r$ and $\hat \Gamma \vdash v_1: \{ r \}~\kw{with} \varnothing$. Then $r.\pi \longrightarrow \kwa{unit}~|~\{ r.\pi \}$ by \textsc{E-OperCall2}.

Case: \textsc{$\varepsilon$-Subsume}. Then $\hat \Gamma \vdash \hat e: \hat \tau'~\kw{with} \varepsilon'$. By inversion, $\hat \Gamma \vdash \hat e: \tau~\kw{with} \varepsilon$, where $\tau' <: \tau$ and $\varepsilon' \subseteq \varepsilon$. These are subderivations, so the result holds by inductive assumption.

Case: \textsc{$\varepsilon$-Module}. Then $\hat e = \import{\varepsilon}{x}{\hat e'}{e}$. If $\hat e'$ is a non-value then $\import{\varepsilon}{x}{\hat e'}{e} \longrightarrow \import{\varepsilon}{x}{\hat e''}{e}~|~\varepsilon'$ by \textsc{E-Module1}. Otherwise $\hat e' = \hat v$ is a value. Then $\import{\varepsilon}{x}{\hat v}{e} \longrightarrow [\hat v/x]\kwa{annot}(e, \varepsilon)~|~\varnothing$ by \textsc{E-Module2}.
\end{proof}



\begin{lemma}[Substitution]
If $\hat \Gamma, x: \hat \tau' \vdash e: \hat \tau~\kw{with} \varepsilon$ and $\hat \Gamma \vdash \hat v: \hat \tau'~\kw{with} \varnothing$ then $\hat \Gamma \vdash [\hat v/x]e: \hat \tau~\kw{with} \varepsilon$.
\end{lemma}

\begin{proof} By induction on $\hat \Gamma, x: \hat \tau' \vdash e: \hat \tau~\kw{with} \varepsilon$. \\

\textit{Case}: \textsc{$\varepsilon$-Var}. Then $\hat e = y$ and either $y = x$ or $y \neq x$. If $y \neq x$. Then $[\hat v/x]y = y$ and $\hat \Gamma \vdash y: \hat \tau~\kw{with} \varnothing$. Therefore $\hat \Gamma \vdash [\hat v/x]y: \hat \tau~\kw{with} \varnothing$. Otherwise $y = x$. By inversion on \textsc{$\varepsilon$-Var}, the typing judgement from the theorem assumption is $\hat \Gamma, x: \hat \tau' \vdash x: \hat \tau'~\kw{with} \varnothing$. Since $[\hat v/x]y = \hat v$, and by assumption $\hat \Gamma \vdash \hat v: \hat \tau'~\kw{with} \varnothing$, then $\hat \Gamma \vdash [\hat v/x]x: \hat \tau'~\kw{with} \varnothing$. \\

\textit{Case}: \textsc{$\varepsilon$-Resource}. Because $\hat e = r$ is a resource literal then $\hat \Gamma \vdash r: \hat \tau~\kw{with} \varnothing$ by canonical forms. By definition $[\hat v/x]r = r$, so $\hat \Gamma \vdash [\hat v/x]r: \hat \tau~\kw{with} \varnothing$. \\

\textit{Case:} \textsc{$\varepsilon$-App} By inversion we know $\hat \Gamma, x: \hat \tau' \vdash \hat e_1:\hat \tau_2 \rightarrow_{\varepsilon_3} \hat \tau_3~\kw{with} \varepsilon_A$ and $\hat \Gamma, x: \hat \tau' \vdash \hat e_2: \hat \tau_2~\kw{with} \varepsilon_B$, where $\varepsilon = \varepsilon_A \cup \varepsilon_B \cup \varepsilon_3$ and $\hat \tau = \hat \tau_3$. By inductive assumption, $\hat \Gamma \vdash [\hat v/x]\hat e_1: \hat \tau_2 \rightarrow_{\varepsilon_3} \hat \tau_3~\kw{with} \varepsilon_A$ and $\hat \Gamma \vdash [\hat v/x]\hat e_2: \hat \tau_2~\kw{with} \varepsilon_B$. By \textsc{$\varepsilon$-App} we have $\hat \Gamma \vdash ([\hat v/x]\hat e_1) ([\hat v/x]\hat e_2) : \hat \tau_3~\kw{with} \varepsilon_A \cup \varepsilon_B \cup \varepsilon_3$. By simplifying and applying the definition of $\kwa{substitution}$, this is the same as $\hat \Gamma \vdash [\hat v/x](\hat e_1 \hat e_2): \hat \tau~\kw{with} \varepsilon$. \\

\textit{Case:} \textsc{$\varepsilon$-OperCall} By inversion we know $\hat \Gamma, x: \hat \tau' \vdash \hat e_1: \{ \bar r \}~\kw{with} \varepsilon_1$, where $\varepsilon = \varepsilon_1 \cup \{ r.\pi \mid r.\pi \in \bar r \times \Pi \}$ and $\hat \tau = \{ \bar r \}$. By applying the inductive assumption, $\hat \Gamma \vdash [\hat v/x]\hat e_1 : \{ \bar r \} ~\kw{with} \varepsilon_1$. Then by \textsc{$\varepsilon$-OperCall}, $\hat \Gamma \vdash ([\hat v/x]\hat e_1).\pi: \{ \bar r \}~\kw{with} \varepsilon_1 \cup \{ r.\pi \mid r.\pi \in \bar r \times \Pi \}$. By simplifying and applying the definition of $\kwa{substitution}$, this is the same as $\hat \Gamma \vdash [\hat v/x](\hat e_1.\pi): \hat \tau~\kw{with} \varepsilon$.\\

\textit{Case:} \textsc{$\varepsilon$-Subsume} By inversion we know $\hat \Gamma, x: \hat \tau' \vdash \hat e: \hat \tau_2~\kw{with} \varepsilon_2$, where $\hat \tau_2 <: \hat \tau$ and $\varepsilon_2 \subseteq \varepsilon$. By inductive hypothesis, $\hat \Gamma \vdash [\hat v/x]\hat e: \hat \tau_2~\kw{with} \varepsilon_2$. Then by \textsc{$\varepsilon$-Subsume} we get $\hat \Gamma \vdash [\hat v/x]\hat e: \hat \tau~\kw{with} \varepsilon$. \\

\textit{Case:} \textsc{$\varepsilon$-Module} Then $\hat \Gamma, x: \hat \tau' \vdash \import : \kwa{annot}(\tau, \varepsilon)~\kw{with} \varepsilon \cup \varepsilon_1$. By inversion we know $\hat \Gamma, x: \hat \tau' \vdash \hat e: \hat \tau_1~\kw{with} \varepsilon_1$. By inductive assumption, $\hat \Gamma \vdash [\hat v/x]\hat e: \hat \tau_1~\kw{with} \varepsilon_1$. Then by \textsc{$\varepsilon$-Module} we have $\hat \Gamma \vdash \import: \kwa{annot}(\tau, \varepsilon)~\kw{with} \varepsilon \cup \varepsilon_1$.
\end{proof}




