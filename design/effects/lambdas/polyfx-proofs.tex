\documentclass{llncs}

\usepackage{listings}
\usepackage{proof}
\usepackage{amssymb}
\usepackage[margin=.9in]{geometry}
\usepackage{amsmath}
\usepackage[english]{babel}
\usepackage[utf8]{inputenc}
\usepackage{enumitem}
\usepackage{filecontents}
\usepackage{calc}
\usepackage[linewidth=0.5pt]{mdframed}
\usepackage{changepage}
\allowdisplaybreaks

\usepackage{fancyhdr}
\renewcommand{\headrulewidth}{0pt}
\pagestyle{fancy}
 \fancyhf{}
\rhead{\thepage}

\lstset{tabsize=3, basicstyle=\ttfamily\small, commentstyle=\itshape\rmfamily, numbers=left, numberstyle=\tiny, language=java,moredelim=[il][\sffamily]{?},mathescape=true,showspaces=false,showstringspaces=false,columns=fullflexible,xleftmargin=5pt,escapeinside={(@}{@)}, morekeywords=[1]{objtype,module,import,let,in,fn,var,type,rec,fold,unfold,letrec,alloc,ref,application,policy,external,component,connects,to,meth,val,where,return,group,by,within,count,connect,with,attr,html,head,title,style,body,div,keyword,unit,def}}
\lstloadlanguages{Java,VBScript,XML,HTML}

\newcommand{\keywadj}[1]{\mathtt{#1}}
\newcommand{\keyw}[1]{\keywadj{#1}~}

\newcommand{\kw}[1]{\keyw{ #1 }}
\newcommand{\kwa}[1]{\keywadj{ #1 }}
\newcommand{\reftt}{\mathtt{ref}~}
\newcommand{\Reftt}{\mathtt{Ref}~}
\newcommand{\inttt}{\mathtt{int}~}
\newcommand{\Inttt}{\mathtt{Int}~}
\newcommand{\stepsto}{\leadsto}
\newcommand{\todo}[1]{\textbf{[#1]}}
\newcommand{\intuition}[1]{#1}
\newcommand{\hyphen}{\hbox{-}}

%\newcommand{\intuition}[1]{}

\newlist{pcases}{enumerate}{1}
\setlist[pcases]{
  label=\fbox{\textit{Case}}\protect\thiscase\textit{:}~,
  ref=\arabic*,
  align=left,
  labelsep=0pt,
  leftmargin=0pt,
  labelwidth=0pt,
  parsep=0pt
}
\newcommand{\pcase}[1][]{

  \if\relax\detokenize{#1}\relax
    \def\thiscase{}
  \else
    \def\thiscase{~\fbox{#1:}}
  \fi
  \item
}

\newcommand{\thm}[3]{
	\begin{large}
		\bf{#1}
	\end{large} \\\\
	\fbox{Statement.} ~ #2
	\fbox{Proof.}~ #3 \qed
}

\newcommand{\proofcase}[2]{
	\begin{adjustwidth}{1.5em}{0pt}
		\fbox{Case.}~~#1. \\ ~#2
	\end{adjustwidth}
}

\newcommand{\subcase}[1] {
	\begin{adjustwidth}{2.2em}{0pt}
		\underline{Subcase.} #1
	\end{adjustwidth}
}

\newcommand{\stmt}[1] {

\begin{adjustwidth}{2.5em}{0pt}
	#1
\end{adjustwidth}

}
\newcommand{\type}[2]{
	#1~\keyw{with} #2
}

\newcommand{\unit}[0]{ \kwa{unit} }

\newcommand{\Unit}[0]{ \kwa{Unit} }

\newcommand{\fx}[1]{ \kwa{effects}(#1) }

\newcommand{\hofx}[1]{ \kwa{ho \hyphen effects}(#1) }

\newcommand{\safe}[2]{ \kwa{safe}(#1, #2) }

\newcommand{\hosafe}[2]{ \kwa{ho \hyphen safe}(#1, #2) }

\newcommand{\arr}[3]{
	#1 \rightarrow_{#3} #2
}

\newcommand{\newd}[0]{
	\keywadj{new}_d~x \Rightarrow \overline{d = e}
}

\newcommand{\newsig}[0]{
	\keywadj{new}_\sigma~x \Rightarrow \overline{\sigma = e}
}

\newcommand{\import}[4]{
	\keywadj{import}(#1)~#2 = #3~\kw{in} #4
}

\newcommand{\annot}[2]{
	\keywadj{annot}(#1, #2)
}

\newcommand{\erase}[1]{
	\keywadj{erase}(#1)
}

\newcommand{\poly}[2]{
	\forall #1. #2
}

\newcommand{\polycap}[3]{
	\forall #1. #2~ \kw{caps} #3
}

\newcommand{\ispoly}[1]{
	\kwa{is \hyphen poly}(#1)
}

\newcommand{\lub}[1]{
	\kwa{lub}(#1)
}

\newcommand{\ub}[1]{
	\kwa{ub}(#1)
}

\newcommand\defn{\mathrel{\overset{\makebox[0pt]{\mbox{\normalfont\tiny\sffamily def}}}{=}}}


\begin{document}


\begin{lemma}[Substitution of Values]
If $\hat \Gamma, x: \hat \tau' \vdash \hat e: \hat \tau~\kw{with} \varepsilon$ and $\hat \Gamma \vdash \hat v: \hat \tau'~\kw{with} \varnothing$, then $\hat \Gamma \vdash [\hat v/x]\hat e: \hat \tau~\kw{with} \varepsilon$
\end{lemma}

\begin{proof} By induction on the derivation of $\hat \Gamma, x: \hat \tau' \vdash \hat e: \hat \tau~\kw{with} \varepsilon$.\\



\textit{Case:} \textsc{$\varepsilon$-PolyTypeAbs}. Then $\hat e = \lambda X <: \hat \tau_1. \hat e_1$, and $[\hat v/x]\hat e = \lambda X <: \hat \tau_1. [\hat v/y]\hat e_1$. By inversion and inductive hypothesis, $[\hat v/x]\hat e_1$ in $\hat \Gamma$ can be typed the same as $\hat e_1$ in $\hat \Gamma, x: \hat \tau'$. Then by applying \textsc{$\varepsilon$-PolyTypeAbs}, we get the conclusion.\\

\textit{Case:} \textsc{$\varepsilon$-PolyFxAbs}. Then $\hat e = \lambda \phi \subseteq \varepsilon_1. \hat e_1$, and $[\hat v/x]\hat e = \lambda \phi \subseteq \varepsilon_1. [\hat v/x]\hat e_1$. By inversion and inductive hypothesis, $[\hat v/x]\hat e_1$ in $\hat \Gamma$ can be typed the same as $\hat e_1$ in $\hat \Gamma, x: \hat \tau'$. Then by applying \textsc{$\varepsilon$-PolyFxAbs}, we get the conclusion. \\


\textit{Case:} \textsc{$\varepsilon$-PolyTypeApp}. Then $\hat e = \hat e_1~\hat \tau_1$, and $[\hat v/x]\hat e = [\hat v/x]\hat e_1~\hat \tau_1$. By inductive hypothesis, $[\hat v/x]\hat e_1$ in $\hat \Gamma$ can be typed the same as $\hat e_1$ in $\hat \Gamma, x: \hat \tau'$. Then by applying \textsc{$\varepsilon$-PolyTypeApp}, we get the conclusion.\\

\textit{Case:} \textsc{$\varepsilon$-PolyFxApp}. Then $\hat e = \hat e_1~\varepsilon$, and $[\hat v/x]\hat e = [\hat v/x]\hat e_1~\varepsilon$. By inductive hypothesis, $[\hat v/x]\hat e_1$ in $\hat \Gamma$ can be typed the same as $\hat e_1$ in $\hat \Gamma, x: \hat \tau'$. Then by applying \textsc{$\varepsilon$-PolyFxApp}, we get the conclusion.\\



\end{proof}


\hrulefill

\begin{lemma}[Substitution of Types]
If $\hat \Gamma, X <: \hat \tau' \vdash \hat e: \hat \tau~\kw{with} \varepsilon$ and $\hat \Gamma \vdash \hat \tau'' <: \hat \tau'$, then $\hat \Gamma \vdash [\hat \tau''/X]\hat e: \hat \tau~\kw{with} \varepsilon$
\end{lemma}

\begin{proof} By induction on the derivation of $\hat \Gamma, X <: \hat \tau' \vdash \hat e: \hat \tau~\kw{with} \varepsilon$.'\\

\textit{Case:} \textsc{$\varepsilon$-Var, $\varepsilon$-Resource}. Then $\hat e = [\hat \tau''/X]\hat e$, so the typing judgement in the antecedent and consequent can be the same.\\

\textit{Case:} \textsc{$\varepsilon$-Abs}. Then $\hat e = \lambda x: \hat \tau_1. \hat e_2$, and $[\hat \tau''/X]\hat e_2 = \lambda x: [\hat \tau''/X]\hat \tau_1. [\hat \tau''/X]\hat e_2$. WLOG assume that $\hat \tau = \hat \tau_1 \rightarrow_{\varepsilon'} \hat \tau_2$. By inductive assumption and inversion, $[\hat \tau''/X]\hat e_2$  in $\hat \Gamma$ can be typed the same as $\hat e_2$ in $\hat \Gamma, X <: \hat \tau'$. By \textsc{$\varepsilon$-Abs}, $\hat \Gamma \vdash [\hat \tau''/X]\hat e : [\hat \tau''/X]\hat \tau_1 \rightarrow_{\varepsilon'} \hat \tau_2$.

\textbf{But now we have to establish that this new type we just derived is a subtype of $\hat \tau_1 \rightarrow_{\varepsilon'} \hat \tau_2$. To do that requires us to show that $\hat \tau_1 <: [\hat \tau''/X]\hat \tau_1$, because function types are contravariant in their input type under the subtyping relation. However, the substitution should intuitively be making the type more precise, so the subtyping is going the wrong way. }

\end{proof}



\hrulefill

\begin{theorem}[Progress]
If $\hat \Gamma \vdash \hat e: \hat \tau~\kw{with} \varepsilon$ and $\hat e$ is not a value, then $\hat e \longrightarrow \hat e'~|~\varepsilon$, for some $\hat e', \varepsilon$.
\end{theorem}

\begin{proof} By induction on the derivation of $\hat \Gamma \vdash \hat e: \hat \tau~\kw{with} \varepsilon$.\\

\textit{Case:} \textsc{$\varepsilon$-PolyTypeAbs}. Trivial; $\hat e$ is a value. \\

\textit{Case:} \textsc{$\varepsilon$-PolyFxAbs}. Trivial; $\hat e$ is a value. \\

\textit{Case:} \textsc{$\varepsilon$-PolyTypeApp}. Then $\hat e= \hat e_1~\hat \tau'$. If $\hat e_1$ is not a value then $\hat e_1 \longrightarrow \hat e_1'~|~\varepsilon$ by inductive hypothesis, and applying \textsc{E-PolyTypeApp1} gives the reduction $\hat e_1~\hat \tau' \longrightarrow \hat e'' \hat \tau'~|~\varepsilon$. Otherwise, $\hat e$ is a value, so $\hat e = \lambda X <: \hat \tau_1. \hat e_2$, and applying \textsc{E-PolyTypeApp2} gives the reduction $(\lambda X <: \hat \tau_1. \hat e_2) \hat \tau' \longrightarrow [\hat \tau'/X]\hat e_2~|~\varnothing$. \\

\textit{Case:} \textsc{$\varepsilon$-PolyFxApp}. Then $\hat e = \hat e_1~\varepsilon'$. If $\hat e_1$ is not a value then $\hat e_1 \longrightarrow \hat e_1'~|~\varepsilon$ by inductive hypothesis, and applying \textsc{E-PolyFxApp1} gives the reduction $\hat e_1~\varepsilon' \longrightarrow \hat e_1'~\varepsilon'~|~\varepsilon$. Otherwise, $\hat e$ is a value, so $\hat e = \lambda \phi \subseteq \varepsilon_1.\hat e_2$, and applying \textsc{E-PolyFxApp2} gives the reduction $(\lambda \phi \subseteq \varepsilon_1.\hat e_2) \varepsilon' \longrightarrow [\varepsilon'/\phi]\hat e_2$.


\end{proof}

\hrulefill

\begin{theorem} [Preservation]
If $\hat \Gamma \vdash \hat e_A: \hat \tau_A~\kw{with} \varepsilon_A$ and $\hat e_A \longrightarrow \hat e_B~|~\varepsilon$, then $\hat \Gamma \vdash \hat e_B: \hat \tau_B~\kw{with} \varepsilon_B$, where $\hat e_B <: \hat e_A$ and $\varepsilon \cup \varepsilon_B \subseteq \varepsilon_A$, for some $\hat e_B, \varepsilon, \hat \tau_B, \varepsilon_B$.
\end{theorem}


\begin{proof} By induction on the derivations of $\hat \Gamma \vdash \hat e_A: \hat \tau_A~\kw{with} \varepsilon_A$ and $\hat e_A \longrightarrow \hat e_B~|~\varepsilon$.\\

\textit{Case:} \textsc{$\varepsilon$-PolyTypeAbs}. Trivial; $\hat e$ is a value.\\

\textit{Case:} \textsc{$\varepsilon$-PolyFxAbs}. Trivial; $\hat e$ is a value.\\

\textit{Case:} \textsc{$\varepsilon$-PolyTypeApp}. Then $\hat e = \hat e_1~\hat \tau'$. Consider which reduction rule was used.

\textbf{Subcase:} \textsc{E-PolyTypeApp1}. Then $\hat e_1~\hat \tau' \longrightarrow \hat e_1'~\hat \tau'~|~\varepsilon$. By inversion, $\hat e_1 \longrightarrow \hat e_1'~|~\varepsilon$. With the inductive hypothesis and subsumption, $\hat e_1'$ can be typed in $\hat \Gamma$ the same as $\hat e_1$. Then by \textsc{$\varepsilon$-PolyTypeApp}, $\hat \Gamma \vdash \hat e_1'~\hat \tau': \hat \tau_A~\kw{with} \varepsilon_A$. That $\varepsilon \cup \varepsilon_B \subseteq \varepsilon_A$ follows by inductive hypothesis.

\textbf{Subcase:} \textsc{E-PolyTypeApp2}. Then $(\lambda X <: \hat \tau_3. \hat e') \hat \tau' \longrightarrow [\hat \tau'/X]\hat e'~|~\varnothing$.

 \textbf{The result follows by the substitution lemma}.\\

\textit{Case:} \textsc{$\varepsilon$-PolyFxApp}. Then $\hat e = \hat e_1~\varepsilon'$. Consider which reduction rule was used.

\textbf{Subcase:} \textsc{E-PolyFxApp1}. Then $\hat e_1~\varepsilon' \longrightarrow \hat e_1'~\varepsilon'~|~\varepsilon$. By inversion, $\hat e_1 \longrightarrow \hat e_1'~|~\varepsilon$. With the inductive hypothesis and subsumption, $\hat e_1'$ can be typed in $\hat \Gamma$ the same as $\hat e_1$. Then by \textsc{$\varepsilon$-PolyFxApp}, $\hat \Gamma \vdash \hat e_1'~\hat \varepsilon': \hat \tau_A~\kw{with} \varepsilon_A$. That $\varepsilon \cup \varepsilon_B \subseteq \varepsilon_A$ follows by inductive hypothesis.

\textbf{Subcase:} \textsc{E-PolyFxApp2}. Then $(\lambda \phi \subseteq \varepsilon_3.\hat e') \varepsilon' \longrightarrow [\varepsilon'/X]\hat e'~|~\varnothing$. \textbf{The result follows by the substitution lemma}.


\end{proof}


\end{document}
