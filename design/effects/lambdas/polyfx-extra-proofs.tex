\documentclass{llncs}

\usepackage{listings}
\usepackage{proof}
\usepackage{amssymb}
\usepackage[margin=.9in]{geometry}
\usepackage{amsmath}
\usepackage[english]{babel}
\usepackage[utf8]{inputenc}
\usepackage{enumitem}
\usepackage{filecontents}
\usepackage{calc}
\usepackage[linewidth=0.5pt]{mdframed}
\usepackage{changepage}
\allowdisplaybreaks

\usepackage{fancyhdr}
\renewcommand{\headrulewidth}{0pt}
\pagestyle{fancy}
 \fancyhf{}
\rhead{\thepage}

\lstset{tabsize=3, basicstyle=\ttfamily\small, commentstyle=\itshape\rmfamily, numbers=left, numberstyle=\tiny, language=java,moredelim=[il][\sffamily]{?},mathescape=true,showspaces=false,showstringspaces=false,columns=fullflexible,xleftmargin=5pt,escapeinside={(@}{@)}, morekeywords=[1]{objtype,module,import,let,in,fn,var,type,rec,fold,unfold,letrec,alloc,ref,application,policy,external,component,connects,to,meth,val,where,return,group,by,within,count,connect,with,attr,html,head,title,style,body,div,keyword,unit,def}}
\lstloadlanguages{Java,VBScript,XML,HTML}

\newcommand{\keywadj}[1]{\mathtt{#1}}
\newcommand{\keyw}[1]{\keywadj{#1}~}

\newcommand{\kw}[1]{\keyw{ #1 }}
\newcommand{\kwa}[1]{\keywadj{ #1 }}
\newcommand{\reftt}{\mathtt{ref}~}
\newcommand{\Reftt}{\mathtt{Ref}~}
\newcommand{\inttt}{\mathtt{int}~}
\newcommand{\Inttt}{\mathtt{Int}~}
\newcommand{\stepsto}{\leadsto}
\newcommand{\todo}[1]{\textbf{[#1]}}
\newcommand{\intuition}[1]{#1}
\newcommand{\hyphen}{\hbox{-}}

%\newcommand{\intuition}[1]{}

\newlist{pcases}{enumerate}{1}
\setlist[pcases]{
  label=\fbox{\textit{Case}}\protect\thiscase\textit{:}~,
  ref=\arabic*,
  align=left,
  labelsep=0pt,
  leftmargin=0pt,
  labelwidth=0pt,
  parsep=0pt
}
\newcommand{\pcase}[1][]{

  \if\relax\detokenize{#1}\relax
    \def\thiscase{}
  \else
    \def\thiscase{~\fbox{#1:}}
  \fi
  \item
}

\newcommand{\thm}[3]{
	\begin{large}
		\bf{#1}
	\end{large} \\\\
	\fbox{Statement.} ~ #2
	\fbox{Proof.}~ #3 \qed
}

\newcommand{\proofcase}[2]{
	\begin{adjustwidth}{1.5em}{0pt}
		\fbox{Case.}~~#1. \\ ~#2
	\end{adjustwidth}
}

\newcommand{\subcase}[1] {
	\begin{adjustwidth}{2.2em}{0pt}
		\underline{Subcase.} #1
	\end{adjustwidth}
}

\newcommand{\stmt}[1] {

\begin{adjustwidth}{2.5em}{0pt}
	#1
\end{adjustwidth}

}
\newcommand{\type}[2]{
	#1~\keyw{with} #2
}

\newcommand{\unit}[0]{ \kwa{unit} }

\newcommand{\Unit}[0]{ \kwa{Unit} }

\newcommand{\fx}[1]{ \kwa{effects}(#1) }

\newcommand{\hofx}[1]{ \kwa{ho \hyphen effects}(#1) }

\newcommand{\safe}[2]{ \kwa{safe}(#1, #2) }

\newcommand{\hosafe}[2]{ \kwa{ho \hyphen safe}(#1, #2) }

\newcommand{\wf}[1]{ \kwa{WF}(#1) }

\newcommand{\fv}[1]{ \kwa{free \hyphen vars}(#1) }

\newcommand{\arr}[3]{
	#1 \rightarrow_{#3} #2
}

\newcommand{\newd}[0]{
	\keywadj{new}_d~x \Rightarrow \overline{d = e}
}

\newcommand{\newsig}[0]{
	\keywadj{new}_\sigma~x \Rightarrow \overline{\sigma = e}
}

\newcommand{\import}[4]{
	\keywadj{import}(#1)~#2 = #3~\kw{in} #4
}

\newcommand{\annot}[2]{
	\keywadj{annot}(#1, #2)
}

\newcommand{\erase}[1]{
	\keywadj{erase}(#1)
}

\newcommand{\poly}[2]{
	\forall #1. #2
}

\newcommand{\polycap}[3]{
	\forall #1. #2~ \kw{caps} #3
}

\newcommand{\ispoly}[1]{
	\kwa{is \hyphen poly}(#1)
}

\newcommand{\lub}[1]{
	\kwa{lub}(#1)
}

\newcommand{\ub}[1]{
	\kwa{ub}(#1)
}

\newcommand\defn{\mathrel{\overset{\makebox[0pt]{\mbox{\normalfont\tiny\sffamily def}}}{=}}}


\begin{document}


\noindent
\textbf{Notation}: $\hat \Gamma \vdash \delta_1, ..., \delta_n$ means $\hat \Gamma \vdash \delta_1$ and $\hat \Gamma \vdash \delta_2$ and ... and $\hat \Gamma \vdash \delta_n$, where each $\delta_i$ is a judgement.

\hrulefill

\begin{lemma}[Narrowing 1 (Subtypes)]
If $\hat \Gamma, X <: \hat \tau, \hat \Delta \vdash \hat \tau_1 <: \hat \tau_2$ and $\hat \Gamma \vdash \hat \tau' <: \hat \tau$ then $\hat \Gamma, X <: \hat \tau', \hat \Delta \vdash \hat \tau_1 <: \hat \tau_2$
\end{lemma}

\begin{proof}

By induction on $\hat \Gamma, X<: \hat \tau, \hat \Delta \vdash \hat \tau_1 <: \hat \tau_2$. The tricky cases are \textsc{S-TypePoly} and \textsc{S-TypeVar}; the others follow by routine application of the inductive hypothesis to subderivations.\\

\fbox{\textit{Case:} \textsc{S-Reflexive}.} Then $\hat \tau_1 = \hat \tau_2$, and $\hat \tau_1 <: \hat \tau_2$ holds in any context, including $\hat \Gamma, X <: \hat \tau', \hat \Delta$.\\

\fbox{\textit{Case:} \textsc{S-Transitive}.} Let $\hat \tau_1 = \hat \tau_A$ and $\hat \tau_2 = \hat \tau_C$. By inversion, there is some $\hat \tau_B$ such that $\hat \Gamma, X <: \hat \tau, \hat \Delta \vdash \hat \tau_A <: \hat \tau_B$ and $\hat \Gamma, X <: \hat \tau, \hat \Delta \vdash \hat \tau_B <: \hat \tau_C$. Applying the inductive assumption, we get the judgements $\hat \Gamma, X <: \hat \tau', \hat \Delta \vdash \hat \tau_A <: \hat \tau_B$ and $\hat \Gamma, X <: \hat \tau', \hat \Delta \vdash \hat \tau_B <: \hat \tau_C$. Then by \textsc{S-Transitive}, $\hat \Gamma, X <: \hat \tau', \hat \Delta \vdash \hat \tau_A <: \hat \tau_C$, which is the same as $\hat \Gamma, X <: \hat \tau', \hat \Delta \vdash \hat \tau_1 <: \hat \tau_2$.\\

\fbox{\textit{Case:} \textsc{S-ResourceSet}.} Follows immediately, since the premises of this rule have nothing to do with the context. That is, if $\hat \Gamma, X <: \hat \tau, \hat \Delta \vdash \{ \overline{r_1} \} <: \{ \overline{r_2} \}$, then by inversion, $r \in \overline{r_1} \implies r \in \overline{r_2}$. Then by \textsc{S-ResourceSet}, $\hat \Gamma, X <: \hat \tau', \hat \Delta \vdash \{ \overline{r_1} \} <: \{ \overline{r_2} \}$.\\

\fbox{\textit{Case:} \textsc{S-Arrow}.} Then $\hat \Gamma, X <: \hat \tau, \hat \Delta \vdash \hat \tau_A \rightarrow_{\varepsilon'} \hat \tau_B <: \hat \tau_A' \rightarrow_{\varepsilon} \hat \tau_B'$. By inversion, $\hat \Gamma, X <: \hat \tau, \hat \Delta \vdash \hat \tau_A' <: \hat \tau_A$ and $\hat \Gamma, X <: \hat \tau, \hat \Delta \vdash \hat \tau_B <: \hat \tau_B'$ and $\varepsilon' \subseteq \varepsilon$. To these first two judgements, apply the inductive assumption, giving $\hat \Gamma, X <: \hat \tau', \hat \Delta \vdash \hat \tau_A' <: \hat \tau_A$ and $\hat \Gamma, X <: \hat \tau', \hat \Delta \vdash \hat \tau_B <: \hat \tau_B'$. Then by \textsc{S-Arrow}, $\hat \Gamma, X <: \hat \tau', \hat \Delta \vdash \hat \tau_A \rightarrow_{\varepsilon'} \hat \tau_B <: \hat \tau_A' \rightarrow_{\varepsilon} \hat \tau_B'$.\\

\fbox{\textit{Case:} \textsc{S-TypePoly}.} Then $\hat \Gamma, X <: \hat \tau, \hat \Delta \vdash (\forall Y <: \hat \tau_A. \hat \tau_B) <: (\forall Z <: \hat \tau_A'. \hat \tau_B')$. By inversion, we have the following two judgements:

\begin{enumerate}
	\item $\hat \Gamma, X <: \hat \tau, \hat \Delta \vdash \hat \tau_A' <: \hat \tau_A$
	\item $\hat \Gamma, X <: \hat \tau, \hat \Delta, Y <: \hat \tau_A' \vdash \hat \tau_B <: \hat \tau_B'$
\end{enumerate}

Using (1) and the assumption $\hat \Gamma \vdash \hat \tau' <: \hat \tau$, the inductive hypothesis can be used to obtain (3).

\begin{enumerate}

  \setcounter{enumi}{2}
	\item $\hat \Gamma, X <: \hat \tau', \hat \Delta \vdash \hat \tau_A' <: \hat \tau_A$

\end{enumerate}

Let $\Delta' = \Delta, Y <:  \hat \tau_A'$. With this, and the assumption $\hat \Gamma \vdash \hat \tau' <: \hat \tau$, we shall apply the inductive hypothesis to obtain (4),

\begin{enumerate}
	\setcounter{enumi}{3}
	\item $\hat \Gamma, X <: \hat \tau', \hat \Delta' \vdash \hat \tau_B <: \hat \tau_B'$ 
\end{enumerate}

Expanding the definition of $\Delta'$, we get (5),

\begin{enumerate}
	\setcounter{enumi}{4}
	\item $\hat \Gamma, X <: \hat \tau', \hat \Delta, Y <: \hat \tau_A' \vdash \hat \tau_B <: \hat \tau_B'$
\end{enumerate}

From (3) and (5), we can use \textsc{S-TypePoly} to obtain (6), which is the theorem conclusion.

\begin{enumerate}
	\setcounter{enumi}{5}
	\item $\hat \Gamma, X <: \hat \tau', \hat \Delta \vdash (\forall Y <: \hat \tau_A. \hat \tau_B) <: (\forall Z <: \hat \tau_A'. \hat \tau_B')$
\end{enumerate}


\fbox{\textit{Case:} \textsc{S-TypeVar}.} Then $\hat \Gamma, X <: \hat \tau, \hat \Delta \vdash Y <: \hat \tau_B$. There are two cases, depending on whether $X = Y$.\\

\textbf{Subcase 1.} $X = Y$. Then $\hat \Gamma, X <: \hat \tau, \hat \Delta \vdash X <: \hat \tau$. It is also true that (1) $\hat \Gamma, X <: \hat \tau', \hat \Delta \vdash X <: \hat \tau'$, by use of \textsc{S-TypeVar}. The assumption $\hat \Gamma \vdash \hat \tau' <: \hat \tau$ can be widened to (2) $\hat \Gamma, X <: \hat \tau', \hat \Delta \vdash \hat \tau' <: \hat \tau$. Then by (1) and (2), we can apply \textsc{S-Transitive} to get $\hat \Gamma, X <: \hat \tau', \hat \Delta \vdash X <: \hat \tau$. \\

\textbf{Subcase 2.} $X \neq Y$. Then $X <: \hat \tau$ is not used in the derivation of $\hat \Gamma, X <: \hat \tau, \hat \Delta \vdash Y <: \hat \tau_B$, so the judgement can be strengthened to $\hat \Gamma, \hat \Delta \vdash Y <: \hat \tau_B$. Then the judgement can be weakened to $\hat \Gamma, X <: \hat \tau', \hat \Delta \vdash Y <: \hat \tau_B$.


\end{proof}

\hrulefill

\begin{lemma}[Narrowing 2 (Effects)]
If $\hat \Gamma, X <: \hat \tau, \hat \Delta \vdash \varepsilon_1 \subseteq \varepsilon_2$ and $\hat \Gamma \vdash \hat \tau' <: \hat \tau$, then $\hat \Gamma, X <: \hat \tau', \hat \Delta \vdash \varepsilon_1 \subseteq \varepsilon_2$.
\end{lemma}

\begin{proof} By induction on $\hat \Gamma, X <: \hat \tau, \hat \Delta \vdash \varepsilon_1 \subseteq \varepsilon_2$.

\end{proof}

\hrulefill

\begin{lemma}[Narrowing 3 (Types)]
If $\hat \Gamma, X <: \hat \tau, \hat \Delta \vdash \hat e: \hat \tau_A~\kw{with} \varepsilon$ and $\hat \Gamma \vdash \hat \tau' <: \hat \tau$ then $\hat \Gamma, X <: \hat \tau', \hat \Delta \vdash \hat e: \hat \tau_A~\kw{with} \varepsilon_A$
\end{lemma}

\begin{proof} By induction on the derivation of $\hat \Gamma, X <: \hat \tau, \hat \Delta \vdash \hat e: \hat \tau_A~\kw{with} \varepsilon_A$. \textsc{$\varepsilon$-Abs, $\varepsilon$-PolyTypeAbs, $\varepsilon$-PolyTypeApp, $\varepsilon$-PolyFxAbs, $\varepsilon$-PolyFxApp} are the tricky cases; they require the use of the inductive hypothesis in a slightly more tricky way. The other cases follow by routine induction. \\

\fbox{\textit{Case:} \textsc{$\varepsilon$-Var}.} Then $\hat \Gamma, X <: \hat \tau, \hat \Delta \vdash x: \hat \tau_A~\kw{with} \varnothing$, where $\hat e = x$. Since $X <: \hat \tau$ is not used in the derivation, we can strengthen the context to get $\hat \Gamma, \hat \Delta \vdash x: \hat \tau_A~\kw{with} \varnothing$. Then by weakening, $\hat \Gamma, X <: \hat \tau', \hat \Delta \vdash x: \hat \tau_A~\kw{with} \varnothing$.\\

\fbox{\textit{Case:} \textsc{$\varepsilon$-Resource}.} Then $\hat \Gamma, X <: \hat \tau, \hat \Delta \vdash r: \{ \bar r \}~\kw{with} \varnothing$, where $\hat e = r$. Since $X <: \hat \tau$ is not used in the derivation, we can strengthen the context to get $\hat \Gamma, \hat \Delta \vdash r: \{ r \}~\kw{with} \varnothing$. Then by weakening, $\hat \Gamma, x <: \hat \tau', \hat \Delta \vdash r: \{ r \}~\kw{with} \varnothing$. \\

\fbox{\textit{Case:} \textsc{$\varepsilon$-OperCall}.} Then $\hat \Gamma, X <: \hat \tau, \hat \Delta \vdash \hat e_1.\pi: \Unit~\kw{with} \varepsilon_1 \cup \{ r.\pi \mid r \in \overline{r} \}$, and $\hat \Gamma, X <: \hat \tau, \hat \Delta \vdash \hat e_1: \{ \overline{r} \}~\kw{with} \varepsilon_1$. To this second judgement we apply the inductive hypothesis, giving $\hat \Gamma, X <: \hat \tau', \hat \Delta \vdash \hat e_1: \{ \overline{r} \}~\kw{with} \varepsilon_1$. With this new judgement, apply \textsc{$\varepsilon$-OperCall} to get $\hat \Gamma, X <: \hat \tau', \hat \Delta \vdash \hat e_1.\pi: \Unit~\kw{with} \varepsilon_1 \cup \{ r.\pi \mid r \in \overline{r} \}$.\\

\fbox{\textit{Case:} \textsc{$\varepsilon$-Subsume}.} Then $\hat \Gamma, X <: \hat \tau, \hat \Delta \vdash \hat e: \hat \tau_A~\kw{with} \varepsilon_A$. By inversion, $\hat \Gamma, X <: \hat \tau, \hat \Delta \vdash \hat \tau_A <: \hat \tau_B, \varepsilon \subseteq \varepsilon'$. By applying Narrowing Lemma 1 to the first judgement, $\hat \Gamma, X <: \hat \tau', \hat \Delta \vdash \hat \tau <: \hat \tau'$. By applying the Narrowing Lemma for effects\footnote{This has yet to be proven}, $\hat \Gamma, X <: \hat \tau', \hat \Delta \vdash \varepsilon \subseteq \varepsilon'$. With these two judgements, \textsc{$\varepsilon$-Subsume} can be used to obtain the judgement $\hat \Gamma, X <: \hat \tau', \hat \Delta \vdash \hat e: \hat \tau_A~\kw{with} \varepsilon_A$.\\

\fbox{\textit{Case:} \textsc{$\varepsilon$-Abs}.} Then $\hat \Gamma, X <: \hat \tau, \hat \Delta \vdash \lambda x: \hat \tau_1. \hat e_2 : \hat \tau_1 \rightarrow_{\varepsilon_2} \hat \tau_2~\kw{with} \varnothing$, where $\hat \Gamma, X<: \hat \tau, \hat \Delta, x: \hat \tau_1 \vdash \hat e_2: \hat \tau_2~\kw{with} \varepsilon_2$. By letting $\hat \Delta' = \hat \Delta, x: \hat \tau_1$, this second judgement can be rewritten as (1),

\begin{enumerate}
	\item $\hat \Gamma, X <: \hat \tau, \hat \Delta' \vdash \hat e_2: \hat \tau_2~\kw{with} \varepsilon_2$. 
\end{enumerate}

Using (1) and the assumption that $\hat \Gamma \vdash \hat \tau' <: \hat \tau$, apply the inductive hypothesis to obtain (2),

\begin{enumerate}
	\setcounter{enumi}{1}
	\item $\hat \Gamma, X <: \hat \tau', \hat \Delta' \vdash \hat e_2: \hat \tau_2~\kw{with} \varepsilon_2$.
\end{enumerate}

Using the definition of $\hat \Delta'$, this can be simplified,

\begin{enumerate}
	\setcounter{enumi}{2}
	\item $\hat \Gamma, X <: \hat \tau', \hat \Delta, x: \hat \tau_1 \vdash \hat e_2: \hat \tau_2~\kw{with} \varepsilon_2$.
\end{enumerate}

Then with (3) we can use \textsc{$\varepsilon$-Abs} to get (4),

\begin{enumerate}
	\setcounter{enumi}{3}
	\item $\hat \Gamma, X <: \hat \tau', \hat \Delta \vdash \lambda x: \hat \tau_1. e_2: \hat \tau_1 \rightarrow_{\varepsilon_2} \hat \tau_2~\kw{with} \varnothing$
\end{enumerate}

\fbox{\textit{Case:} \textsc{$\varepsilon$-App}.} Then $\hat \Gamma, X <: \hat \tau, \hat \Delta \vdash \hat e_1~\hat e_2: \hat \tau_3~\kw{with} \varepsilon_1 \cup \varepsilon_2 \cup \varepsilon_3$, where the following judgements are true from inversion:

\begin{enumerate}
	\item $\hat \Gamma, X <: \hat \tau, \hat \Delta \vdash \hat e_1: \hat \tau_2 \rightarrow_{\varepsilon_3} \hat \tau_3 ~\kw{with} \varepsilon_1$
	\item $\hat \Gamma, X <: \hat \tau, \hat \Delta \vdash \hat e_2: \hat \tau_2~\kw{with} \varepsilon_2$
\end{enumerate}

By applying the inductive assumption to (1) and (2), we get (3) and (4),

\begin{enumerate}
	\setcounter{enumi}{2}
	\item $\hat \Gamma, X <: \hat \tau', \hat \Delta \vdash \hat e_1: \hat \tau_2 \rightarrow_{\varepsilon_3} \hat \tau_3 ~\kw{with} \varepsilon_1$
	\item $\hat \Gamma, X <: \hat \tau', \hat \Delta \vdash \hat e_2: \hat \tau_2~\kw{with} \varepsilon_2$
\end{enumerate}

Then by \textsc{$\varepsilon$-App}, we get $\hat \Gamma, X <: \hat \tau', \hat \Delta \vdash \hat e_1~\hat e_2: \hat \tau_3~\kw{with} \varepsilon_1 \cup \varepsilon_2 \cup \varepsilon_3$.\\

\fbox{\textit{Case:} \textsc{$\varepsilon$-PolyTypeAbs}.} Then $\hat \Gamma, X <: \hat \tau, \hat \Delta \vdash \lambda Y <: \hat \tau_1. \hat e_2: \forall Y <: \hat \tau_1. \hat \tau_2~\kw{caps} \varepsilon_2~\kw{with} \varnothing$. From inversion, we have $\hat \Gamma, X <: \hat \tau, \hat \Delta, Y <: \hat \tau_1 \vdash \hat e_2: \hat \tau_2~\kw{with} \varepsilon_2$. By letting $\Delta' = \Delta, Y <: \hat \tau_1$, the second judgement can be rewritten,

\begin{enumerate}
	\item $\hat \Gamma, X <: \hat \tau, \hat \Delta' \vdash \hat e_2: \hat \tau_2~\kw{with} \varepsilon_2$
\end{enumerate}

By applying the inductive hypothesis to (1), we get judgement (2), which further simplifies to (3) by simplifying $\hat \Delta'$,

\begin{enumerate}
	\setcounter{enumi}{1}
	\item $\hat \Gamma, X <: \hat \tau', \hat \Delta' \vdash \hat e_2: \hat \tau_2~\kw{with} \varepsilon_2$
	\item $\hat \Gamma, X <: \hat \tau', \hat \Delta, Y <: \hat \tau_1 \vdash \hat e_2: \hat \tau_2~\kw{with} \varepsilon_2$
\end{enumerate}

Then by \textsc{$\varepsilon$-PolyTypeAbs}, we get $\hat \Gamma, X <: \hat \tau', \hat \Delta \vdash \lambda Y <: \hat \tau_1. \hat e_2: \forall Y <: \hat \tau_1. \hat \tau_2~\kw{caps} \varepsilon_2~\kw{with} \varnothing$.\\

\fbox{\textit{Case:} \textsc{$\varepsilon$-PolyFxAbs}.} Then $\hat \Gamma, X <: \hat \tau, \hat \Delta \vdash \lambda \phi \subseteq \varepsilon. \hat e_1: \forall \phi \subseteq \varepsilon. \hat \tau_1~\kw{caps} \varepsilon_1~\kw{with} \varnothing$. By inversion, $\hat \Gamma, X <: \hat \tau, \hat \Delta, \phi \subseteq \varepsilon \vdash \hat e_1: \hat \tau_1~\kw{with} \varepsilon_1$. By letting $\hat \Delta' = \hat \Delta, \phi \subseteq \varepsilon$, the second judgement can be rewritten as (1),

\begin{enumerate}
	\item $\hat \Gamma, X <: \hat \tau, \hat \Delta' \vdash \hat e_1: \hat \tau_1~\kw{with} \varepsilon_1$
\end{enumerate}

Using (1) and the assumption that $\hat \Gamma \vdash \hat \tau' <: \hat \tau$, the inductive hypothesis gives judgement (2), which further simplifies to (3) by expanding the definition of $\hat \Delta'$,

\begin{enumerate}
	\setcounter{enumi}{1}
	\item $\hat \Gamma, X <: \hat \tau', \hat \Delta' \vdash \hat e_1: \hat \tau_1~\kw{with} \varepsilon_1$
	\item $\hat \Gamma, X <: \hat \tau', \hat \Delta, \phi \subseteq \varepsilon \vdash \hat e_1: \hat \tau_1~\kw{with} \varepsilon_1$
\end{enumerate}

Then from (2), we can apply \textsc{$\varepsilon$-PolyFxAbs}, giving the judgement $\hat \Gamma, X <: \hat \tau', \hat \Delta \vdash \lambda \phi \subseteq \varepsilon. \hat e_1: \forall \phi \subseteq \varepsilon. \hat \tau_1~\kw{caps} \varepsilon_1~\kw{with} \varnothing$. \\

\fbox{\textit{Case:} \textsc{$\varepsilon$-PolyTypeApp}.} Then $\hat \Gamma, X <: \hat \tau, \hat \Delta \vdash \hat e_A~\hat \tau_1': [\hat \tau_1'/Y]\hat \tau_2~\kw{with} [\hat \tau_1'/Y]\varepsilon_1 \cup \varepsilon_2$, where the following judgements are from inversion:

\begin{enumerate}
	\item $\hat \Gamma, X<: \hat \tau, \hat \Delta \vdash \hat e_A: \forall Y <: \hat \tau_1. \hat \tau_2~\kw{caps} \varepsilon_1~\kw{with} \varepsilon_2$
	\item $\hat \Gamma, X <: \hat \tau, \hat \Delta \vdash \hat \tau_1' <: \hat \tau_1$
\end{enumerate}

With the assumption that $\hat \Gamma, X <: \hat \tau, \hat \Delta \vdash \hat \tau' <: \hat \tau$ and (1), we can apply the inductive hypothesis to get (3). With the same assumption and (2), we can apply Narrowing Lemma 1 (Subtypes) to get (4),

\begin{enumerate}
	\setcounter{enumi}{2}
	\item $\hat \Gamma, X <:  \hat \tau', \hat \Delta \vdash \hat e_A: \forall Y <: \hat \tau_1. \hat \tau_2~\kw{caps} \varepsilon_1~\kw{with} \varepsilon_2$
	\item $\hat \Gamma, X <: \hat \tau', \hat \Delta \vdash \hat \tau_1' <: \hat \tau_1$
\end{enumerate}

From (3) and (4), \textsc{$\varepsilon$-PolyTypeApp} gives the judgement $\hat \Gamma, X <: \hat \tau', \hat \Delta \vdash \hat e_A~\hat \tau_1': [\hat \tau_1'/Y]\hat \tau_2~\kw{with} [\hat \tau_1'/Y]\varepsilon_1 \cup \varepsilon_2$.\\

\fbox{\textit{Case:} \textsc{$\varepsilon$-PolyFxApp}.} Then $\hat \Gamma, X <: \hat \tau, \hat \Delta \vdash \hat e_A~\varepsilon': [\varepsilon'/\phi]\hat \tau_2~\kw{with} [\varepsilon'/\phi]\varepsilon_1 \cup \varepsilon_2$, where the following are true by inversion:

\begin{enumerate}
	\item $\hat \Gamma, X <: \hat \tau, \hat \Delta \vdash \hat e_A: \forall \phi \subseteq \varepsilon. \hat \tau_2~\kw{caps} \varepsilon_1~\kw{with} \varepsilon_2$
	\item $\hat \Gamma, X <: \hat \tau, \hat \Delta \vdash \varepsilon' \subseteq \varepsilon$
\end{enumerate}

With the assumption that $\hat \Gamma, X <: \hat \tau, \hat \Delta \vdash \hat \tau' <: \hat \tau$ and (1), we can apply the inductive hypothesis to obtain (3). With the same assumption and (2), we can apply the Narrowing Lemma for Effect Judgements\footnote{Doesn't actually exist yet} to get (4),

\begin{enumerate}
	\setcounter{enumi}{2}
	\item $\hat \Gamma, X <: \hat \tau', \hat \Delta \vdash \hat e_A: \forall \phi \subseteq \varepsilon. \hat \tau_2~\kw{caps} \varepsilon_1~\kw{with} \varepsilon_2$
	\item $\hat \Gamma, X <: \hat \tau', \hat \Delta \vdash \varepsilon' \subseteq \varepsilon$
\end{enumerate}

With (3) and (4) we can apply \textsc{$\varepsilon$-PolyFxApp} to get $\hat \Gamma, X <: \hat \tau', \hat \Delta \vdash \hat e_A~\varepsilon': [\varepsilon'/\phi]\hat \tau_2~\kw{with} [\varepsilon'/\phi]\varepsilon_1 \cup \varepsilon_2$. \\

\fbox{\textit{Case:} \textsc{$\varepsilon$-Import}.} (We prove for a single import). Then $\hat \Gamma, X <: \hat \tau, \hat \Delta \vdash \kwa{import}(\varepsilon_s)~x_1 = \hat e_1~\kw{in} e: \kw{annot}(\tau, \varepsilon_s)~\kw{with} \varepsilon_s \cup \varepsilon_1$. By inversion, $\hat \Gamma, X <: \hat \tau, \hat \Delta \vdash \hat e_1: \hat \tau_1~\kw{with} \varepsilon_1$. By inductive hypothesis, $\hat \Gamma, X <: \hat \tau', \hat \Delta \vdash \hat e_1: \hat \tau_1~\kw{with} \varepsilon_1$. This, together with the other premises obtained by inversion, gives the judgement $\hat \Gamma, X <: \hat \tau', \hat \Delta \vdash \kwa{import}(\varepsilon_s)~x_1 = \hat e_1~\kw{in} e: \kw{annot}(\tau, \varepsilon_s)~\kw{with} \varepsilon_s \cup \varepsilon_1$.

\end{proof}




\end{document}