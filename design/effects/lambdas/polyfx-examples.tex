\documentclass{llncs}

\usepackage{listings}
\usepackage{proof}
\usepackage{amssymb}
\usepackage[margin=.9in]{geometry}
\usepackage{amsmath}
\usepackage[english]{babel}
\usepackage[utf8]{inputenc}
\usepackage{enumitem}
\usepackage{filecontents}
\usepackage{calc}
\usepackage[linewidth=0.5pt]{mdframed}
\usepackage{changepage}
\allowdisplaybreaks

\usepackage{fancyhdr}
\renewcommand{\headrulewidth}{0pt}
\pagestyle{fancy}
 \fancyhf{}
\rhead{\thepage}

\lstset{tabsize=3, basicstyle=\ttfamily\small, commentstyle=\itshape\rmfamily, numbers=left, numberstyle=\tiny, language=java,moredelim=[il][\sffamily]{?},mathescape=true,showspaces=false,showstringspaces=false,columns=fullflexible,xleftmargin=5pt,escapeinside={(@}{@)}, morekeywords=[1]{objtype,module,import,let,in,fn,var,type,rec,fold,unfold,letrec,alloc,ref,application,policy,external,component,connects,to,meth,val,where,return,group,by,within,count,connect,with,attr,html,head,title,style,body,div,keyword,unit,def}}
\lstloadlanguages{Java,VBScript,XML,HTML}

\newcommand{\keywadj}[1]{\mathtt{#1}}
\newcommand{\keyw}[1]{\keywadj{#1}~}

\newcommand{\kw}[1]{\keyw{ #1 }}
\newcommand{\kwa}[1]{\keywadj{ #1 }}
\newcommand{\reftt}{\mathtt{ref}~}
\newcommand{\Reftt}{\mathtt{Ref}~}
\newcommand{\inttt}{\mathtt{int}~}
\newcommand{\Inttt}{\mathtt{Int}~}
\newcommand{\stepsto}{\leadsto}
\newcommand{\todo}[1]{\textbf{[#1]}}
\newcommand{\intuition}[1]{#1}
\newcommand{\hyphen}{\hbox{-}}

%\newcommand{\intuition}[1]{}

\newlist{pcases}{enumerate}{1}
\setlist[pcases]{
  label=\fbox{\textit{Case}}\protect\thiscase\textit{:}~,
  ref=\arabic*,
  align=left,
  labelsep=0pt,
  leftmargin=0pt,
  labelwidth=0pt,
  parsep=0pt
}
\newcommand{\pcase}[1][]{

  \if\relax\detokenize{#1}\relax
    \def\thiscase{}
  \else
    \def\thiscase{~\fbox{#1:}}
  \fi
  \item
}

\newcommand{\thm}[3]{
	\begin{large}
		\bf{#1}
	\end{large} \\\\
	\fbox{Statement.} ~ #2
	\fbox{Proof.}~ #3 \qed
}

\newcommand{\proofcase}[2]{
	\begin{adjustwidth}{1.5em}{0pt}
		\fbox{Case.}~~#1. \\ ~#2
	\end{adjustwidth}
}

\newcommand{\subcase}[1] {
	\begin{adjustwidth}{2.2em}{0pt}
		\underline{Subcase.} #1
	\end{adjustwidth}
}

\newcommand{\stmt}[1] {

\begin{adjustwidth}{2.5em}{0pt}
	#1
\end{adjustwidth}

}
\newcommand{\type}[2]{
	#1~\keyw{with} #2
}

\newcommand{\unit}[0]{ \kwa{unit} }

\newcommand{\Unit}[0]{ \kwa{Unit} }

\newcommand{\fx}[1]{ \kwa{effects}(#1) }

\newcommand{\hofx}[1]{ \kwa{ho \hyphen effects}(#1) }

\newcommand{\safe}[2]{ \kwa{safe}(#1, #2) }

\newcommand{\hosafe}[2]{ \kwa{ho \hyphen safe}(#1, #2) }

\newcommand{\arr}[3]{
	#1 \rightarrow_{#3} #2
}

\newcommand{\newd}[0]{
	\keywadj{new}_d~x \Rightarrow \overline{d = e}
}

\newcommand{\newsig}[0]{
	\keywadj{new}_\sigma~x \Rightarrow \overline{\sigma = e}
}

\newcommand{\import}[4]{
	\keywadj{import}(#1)~#2 = #3~\kw{in} #4
}

\newcommand{\annot}[2]{
	\keywadj{annot}(#1, #2)
}

\newcommand{\erase}[1]{
	\keywadj{erase}(#1)
}

\newcommand{\poly}[2]{
	\forall #1. #2
}

\newcommand{\polycap}[3]{
	\forall #1. #2~ \kw{caps} #3
}

\newcommand\defn{\mathrel{\overset{\makebox[0pt]{\mbox{\normalfont\tiny\sffamily def}}}{=}}}


\begin{document}











\section{Basic Effect Polymorphism}

\subsection*{Pseudo-Wyvern}
\begin{lstlisting}
def polymorphicWriter(x: T <: {File, Socket}): Unit with T.write =
    x.write
 
/* below invocation should typecheck with File.write as its only effect */
polymorphicWriter File
    
\end{lstlisting}

\subsection*{$\lambda$-Calculus}
\begin{lstlisting}
let pw = $\lambda \phi \subseteq$ {File.write, Socket.write}.
   $\lambda$f: Unit $\rightarrow_{\phi}$ Unit.
      f unit

in let makeWriter = $\lambda$r: {File, Socket}.
   $\lambda$x: Unit. r.write

in (pw {File.write}) (makeWriter File)
\end{lstlisting}


\subsection*{Typing}

To type the definition of $\kwa{polymorphicWriter}$:
\begin{enumerate}
	\item By \textsc{$\varepsilon$-App}\\ $\phi \subseteq \{ \kwa{F.w, S.w} \}$, x: $\Unit \rightarrow_{\phi} \Unit \vdash x~\unit: \Unit~\kw{with} \phi$.
	\item By \textsc{$\varepsilon$-Abs}\\ $\phi \subseteq \{ \kwa{F.w, S.w} \} \vdash \lambda x: \Unit \rightarrow_{\phi} \Unit. x~\unit: (\Unit \rightarrow_{\phi} \Unit) \rightarrow_{\phi} \Unit~\kw{with} \varnothing$
	\item By \textsc{$\varepsilon$-PolyFxAbs}, \\ $\vdash \forall \phi \subseteq \{ \kwa{S.w, F.w} \}. \lambda x: \Unit \rightarrow_{\phi} \Unit. x~\unit:\polycap{\phi \subseteq \{ \kwa{F.w, S.w} \}}{(\Unit \rightarrow_{\phi} \Unit) \rightarrow_{\phi} \Unit}{\varnothing}~\kw{with} \varnothing$
\end{enumerate}

\noindent
Then $\kwa{(pw~\{ File.write \})}$ can be typed as such:

\begin{enumerate}
  \setcounter{enumi}{3}
  \item By \textsc{$\varepsilon$-PolyFxApp}, \\ $\vdash \kwa{pw~\{ F.w \}}: [\{ \kwa{F.w} \}/\phi]( (\Unit \rightarrow_{\phi} \Unit) \rightarrow_{\phi} \Unit) ~\kw{with} [\{ \kwa{F.w} \}/\phi]\varnothing \cup \varnothing$
\end{enumerate}

\noindent
The judgement can be simplified to:

\begin{enumerate}
	\setcounter{enumi}{4}
	\item $\vdash \kwa{pw~\{ F.w \}}: (\Unit \rightarrow_{\{\kwa{F.w}\}} \Unit) \rightarrow_{\{\kwa{F.w}\}} \Unit ~\kw{with} \varnothing$
\end{enumerate}

\noindent
Any application of this function, as in $\kwa{(pw~\{File.write\}) (makeWriter~ File)}$, will therefore type as having the single effect $\kwa{F.w}$ by applying \textsc{$\varepsilon$-App} to judgement (5).




























\section{Dependency Injection}

\subsection*{Pseudo-Wyvern}

An HTTPServer module provides a single $\kwa{init}$ method which returns a $\kwa{Server}$ that responds to HTTP requests on the supplied socket.
\begin{lstlisting}
module HTTPServer

def init(out: A <: {File, Socket}): $\kwa{Str}\rightarrow_{A.write}\Unit$ with $\varnothing$ =
   $\lambda$ msg: Str.
      if (msg == ``POST'') then out.write(``post response'')
      else if (msg == ``GET'') then out.write(``get response'')
      else out.write(``client error 400'')
\end{lstlisting}

\noindent
The main module calls $\kwa{HTTPServer.init}$ with the $\kwa{Socket}$ it should be writing to.

\begin{lstlisting}
module Main
require HTTPServer, Socket

def main(): Unit =
   HTTPServer.init(Socket) ``GET /index.html''
\end{lstlisting}

\noindent
The testing module calls $\kwa{HTTPServer.init}$ with a $\kwa{LogFile}$, perhaps so the responses of the server can be tested offline.

\begin{lstlisting}
module Testing
require HTTPServer, LogFile

def testSocket():  =
   HTTPServer.init(LogFile) ``GET /index.html''
\end{lstlisting}


\subsection*{$\lambda$-Calculus}

\noindent
The HTTPServer module:
\begin{lstlisting}
MakeHTTPServer = $\lambda$x: Unit.
   $\lambda \phi \subseteq \{ \kwa{LogFile.write, Socket.write} \}$.
      $\lambda$f: Str $\rightarrow_{\phi}$ Unit.
         $\lambda$msg: Str.
            f msg
\end{lstlisting}

\noindent
The Main module:

\begin{lstlisting}
MakeMain = $\lambda$hs: HTTPServer. $\lambda$sock: {Socket}.
   $\lambda$x: Unit.
      let socketWriter = ($\lambda$s: {Socket}. $\lambda$x: Unit. s.write) sock in
      let theServer = hs {Socket.write} socketWriter in
      theServer ``GET/index.html''
\end{lstlisting}

\noindent
The Testing module:

\begin{lstlisting}
MakeTest = $\lambda$hs: HTTPserver. $\lambda$lf: {LogFile}.
   $\lambda$x: Unit.
      let logFileWriter = ($\lambda$l: {LogFile}. $\lambda$x: Unit. l.write) lf in
      let theServer = hs {LogFile.write} logFileWriter in
      theServer ``GET/index.html''
\end{lstlisting}

\noindent
A single, desugared program for production would be:

\begin{lstlisting}
let MakeHTTPServer = $\lambda$x: Unit.
   $\lambda \phi \subseteq \{ \kwa{LogFile.write, Socket.write} \}$.
      $\lambda$f: Str $\rightarrow_{\phi}$ Unit.
         $\lambda$msg: Str.
            f msg

in let Run = $\lambda$Socket: {Socket}.
   let HTTPServer = MakeHTTPServer unit in
   let Main = MakeMain HTTPServer Socket in
   Main unit

in Run Socket 
\end{lstlisting}

\noindent
A single, desugared program for testing would be:

\begin{lstlisting}
let MakeHTTPServer = $\lambda$x: Unit.
   $\lambda \phi \subseteq \{ \kwa{LogFile.write, Socket.write} \}$.
      $\lambda$f: Str $\rightarrow_{\phi}$ Unit.
         $\lambda$msg: Str.
            f msg

in let Run = $\lambda$LogFile: {LogFile}.
   let HTTPServer = MakeHTTPServer unit in
   let Main = MakeMain HTTPServer LogFile in
   Main unit

in Run LogFile
\end{lstlisting}

\noindent
Note how the HTTPServer code is identical in the testing and production examples.

\subsection*{Typing}


\begin{lstlisting}
let MakeHTTPServer = $\lambda$x: Unit.
   $\lambda \phi \subseteq \{ \kwa{LogFile.write, Socket.write} \}$.
      $\lambda$f: Str $\rightarrow_{\phi}$ Unit.
         $\lambda$msg: Str.
            f msg
\end{lstlisting}

\noindent
To type $\kwa{MakeHTTPServer}$:

\begin{enumerate}
	\item By \textsc{$\varepsilon$-App}, \\
	$\kwa{x: Unit,~ \phi \subseteq \{LF.w, S.w\}, f: Str \rightarrow_{\phi} Unit,~ msg: Str}$ \\
	$\vdash \kwa{f~msg: Unit~\kw{with} \phi}$
	
	\item By \textsc{$\varepsilon$-Abs}, \\
	$\kwa{x: Unit,~ \phi \subseteq \{LF.w, S.w\}, f: Str \rightarrow_{\phi} Unit}$ \\
	$\vdash \kwa{\lambda msg: Str.~f~msg: Str \rightarrow_{\phi} Unit~\kw{with} \varnothing }$

	\item By \textsc{$\varepsilon$-Abs}, \\
	$\kwa{x: Unit,~ \phi \subseteq \{LF.w, S.w\}}$ \\
	$\vdash \kwa{\lambda f: Str \rightarrow_{\phi} Unit.~\lambda msg: Str.~f~msg:}\\
	\kwa{ (Str \rightarrow_{\phi} Unit) \rightarrow_{\varnothing} (Str \rightarrow_{\phi} Unit)~\kw{with} \varnothing }$

	\item By \textsc{$\varepsilon$-PolyFxAbs}, \\
	$\kwa{x: Unit}$ \\
	$\vdash \kwa{\lambda \phi \subseteq \{ LF.w, S.w \}.~ \lambda f: Str \rightarrow_{\phi} Unit.~\lambda msg: Str.~f~msg:}$\\
	$\kwa{\polycap{\phi \subseteq \{ LF.w, S.w\}}{(Str \rightarrow_{\phi} Unit) \rightarrow_{\varnothing} (Str \rightarrow_{\phi} Unit)}{\varnothing}~\kw{with} \varnothing}$

	\item By \textsc{$\varepsilon$-Abs}, \\
	$\vdash \kwa{\lambda x: Unit. ~\lambda \phi \subseteq \{ LF.w, S.w \}.~ \lambda f: Str \rightarrow_{\phi} Unit.~\lambda msg: Str.~f~msg:}$\\
	$\kwa{Unit} \rightarrow_{\varnothing} \kwa{\polycap{\phi \subseteq \{ LF.w, S.w\}}{(Str \rightarrow_{\phi} Unit) \rightarrow_{\varnothing} (Str \rightarrow_{\phi} Unit)}{\varnothing}~\kw{with} \varnothing}$
\end{enumerate}

\noindent
Note that after two applications of $\kwa{MakeHTTPServer}$, as in $\kwa{MakeHTTPServer~unit~\{Socket.write\}}$, it would type as follows:

\begin{enumerate}
	\setcounter{enumi}{5}
	\item By \textsc{$\varepsilon$-PolyFxApp}, \\
	$\kwa{x: Unit}$\\
	$\kwa{\vdash MakeHTTPServer~unit~\{S.w\}:}$\\
	$\kwa{(Str \rightarrow_{\{S.w\}} Unit) \rightarrow_{\varnothing} (Str \rightarrow_{\{S.w\}} Unit)~\kw{with} \varnothing}$
\end{enumerate}

\noindent
After fixing the polymorphic set of effects, possessing this function only gives you access to the $\kwa{Socket.write}$ effect.












\section{Map Function}

\subsection*{Pseudo-Wyvern}
\begin{lstlisting}
def map(f: A $\rightarrow_{\phi}$ B, l: List[A]): List[B] with $\phi$ =
	if isnil l then []
	else cons (f (head l)) (map (tail l f))
\end{lstlisting}

\subsection*{$\lambda$-Calculus}
\begin{lstlisting}
map = $\lambda \phi$. $\lambda$A. $\lambda$B.
  $\lambda$f: A$\rightarrow_{\phi}$B.
    (fix ($\lambda$map: List[A] $\rightarrow$ List[B]).
      $\lambda$l: List[A].
        if isnil l then []
        else cons (f (head l)) (map (tail l f)))
\end{lstlisting}

\subsection*{Typing}

\begin{itemize}
	\item This has the type: $\forall \phi. \forall A. \forall B. (A \rightarrow_{\phi} B)  \rightarrow_{\varnothing} \kwa{List}[A] \rightarrow_{\phi} \kwa{List}[B]~ \kw{with} \varnothing$.
	\item $\kwa{map}~\varnothing$ is a pure version of map.
	\item $\kwa{map}~\{ \kwa{File.*} \}$ is a version of map which can perform operations on $\kwa{File}$.
\end{itemize}













%\section{File-Backed Data Structure}

%\section{Accessing Database via Expert}

























\end{document}





