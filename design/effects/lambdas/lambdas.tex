\documentclass{llncs}

\usepackage{listings}
\usepackage{proof}
\usepackage{amssymb}
\usepackage[margin=.9in]{geometry}
\usepackage{amsmath}
\usepackage[english]{babel}
\usepackage[utf8]{inputenc}
\usepackage{enumitem}
\usepackage{filecontents}
\usepackage{calc}
\usepackage[linewidth=0.5pt]{mdframed}
\usepackage{changepage}
\usepackage{tabto}
\allowdisplaybreaks

\usepackage{fancyhdr}
\renewcommand{\headrulewidth}{0pt}
\pagestyle{fancy}
 \fancyhf{}
\rhead{\thepage}

\lstset{tabsize=3, basicstyle=\ttfamily\small, commentstyle=\itshape\rmfamily, numbers=left, numberstyle=\tiny, language=java,moredelim=[il][\sffamily]{?},mathescape=true,showspaces=false,showstringspaces=false,columns=fullflexible,xleftmargin=5pt,escapeinside={(@}{@)}, morekeywords=[1]{objtype,module,import,let,in,fn,var,type,rec,fold,unfold,letrec,alloc,ref,application,policy,external,component,connects,to,meth,val,where,return,group,by,within,count,connect,with,attr,html,head,title,style,body,div,keyword,unit,def}}
\lstloadlanguages{Java,VBScript,XML,HTML}

\newcommand{\keywadj}[1]{\mathtt{#1}}
\newcommand{\keyw}[1]{\keywadj{#1}~}

\newcommand{\kw}[1]{\keyw{ #1 }}
\newcommand{\kwa}[1]{\keywadj{ #1 }}
\newcommand{\reftt}{\mathtt{ref}~}
\newcommand{\Reftt}{\mathtt{Ref}~}
\newcommand{\inttt}{\mathtt{int}~}
\newcommand{\Inttt}{\mathtt{Int}~}
\newcommand{\stepsto}{\leadsto}
\newcommand{\todo}[1]{\textbf{[#1]}}
\newcommand{\intuition}[1]{#1}
\newcommand{\hyphen}{\hbox{-}}

%\newcommand{\intuition}[1]{}

\newlist{pcases}{enumerate}{1}
\setlist[pcases]{
  label=\fbox{\textit{Case}}\protect\thiscase\textit{:}~,
  ref=\arabic*,
  align=left,
  labelsep=0pt,
  leftmargin=0pt,
  labelwidth=0pt,
  parsep=0pt
}
\newcommand{\pcase}[1][]{

  \if\relax\detokenize{#1}\relax
    \def\thiscase{}
  \else
    \def\thiscase{~\fbox{#1:}}
  \fi
  \item
}

\newcommand{\thm}[3]{
	\begin{large}
		\bf{#1}
	\end{large} \\\\
	\fbox{Statement.} ~ #2
	\fbox{Proof.}~ #3 \qed
}

\newcommand{\proofcase}[2]{
	\begin{adjustwidth}{1.5em}{0pt}
		\fbox{Case.}~~#1. \\ ~#2
	\end{adjustwidth}
}

\newcommand{\subcase}[1] {
	\begin{adjustwidth}{2.2em}{0pt}
		\underline{Subcase.} #1
	\end{adjustwidth}
}

\newcommand{\stmt}[1] {

\begin{adjustwidth}{2.5em}{0pt}
	#1
\end{adjustwidth}

}
\newcommand{\type}[2]{
	#1~\keyw{with} #2
}

\newcommand{\unit}[0]{ \kwa{unit} }

\newcommand{\Unit}[0]{ \kwa{Unit} }

\newcommand{\arr}[3]{
	#1 \rightarrow_{#3} #2
}

\newcommand{\module}[0]{
\kwa{import}(\varepsilon)~x = \hat e~\kwa{in}~e
}

\newcommand{\newd}[0]{
	\keywadj{new}_d~x \Rightarrow \overline{d = e}
}

\newcommand{\newsig}[0]{
	\keywadj{new}_\sigma~x \Rightarrow \overline{\sigma = e}
}


\begin{document}

\section{Grammar}

\[
\begin{array}{lll}

\begin{array}{lllr}

e & ::= & x & exprs. \\
	& | & r \\
	& | & \lambda x: \tau.e \\
	& | & e ~ e \\
	& | & e.\pi \\
	& | & \kwa{unit} \\
	&&\\

\hat e & ::= & x & labelled~exprs. \\
	& | & r \\
	& | & \lambda x: \hat \tau.\hat e \\
	& | & \hat e ~ \hat e \\
	& | & \hat e.\pi \\
	& | & \kwa{unit} \\
	& | & \kwa{import}(\varepsilon)~x = \hat e~\kwa{in}~e \\
	&&\\

v & ::= & r & values. \\
	& | & \lambda x: \tau.e \\
	& | & \kwa{unit} \\
	&&\\
	
\hat v & ::= & r & labelled~values \\
	& | & \lambda x: \hat \tau.\hat e\\
	& | & \kwa{unit} \\
	&&\\

\end{array}

& ~~~~~~~~&

\begin{array}{lllr}

\varepsilon & ::= & \{ \overline{r.\pi} \} & effects \\
	&&\\

\tau & ::= & \{ \bar r \} & types \\
		& | & \tau \rightarrow \tau \\ 
		& | & \kwa{Unit} \\
		&&\\

\hat \tau & ::= & \{ \bar r \} & labelled ~types \\
		& | & \hat \tau \rightarrow_{\varepsilon} \hat \tau \\
		& | & \kwa{Unit} \\
		&&\\

\Gamma & ::= & \varnothing & type~ctx. \\
				& | & \Gamma, x: \tau \\
				&&\\
				
\hat \Gamma & ::= & \varnothing & labelled~type~ctx.\\
				& | & \hat \Gamma, x: \hat \tau \\
				&&\\

\end{array}

\end{array}
\]

\section{Functions}

\subsection*{Definition ($\kwa{annot :: \tau \times \varepsilon \rightarrow \hat \tau}$)}

\begin{enumerate}
	\item $\kwa{annot}(\{ \bar r \}, \_) = \{ \bar r \}$
	\item $\kwa{annot}(\kwa{Unit}, \_) = \kwa{Unit}$
	\item $\kwa{annot}(\tau_1 \rightarrow \tau_2, \varepsilon) = \kwa{annot}(\tau_1, \varepsilon) \rightarrow_{\varepsilon} \kwa{annot}(\tau_2, \varepsilon)$
\end{enumerate}


\subsection*{Definition ($\kwa{annot :: e \times \varepsilon \rightarrow \hat e}$)}

\begin{enumerate}
	\item $\kwa{annot}(x, \_) = e$
	\item $\kwa{annot}(r, \_) = r$
	\item $\kwa{annot}(\kwa{unit}, \_) = \kwa{unit}$
	\item $\kwa{annot}(e_1 e_2, \varepsilon) = \kwa{annot}(e_1) \kwa{annot}(e_2)$
	\item $\kwa{annot}(e.\pi, \varepsilon) = \kwa{annot}(e).\pi$
	\item $\kwa{annot}(\lambda x: \tau.e, \varepsilon) = \lambda x: \kwa{annot}(\tau, \varepsilon) . \kwa{annot}(e, \varepsilon)$
\end{enumerate}

\subsection*{Definition ($\kwa{annot :: \Gamma \times \varepsilon \rightarrow \hat \Gamma}$)}

\begin{enumerate}
	\item $\kwa{annot}(\varnothing, \_) = \varnothing$
	\item $\kwa{annot}((\Gamma, x: \tau), \varepsilon) = \kwa{annot}(\Gamma, \varepsilon), x: \kwa{annot}(\tau, \varepsilon)$
\end{enumerate}

\subsection*{Definition ($\kwa{erase :: \hat \tau \rightarrow \tau}$)}

\begin{enumerate}
	\item $\kwa{erase}(\{ \bar r \}, \_) = \{ \bar r \}$
	\item $\kwa{erase}(\kwa{Unit}, \_) = \kwa{Unit}$
	\item $\kwa{erase}(\hat \tau_1 \rightarrow_{\varepsilon} \hat \tau_2) = \kwa{erase}(\hat \tau_1) \rightarrow \kwa{erase}(\hat \tau_2)$
\end{enumerate}

\subsection*{Definition ($\kwa{erase :: \hat e \rightarrow e}$)}

\begin{enumerate}
	\item $\kwa{erase}(x) = x$
	\item $\kwa{erase}(r) = r$
	\item $\kwa{erase}(\kwa{unit}) = \kwa{unit}$
	\item $\kwa{erase}(e_1 e_2) = \kwa{erase}(e_1) \kwa{erase}(e_2)$
	\item $\kwa{erase}(e.\pi) = \kwa{erase}(e).\pi$
	\item $\kwa{erase}(\lambda x: \hat \tau.\hat e) = \lambda x: \kwa{erase}(\hat \tau).\kwa{erase}(\hat e)$
\end{enumerate}

\subsection*{Definition ($\kwa{effects :: \tau \rightarrow \varepsilon}$)}

\begin{enumerate}
	\item $\kwa{effects}(\kwa{Unit}) = \varnothing$
	\item $\kwa{effects}(\{ \bar r \}) = \{ r.\pi \mid r \in \bar r, \pi \in \Pi \}$
	\item $\kwa{effects}(\hat \tau_1 \rightarrow_{\varepsilon} \hat \tau_2) = \kwa{ho \hyphen effects}(\hat \tau_1) \cup \varepsilon \cup \kwa{effects}(\hat \tau_2)$
\end{enumerate}

\subsection*{Definition ($\kwa{ho \hyphen effects :: \tau \rightarrow \varepsilon}$)}

\begin{enumerate}
	\item $\kwa{ho \hyphen effects}(\kwa{Unit}) = \varnothing$
	\item $\kwa{ho \hyphen effects}(\{ \bar r \}) = \varnothing$
	\item $\kwa{ho \hyphen effects}(\hat \tau_1 \rightarrow_{\varepsilon} \hat \tau_2) = \kwa{effects}(\tau_1) \cup \kwa{ho \hyphen effects}(\hat \tau_2)$
\end{enumerate}

\section{Static Rules}

\fbox{$\Gamma \vdash e: \tau$}

\[
\begin{array}{c}


\infer[\textsc{(T-Var)}]
	{\Gamma, x: \tau \vdash x: \tau}
	{}
~~~~~~
\infer[\textsc{(T-Resource)}]
	{\Gamma, r: \{ r \} \vdash r : \{ r \}}
	{} \\[4ex]

\infer[\textsc{(T-Unit)}]
	{\Gamma \vdash \kwa{unit} : \kwa{Unit}}
	{}
~~~~~~
\infer[\textsc{(T-Abs)}]
	{\Gamma \vdash \lambda x: \tau_1.e : \tau_1 \rightarrow \tau_2}
	{\Gamma, x: \tau_1 \vdash e: \tau_2}\\[4ex]
	
\infer[\textsc{(T-App)}]
	{\Gamma \vdash e_1~e_2: \tau_3}
	{\Gamma \vdash e_1: \tau_2 \rightarrow \tau_3 & \Gamma \vdash e_2: \tau_2}
~~~~~~
\infer[\textsc{(T-OperCall)}]
	{\Gamma \vdash e.\pi: \kwa{Unit}}
	{\Gamma \vdash e: \{ \bar r \} & \forall r \in \bar r \mid r \in R & \pi \in \Pi}

\end{array}
\]

\noindent
\fbox{$\hat \Gamma \vdash \hat e: \hat \tau~\kw{with} \varepsilon$}

\[
\begin{array}{c}

\infer[\textsc{($\varepsilon$-Var)}]
	{ \Gamma, x:\tau \vdash x: \tau~\kw{with} \varnothing }
	{}
~~~~~~
\infer[\textsc{($\varepsilon$-Resource)}]
 	{ \Gamma, r: \{ r \} \vdash r : \{ r \}~\kw{with} \varnothing }
 	{}\\[4ex]
 	
\infer[\textsc{($\varepsilon$-Unit)}]
	{\Gamma \vdash \kwa{unit} : \kwa{Unit}~\kw{with} \varnothing}
	{}
~~~~~~
	\infer[\textsc{($\varepsilon$-Abs)}]
	{ \Gamma \vdash \lambda x:\tau_2 . \hat e : \tau_2 \rightarrow_{\varepsilon} \tau_3~\kw{with} \varnothing }
	{ \Gamma, x: \tau_2 \vdash \hat e: \tau_3~\kw{with} \varepsilon } \\[4ex]
	
	
\infer[\textsc{($\varepsilon$-App)}]
	{ \Gamma \vdash \hat e_1 \hat e_2 : \tau_3~\kw{with} \varepsilon_1 \cup \varepsilon_2 \cup \varepsilon  }
	{ \Gamma \vdash \hat e_1: \tau_2 \rightarrow_{\varepsilon} \tau_3~\kw{with} \varepsilon_1 & \Gamma \vdash \hat e_2: \tau_2~\kw{with} \varepsilon_2 }
	~~~~~~
	
\infer[\textsc{($\varepsilon$-OperCall)}]
	{ \Gamma \vdash \hat e.\pi: \kw{Unit} \kw{with} \{ \bar r.\pi \} }
	{ \Gamma \vdash \hat e: \{ \bar r \} & \forall r \in \bar r \mid r \in R & \pi \in \Pi }\\[4ex]

\infer[\textsc{($\varepsilon$-Module)}]
	{ \hat \Gamma \vdash \kwa{import}(\varepsilon)~x = \hat e~\kw{in} e: \kwa{annot}(\tau, \varepsilon)~\kw{with} \varepsilon \cup \varepsilon_1 }
{{\def\arraystretch{1.4}
  \begin{array}{c}
\hat \Gamma \vdash \hat e: \hat \tau~\kw{with} \varepsilon_1
~~~~~~
\varepsilon = \kwa{effects}(\hat \tau) \\
\kwa{ho \hyphen safe}(\hat \tau, \varepsilon) ~~~~~~ x: \kwa{erase}(\hat \tau) \vdash e: \tau
  \end{array}}}
 
\end{array}
\]

\noindent
$\fbox{$\kwa{safe(\tau, \varepsilon)}$}$

\[
\begin{array}{c}

\infer[\textsc{(Safe-Resource)}]
	{\kwa{safe}(\{ \bar r \}, \varepsilon)}
	{}
~~~~~
\infer[\textsc{(Safe-Unit)}]
	{\kwa{safe}(\kwa{Unit}, \varepsilon)}
	{} \\[3ex]

\infer[\textsc{(Safe-Arrow)}]
	{\kwa{safe}(\hat \tau_1 \rightarrow_{\varepsilon_2} \hat \tau_2, \varepsilon)}
	{\varepsilon \subseteq \varepsilon_2 & \kwa{ho \hyphen safe}(\hat \tau_1, \varepsilon) & \kwa{safe}(\hat \tau_2, \varepsilon)} \\[3ex]

\end{array}
\]

\noindent
$\fbox{$\kwa{ho \hyphen safe(\hat \tau, \varepsilon)}$}$

\[
\begin{array}{c}

\infer[\textsc{(HOSafe-Resource)}]
	{ \kwa{ho \hyphen safe}( \{ \bar r \}, \varepsilon)} 
	{}
	~~~~~~
\infer[\textsc{(HOSafe-Unit)}]
	{ \kwa{ho \hyphen safe}( \kwa{Unit}, \varepsilon)} 
	{}\\[3ex]

\infer[\textsc{(HOSafe-Arrow)}]
	{ \kwa{ho \hyphen safe}( \hat \tau_1 \rightarrow_{\varepsilon_2} \hat \tau_2, \varepsilon ) }
	{ \kwa{safe}(\hat \tau_1, \varepsilon)  & \kwa{ho \hyphen safe}(\hat \tau_2, \varepsilon) }\\[3ex]

\end{array}
\]

\noindent
$\fbox{$\hat \tau <: \hat \tau$}$

\[
\begin{array}{c}

\infer[(\textsc{S-Effects})]
	{\hat \tau_1 \rightarrow_{\varepsilon} \hat \tau_2 <: \hat \tau_1' \rightarrow_{\varepsilon'} \hat \tau_2'}
	{\varepsilon \subseteq \varepsilon' & \hat \tau_2 <: \hat \tau_2' & \hat \tau_1' <: \hat \tau_1}

\end{array}
\]

\section{Dynamic Rules}

\noindent
\fbox{$\hat e \longrightarrow \hat e~|~\varepsilon$}

\[
\begin{array}{c}

\infer[\textsc{(E-App1)}]
	{\hat e_1 \hat e_2 \longrightarrow \hat e_1' \hat e_2~|~\varepsilon}
	{\hat e_1 \longrightarrow \hat e_1'~|~\varepsilon}
	~~~~~~
\infer[\textsc{(E-App2)}]
	{\hat v_1 \hat e_2 \longrightarrow \hat v_1 \hat e_2'~|~\varepsilon} 
	{\hat e_2 \longrightarrow \hat e_2'~|~\varepsilon}
~~~~~~
\infer[\textsc{(E-App3)}]
	{ (\lambda x: \tau.\hat e) \hat v_2 \longrightarrow [\hat v_2/x]\hat e~|~\varnothing }
	{}\\[4ex]
	
\infer[\textsc{(E-OperCall1)}]
	{\hat e.\pi \longrightarrow \hat e'.\pi~|~\varepsilon }
	{\hat e \rightarrow \hat e'~|~\varepsilon}
		
	~~~~~~
	
\infer[\textsc{(E-OperCall2)}]
	{r.\pi \longrightarrow \kwa{unit}~|~\{ r.\pi \} }
	{r \in R & \pi \in \Pi}
	 \\[4ex]
	 
%	\infer[\textsc{(E-Module1)}]
%	{\kwa{import}(\varepsilon)~x = \hat e~\kw{in} e \longrightarrow \kwa{import}(\varepsilon)~x = \hat e'~\kw{in} e~|~\varepsilon'}
%	{\hat e \longrightarrow \hat e'~|~\varepsilon'}\\[4ex]

\infer[\textsc{(E-Module2)}]
	{\kwa{import}(\varepsilon)~x = \hat v~\kw{in} e \longrightarrow [\hat v/x]\kwa{annot}(e, \varepsilon)~|~\varnothing}
	{}

\end{array}
\]

\section{Proofs}

\begin{lemma}
If $\varepsilon \subseteq \kwa{effects}(\hat \tau)$ and $\kwa{ho \hyphen safe}(\hat \tau, \varepsilon)$ then $\hat \tau <: \kwa{annot}(\kwa{erase}(\hat \tau), \varepsilon)$.
\end{lemma}

\begin{lemma}
If $\varepsilon \subseteq \kwa{ho \hyphen effects}(\hat \tau)$ and $\kwa{safe}(\hat \tau, \varepsilon)$ then $\kwa{annot}(\kwa{erase}(\hat \tau), \varepsilon) <: \hat \tau$.
\end{lemma}

\begin{theorem}
If $\Gamma, x: \kwa{erase}(\hat \tau) \vdash e: \tau$ and $\varepsilon = \kwa{effects}(\hat \tau)$ and $\kwa{ho \hyphen safe}(\hat \tau, \varepsilon)$ then $\kwa{annot}(\Gamma), x: \hat \tau \vdash \kwa{annot}(e, \varepsilon): \kwa{annot}(\tau, \varepsilon)~\kw{with} \varepsilon$.
\end{theorem}

\begin{theorem}[Progress]
If $\Gamma \vdash \hat e_A: \tau_A~\kw{with} \varepsilon_A$ then $\hat e_A$ is a value or $\hat e_A \longrightarrow \hat e_B~|~\varepsilon$.
\end{theorem}

\begin{proof}
By induction on $\Gamma \vdash \hat e_A: \tau_A~\kw{with} \varepsilon_A$.\\

\noindent
\fbox{Case: \textsc{$\varepsilon$-Resource}, \textsc{$\varepsilon$-Unit}, \textsc{$\varepsilon$-Abs}}
Then $\hat e_A$ is a value. \\

\noindent
\fbox{Case: \textsc{$\varepsilon$-App}}
Then $\hat e_A = \hat e_1~\hat e_2$. We consider the cases in which $\hat e_1$ and $\hat e_2$ are values.

If $\hat e_1$ is not a value then by inductive assumption there is a reduction $\hat e_1 \longrightarrow \hat e_1'~|~\varepsilon$. Then $\hat e_1~\hat e_2$ reduces by the rule \textsc{E-App1}, giving $\hat e_1~\hat e_2 \longrightarrow \hat e_1'~\hat e_2~|~\varepsilon$.

If $\hat e_2$ is not a value then WLOG $\hat e_1$ is a value. By inductive assumption $\hat e_2 \longrightarrow \hat e_2'~|~\varepsilon$. Then $\hat v_1~\hat e_2$ reduces by the rule \textsc{E-App2}, giving $\hat v_1~\hat e_2 \longrightarrow \hat v_1~\hat e_2'~|~\varepsilon$.

If $\hat e_1$ and $\hat e_2$ are both values then by canonical forms $\hat e_1 = \hat v_1 = \lambda x: \tau_2.e$. Then $\hat v_1~\hat v_2$ reduces by the rule \textsc{E-App3}, giving $\hat v_1~\hat v_2 \longrightarrow [\hat v_2/x]\hat e ~|~\varnothing$. \\

\noindent
\fbox{Case: \textsc{$\varepsilon$-OperCall}} Then $\hat e_A = \hat e_1.\pi$. We consider whether $\hat e_1$ is a value.

If $\hat e_1$ is not a value then by inductive assumption there is a reduction $\hat e_1 \longrightarrow \hat e_1'~|~\varepsilon$. Then $\hat e_1.\pi$ reduces by the rule \textsc{E-OperCall1}, giving $\hat e_1.\pi \longrightarrow \hat e_1'.\pi~|~\varepsilon$.

If $\hat e_1$ is a value then $\hat e_1 = r$ by canonical forms. By the assumption that $r.\pi$ is closed under $\Gamma$, we know $r \in R$ and $\pi \in \Pi$. Then $\hat e_1.\pi$ reduces by the rule \textsc{E-OperCall2}, giving $r.\pi \longrightarrow \kwa{unit}~|~\varepsilon$. \\

\noindent
\fbox{Case: \textsc{$\varepsilon$-Module}}
Then $e_A = \module$ which reduces by the rule \textsc{E-Module}, giving $\module \longrightarrow [\hat v/x]\kwa{annot}(e, \varepsilon)~|~\varnothing$.

\end{proof}

\end{document}










