\documentclass{llncs}

\usepackage{listings}
\usepackage{proof}
\usepackage{amssymb}
\usepackage[margin=.9in]{geometry}
\usepackage{amsmath}
\usepackage[english]{babel}
\usepackage[utf8]{inputenc}
\usepackage{enumitem}
\usepackage{filecontents}
\usepackage{calc}
\usepackage[linewidth=0.5pt]{mdframed}
\usepackage{changepage}
\usepackage{tabto}
\allowdisplaybreaks

\usepackage{fancyhdr}
\renewcommand{\headrulewidth}{0pt}
\pagestyle{fancy}
 \fancyhf{}
\rhead{\thepage}

\lstset{tabsize=3, basicstyle=\ttfamily\small, commentstyle=\itshape\rmfamily, numbers=left, numberstyle=\tiny, language=java,moredelim=[il][\sffamily]{?},mathescape=true,showspaces=false,showstringspaces=false,columns=fullflexible,xleftmargin=5pt,escapeinside={(@}{@)}, morekeywords=[1]{objtype,module,import,let,in,fn,var,type,rec,fold,unfold,letrec,alloc,ref,application,policy,external,component,connects,to,meth,val,where,return,group,by,within,count,connect,with,attr,html,head,title,style,body,div,keyword,unit,def}}
\lstloadlanguages{Java,VBScript,XML,HTML}

\newcommand{\keywadj}[1]{\mathtt{#1}}
\newcommand{\keyw}[1]{\keywadj{#1}~}

\newcommand{\kw}[1]{\keyw{ #1 }}
\newcommand{\kwa}[1]{\keywadj{ #1 }}
\newcommand{\reftt}{\mathtt{ref}~}
\newcommand{\Reftt}{\mathtt{Ref}~}
\newcommand{\inttt}{\mathtt{int}~}
\newcommand{\Inttt}{\mathtt{Int}~}
\newcommand{\stepsto}{\leadsto}
\newcommand{\todo}[1]{\textbf{[#1]}}
\newcommand{\intuition}[1]{#1}
\newcommand{\hyphen}{\hbox{-}}

%\newcommand{\intuition}[1]{}

\newlist{pcases}{enumerate}{1}
\setlist[pcases]{
  label=\fbox{\textit{Case}}\protect\thiscase\textit{:}~,
  ref=\arabic*,
  align=left,
  labelsep=0pt,
  leftmargin=0pt,
  labelwidth=0pt,
  parsep=0pt
}
\newcommand{\pcase}[1][]{

  \if\relax\detokenize{#1}\relax
    \def\thiscase{}
  \else
    \def\thiscase{~\fbox{#1:}}
  \fi
  \item
}

\newcommand{\thm}[3]{
	\begin{large}
		\bf{#1}
	\end{large} \\\\
	\fbox{Statement.} ~ #2
	\fbox{Proof.}~ #3 \qed
}

\newcommand{\proofcase}[2]{
	\begin{adjustwidth}{1.5em}{0pt}
		\fbox{Case.}~~#1. \\ ~#2
	\end{adjustwidth}
}

\newcommand{\subcase}[1] {
	\begin{adjustwidth}{2.2em}{0pt}
		\underline{Subcase.} #1
	\end{adjustwidth}
}

\newcommand{\stmt}[1] {

\begin{adjustwidth}{2.5em}{0pt}
	#1
\end{adjustwidth}

}
\newcommand{\type}[2]{
	#1~\keyw{with} #2
}

\newcommand{\unit}[0]{ \kwa{unit} }

\newcommand{\Unit}[0]{ \kwa{Unit} }

\newcommand{\arr}[3]{
	#1 \rightarrow_{#3} #2
}

\newcommand{\module}[0]{
\kwa{import}(\varepsilon)~x = \hat e~\kwa{in}~e
}

\newcommand{\newd}[0]{
	\keywadj{new}_d~x \Rightarrow \overline{d = e}
}

\newcommand{\newsig}[0]{
	\keywadj{new}_\sigma~x \Rightarrow \overline{\sigma = e}
}


\begin{document}

\section{Grammar}

\[
\begin{array}{lll}

\begin{array}{lllr}

e & ::= & x & exprs. \\
	& | & r \\
	& | & \lambda x: \tau.e \\
	& | & e ~ e \\
	& | & e.\pi \\
	&&\\

\hat e & ::= & x & labelled~exprs. \\
	& | & r \\
	& | & \lambda x: \hat \tau.\hat e \\
	& | & \hat e ~ \hat e \\
	& | & \hat e.\pi \\
	& | & \kwa{import}(\varepsilon)~x = \hat e~\kwa{in}~e \\
	&&\\

v & ::= & r & values. \\
	& | & \lambda x: \tau.e \\
	&&\\
	
\hat v & ::= & r & labelled~values \\
	& | & \lambda x: \hat \tau.\hat e\\
	&&\\

\end{array}

& ~~~~~~~~&

\begin{array}{lllr}

\varepsilon & ::= & \{ \overline{r.\pi} \} & effects \\
	&&\\

\tau & ::= & \{ \bar r \} & types \\
		& | & \tau \rightarrow \tau \\ 
		&&\\

\hat \tau & ::= & \{ \bar r \} & labelled ~types \\
		& | & \hat \tau \rightarrow_{\varepsilon} \hat \tau \\
		&&\\

\Gamma & ::= & \varnothing & type~ctx. \\
				& | & \Gamma, x: \tau \\
				&&\\
				
\hat \Gamma & ::= & \varnothing & labelled~type~ctx.\\
				& | & \hat \Gamma, x: \hat \tau \\
				&&\\

\end{array}

\end{array}
\]

\section{Functions}

\subsection*{Definition ($\kwa{annot :: \tau \times \varepsilon \rightarrow \hat \tau}$)}

\begin{enumerate}
	\item $\kwa{annot}(\{ \bar r \}, \_) = \{ \bar r \}$
	\item $\kwa{annot}(\tau_1 \rightarrow \tau_2, \varepsilon) = \kwa{annot}(\tau_1, \varepsilon) \rightarrow_{\varepsilon} \kwa{annot}(\tau_2, \varepsilon)$
\end{enumerate}


\subsection*{Definition ($\kwa{annot :: e \times \varepsilon \rightarrow \hat e}$)}

\begin{enumerate}
	\item $\kwa{annot}(x, \_) = e$
	\item $\kwa{annot}(r, \_) = r$
	\item $\kwa{annot}(e_1 e_2, \varepsilon) = \kwa{annot}(e_1) \kwa{annot}(e_2)$
	\item $\kwa{annot}(e.\pi, \varepsilon) = \kwa{annot}(e).\pi$
	\item $\kwa{annot}(\lambda x: \tau.e, \varepsilon) = \lambda x: \kwa{annot}(\tau, \varepsilon) . \kwa{annot}(e, \varepsilon)$
\end{enumerate}

\subsection*{Definition ($\kwa{annot :: \Gamma \times \varepsilon \rightarrow \hat \Gamma}$)}

\begin{enumerate}
	\item $\kwa{annot}(\varnothing, \_) = \varnothing$
	\item $\kwa{annot}((\Gamma, x: \tau), \varepsilon) = \kwa{annot}(\Gamma, \varepsilon), x: \kwa{annot}(\tau, \varepsilon)$
\end{enumerate}

\subsection*{Definition ($\kwa{erase :: \hat \tau \rightarrow \tau}$)}

\begin{enumerate}
	\item $\kwa{erase}(\{ \bar r \}, \_) = \{ \bar r \}$
	\item $\kwa{erase}(\hat \tau_1 \rightarrow_{\varepsilon} \hat \tau_2) = \kwa{erase}(\hat \tau_1) \rightarrow \kwa{erase}(\hat \tau_2)$
\end{enumerate}

\subsection*{Definition ($\kwa{erase :: \hat e \rightarrow e}$)}

\begin{enumerate}
	\item $\kwa{erase}(x) = x$
	\item $\kwa{erase}(r) = r$
	\item $\kwa{erase}(e_1 e_2) = \kwa{erase}(e_1) \kwa{erase}(e_2)$
	\item $\kwa{erase}(e.\pi) = \kwa{erase}(e).\pi$
	\item $\kwa{erase}(\lambda x: \hat \tau.\hat e) = \lambda x: \kwa{erase}(\hat \tau).\kwa{erase}(\hat e)$
\end{enumerate}

\subsection*{Definition ($\kwa{effects :: \tau \rightarrow \varepsilon}$)}

\begin{enumerate}
	\item $\kwa{effects}(\{ \bar r \}) = \{ r.\pi \mid r \in \bar r, \pi \in \Pi \}$
	\item $\kwa{effects}(\hat \tau_1 \rightarrow_{\varepsilon} \hat \tau_2) = \kwa{ho \hyphen effects}(\hat \tau_1) \cup \varepsilon \cup \kwa{effects}(\hat \tau_2)$
\end{enumerate}

\noindent
This function computes those effects:
\begin{itemize}
	\item Directly invoked by a function of type $\hat \tau$.
	\item Captured by functions created/returned by $\hat \tau$.
\end{itemize}

\subsection*{Definition ($\kwa{ho \hyphen effects :: \tau \rightarrow \varepsilon}$)}

\begin{enumerate}
	\item $\kwa{ho \hyphen effects}(\{ \bar r \}) = \varnothing$
	\item $\kwa{ho \hyphen effects}(\hat \tau_1 \rightarrow_{\varepsilon} \hat \tau_2) = \kwa{effects}(\hat \tau_1) \cup \kwa{ho \hyphen effects}(\hat \tau_2)$
\end{enumerate}

\noindent
This function computes those effects:
\begin{itemize}
	\item Captured by functions passed into $\hat \tau$.
\end{itemize}

\subsection*{Examples}

Suppose $a$ is a base type that captures no effects. \\

\noindent
Consider $\hat \tau = (a \rightarrow_b (a \rightarrow_c a)) \rightarrow_d (a \rightarrow_e a)$.

$\kwa{effects}(\hat \tau) = \{ d, e \}$

$\kwa{ho \hyphen effects}(\hat \tau) = \{ b, c \}$ \\

\noindent
Consider $\hat \tau = ((a \rightarrow_b a) \rightarrow_c (a \rightarrow_d a)) \rightarrow_e ((a \rightarrow_f a) \rightarrow_g (a \rightarrow_h a))$

$\kwa{effects}(\hat \tau) = \{ e, b, g, h \}$

$\kwa{ho \hyphen effects}(\hat \tau) = \{ c, d, f \}$

\section{Static Rules}

\fbox{$\Gamma \vdash e: \tau$}

\[
\begin{array}{c}


\infer[\textsc{(T-Var)}]
	{\Gamma, x: \tau \vdash x: \tau}
	{}
~~~~~~
\infer[\textsc{(T-Resource)}]
	{\Gamma, r: \{ r \} \vdash r : \{ r \}}
	{}

~~~~~~
\infer[\textsc{(T-Abs)}]
	{\Gamma \vdash \lambda x: \tau_1.e : \tau_1 \rightarrow \tau_2}
	{\Gamma, x: \tau_1 \vdash e: \tau_2}\\[4ex]
	
\infer[\textsc{(T-App)}]
	{\Gamma \vdash e_1~e_2: \tau_3}
	{\Gamma \vdash e_1: \tau_2 \rightarrow \tau_3 & \Gamma \vdash e_2: \tau_2}
~~~~~~
\infer[\textsc{(T-OperCall)}]
	{\Gamma \vdash e.\pi: \kwa{Unit}}
	{\Gamma \vdash e: \{ \bar r \} & \forall r \in \bar r \mid r \in R & \pi \in \Pi}

\end{array}
\]

\noindent
\fbox{$\hat \Gamma \vdash \hat e: \hat \tau~\kw{with} \varepsilon$}

\[
\begin{array}{c}

\infer[\textsc{($\varepsilon$-Var)}]
	{ \hat \Gamma, x:\tau \vdash x: \tau~\kw{with} \varnothing }
	{}
~~~~~~
\infer[\textsc{($\varepsilon$-Resource)}]
 	{ \hat \Gamma, r: \{ r \} \vdash r : \{ r \}~\kw{with} \varnothing }
 	{}\\[4ex]
 	
~~~~~~
	\infer[\textsc{($\varepsilon$-Abs)}]
	{ \hat \Gamma \vdash \lambda x:\tau_2 . \hat e : \hat \tau_2 \rightarrow_{\varepsilon_3} \hat \tau_3~\kw{with} \varnothing }
	{ \hat \Gamma, x: \hat \tau_2 \vdash \hat e: \hat \tau_3~\kw{with} \varepsilon_3 }
	
	~~~~~~
	
\infer[\textsc{($\varepsilon$-App)}]
	{ \hat \Gamma \vdash \hat e_1 \hat e_2 : \hat \tau_3~\kw{with} \varepsilon_1 \cup \varepsilon_2 \cup \varepsilon  }
	{ \hat \Gamma \vdash \hat e_1: \hat \tau_2 \rightarrow_{\varepsilon} \hat \tau_3~\kw{with} \varepsilon_1 & \hat \Gamma \vdash \hat e_2: \hat \tau_2~\kw{with} \varepsilon_2 } \\[4ex]
	
\infer[\textsc{($\varepsilon$-OperCall)}]
	{ \hat \Gamma \vdash \hat e.\pi: \kw{Unit} \kw{with} \{ \bar r.\pi \} }
	{ \hat \Gamma \vdash \hat e: \{ \bar r \} & \forall r \in \bar r \mid r: \{ r \} \in \Gamma & \pi \in \Pi }
	~~~~~~

\infer[\textsc{($\varepsilon$-Subsume)}]
	{ \hat \Gamma \vdash e: \tau' ~\kw{with} \varepsilon'}
	{ \hat \Gamma \vdash e: \tau ~\kw{with} \varepsilon & \tau' <: \tau & \varepsilon' \subseteq \varepsilon}\\[4ex]

\infer[\textsc{($\varepsilon$-Module)}]
	{ \hat \Gamma \vdash \kwa{import}(\varepsilon)~x = \hat e~\kw{in} e: \kwa{annot}(\tau, \varepsilon)~\kw{with} \varepsilon \cup \varepsilon_1 }
{{\def\arraystretch{1.4}
  \begin{array}{c}
\hat \Gamma \vdash \hat e: \hat \tau~\kw{with} \varepsilon_1
~~~~~~
\varepsilon = \kwa{effects}(\hat \tau) \\
\kwa{ho \hyphen safe}(\hat \tau, \varepsilon) ~~~~~~ x: \kwa{erase}(\hat \tau) \vdash e: \tau
  \end{array}}}
 
\end{array}
\]

\noindent
$\fbox{$\kwa{safe(\tau, \varepsilon)}$}$

\[
\begin{array}{c}

\infer[\textsc{(Safe-Resource)}]
	{\kwa{safe}(\{ \bar r \}, \varepsilon)}
	{}
~~~~~
\infer[\textsc{(Safe-Unit)}]
	{\kwa{safe}(\kwa{Unit}, \varepsilon)}
	{} \\[3ex]

\infer[\textsc{(Safe-Arrow)}]
	{\kwa{safe}(\hat \tau_1 \rightarrow_{\varepsilon'} \hat \tau_2, \varepsilon)}
	{\varepsilon \subseteq \varepsilon' & \kwa{ho \hyphen safe}(\hat \tau_1, \varepsilon) & \kwa{safe}(\hat \tau_2, \varepsilon)} \\[3ex]

\end{array}
\]

\noindent
$\fbox{$\kwa{ho \hyphen safe(\hat \tau, \varepsilon)}$}$

\[
\begin{array}{c}

\infer[\textsc{(HOSafe-Resource)}]
	{ \kwa{ho \hyphen safe}( \{ \bar r \}, \varepsilon)} 
	{}
	~~~~~~
\infer[\textsc{(HOSafe-Unit)}]
	{ \kwa{ho \hyphen safe}( \kwa{Unit}, \varepsilon)} 
	{}\\[3ex]

\infer[\textsc{(HOSafe-Arrow)}]
	{ \kwa{ho \hyphen safe}( \hat \tau_1 \rightarrow_{\varepsilon'} \hat \tau_2, \varepsilon ) }
	{ \kwa{safe}(\hat \tau_1, \varepsilon)  & \kwa{ho \hyphen safe}(\hat \tau_2, \varepsilon) }\\[3ex]

\end{array}
\]

\noindent
$\fbox{$\hat \tau <: \hat \tau$}$

\[
\begin{array}{c}

\infer[(\textsc{S-Effects})]
	{\hat \tau_1 \rightarrow_{\varepsilon} \hat \tau_2 <: \hat \tau_1' \rightarrow_{\varepsilon'} \hat \tau_2'}
	{\varepsilon \subseteq \varepsilon' & \hat \tau_2 <: \hat \tau_2' & \hat \tau_1' <: \hat \tau_1}

\end{array}
\]

\section{Dynamic Rules}

\noindent
\fbox{$\hat e \longrightarrow \hat e~|~\varepsilon$}

\[
\begin{array}{c}

\infer[\textsc{(E-App1)}]
	{\hat e_1 \hat e_2 \longrightarrow \hat e_1' \hat e_2~|~\varepsilon}
	{\hat e_1 \longrightarrow \hat e_1'~|~\varepsilon}
	~~~~~~
\infer[\textsc{(E-App2)}]
	{\hat v_1 \hat e_2 \longrightarrow \hat v_1 \hat e_2'~|~\varepsilon} 
	{\hat e_2 \longrightarrow \hat e_2'~|~\varepsilon}
~~~~~~
\infer[\textsc{(E-App3)}]
	{ (\lambda x: \hat \tau.\hat e) \hat v_2 \longrightarrow [\hat v_2/x]\hat e~|~\varnothing }
	{}\\[4ex]
	
\infer[\textsc{(E-OperCall1)}]
	{\hat e.\pi \longrightarrow \hat e'.\pi~|~\varepsilon }
	{\hat e \rightarrow \hat e'~|~\varepsilon}
		
	~~~~~~
	
\infer[\textsc{(E-OperCall2)}]
	{r.\pi \longrightarrow \kwa{unit}~|~\{ r.\pi \} }
	{r \in R & \pi \in \Pi}
	 \\[4ex]
	 
\infer[\textsc{(E-Module1)}]
	{\kwa{import}(\varepsilon)~x = \hat e~\kw{in} e \longrightarrow \kwa{import}(\varepsilon)~x = \hat e'~\kw{in} e~|~\varepsilon'}
	{\hat e \longrightarrow \hat e'~|~\varepsilon'}\\[4ex]

\infer[\textsc{(E-Module2)}]
	{\kwa{import}(\varepsilon)~x = \hat v~\kw{in} e \longrightarrow [\hat v/x]\kwa{annot}(e, \varepsilon)~|~\varnothing}
	{}

\end{array}
\]


\section{Encodings}

\subsection{$\bot$}
We can define the bottom type as $\kwa{\bot = \{\}}$, because there is no empty-set literal.

\subsection{$\unit,~\Unit$}

\noindent
Define $\kwa{unit = \lambda x: \{\}.x}$, i.e. the function which takes an empty set of resources and returns it. We shall refer to its type, which is $\kwa{\{\} \rightarrow_{\varnothing} \{\}}$, as $\Unit$. It has various properties befitting $\unit$.

\begin{enumerate}
	\item $\unit$ cannot be invoked, as $\{\}$ is uninhabited.
	\item $\unit$ is a value.
	\item The only term with type $\Unit$ is $\unit$.
	\item $\vdash \unit : \Unit$, by using \textsc{$\varepsilon$-Abs} and \textsc{$\varepsilon$-Var}.
	\item $\kwa{effects}(\Unit) = \kwa{ho \hyphen effects}(\Unit) = \varnothing$
	\item $\kwa{safe}(\Unit, \varepsilon)$ and $\kwa{ho \hyphen safe}(\Unit, \varepsilon)$
\end{enumerate}

\section{Proofs}

\begin{theorem}[Progress]
If $\hat \Gamma \vdash \hat e_A: \hat \tau_A~\kw{with} \varepsilon_A$ then $\hat e_A$ is a value or $\hat e_A \longrightarrow \hat e_B~|~\varepsilon$.
\end{theorem}

\begin{proof}
By induction on $\hat \Gamma \vdash \hat e_A: \hat \tau_A~\kw{with} \varepsilon_A$.\\

\noindent
\fbox{Case: \textsc{$\varepsilon$-Resource}, \textsc{$\varepsilon$-Unit}, \textsc{$\varepsilon$-Abs}}
Then $\hat e_A$ is a value. \\

\noindent
\fbox{Case: \textsc{$\varepsilon$-Subsume}}
Then $\hat \Gamma \vdash e: \tau'~\kw{with} \varepsilon'$, and $\hat \Gamma \vdash e: \tau~\kw{with} \varepsilon$, where $\tau' <: \tau$ and $\varepsilon' \subseteq \varepsilon$ are subderivations. The theorem conclusion holds by inductive assumption applied to $\hat \Gamma \vdash e: \tau~\kw{with} \varepsilon$.\\

\noindent
\fbox{Case: \textsc{$\varepsilon$-App}}
Then $\hat e_A = \hat e_1~\hat e_2$. We consider the cases in which $\hat e_1$ and $\hat e_2$ are values.

If $\hat e_1$ is not a value then by inductive assumption there is a reduction $\hat e_1 \longrightarrow \hat e_1'~|~\varepsilon$. Then $\hat e_1~\hat e_2$ reduces by the rule \textsc{E-App1}, giving $\hat e_1~\hat e_2 \longrightarrow \hat e_1'~\hat e_2~|~\varepsilon$.

If $\hat e_2$ is not a value then WLOG $\hat e_1$ is a value. By inductive assumption $\hat e_2 \longrightarrow \hat e_2'~|~\varepsilon$. Then $\hat v_1~\hat e_2$ reduces by the rule \textsc{E-App2}, giving $\hat v_1~\hat e_2 \longrightarrow \hat v_1~\hat e_2'~|~\varepsilon$.

If $\hat e_1$ and $\hat e_2$ are both values then by canonical forms $\hat e_1 = \hat v_1 = \lambda x: \tau_2.e$. Then $\hat v_1~\hat v_2$ reduces by the rule \textsc{E-App3}, giving $\hat v_1~\hat v_2 \longrightarrow [\hat v_2/x]\hat e ~|~\varnothing$. \\

\noindent
\fbox{Case: \textsc{$\varepsilon$-OperCall}} Then $\hat e_A = \hat e_1.\pi$. We consider whether $\hat e_1$ is a value.

If $\hat e_1$ is not a value then by inductive assumption there is a reduction $\hat e_1 \longrightarrow \hat e_1'~|~\varepsilon$. Then $\hat e_1.\pi$ reduces by the rule \textsc{E-OperCall1}, giving $\hat e_1.\pi \longrightarrow \hat e_1'.\pi~|~\varepsilon$.

If $\hat e_1$ is a value then $\hat e_1 = r$ by canonical forms. By the assumption that $r.\pi$ is closed under $\Gamma$, we know $r \in R$ and $\pi \in \Pi$. Then $\hat e_1.\pi$ reduces by the rule \textsc{E-OperCall2}, giving $r.\pi \longrightarrow \kwa{unit}~|~\varepsilon$. \\

\noindent
\fbox{Case: \textsc{$\varepsilon$-Module}}
Then $e_A = \module$. If $\hat e$ is an expression then it can be reduced, so $\hat e \longrightarrow \hat e' ~|~\varepsilon'$, and so by \textsc{E-Module1} we get $\module \longrightarrow \kwa{import}(\varepsilon)~x = \hat e'~\kw{in} e ~|~\varepsilon'$. Otherwise $\hat e = \hat v$ is a value. Then by \textsc{E-Module2} we get $\kwa{import}(\varepsilon)~x = \hat v \longrightarrow [\hat v/x]\kwa{annot}(e, \varepsilon)~|~\varnothing$.
\end{proof}

\hrulefill

\begin{lemma}[Substitution]
If $\hat \Gamma, x: \hat \tau' \vdash e: \hat \tau~\kw{with} \varepsilon$ and $\hat \Gamma \vdash v: \hat \tau'~\kw{with} \varnothing$ then $\hat \Gamma \vdash [v/x]e: \hat \tau~\kw{with} \varepsilon$.
\end{lemma}

\hrulefill

\begin{lemma}
$\kwa{ho \hyphen safe}(\hat \tau_1 \rightarrow_{\varepsilon'} \hat \tau_2, \varepsilon) \implies \varepsilon \subseteq \kwa{ho \hyphen effects}(\hat \tau_1 \rightarrow_{\varepsilon'} \hat \tau_2)$
\end{lemma}

\begin{lemma}
$\kwa{safe}(\hat \tau_1 \rightarrow_{\varepsilon'} \hat \tau_2, \varepsilon) \implies \varepsilon \subseteq \kwa{effects}(\hat \tau_1 \rightarrow_{\varepsilon'} \hat \tau_2)$
\end{lemma}

\begin{proof}
By simultaneous induction on derivations. \\

\noindent
\fbox{$\kwa{ho \hyphen safe}(\hat \tau_1 \rightarrow_{\varepsilon'} \hat \tau_2, \varepsilon)$} By definition, $\kwa{safe}(\hat \tau_1, \varepsilon)$ and $\kwa{ho \hyphen safe}(\hat \tau_2, \varepsilon)$. By applying inductive assumptions, $\varepsilon \subseteq \kwa{effects}(\hat \tau_1)$ and $\varepsilon \subseteq \kwa{ho \hyphen effects}(\hat \tau_2)$. Then $\varepsilon \subseteq \kwa{effects}(\hat \tau_1) \cup \kwa{ho \hyphen effects}(\hat \tau_2) = \kwa{ho \hyphen effects}(\hat \tau_1 \rightarrow_{\varepsilon'} \hat \tau_2)$. \\

\noindent
\fbox{$\kwa{safe}(\hat \tau_1 \rightarrow_{\varepsilon'} \hat \tau_2, \varepsilon)$} By definition, $\varepsilon \subseteq \varepsilon'$ and $\kwa{ho \hyphen safe}(\hat \tau_1, \varepsilon)$ and $\kwa{safe}(\hat \tau_2, \varepsilon)$. By applying inductive assumptions, $\varepsilon \subseteq \kwa{ho \hyphen effects}(\hat \tau_1)$ and $\varepsilon \subseteq \kwa{effects}(\hat \tau_2)$. Then $\varepsilon \subseteq \varepsilon' \cup \kwa{ho \hyphen effects}(\hat \tau_1) \cup \kwa{effects}(\hat \tau_2) = \kwa{effects}(\hat \tau_1 \rightarrow_{\varepsilon'} \hat \tau_2)$.
\end{proof}

\hrulefill

\begin{lemma}
If $\varepsilon \subseteq \kwa{effects}(\hat \tau)$ and $\kwa{ho \hyphen safe}(\hat \tau, \varepsilon)$ then $\hat \tau <: \kwa{annot}(\kwa{erase}(\hat \tau), \varepsilon)$.
\end{lemma}

\begin{lemma}
If $\varepsilon \subseteq \kwa{ho \hyphen effects}(\hat \tau)$ and $\kwa{safe}(\hat \tau, \varepsilon)$ then $\kwa{annot}(\kwa{erase}(\hat \tau), \varepsilon) <: \hat \tau$.
\end{lemma}

\begin{proof}

By simultaneous induction on derivations.\\

\noindent
\fbox{Case: $\hat \tau = \{ \bar r \}$}
Then $\kwa{annot(erase(\hat \tau), \varepsilon)} = \hat \tau$.\\

\noindent
\fbox{Case: $\hat \tau = \hat \tau_1 \rightarrow_{\varepsilon'} \hat \tau_2$, $\varepsilon \subseteq \kwa{effects}(\hat \tau)$, $\kwa{ho \hyphen safe}(\hat \tau, \varepsilon)$} By definition of $\kwa{annot}$ and $\kwa{erase}$ we know: 

\noindent
 $\kwa{annot(erase(\hat \tau_1 \rightarrow_{\varepsilon'} \hat \tau_2), \varepsilon = annot(erase(\hat \tau_1), \varepsilon) \rightarrow_{\varepsilon} annot(erase(\hat \tau_2), \varepsilon)}$ \\

\noindent
By definition we have the following subderivations.
\begin{enumerate}
	\item $\kwa{effects}(\hat \tau_1 \rightarrow_{\varepsilon'} \hat \tau_2) = \kwa{ho \hyphen effects}(\hat \tau_1)~\cup~\varepsilon'~\cup~\kwa{effects}(\hat \tau_2)$
	\item  $\kwa{ho \hyphen safe}(\hat \tau_1 \rightarrow_{\varepsilon'} \hat \tau_2, \varepsilon) = \kwa{safe}(\hat \tau_1, \varepsilon) \land \kwa{ho \hyphen safe}(\hat \tau_2, \varepsilon)$
\end{enumerate}

\noindent
$\kwa{safe}(\hat \tau_1, \varepsilon)$ is a subderivation of $\kwa{ho \hyphen safe}(\hat \tau_1 \rightarrow_{\varepsilon'} \hat \tau_2)$. By \textbf{Lemma 2}, $\kwa{ho \hyphen safe}(\hat \tau_1 \rightarrow_{\varepsilon'} \hat \tau_2)$ implies $\varepsilon \subseteq \kwa{ho \hyphen effects}(\hat \tau_1 \rightarrow_{\varepsilon'} \hat \tau_2) = \kwa{effects}(\hat \tau_1) \cup \kwa{ho \hyphen effects}(\hat \tau_2)$.


\noindent
\fbox{Case: $\hat \tau = \hat \tau_1 \rightarrow_{\varepsilon'} \hat \tau_2$, $\varepsilon \subseteq \kwa{ho \hyphen effects}(\hat \tau)$, $\kwa{safe}(\hat \tau, \varepsilon)$}

\noindent
$\kwa{ho \hyphen effects}(\hat \tau_1 \rightarrow_{\varepsilon'} \hat \tau_2) = \kwa{effects}(\hat \tau_1) \cup \kwa{ho \hyphen effects}(\hat \tau_2)$

\noindent
$\kwa{safe}(\hat \tau_1 \rightarrow_{\varepsilon'} \hat \tau_2) =
\varepsilon \subseteq \varepsilon' \land \kwa{ho \hyphen safe}(\hat \tau_1, \varepsilon) \land \kwa{safe}(\hat \tau_2, \varepsilon)$

\end{proof}


\hrulefill

\begin{theorem}[Preservation]
If $\hat \Gamma \vdash \hat e_A: \hat \tau_A~\kw{with} \varepsilon_A$ and $e_A \longrightarrow e_B~|~\varepsilon$, then $\hat \Gamma \vdash e_B: \tau_B~\kw{with} \varepsilon_B$, where $e_B <: e_B$ and $\varepsilon \cup \varepsilon_B \subseteq \varepsilon_A$.
\end{theorem}

\begin{proof}
By induction on $\hat \Gamma \vdash \hat e_A: \tau_A~\kw{with }\varepsilon_A$, and then on $e_A \longrightarrow e_B~|~\varepsilon$. \\

\noindent
\fbox{\textsc{$\varepsilon$-Var}, \textsc{$\varepsilon$-Resource}, \textsc{$\varepsilon$-Unit}, \textsc{$\varepsilon$-Abs}} Then $e_A$ cannot be reduced and so the theorem statement vacuously holds. \\

\noindent
\fbox{\textsc{$\varepsilon$-App}}
Then $e_A = \hat e_1 \hat e_2$ and $\hat e_1: \hat \tau_2 \rightarrow_{\varepsilon} \hat \tau_3~\kw{with} \varepsilon_1$ and $\hat \Gamma \vdash \hat e_2: \hat \tau_2~\kw{with} \varepsilon_2$. If the reduction rule used was \textsc{E-App1} or \textsc{E-App2}, then the result follows by applying the inductive hypothesis to $\hat e_1$ and $\hat e_2$ respectively.

Otherwise the rule used was \textsc{E-App3}. Then $(\lambda x: \hat \tau_2.\hat e)\hat v_2 \longrightarrow [\hat v_2/x]\hat e~|~\varnothing$. By inversion on the typing rule for $\lambda x: \hat \tau_2.\hat e$ we know $\Gamma, x: \hat \tau_2 \vdash \hat e: \hat \tau_3~\kw{with} \varepsilon_3$. By canonical forms, $\varepsilon_2 = \varnothing$ because $\hat e_2 = \hat v_2$ is a value. Then by the substitution lemma, $\hat \Gamma \vdash [\hat v_2/x]\hat e : \hat \tau_3~\kw{with} \varepsilon_3$. By canonical forms, $\varepsilon_1 = \varepsilon_2 = \varnothing = \varepsilon$. Therefore $\varepsilon_A = \varepsilon_3 = \varepsilon_B \cup \varepsilon$.\\

\noindent
\fbox{$\varepsilon$-OperCall}
Then $e_A = e_1.\pi$ and $\hat \Gamma \vdash e_1 : \{ \bar r \}~\kw{with} \varepsilon_1$. If the reduction rule used was \textsc{E-OperCall1} then the result follows by applying the inductive hypothesis to $\hat e_1$.

Otherwise the reduction rule used was \textsc{E-OperCall2} and $v_1.\pi \longrightarrow \kwa{unit}~|~\{ r.\pi \}$. By canonical forms, $\hat \Gamma \vdash v_1: \kwa{unit}~\kw{with} \{ r.\pi \}$. Also, $\hat \Gamma \vdash \kwa{unit}: \kwa{Unit}~\kw{with} \varnothing$. Then $\tau_B = \tau_A$. Also, $\varepsilon \cup \varepsilon_B = \{ r.\pi \} = \varepsilon_A$.\\

\noindent
\fbox{$\varepsilon$-Module}
Then $e_A = \module$. If the reduction rule used was \textsc{E-ModuleCall1} then the result follows by applying the inductive hypothesis to $\hat e$.

Otherwise the following are true:
\begin{enumerate}
	\item $e_A = \kwa{import}(\varepsilon)~x = \hat v~\kw{in} e$
	\item $\hat \Gamma \vdash e_A: \kwa{annot}(\tau, \varepsilon)~\kw{with} \varepsilon \cup \varepsilon_1$
	\item $\kwa{import}(\varepsilon)~x = \hat v~\kw{in} e \longrightarrow [\hat v/x]\kwa{annot}(e, \varepsilon)~|~\varnothing$
	\item $\hat \Gamma \vdash \hat v: \hat \tau~\kw{with} \varnothing$
	\item $\varepsilon = \kwa{effects}(\hat \tau)$
	\item $\kwa{ho \hyphen safe}(\hat \tau, \varepsilon)$
	\item $x: \kwa{erase}(\hat \tau) \vdash e: \tau$
\end{enumerate}

\end{proof}

\noindent
Use the \textbf{annotation lemma} to get $\hat \Gamma, x: \hat \tau \vdash \kwa{annot}(e, \varepsilon): \kwa{annot}(\tau, \varepsilon)~\kw{with} \varepsilon$. \\

\noindent
By \textbf{3} we have $\hat \Gamma \vdash \hat v: \hat \tau~\kw{with} \varnothing$. \\

\noindent
By \textbf{substitution lemma}, $\hat \Gamma \vdash [\hat v/x]\kwa{annot}(e, \varepsilon): \kwa{annot}(\tau, \varepsilon)~\kw{with} \varepsilon$.

\hrulefill

\begin{lemma}[Annotation]
If the following are true for every $\Gamma$:

\begin{itemize}
	\item $\hat \Gamma \vdash \hat v : \hat \tau~\kw{with} \varnothing$
	\item $\Gamma, y: \kwa{erase}(\hat \tau) \vdash e: \tau$
	\item $\varepsilon = \kwa{effects}(\hat \tau)$
	\item $\kwa{ho \hyphen safe}(\hat \tau, \varepsilon)$
\end{itemize}

\noindent
Then $\hat \Gamma, \kwa{annot}(\Gamma, \varepsilon), y: \hat \tau \vdash \kwa{annot}(e, \varepsilon) : \kwa{annot}(\tau, \varepsilon)~\kw{with} \varepsilon \cup \kwa{effects}(\kwa{annot}(\Gamma, \varepsilon))$.
\end{lemma}

\begin{proof}
By induction on $\Gamma, y: \kwa{erase}(\hat \tau) \vdash e: \tau$.\\

\noindent
\fbox{Case: \textsc{T-Var}} Then $e=x$ and $\Gamma, y: \kwa{erase}(\hat \tau) \vdash x: \tau$. There are two cases: either $x=y$ or $x \neq y$. \\

\noindent
\textbf{Subcase 1: $x = y$}. Then by \textsc{$\varepsilon$-Var} we get $\hat \Gamma, \kwa{annot}(\Gamma, \varepsilon), y: \hat \tau \vdash x: \hat \tau~\kw{with} \varnothing$. \textbf{Need to justify why $\kwa{annot}(\tau, \varepsilon) = \hat \tau$. Note that $\varepsilon = \kwa{effects}(\hat \tau)$. Maybe show that $\kwa{annot}(\tau, \kwa{effects}(\tau)) <: \tau$. Can you use the lemma here?}\\

\noindent
\textbf{Subcase 2: $x \neq y$}. Then $x: \tau \in \Gamma$. \\

\noindent
\fbox{Case: \textsc{T-Resource}} Then $\Gamma, y: \kwa{erase}(\hat \tau) \vdash r : \{ r \}$. By definition, $\kwa{annot}(r, \varepsilon) = r$ and $\kwa{annot}(\{ r \}, \varepsilon$). By \textsc{$\varepsilon$-Resource}  $\hat \Gamma, \kwa{annot}(\Gamma, \varepsilon), y: \hat \tau \vdash r: \{ r \}~\kw{with} \varnothing$. By \textsc{$\varepsilon$-Subsume}, $\hat \Gamma, \kwa{annot}(\Gamma, \varepsilon), y: \hat \tau \vdash r: \{ r \}~\kw{with} \varepsilon \cup \kwa{effects}(\kwa{annot}(\Gamma, \varepsilon))$. \\

\noindent
\fbox{Case: \textsc{T-Abs}} Then $\Gamma, y: \kwa{erase}(\hat \tau) \vdash \lambda x: \tau_1.e_{body} : \tau_1 \rightarrow \tau_2$. \\

\noindent
By inversion, we get the sub-derivation $\Gamma, y: \kwa{erase}(\hat \tau), x:\tau_1  \vdash e_2: \tau_2$. \\

\noindent
By definition, $\kwa{annot}(e, \varepsilon) = \kwa{annot}(\lambda x: \tau_1.e_2, \varepsilon) = \lambda x: \kwa{annot}(\tau_1, \varepsilon).\kwa{annot}(e_2, \varepsilon)$. \\

\noindent
By definition, $\kwa{annot}(\tau, \varepsilon) = \kwa{annot}(\tau_1 \rightarrow \tau_2, \varepsilon) = \kwa{annot}(\tau_1, \varepsilon) \rightarrow_{\varepsilon} \kwa{annot}(\tau_2, \varepsilon)$. \\

\noindent
To apply the inductive assumption to $e_2$ we use the unlabelled context $\Gamma, x: \tau_1$. The inductive assumption tells us $\hat \Gamma, \kwa{annot}(\Gamma, \varepsilon), y: \hat \tau, x: \kwa{annot}(\tau_1, \varepsilon) \vdash \kwa{annot}(e_2, \varepsilon): \kwa{annot}(\tau_2, \varepsilon)~\kw{with} \varepsilon \cup \kwa{effects}(\kwa{annot}(\Gamma, \varepsilon)) \cup \kwa{effects}(\kwa{annot}(\tau_1, \varepsilon))$. Call this last effect-set $\varepsilon'$. \\

\noindent
By \textsc{$\varepsilon$-Abs}, we get $\hat \Gamma, \kwa{annot}(\Gamma, \varepsilon), y: \hat \tau \vdash \lambda x: \kwa{annot}(\tau_1, \varepsilon) . \kwa{annot}(e_2, \varepsilon) : \kwa{annot}(\hat \tau_1) \rightarrow_{\varepsilon'} \kwa{annot}(\hat \tau_2)~\kw{with} \varnothing$. \\

\noindent
By \textsc{$\varepsilon$-Subsume}, we get $\hat \Gamma, \kwa{annot}(\Gamma, \varepsilon), y: \hat \tau \vdash \kwa{annot}(e, \varepsilon) : \kwa{annot}(\hat \tau_1) \rightarrow_{\varepsilon} \kwa{annot}(\hat \tau_2)~\kw{with} \varepsilon \cup \kwa{effects}(\kwa{annot}(\Gamma), \varepsilon) $. \\

\noindent
\fbox{Case: \textsc{T-App}} TODO\\

\noindent
\fbox{Case: \textsc{T-OperCall}} Then $\Gamma, y: \kwa{erase}(\hat \tau) \vdash e_1.\pi : \Unit$. \\

\noindent
By inversion we get the sub-derivation $\Gamma, y: \kwa{erase}(\hat \tau) \vdash e_1: \{ \bar r \}$. \\

\noindent
By definition, $\kwa{annot}(\{ \bar r \}, \varepsilon) = \{ \bar r \}$. \\

\noindent
By inductive assumption, $\hat \Gamma, \kwa{annot}(\Gamma, \varepsilon), y: \hat \tau \vdash e_1: \{ \bar r \} ~\kw{with} \varepsilon \cup \kwa{effects}(\kwa{annot}(\Gamma, \varepsilon))$. \\

\noindent
By \textsc{$\varepsilon$-OperCall}, $\hat \Gamma, \kwa{annot}(\Gamma, \varepsilon), y: \hat \tau \vdash e_1.\pi: \{ \bar r \} ~\kw{with} \varepsilon \cup \{ \bar r.\pi \}$. \\

\noindent
It remains to show $\{ \bar r.\pi \} \subseteq \varepsilon$. We shall do this by considering where $r$ must have come from (which subcontext left of the turnstile). \\

\noindent
\textbf{Subcase 1.} $r = \hat \tau$. As $\varepsilon = \kwa{effects}(\hat \tau)$, then $r.\pi \in \kwa{effects}(\hat \tau)$. \\

\noindent
\textbf{Subcase 2.} $r: \{ r \} \in \Gamma$. As $\kwa{annot}(r, \varepsilon) = r$, then $r.\pi \in \kwa{annot}(\Gamma, \varepsilon)$. \\

\noindent
\textbf{Subcase 3.} $r: \{ r \} \in \hat \Gamma$. Then because $\Gamma, y: \kwa{erase}(\hat \tau) \vdash e_1: \{ \bar r \}$, then $r \in \Gamma$ or $r = \kwa{erase}(\hat \tau) = \hat \tau$ and one of the above subcases must also hold. \\


\end{proof}


\end{document}










