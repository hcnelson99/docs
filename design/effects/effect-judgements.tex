\documentclass{llncs}

\usepackage{listings}
\usepackage{proof}
\usepackage{amssymb}
\usepackage[margin=.9in]{geometry}
\usepackage{amsmath}
\usepackage[english]{babel}
\usepackage[utf8]{inputenc}
\usepackage{enumitem}
\usepackage{filecontents}
\usepackage{calc}
\usepackage[linewidth=0.5pt]{mdframed}
\allowdisplaybreaks

\usepackage{fancyhdr}
\renewcommand{\headrulewidth}{0pt}
\pagestyle{fancy}
 \fancyhf{}
\rhead{\thepage}

\lstset{tabsize=3, basicstyle=\ttfamily\small, commentstyle=\itshape\rmfamily, numbers=left, numberstyle=\tiny, language=java,moredelim=[il][\sffamily]{?},mathescape=true,showspaces=false,showstringspaces=false,columns=fullflexible,xleftmargin=5pt,escapeinside={(@}{@)}, morekeywords=[1]{objtype,module,import,let,in,fn,var,type,rec,fold,unfold,letrec,alloc,ref,application,policy,external,component,connects,to,meth,val,where,return,group,by,within,count,connect,with,attr,html,head,title,style,body,div,keyword,unit,def}}
\lstloadlanguages{Java,VBScript,XML,HTML}


\newcommand{\keywadj}[1]{\mathtt{#1}}
\newcommand{\keyw}[1]{\keywadj{#1}~}
\newcommand{\reftt}{\mathtt{ref}~}
\newcommand{\Reftt}{\mathtt{Ref}~}
\newcommand{\inttt}{\mathtt{int}~}
\newcommand{\Inttt}{\mathtt{Int}~}
\newcommand{\stepsto}{\leadsto}
\newcommand{\todo}[1]{\textbf{[#1]}}
\newcommand{\intuition}[1]{#1}
%\newcommand{\intuition}[1]{}

\newlist{pcases}{enumerate}{1}
\setlist[pcases]{
  label=\textit{Case}\protect\thiscase\textit{:}~,
  ref=\arabic*,
  align=left,
  labelsep=0pt,
  leftmargin=0pt,
  labelwidth=0pt,
  parsep=0pt
}
\newcommand{\pcase}[1][]{
  \if\relax\detokenize{#1}\relax
    \def\thiscase{}
  \else
    \def\thiscase{~#1}
  \fi
  \item
}
\begin{document}

\today
\section{Effects}

\paragraph{}
Fix some set of resources $R$. A resource is some language primitive that has the authority to directly perform I/O operations. Elements of the set $R$ are denoted by $r$. $M$ is another fixed set of methods on resources. An effect is a member of the set of pairs $R \times M$. A set of effects is denoted by $\varepsilon$. In this system we cannot dynamically create resources or resource-operations.

\paragraph{}
Throughout we refer to the notions of effects and captures. A piece of code $C$ has the effect $(r, m)$ if operation $m$ is performed on resource $r$ during execution of $C$. $C$ captures the effect $(r, m)$ if it has the authority to perform operation $m$ on resource $r$ at some point during its execution.

\paragraph{}
We use $r.m$ as syntactic sugar for the effect $(r,m)$. For example, $FileIO.append$ instead of $(FileIO, append)$.

\paragraph{}
Non-resource types may be structural or primitive (is this right?). The set of primitive types $P$ contains familiar types. For example, $Int \in P$. If a type is not a resource and not primitive, we call it composite. Structural types are sets of method declarations. An example is $\{ \keyw{def} double(x : Int) : Int \}$. An interesting composite type is $\keywadj{Unit}$, which is equivalent to the empty set $\varnothing$.

\section{Fully-Annotated Programs}

In this first system every method in the program is explicitly annotated with its set of effects.

\subsection{Grammar}

\[
\begin{array}{lll}
\begin{array}{lllr}

	e & ::= & x & expressions \\
  		& | & \keywadj{new}~x \Rightarrow \overline{\sigma = e} \\
  		& | & e.m(e)\\
  		& | & r \\
		&&\\

	\tau & ::= & \{ \bar \sigma \} ~ | ~ \{ \bar r \} & types \\
		&&\\

	\sigma & ::= & d~\keyw{with}\varepsilon  & labeled~ decls.\\
		&&\\
		
	\Gamma & ::= & \varnothing \\
		& | & \Gamma,~x : \tau\\
		&&\\

\end{array}
& ~~~~~~
&
\end{array}
\]


\noindent \textbf{Notes:}

\begin{itemize}
	\item Declarations ($\sigma$-terms) are annotated by what effects they have.
	\item All methods take exactly one argument. If a method specifies no argument, then the argument is implicitly oaf type $\keywadj{Unit}$.
\end{itemize}

\subsection{Rules}

\fbox{$\Gamma \vdash e : \tau~\keyw{with} \varnothing$}

\[
\begin{array}{c}
\infer[\textsc{($\varepsilon$-Var)}]
  {\Gamma,~x:\tau \vdash x:\tau~ \keyw{with} \varnothing}
  {} 
~~~~~~~~~~
\infer[\textsc{($\varepsilon$-Resource)}]
  {\Gamma,~r :\{r\} \vdash r:\{ r \} ~ \keyw{with} \varnothing}
  {} \\[5ex]

\infer[\textsc{($\varepsilon$-NewObj)}]
	{\Gamma \vdash \keywadj{new}~x \Rightarrow \overline{\sigma = e} : \{ \bar \sigma \}~\keyw{with} \varnothing}
	{\Gamma,~x : \{ \bar \sigma \} \vdash \overline{\sigma = e}~\keywadj{OK}} \\[5ex]
\end{array}
\]

$\fbox{$\Gamma \vdash \sigma = e~\keyw{OK}$}$

\[
\begin{array}{c}
\infer[\textsc{($\varepsilon$-ValidImpl$_\sigma$)}]
	{\Gamma \vdash \sigma = e~\keywadj{OK}}
	{\Gamma,~x:\tau \vdash e: \tau'~\keyw{with} \varepsilon & \sigma = \keyw{def} m(x:\tau):\tau'~\keyw{with} \varepsilon} \\[5ex]
\end{array}
\]

$\fbox{$\Gamma \vdash e_1.m(e_2) : \tau~\keyw{with} \varepsilon$}$
\[
\begin{array}{c}
\infer[\textsc{($\varepsilon$-MethCallResource)}]
	{\Gamma \vdash e_1.m(e_2) : \{ \bar r \}~\keyw{with} \{ \bar r, m\} \cup \varepsilon_1 \cup \varepsilon_2}
	{\Gamma \vdash e_1 : \{ \bar r \}~ \keyw{with} \varepsilon_1 & \Gamma \vdash e_2:\tau_2~\keyw{with} \varepsilon_2 & m \in M} \\[5ex]
	
\infer[\textsc{($\varepsilon$-MethCallObj)}]
	{\Gamma \vdash e_1.m_i(e_2):\tau~\keyw{with} \varepsilon_1 \cup \varepsilon_2 \cup \varepsilon}
	{\Gamma \vdash e_1 : \{ \bar \sigma \}~\keyw{with} \varepsilon_1 & \Gamma \vdash e_2 : \tau_2~\keyw{with} \varepsilon_2 & \sigma_i = \keyw{def} m_i(y:\tau_2):\tau~\keyw{with} \varepsilon}

\end{array}
\]

\noindent \textbf{Notes:}

\begin{itemize}
	\item Every expression in the program must be explicitly annotated; either as $\sigma$-terms or by what they capture.
	\item The rules \textsc{$\varepsilon$-Var}, \textsc{$\varepsilon$-Resource}, and \textsc{$\varepsilon$-NewObj} have in their consequents an expression typed with no effect: merely having an object or resource is not an effect; you must do something with it, like a call a method on it, in order for it to be an effect.
	\item \textsc{$\varepsilon$-ValidImpl} says that the return type and effects of the body of a method must agree with what its signature says.
	\item In \textsc{$\varepsilon$-MethCallResource}, we may only call a method $m$ on a resource $r$ if $m$ is a predefined operation in the set $M$. Invoking $m$ returns the resource $r$ you called it upon (which has potentially different state afterwards).
\end{itemize}





\newpage


% ========================================
% Partially-Annotated Programs
% ========================================

\section{Partially-Annotated Programs}

In this second system methods may either be fully labeled with their effects or have no labels. When they have no labels a conservative effect inference is performed using rules which provide an upper-bound (not necessarily tight) on the effects of the code when executed.

\subsection{Grammar}

\[
\begin{array}{lll}

\begin{array}{lllr}

e & ::= & x & expressions \\
		& | & \keywadj{new}_\sigma~x \Rightarrow \overline{\sigma = e} \\
 		& | & \keywadj{new}_d~x \Rightarrow \overline{d = e} \\
 		& | & e.m(e)\\
  		& | & r \\
		&&\\

\tau & ::= & \{ \bar \sigma \} & types \\
		& | & \{ \bar r \} \\
		& | & \{ \bar d \} \\
		& | & \{ \bar d ~\keyw{captures} \varepsilon \} \\
		&&\\
		
\sigma & ::= & d~\keyw{with}\varepsilon  & labeled~ decls.\\
		&&\\

d & ::= & \keyw{def} m(x:\tau):\tau & unlabeled~decls.\\
		&&\\

\end{array}
& ~~~~~~
&
\end{array}
\]

\noindent \textbf{Notes:}
\begin{itemize}
	\item $\sigma$ denotes a declaration with effect labels. $d$ denotes a declaration without effect labels.
	\item There are two new expressions: $\keywadj{new}_\sigma$ for objects whose declarations are annotated; $\keywadj{new}_d$ for objects whose declarations aren't.
	\item $\{ \bar d ~\keyw{captures} \varepsilon \}$ is a special kind of type that doesn't appear in the source program, but may be assigned as a consequence of the capture rules.
\end{itemize}

\subsection{Rules}

In addition to the rules from the previous system, the partially-annotated system has the following rules.\\

$\fbox{$\Gamma \vdash e : \tau$}$

\[
\begin{array}{c}
\infer[\textsc{(T-Var)}]
	{\Gamma,~x : \tau \vdash x : \tau}
	{}
~~~~~~~~~~
\infer[\textsc{(T-Resource)}]
	{\Gamma,~r : \{ \bar r \} \vdash r : \{ \bar r \}}
	{} \\[5ex]
~~~~~~~~~~
\infer[\textsc{(T-MethCall$_r$)}]
	{\Gamma \vdash r.m(e_1) : \{ \bar r\}}
	{\Gamma \vdash r : \{ \bar r \} & \Gamma \vdash e : \tau & m \in M} \\[5ex]
~~~~~~~~~~
\infer[\textsc{(T-MethCall$_\sigma$)}]
	{\Gamma \vdash e_1.m(e_2) : \tau_2}
	{\Gamma \vdash e_1 : \{ \bar \sigma \},~\keyw{def} m(x : \tau_1) : \tau_2~\keyw{with} \varepsilon \in \{ \bar \sigma \} &  \Gamma \vdash e_2 : \tau_1 }\\[5ex]
~~~~~~~~~~
\infer[\textsc{(T-MethCall$_d$)}]
	{\Gamma \vdash e_1.m(e_2) : \tau_2}
	{\Gamma \vdash e_1 : \{ \bar d\},~\keyw{def} m(x : \tau_1) : \tau_2 \in \{ \bar d \} &  \Gamma \vdash e_2 : \tau_1 }\\[5ex]
~~~~~~~~~~

~~~~~~~~~~
\infer[\textsc{(T-New$_\sigma$)}]
	{\Gamma \vdash~\keywadj{new}_\sigma~x \Rightarrow \overline{\sigma = e} : \{ \bar \sigma \}}
	{\Gamma \vdash \sigma_i = e_i~\keywadj{OK}}
~~~~~~~~~~
\infer[\textsc{(T-New$_d$)}]
	{\Gamma \vdash~\keywadj{new}_d~x \Rightarrow \overline{d = e} : \{ \bar d \}}
	{\Gamma \vdash d_i = e_i~\keywadj{OK}}\\[5ex]
\end{array}
\]

$\fbox{$\Gamma \vdash d = e~\keyw{OK}$}$
\[
\begin{array}{c}
\infer[\textsc{($\varepsilon$-ValidImpl$_d$)}]
	{\Gamma \vdash d = e~\keywadj{OK}}
	{d = \keyw{def} m(x : \tau_1) : \tau_2 & \Gamma \vdash e : \tau_2}
	\\[5ex]
\end{array}
\]

\fbox{$\Gamma \vdash e_1.m(e_2) : \tau~\keyw{with} \varepsilon$}

\[
\begin{array}{c}
\infer[\textsc{(C-NewObj)}]
	{\Gamma \vdash~\keywadj{new}_d~x \Rightarrow \overline{d = e} : \{  \bar d~\keyw{captures} \varepsilon \} ~\keyw{with} \varnothing }
	{\varepsilon = effects(\Gamma') & \Gamma' \subseteq \Gamma & \Gamma', x : \{ \bar d~\keyw {captures} \varepsilon \} \vdash \overline{d = e}~\keyw{OK}} \\[5ex]
\end{array}
\]

\[
\begin{array}{c}
\infer[\textsc{(C-MethCall)}]
	{\Gamma \vdash e_1.m_i(e_2) : \tau~ \keyw{with} \varepsilon_1 \cup \varepsilon_2 \cup effects(\tau_2) \cup \varepsilon}
	{\Gamma \vdash e_1 : \{ \bar d~\keyw{captures} \varepsilon \}~\keyw{with} \varepsilon_1 & \Gamma \vdash e_2:\tau_2~\keyw{with} \varepsilon_2&d_i :=~\keyw{def} m_i(y:\tau_2):\tau } \\[5ex]
\end{array}
\]


\noindent \textbf{Notes:}

\begin{itemize}
	\item The \textsc{$\varepsilon$} judgements are to be applied to annotated parts of the program; the \textsc{C} rules for unannotated parts. 
	\item The rules \textsc{$\varepsilon$-Var}, \textsc{$\varepsilon$-Resource}, and \textsc{$\varepsilon$-NewObj} have in their antecedents an expression typed with no effect. Merely having an object or resource is not an effect; you must do something with it, like a call a method on it, in order for your program to have effects.
	\item The \textsc{T} judgements before standard typechecking, but they operate on annotated terms. They are needed to apply the \textsc{$\varepsilon$-ValidImpl$_d$)} rule.
	\item In applying \textsc{C-NewObj} the variable $\Gamma$ is the current context. The variable $\Gamma'$ is some sub-context. A good choice of sub-context is $\Gamma$ restricted to the free variables in the method-body being typechecked. This means we only consider the effects used in the method-body and gives a better approximation of its effects.
	\item When an unannotated $d$-declaration is encountered it is first assigned a $\gamma$-type by \textsc{C-NewObj}. This annotates it as capturing a certain set of effects. \textsc{C-MethCall} can then conclude its effects to be what it captures.
\end{itemize}

\subsection{Effects Function}

\paragraph{}
The $\keywadj{effects}$ function returns the set of effects in a particular typing context.

\paragraph{}
A method $m$ can return a resource $r$ (or an object that returns $r$, and so on). Returning a resource isn't an effect but it means any unannotated program using $m$ also captures $r$. To account for this, $\keywadj{effects}$ also uses $\keywadj{escapes}$.

\begin{itemize}
	\item $\keywadj{effects}(\cdot) = \varnothing$
	\item $\keywadj{effects}(\{\bar r\}) = \{ (r, m) \mid r \in \bar r, m \in M \}$
	\item $\keywadj{effects}(\{ \bar d~\keyw{captures} \varepsilon_1 \}~\keyw{with} \varepsilon_2) = \varepsilon_1 \cup \varepsilon_2 \cup \keywadj{escapes}(\bar d)$
	\item $\keywadj{effects}(d~\keyw{with} \varepsilon) = \varepsilon \cup \keywadj{escapes}(d~\keyw{with} \varepsilon)$
	\item $\keywadj{effects}(\{\bar \sigma\}) = \bigcup_{\sigma \in \bar \sigma}~\keywadj{effects}(\sigma)$
\end{itemize}

\subsection{Escapes Function}

\fbox{$\keywadj{escapes(\tau)}$}

\[
\begin{array}{c}
\infer[\textsc{(Escapes-Resource)}]
	{\keyw{escapes(def} m(x : \tau) : \{ \bar r \}) = \{ (r, m) \mid r \in \bar r, m \in M \}}
	{} \\[4ex]
	
\infer[\textsc{(Escapes-Structural)}]
	{\keywadj{escapes(def}~m(x : \tau_1) : \tau_2) = escapes(\tau_2)}
	{\tau_2 \notin P & \tau_2 \notin R} \\[4ex]

\infer[\textsc{(Escapes-Primitive)}]
	{\keyw{escapes(def} m(x : \tau_1) : \tau_2) = \varnothing}
	{\tau_2 \in P} \\[4ex]

\infer[\textsc{(Escapes$_\sigma$)}]
	{\keywadj{escapes(def}~m(x : \tau_1) : \tau_2~ with~\varepsilon) = \keywadj{escapes(def}~m(x : \tau_1) : \tau_2)}
	{} \\[4ex]

\infer[\textsc{(Escapes$_{\bar d}$)}]
	{\keywadj{escapes}(\{\bar d\}) = \bigcup_{d \in \bar d}~escapes(d)}
	{}
~~~~~~~~
\infer[\textsc{(Escapes$_{\bar \sigma}$)}]
	{\keywadj{escapes}(\{\bar \sigma\}) = \bigcup_{\sigma \in \bar \sigma}~escapes(\sigma)}
	{} \\[4ex]
	
%\infer[\textsc{(Escapes-Recursive)}]
%	{\keyw{escapes(def} m(x : \tau_1) : \{ \bar \sigma \}) = \bigcup_{\sigma \in %\bar \sigma}~\keyw{escapes(\sigma)}}
%	{\tau_2 \notin P & \tau_2 \notin R} \\[2ex]
\end{array}
\]

\newpage

\end{document}
