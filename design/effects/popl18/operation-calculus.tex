\section{Operation Calculus ($\opercalc$)}

$\opercalc$ extends the simply-typed lambda calculus with a notion of primitive capabilities and their operations.
Every function is annotated with the effects it may incur.
Its static rules associate a type and a set of effects to well-formed programs.
Defining $\opercalc$ will introduce the notations and concepts needed to understand $\epscalc$, which allows developers to omit annotations from some expressions and uses capability-based reasoning to bound the effects of those expressions.

In a capability-safe language, ``only connectivity begets connectivity'' \cite{miller06}: all access to a capability must derive from previous access.
To prevent an infinite regress, there are a set of primitive capabilities passed into the program by the system environment.
These primitive capabilities provide operations for manipulating resources in the system environment.
For example, $\kwa{File}$ might provide read/write operations on a particular file in the file system.
For convenience, we often conflate primitive capabilities with the resources they manipulate, referring to both as resources.
An effect in $\opercalc$ is a particular operation invoked on some resource; for example, $\kwa{File.write}$.
Functions in an $\opercalc$ program are (conservatively) annotated with the effects they may incur when invoked. Annotations might be given in accordance with the principle of least authority to specify the maximum authority a component may exercise.
When this authority is exceeded, an effect system like that of $\opercalc$ will reject the program, signaling an unsafe implementation.
For example, consider the pair of modules\footnote{Our formal grammar, below, does not include this \textit{Wyvern}-like module syntax, but we can model the \kwat{logger} functor as a function and the \kwat{client} module as a record (which is itself encodable using functions).  See section 4 for details.} in Figure \ref{fig:opercalc_motivating}: the functor $\kwa{logger}$ (declared with \kwat{module} \kwat{def}) must be instantiated with a $\File$ capability, and the resulting module exposes a single function $\kwa{log}$.
The $\kwa{client}$ module has a single function $\kwa{run}$ which, when passed a $\kwa{Logger}$, will invoke $\kwa{Logger.log}$.

\begin{figure}[h]
\vspace{-5pt}

\begin{lstlisting}
module def logger(f:{File}):Logger

def log(): Unit with {File.append} =
    f.read
\end{lstlisting}

\begin{lstlisting}
module client

def run(l: Logger): Unit with {File.append} =
    l.log()
\end{lstlisting}

\vspace{-7pt}
\caption{The implementation of $\kwa{logger.log}$ exceeds its specified authority.}
\label{fig:opercalc_motivating}
\end{figure}

$\kwa{client.run}$ and $\kwa{logger.log}$ are both annotated with $\{ \kwa{File.append} \}$, but the (potentially malicious) implementation of $\kwa{logger.log}$ incurs the $\kwa{File.read}$ effect.
In this section we develop rules for $\opercalc$ that can determine such mismatches between specification and implementation in annotated code.

$\opercalc$ makes some simplifying assumptions.
The semantics of particular operations are not modeled --- our only interest is in what operations could be invoked, and by whom.
Therefore, we assume all operations are null-ary and return a dummy $\unit$ value; $\kwa{File.write(``hello, world!'')}$ becomes $\kwa{File.write}$.
Primitive capabilities and operations are fixed throughout run time and cannot be created or destroyed.

\subsection{Grammar ($\opercalc$)}

A grammar for $\opercalc$ programs is given in Figure \ref{fig:opercalc_grammar}. In addition to those from the lambda calculus, there are two new forms. A resource literal $r$ is a variable drawn from a fixed set $R$. Resources model the primitive capabilities that the system passes into the program. $\kwa{File}$ and $\kwa{Socket}$ are examples of resource literals. An operation call $e.\pi$ is the invocation of an operation $\pi$ on $e$. For example, invoking the $\kwa{open}$ operation on the $\kwa{File}$ resource would be $\kwa{File.open}$. Operations are drawn from a fixed set $\Pi$.

\begin{figure}[h]
\vspace{-5pt}

\[
\begin{array}{lll}

\begin{array}{lllr}

e & ::= & ~ & exprs: \\
	& | & x & variable \\
	& | & v & value \\
	& | & e ~ e & application \\
	& | & e.\pi & operation~call \\
	&&\\

\end{array}

	\hspace{5ex}

\begin{array}{lllr}

v & ::= & ~ & values: \\ 
	& | & r & resource~literal \\
	& | & \lambda x: \tau.e & abstraction \\
	&&\\

\end{array}

\end{array}
\]

\vspace{-7pt}
\caption{Grammar for $\opercalc$ programs.}
\label{fig:opercalc_grammar}
\end{figure}

An effect is a pair $(r, \pi) \in R \times \Pi$. Sets of effects are denoted $\varepsilon$. As a shorthand, we write $r.\pi$ instead of $(r, \pi)$. Effects should be distinguished from operation calls: an operation call is the invocation of a particular operation on a particular resource in a program, while an effect is a mathematical object describing this behaviour. The notation $r.*$ is a short-hand for the set $\{ r.\pi \mid \pi \in \Pi \}$, which contains every effect on $r$. Sometimes we abuse notation by conflating the effect $r.\pi$ with the singleton $\{ r.\pi \}$. We may also write things like $\{ r_1.*, r_2.* \}$, which should be understood as the set of all operations on $r_1$ and $r_2$.

\subsection{Semantics ($\opercalc$)}

During reduction an operation call may be evaluated. When this happens we say that a run time effect has taken place. Reflecting this, the form of the single-step reduction judgement is $e \longrightarrow e'~|~\varepsilon$, meaning $e$ reduces to $e'$, incurring the set of effects $\varepsilon$ in the process. In the case of single-step reduction, $\varepsilon$ is at most a single effect. Rules for single-step reductions are given in Figure \ref{fig:opercalc_singlestep}.

\begin{figure}[h]

\noindent
\fbox{$e \longrightarrow e~|~\varepsilon$}

\[
\begin{array}{c}

\infer[\textsc{(E-App1)}]
	{e_1 e_2 \longrightarrow e_1'~ e_2~|~\varepsilon}
	{e_1 \longrightarrow e_1'~|~\varepsilon}
    
	\hspace{5ex}
    
\infer[\textsc{(E-App2)}]
	{v_1 ~ e_2 \longrightarrow v_1 ~ e_2'~|~\varepsilon} 
	{e_2 \longrightarrow e_2'~|~\varepsilon}
    
	\hspace{5ex}
    
\infer[\textsc{(E-App3)}]
	{ (\lambda x: \tau. e) v_2 \longrightarrow [ v_2/x] e~|~\varnothing }
	{}\\[4ex]
	
\infer[\textsc{(E-OperCall1)}]
	{ e.\pi \longrightarrow  e'.\pi~|~\varepsilon }
	{ e \rightarrow  e'~|~\varepsilon}
		
	\hspace{5ex}
	
\infer[\textsc{(E-OperCall2)}]
	{r.\pi \longrightarrow \kwa{unit}~|~\{ r.\pi \} }
	{}
	 \\[4ex]
	 
\end{array}
\]


\vspace{-7pt}
\caption{Single-step reductions in $\opercalc$.}
\label{fig:opercalc_singlestep}
\end{figure}

The first three rules are analogous to reductions in the lambda calculus. \textsc{E-App1} and \textsc{E-App2} incur the effects of reducing their subexpressions. \textsc{E-App3} replaces free occurrences of the formal name $x$ in $e$ with the actual value $v_2$ being passed as an argument, which incurs no effects. The notation for this is $[v_2/x]e$. It is significant that variables are only substituted for values: if $x$ is replaced by an arbitrary expression, the substitution could be introducing arbitrary effects. However, values incur no effects, so replacing $x$ by a value will not introduce any extra effects. Thus $\opercalc$ is a call-by-value language.

The first new rule is \textsc{E-OperCall1}, which reduces the receiver of an operation call; the effects incurred are the effects incurred by reducing the receiver.  When an operation $\pi$ is invoked on a resource literal $r$, \textsc{E-OperCall2} will reduce it to $\unit$,
%\footnote{Our formal grammar does not include $\unit$, but it can be defined as \lambda x: {}. x}
incurring $\{ r.\pi \}$ as a result. For example, $\kwa{File.write} \longrightarrow \unit~|~\{ \kwa{File.write} \}$ by \textsc{E-OperCall2}. $\unit$ can be treated as a derived form; an explanation is given in section 4.

A multi-step reduction is a sequence of zero
%\footnote{We permit multi-step reductions of length zero to be consistent with Pierce, who defines multi-step reduction as a reflexive relation \cite[p. 39]{pierce02}.}
or more single-step reductions. The resulting set of run time effects is the union of all the run time effects from the intermediate single-steps. Rules for multi-step reductions are given in Figure \ref{fig:opercalc_multistep_defn}. By \textsc{E-MultiStep1}, any expression can ``reduce'' to itself with no run time effects. By \textsc{E-MultiStep2}, any single-step reduction is also a multi-step reduction. If $e \longrightarrow e'~|~\varepsilon_1$ and $e' \longrightarrow e''~|~\varepsilon_2$ are sequences of reductions, then so is $e \longrightarrow e''~|~\varepsilon_1 \cup \varepsilon_2$, by \textsc{E-MultiStep3}.

\begin{figure}[h]

\noindent
\fbox{$ e \longrightarrow^{*}  e~|~\varepsilon$}

\[
\begin{array}{c}

\infer[\textsc{(E-MultiStep1)}]
	{ e \rightarrow^{*}  e~|~\varnothing}
	{}
	\hspace{5ex}
\infer[\textsc{(E-MultiStep2)}]
	{ e \rightarrow^{*}  e'~|~\varepsilon}
	{ e \rightarrow  e'~|~\varepsilon}
	\hspace{5ex}
\infer[\textsc{(E-MultiStep3)}]
	{ e \rightarrow^{*}  e''~|~\varepsilon_1 \cup \varepsilon_2}
	{ e \rightarrow^{*}  e'~|~\varepsilon_1 &  e' \rightarrow^{*}  e''~|~\varepsilon_2}
\end{array}
\]

\vspace{-7pt}
\caption{Multi-step reductions in $\opercalc$.}
\label{fig:opercalc_multistep_defn}
\end{figure}

\subsection{Static Rules ($\opercalc$)}

A grammar for types, contexts, and sets of effects is given in Figure \ref{fig:opercalc_types}. The base types of $\opercalc$ are sets of resources, denoted $\{ \bar r\}$. If an expression $e$ is associated with type $\{ \bar r \}$, it means $e$ will reduce to one of the literals in $\bar r$ (assuming $e$ terminates). The set of empty resources (denoted $\varnothing$) is also a valid type, but has no inhabitants. There is a single type constructor $\rightarrow_{\varepsilon}$, where $\varepsilon$ is a concrete set of effects. $\tau_1 \rightarrow_{\varepsilon} \tau_2$ is the type of a function which takes a $\tau_1$ as input, returns a $\tau_2$ as output, and whose body incurs no more than those effects in $\varepsilon$. $\varepsilon$ is a conservative bound: if an effect $r.\pi \in \varepsilon$, it is not guaranteed to happen at run time, but if $r.\pi \notin \varepsilon$, it cannot happen at run time. A typing context $\Gamma$ maps variables to types. 

\begin{figure}[h]
\vspace{-5pt}

\[
\begin{array}{lll}

\begin{array}{lllr}

\tau & ::= & ~ & types: \\
		& | & \{ \bar r \} & resource~set \\
		& | & \tau \rightarrow_{\varepsilon} \tau & annotated~arrow \\ 
		&&\\
		
\varepsilon & ::= & ~ & effects: \\
		& | & \{ \overline{r.\pi} \} & effect~set \\
		&&\\
		
\end{array}

	\hspace{5ex}
    
\begin{array}{lllr}

\Gamma & ::= & ~ & type~ctx: \\
				& | & \varnothing & empty~ctx. \\
				& | & \Gamma, x: \tau & var.~binding \\
				&&\\
\end{array}

\end{array}
\]

\vspace{-7pt}
\caption{Grammar for types in $\opercalc$.}
\label{fig:opercalc_types}
\end{figure}

To illustrate the types of some functions, if $\kwa{log_1}$ has the type $\{ \kwa{File} \} \rightarrow_{\{\kwa{File.append}\}} \Unit$, then invoking $\kwa{log_1}$ will either incur $\kwa{File.append}$ or no effects. If $\kwa{log_2}$ has the type $\{ \kwa{File} \} \rightarrow_{\{\kwa{File.*}\}} \Unit$, then invoking $\kwa{log_2}$ could incur any effect on $\kwa{File}$, or no effects.

Knowing approximately what effects a piece of code may incur helps a developer determine whether it can be trusted. For example, consider $\kwa{log_3} = \lambda f: \{ \kwa{File} \}.~e$, which is a logging function that takes a $\kwa{File}$ as an argument and then executes $e$. Suppose this function were to typecheck as $\{ \File \} \rightarrow_{\{ \kwa{File.*} \}} \Unit$ --- seeing that invoking this function could incur any effect on $\kwa{File}$, and not just its expected least authority $\kwa{File.append}$, a developer may therefore decide this implementation cannot be trusted and choose not to execute it. In this spirit, the static rules of $\opercalc$ associate well-typed programs with a type and a set of effects: the judgement $\Gamma \vdash e: \tau~\kw{with} \varepsilon$, means $e$ will reduce to a term of type $\tau$ (assuming it terminates), incurring no more effects than those in $\varepsilon$. Judgements are given in Figure \ref{fig:opercalc_static_rules}.

\begin{figure}[h]

\noindent
\fbox{$\Gamma \vdash e: \tau~\kw{with} \varepsilon$}

\[
\begin{array}{c}

\infer[\textsc{($\varepsilon$-Var)}]
	{ \Gamma, x:\tau \vdash x: \tau~\kw{with} \varnothing }
	{}
	
	\hspace{5ex}
	
\infer[\textsc{($\varepsilon$-Resource)}]
 	{ \Gamma, r: \{ r \} \vdash r : \{ r \}~\kw{with} \varnothing }
 	{} \\[3ex]
 	
 	~~~
	\infer[\textsc{($\varepsilon$-Abs)}]
	{ \Gamma \vdash \lambda x:\tau_2 . e : \tau_2 \rightarrow_{\varepsilon_3} \tau_3~\kw{with} \varnothing }
	{ \Gamma, x: \tau_2 \vdash e: \tau_3~\kw{with} \varepsilon_3 }
	
	\hspace{5ex}
	
\infer[\textsc{($\varepsilon$-App)}]
	{ \Gamma \vdash e_1~e_2 : \tau_3~\kw{with} \varepsilon_1 \cup \varepsilon_2 \cup \varepsilon  }
	{ \Gamma \vdash e_1: \tau_2 \rightarrow_{\varepsilon} \tau_3~\kw{with} \varepsilon_1 & \Gamma \vdash e_2: \tau_2~\kw{with} \varepsilon_2 } \\[3ex]
	
\infer[\textsc{($\varepsilon$-OperCall)}]
	{ \Gamma \vdash e.\pi: \kw{Unit} \kw{with} \varepsilon \cup \{ \bar r.\pi \} }
	{ \Gamma \vdash e: \{ \bar r \}~ \kw{with} \varepsilon}

	\hspace{5ex}
    
\infer[\textsc{($\varepsilon$-Subsume)}]
	{ \Gamma \vdash e: \tau' ~\kw{with} \varepsilon'}
	{ \Gamma \vdash e: \tau ~\kw{with} \varepsilon & \tau <: \tau' & \varepsilon \subseteq \varepsilon'}\\[3ex]
	
\end{array}
\]


\vspace{-7pt}
\caption{Type-with-effect judgements in $\opercalc$.}
\label{fig:opercalc_static_rules}
\end{figure}



\textsc{$\varepsilon$-Var} approximates the run time effects of a variable as $\varnothing$. \textsc{$\varepsilon$-Resource} does the same for resource literals. Though a resource captures several effects (namely, every possible operation on itself), attempting to ``reduce'' a resource will incur no effects; something must be done with the resource, such as an operation call, in order to have an effect. For a similar reason, \textsc{$\varepsilon$-Abs} approximates the effects of a function literal as $\varnothing$, and ascribes an arrow type annotated with those effects captured by the function. \textsc{$\varepsilon$-App} approximates a lambda application as incurring those effects from evaluating the subexpressions and the effects incurred by executing the body of the function to which the left-hand side evaluates. The effects of the function body are taken from the function's arrow type. An operation call on a resource literal reduces to $\unit$, so \textsc{$\varepsilon$-OperCall} ascribes its type as $\Unit$.
The approximate effects of an operation call are: the effects of reducing the subexpression, and then the operation $\pi$ on every possible resource which that subexpression to which that subexpression might reduce. For example, consider $e.\pi$, where $\Gamma \vdash e: \{ \kwa{File, Socket} \}~\kw{with} \varnothing$. Then $e$ could evaluate to $\kwa{File}$, in which case the actual run time effect is $\kwa{File}.\pi$, or it could evaluate to $\kwa{Socket}$, in which case the actual run time effect is $\kwa{Socket.\pi}$. Determining which will happen is, in general, undecidable; the safe approximation is to treat them both as happening. The last rule \textsc{$\varepsilon$-Subsume} produces a new judgement by widening the type or approximate effects on an existing one. Subtyping judgements are given in Figure \ref{fig:opercalc_subtype_rules}.


\begin{figure}[h]
\vspace{-5pt}

\fbox{$\tau <: \tau$}

\[
\begin{array}{c}

\infer[\textsc{(S-Arrow)}]
	{ \tau_1 \rightarrow_{\varepsilon} \tau_2 <: \tau_1' \rightarrow_{\varepsilon'} \tau_2' }
	{ \tau_1' <: \tau_1 & \tau_2 <: \tau_2' & \varepsilon \subseteq \varepsilon' }
	\hspace{5ex}
\infer[\textsc{(S-Resource)}]
	{ \{ \bar r_1 \} <: \{ \bar r_2 \} }
	{ r \in r_1 \implies r \in r_2 }

\end{array}
\]

\vspace{-7pt}
\caption{Subtyping judgements of $\opercalc$.}
\label{fig:opercalc_subtype_rules}
\end{figure}

\textsc{S-Arrow} is the standard rule for arrow types, but also stipulates that the effects on the arrow of the subtype must be contained in the effects on the arrow of the supertype: a valid subtype should not invoke any effects the supertype does not already know about. \textsc{S-Resource} says that a subset of resources is a subtype. To illustrate, consider $\{ \bar r_1 \} <: \{ \bar r_2 \}$ --- any value with type $\{ \bar r_1 \}$ can reduce to any resource literal in $\bar r_1$, so to be compatible with an interface $\{ \bar r_2 \}$, the resource literals in $\bar r_1$ must also be in $\bar r_2$.

These rules let us determine what sort of effects might be incurred when a piece of code is executed. For example, consider $rw = \lambda x: \{ \kwa{File, Socket} \}.~\kwa{x.write}$, which takes either a $\File$ or a $\Socket$ and writes to it. If $rw$ is applied, it could incur $\kwa{Socket.write}$ or $\kwa{File.write}$, depending on what had been passed. In general, there is no way to statically determine what this will be, so the safe approximation is $\{ \kwa{File.write, Socket.write} \}$. This is the approximation given by a judgement like $\vdash rw~\File: \Unit~\kw{with} \{ \kwa{File.write, Socket.write} \}$. A derivation of this judgement is given in Figure~\ref{fig:opercalc_tree}. To fit on the page, all resources and operations have been abbreviated to their first letter. A developer who only expects $rw$ to be incurring $\kwa{File.write}$ can typecheck $rw$, see that it could also be writing to $\kwa{Socket}$, and decide it should not be used. If client code was annotated with $\kwa{ \{ \kwa{File.write} \} }$ and tried to use this function, the type system would reject it.

\begin{figure}[h]


    \begin{prooftree*}

    		\Infer0[\textsc{($\varepsilon$-Var)}]{x: \{ \kwa{F}, \kwa{S} \} \vdash x: \{ \kwa{F}, \kwa{S} \}}
    		
    		\Infer1[\textsc{($\varepsilon$-OperCall)}]{x: \{ \kwa{F}, \kwa{S} \} \vdash \kwa{x.w} : \Unit~\kw{with} \{ \kwa{F.w, S.w} \} }
    		
    		\Infer1[\textsc{($\varepsilon$-Abs)}]{ \lambda x: \{ \kwa{F}, \kwa{S} \}.~\kwa{x.w} : \{ \kwa{F, S} \} \rightarrow_{\{\kwa{F.w, S.w}\}} \Unit~\kw{with} \varnothing }
    		
    
       \Infer0[\textsc{($\varepsilon$-Resource)}]{\vdash \kwa{F}: \{ \kwa{F} \}~\kw{with} \varnothing}
    
    \Hypo{F \in \{ F, S \}}
    
    \Infer1[\textsc{(S-Resource)}]{\{ F \} <: \{ F, S \}}
    
    \Infer2[\textsc{($\varepsilon$-Subsume)}]{\vdash \kwa{F}: \{ \kwa{F, S} \}}
    
    		\Infer2[\textsc{($\varepsilon$-App)}]{ \vdash (\lambda x: \{ \kwa{F, S} \}. ~\kwa{x.w})~\kwa{F} : \Unit~\kw{with} \{ \kwa{F.w, S.w} \}  }
    		
 	\end{prooftree*}
 	
\vspace{-12pt}
\caption{Derivation tree for $\vdash rw~\File: \Unit~\kw{with} \{ \kwa{File.write, Socket.write} \}$.}
\label{fig:opercalc_tree}
\end{figure}

\subsection{Soundness ($\opercalc$)}

To show the rules of $\opercalc$ are sound requires an appropriate notion of static approximations being safe with respect to the reductions. If a judgement like $\Gamma \vdash e: \tau~\kw{with} \varepsilon$ were correct, successive reductions on $e$ should never incur effects not in $\varepsilon$. Furthermore, as $e$ is reduced, we learn more about what it is, so approximations on the reduced forms can only get more specific; compare this with how the type of reduced expressions can only get more specific. Adding this to the standard definition of soundness yields the following theorem statement.

\begin{theorem}[$\opercalc$ Single-step Soundness]
If $ \Gamma \vdash  e_A:  \tau_A~\kw{with} \varepsilon_A$ and $ e_A$ is not a value or variable, then $e_A \longrightarrow e_B~|~\varepsilon$, where $ \Gamma \vdash e_B:  \tau_B~\kw{with} \varepsilon_B$ and $ \tau_B <:  \tau_A$ and $\varepsilon_B \cup \varepsilon \subseteq \varepsilon_A$, for some $e_B, \varepsilon, \tau_B, \varepsilon_B$.
\end{theorem}

Our approach to proving soundness is to show progress and preservation. Noting that the rules for values give $\varnothing$ as their approximate effects, the proof of the progress theorem is routine.

\begin{theorem}[$\opercalc$ Progress]
If $ \Gamma \vdash  e:  \tau~\kw{with} \varepsilon$ and $ e$ is not a value or variable, then $ e \longrightarrow  e'~|~\varepsilon'$, for some $e', \varepsilon' \subseteq \varepsilon$.
\end{theorem}

\begin{proof} By induction on derivations of $ \Gamma \vdash  e:  \tau~\kw{with} \varepsilon$.
\end{proof}

To show preservation we need to know that effect safety is preserved by the substitution in \textsc{E-App3}. The semantics are call-by-value, so the name of a function argument is only ever replaced with a value, and we know that the approximate effects of values are $\varnothing$, so the substitution does not introduce more effects. Beyond this observation, the proof is routine.

\begin{theorem}[$\opercalc$ Preservation]
If $\Gamma \vdash e_A: \tau_A~\kw{with} \varepsilon_A$ and $e_A \longrightarrow e_B~|~\varepsilon$, then $\Gamma \vdash e_B: \tau_B~\kw{with} \varepsilon_B$, where $\tau_B <: \tau_A$ and $\varepsilon_B \cup \varepsilon \subseteq \varepsilon_A$, for some $e_B, \varepsilon, \tau_B, \varepsilon_B$.
\end{theorem}

\begin{proof}  By induction on the derivations of $\Gamma \vdash e_A: \tau_A~\kw{with} \varepsilon_A$ and $e_A \longrightarrow e_B~|~\varepsilon$.
\end{proof}

The single-step soundness theorem now holds by combining progress and preservation. The soundness of multi-step reductions follows by inducting on the length of a multi-step and appealing to single-step soundness.

\begin{theorem}[$\opercalc$ Single-step Soundness]
If $ \Gamma \vdash  e_A:  \tau_A~\kw{with} \varepsilon_A$ and $ e_A$ is not a value or variable, then $e_A \longrightarrow e_B~|~\varepsilon$, where $ \Gamma \vdash e_B:  \tau_B~\kw{with} \varepsilon_B$ and $ \tau_B <:  \tau_A$ and $\varepsilon_B \cup \varepsilon \subseteq \varepsilon_A$, for some $e_B, \varepsilon, \tau_B, \varepsilon_B$.
\end{theorem}
\begin{proof}
If $ e_A$ is not a value or variable then the reduction exists by the progress theorem. The rest follows by the preservation theorem.
\end{proof}

\begin{theorem}[$\opercalc$ Multi-step Soundness]
If $ \Gamma \vdash  e_A:  \tau_A~\kw{with} \varepsilon_A$ and $e_A \longrightarrow^{*} e_B~|~\varepsilon$, then $\Gamma \vdash e_B: \tau_B~\kw{with} \varepsilon_B$, where $ \tau_B <: \tau_A$ and $\varepsilon_B \cup \varepsilon \subseteq \varepsilon_A$.
\end{theorem}

\begin{proof} By induction on the length of the multi-step reduction.
\end{proof}