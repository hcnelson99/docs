\section{Translations}

In this section we develop notation and techniques so our calculi can express the practical examples of the next section. To do this, we show how to encode $\unit$ and $\kwa{let}$ in $\epscalc$, make some simplifying assumptions, and show how Wyvern-like programs can be translated into $\epscalc$. With these, we hope to convince the reader that $\epscalc$ adequately captures the properties of capability-safe languages.

\subsection{Unit, Let}

The $\unit$ literal is defined as $\unit \defn \lambda x: \varnothing.~x$. It is the same in both annotated and unannotated code. In annotated code, it has the type $\Unit \defn \varnothing \rightarrow_{\varnothing} \varnothing$, while in unannotated code it has the type $\Unit \defn \varnothing \rightarrow \varnothing$. These are technically two separate types, but we will not distinguish between them. Note that $\unit$ is a value, and because $\varnothing$ is uninhabited (there is no empty resource literal), $\unit$ cannot be applied to anything. Furthermore, $\vdash \unit: \Unit~\kw{with} \varnothing$ by \textsc{$\varepsilon$-Abs}, and $\vdash \unit: \Unit$ by \textsc{T-Abs}. We use $\Unit$ to represent the absence of information, such as when a function takes no input or returns no value

The expression $\letxpr{x}{\hat e_1}{\hat e_2}$ reduces $\hat e_1$ to a value $\hat v_1$, binds it to the name $x$ in $\hat e_2$, and then executes $[\hat v_1/x]\hat e_2$. If $\hat \Gamma \vdash \hat e_1: \hat \tau_1~\kw{with} \varepsilon_1$, then $\letxpr{x}{\hat e_1}{\hat e_2} \defn (\lambda x: \hat \tau_1 . \hat e_2) \hat e_1$\footnote{We could also define an unannotated version of $\kwa{let}$, but we only need the annotated version.}. If $\hat e_1$ is a non-value, we can reduce the $\kwa{let}$ by \textsc{E-App2}. If $\hat e_1$ is a value, we may apply \textsc{E-App3}, which binds $\hat e_1$ to $x$ in $\hat e_2$. $\kwa{let}$ expressions can be typed using \textsc{$\varepsilon$-App}.

\subsection{Modules}

Wyvern's modules are first-class, desugaring into objects --- invoking a module's function is no different from invoking an object's method.
%For the purposes of this paper, Wyvern's objects can be viewed as records, with methods represented as functions.
%There are two kinds of modules: pure and resourceful. For our purposes, a pure module is one with no (transitive) authority over any resources, while a resource module has (transitive) authority over some resource. A pure module may still be given a capability, for example as a function argument, but it may not possess or capture the capability for longer than the duration of the method call.
Figure \ref{fig:wyv_modules} shows an example of two modules.
The first defines a single operation \kwat{tick} that takes an argument file and appends to it; the second is actually a \textit{functor} that takes the file as a module-level argument and uses that in the operation defined.
Modules are declared with the $\kwa{module}$ keyword, and we use \kwat{module} \kwat{def} for functors.

\begin{figure}[h]

\begin{lstlisting}
module Mod: FileTicker

def tick(f: {File}): Unit with {File.append}
   f.append

\end{lstlisting}

\begin{lstlisting}
module def Ftor(f: {File}): UnitTicker

def tick(): Unit with {File.append}
   f.append
\end{lstlisting}

\caption{Definition of two modules, the second of which is a functor.}
\label{fig:wyv_modules}
\end{figure}

Functors must be instantiated with appropriate arguments in order to produce a usable module.
When they are instantiated they are given the capabilities they require.
In Figure \ref{fig:wyv_modules}, $\kwa{Ftor}$ requests the use of a $\kwa{File}$ capability.
Figure \ref{fig:wyv_module_instantiation} demonstrates how the two modules above would be used. To prevent infinite regress the $\kwa{File}$ must, at some point, be introduced into the program. This happens in the client program. When the program begins execution, the $\kwa{File}$ capability is passed into the program from the system environment. The program then imports modules and instantiates functors with the capabilities they require. If a module is annotated, its function signatures will have effect annotations. For example, in Figure \ref{fig:wyv_modules}, $\kwa{Mod.tick}$ has the $\kwa{File.append}$ annotation, meaning it should typecheck as $\kwa{ \{ File \} \rightarrow_{\{\kwa{File.append}\}} \Unit }$. Both $\kwa{Mod}$ and $\kwa{Ftor}$ are annotated. 


\begin{figure}[h]

\begin{lstlisting}
require File

import Mod
instantiate Ftor(File)

Mod.tick(File)
Ftor.tick()
\end{lstlisting}

\caption{The client program instantiates $\kwa{Mod}$ and $\kwa{Ftor}$ and then invokes $\kwa{tick}$ on each.}
\label{fig:wyv_module_instantiation}
\end{figure}

Our Wyvern examples are simplified in several ways so they can be expressed in $\epscalc$. The only objects used are modules. The modules only ever contain one function and the capabilities they require; they have no mutable fields. There are no self-referencing modules or recursive functions. Modules do not reference each other cyclically. These simplifications enable us to model each module as a function. Applying the function will be equivalent to applying the single function defined by the module. A collection of modules is translated into $\epscalc$ as follows. First, a sequence of let-bindings are used to associate the name of a module with the function defined in it, and to associate the name of a functor with a constructor function that, when given the capabilities requested by a functor, will return the function representing a module instance. The constructor for a functor $\kwa{F}$ is called $\kwa{MakeF}$. If the module does not require any capabilities it takes $\Unit$ as its argument. A function is then defined which represents the body of code in the main program. When invoked, this function will instantiate all the functors by invoking their constructors and then will execute the code in from the main program. Finally, the function representing the program is invoked with the primitive capabilities that are passed in from the system environment.

Figure \ref{fig:wyv_tutorial_desugaring} shows how the examples above translate. Lines 1-2 define the module $\kwa{Mod}$.  Lines 4-6 define the constructor for $\kwa{Ftor}$. It requires a $\kwa{File}$ capability, so the constructor takes $\kwa{\{File\}}$ as its input type on line 5. The constructor for the main program is defined on lines 8-12, which instantiates \kwat{Ftor} and runs the main program code. Line 14 starts execution by invoking $\kwa{MakeMain}$ with the initial set of capabilities, which in this case is just $\kwa{File}$.

\begin{figure}[h]

\begin{lstlisting}
let Mod =
   $\lambda$f: {File}. f.append in

let MakeFtor =
   $\lambda$f: {File}.
      $\lambda$x: Unit. f.append in

let MakeMain =
   $\lambda$f: {File}.
      let Ftor = (MakeFtor f) in
      let r1 = (Mod f) in
      Ftor unit

MakeMain File
\end{lstlisting}

\caption{Translation of $\kwa{Mod}$ and $\kwa{Ftor}$ into $\epscalc$.}
\label{fig:wyv_tutorial_desugaring}
\end{figure}

When an unannotated module is translated into $\epscalc$, the translated contents will be encapsulated with an $\kwa{import}$ expression. The selected authority on the $\kwa{import}$ expression will be that we expect of the unannotated code according to the principle of least authority in the particular example under consideration. For example, if the client only expects the unannotated code to have the $\kwa{File.append}$ effect, the corresponding $\kwa{import}$ expression will select $\kwa{\{File.append\}}$.