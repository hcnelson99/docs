\documentclass{llncs}

\usepackage{listings}
\usepackage{proof}
\usepackage{amssymb}
\usepackage[margin=.9in]{geometry}
\usepackage{amsmath}
\usepackage[english]{babel}
\usepackage[utf8]{inputenc}
\usepackage{enumitem}
\usepackage{filecontents}
\usepackage{calc}
\usepackage[linewidth=0.5pt]{mdframed}
\allowdisplaybreaks

\usepackage{fancyhdr}
\renewcommand{\headrulewidth}{0pt}
\pagestyle{fancy}
 \fancyhf{}
\rhead{\thepage}

\lstset{tabsize=3, basicstyle=\ttfamily\small, commentstyle=\itshape\rmfamily, numbers=left, numberstyle=\tiny, language=java,moredelim=[il][\sffamily]{?},mathescape=true,showspaces=false,showstringspaces=false,columns=fullflexible,xleftmargin=5pt,escapeinside={(@}{@)}, morekeywords=[1]{objtype,module,import,let,in,fn,var,type,rec,fold,unfold,letrec,alloc,ref,application,policy,external,component,connects,to,meth,val,where,return,group,by,within,count,connect,with,attr,html,head,title,style,body,div,keyword,unit,def}}
\lstloadlanguages{Java,VBScript,XML,HTML}


\newcommand{\keywadj}[1]{\mathtt{#1}}
\newcommand{\keyw}[1]{\keywadj{#1}~}
\newcommand{\reftt}{\mathtt{ref}~}
\newcommand{\Reftt}{\mathtt{Ref}~}
\newcommand{\inttt}{\mathtt{int}~}
\newcommand{\Inttt}{\mathtt{Int}~}
\newcommand{\stepsto}{\leadsto}
\newcommand{\todo}[1]{\textbf{[#1]}}
\newcommand{\intuition}[1]{#1}
%\newcommand{\intuition}[1]{}

\newlist{pcases}{enumerate}{1}
\setlist[pcases]{
  label=\textit{Case}\protect\thiscase\textit{:}~,
  ref=\arabic*,
  align=left,
  labelsep=0pt,
  leftmargin=0pt,
  labelwidth=0pt,
  parsep=0pt
}
\newcommand{\pcase}[1][]{
  \if\relax\detokenize{#1}\relax
    \def\thiscase{}
  \else
    \def\thiscase{~#1}
  \fi
  \item
}

% definition of a configuration
\newcommand{\config}[1] { \langle #1 \rangle }

\begin{document}

\today

\section{Grammar}

\[
\begin{array}{lll}
\begin{array}{lllr}
	e & ::= & x & expressions \\
  		& | & r \\
  		& | & \keywadj{new}~x \Rightarrow \overline{\sigma = e} \\
  		& | & e.m(e)\\
  		& | & e.\pi(e)\\
		& | & l & (runtime form) \\
		&&\\
		
	v & ::= & r & values \\
		& | & l \\
		&&\\

	d & ::= & \keyw{def} m(x:\tau):\tau & unlabeled~decls.\\		&&\\
		
	\sigma & ::= & d~\keyw{with}\varepsilon  & labeled~ decls.\\
		&&\\
\end{array}
& ~~~~~~
&

\begin{array}{lllr}

	\tau & ::= & \{ \bar \sigma \}  & types \\
		& | & \{ \bar r \} \\
		&&\\

	\Gamma & :: = & \varnothing & contexts\\
		& | & \Gamma,~x : \tau\\
		&&\\

	\mu & :: = & \varnothing & store\\
		& | & \mu,~l \mapsto \{ x \Rightarrow \overline{\sigma = e}\} \\
		& | & \mu,~l \mapsto \{ x \Rightarrow \bar r \} \\
&&\\

\end{array}
\end{array}
\]


\begin{itemize}
	\item The contents of $\mu$ have the form $l \mapsto \{ x \Rightarrow \overline{\sigma = e}\}$, which means the memory address $l$ points to an object with the type $\{ \bar \sigma \}$. $x$ is the 'this' variable.
	\item For an expression $e$, $e[e_1/x_1, ..., e_n/x_n]$ is a new expression with the structure of $e$, where every free occurrence of $x_i$ is replaced with $e_i$.
	\item $\varnothing$ refers to the empty set. The empty type, consisting of zero method declarations, is denoted $\keywadj{Unit}$.
	\item A configuration is a pair $\config{\mu~;~ e}$.
	\item $\config{\mu~;~e} \longrightarrow \config{\mu'~;~e'}$ if, after one reduction step on $e$ in heap $\mu$, the program ends in heap $\mu'$ and ready to execute $e'$.
	\item To execute a program $e$ is to perform reduction steps starting from the configuration $\config{\varnothing, e}$.
	\item $\config{\mu_1~;~e_1} \longrightarrow_* \config{\mu_2~;~e_2}$ if the configuration $\config{\mu_2~;~e_2}$ can be reached from the configuration $\config{\mu_1, e_1}$ by the application of one or more reduction rules.
	\item If $\config{\mu_1~;~e_1} \longrightarrow_* \config{\mu_2~;~v}$, for some value $v$, then we say that $\config{\mu_1~;~e_1}$ terminates.
\end{itemize}

\section{Dynamic Semantics}

This first section introduces a basic dynamic semantics that has no notion of an effect.
\\

\fbox{$\config{\mu, e} \longrightarrow \config{\mu, e}$}

\[
\begin{array}{lll}
\begin{array}{lllr}
	\infer[\textsc{(E-MethCall1)}]
		{\config{\mu~;~e_1.m(e_2)} \longrightarrow \config{\mu'~;~e_1'.m(e_2)}}
		{\config{\mu~;~e_1} \longrightarrow \config{\mu~;~e_1'}}
		
		~~~~~~
		
	\infer[\textsc{(E-MethCall2)}]
		{\config{\mu~;~l.m(e_2)} \longrightarrow \config{\mu'~;~l.m(e_2')}}
		{\config{\mu~;~e_2} \longrightarrow \config{\mu'~;~e_2'}}\\[5ex]
		
	\infer[\textsc{(E-MethCall3)}]
		{\config{\mu~;~l.m(v)}
			\longrightarrow
		 \config{\mu~;~e'[l/x, v/y]}}
  		{l \mapsto \{ x \Rightarrow \overline{\sigma = e} \} \in \mu & \keywadj{def~m}(y: \tau_1) : \tau_2~\keyw{with} \varepsilon = e' \in \overline {\sigma = e}} \\[5ex]

			
	\infer[\textsc{(E-OperCall1)}]
		{\config{\mu~;~e_1.\pi(e_2)}
			\longrightarrow
		\config{\mu'~;~e_1'.\pi(e_2)}}
		{\config{\mu~;~e_1} \longrightarrow \config{\mu'~;~e_1'}}
~~~~~~
			\infer[\textsc{(E-OperCall2)}]
		{\config{\mu~;~r.\pi(e_2)}
			\longrightarrow
		\config{\mu'~;~r.\pi(e_2')}}
		{\config{\mu~;~e_2} \longrightarrow \config{\mu'~;~e_2'}} \\[5ex]
			
			\infer[\textsc{(E-OperCall3)}]
		{\config{\mu~;~r.\pi(v)}
			\longrightarrow
		\config{\mu'~;~\keywadj{Unit}}}
		{r \in R & \pi \in \Pi} \\[5ex]
			
	\infer[\textsc{(E-New$_\sigma$)}]
		{\config{\mu~;~\keywadj{new~x} \Rightarrow \overline{\sigma = e} }
			\longrightarrow
		 \config{\mu, l \mapsto \keywadj{new~x} \Rightarrow \overline{\sigma = e}~;~l }}
		{l \notin dom(\mu)} \\[5ex]
		
\end{array}
\end{array}
\]

\section{Dynamic Semantics With Effects}

We amend the definition of configuration. A configuration is a triple $\config{\mu, e, \varepsilon}$, where $\varepsilon$ is an accumulated set of effects (i.e. pairs from $R \times \Pi$) from the computation so far. A program $e$ has the effect $(r, \pi)$ if $\config{\varnothing~;~e~;~\varnothing} \longrightarrow_* \config{\mu~;~e'~;~\varepsilon}$, where $(r, \pi) \in \varepsilon$. 
\\

\fbox{$\config{\mu, e, \varepsilon} \longrightarrow \config{\mu, e, \varepsilon}$}

\[
\begin{array}{lll}
\begin{array}{lllr}
	\infer[\textsc{(E-MethCall1)}]
		{\config{\mu~;~e_1.m(e_2)~;~\varepsilon} \longrightarrow \config{\mu'~;~e_1'.m(e_2)~;~\varepsilon'}}
		{\config{\mu~;~e_1~;~\varepsilon} \longrightarrow \config{\mu~;~e_1'~;~\varepsilon'}}
		
		~~~~~~
		
	\infer[\textsc{(E-MethCall2)}]
		{\config{\mu~;~l.m(e_2)~;~\varepsilon} \longrightarrow \config{\mu'~;~l.m(e_2')~;~\varepsilon'}}
		{\config{\mu~;~e_2~;~\varepsilon} \longrightarrow \config{\mu'~;~e_2'~;~\varepsilon'}}\\[5ex]
		
	\infer[\textsc{(E-MethCall3)}]
		{\config{\mu~;~l.m(v)~;~\varepsilon}
			\longrightarrow
		 \config{\mu~;~e'[l/x, v/y]~;~\varepsilon}}
  		{l \mapsto \{ x \Rightarrow \overline{\sigma = e} \} \in \mu & \keywadj{def~m}(y: \tau_1) : \tau_2~\keyw{with} \varepsilon_2 = e' \in \overline {\sigma = e}} \\[5ex]

	\infer[\textsc{(E-OperCall1)}]
		{\config{\mu~;~e_1.\pi(e_2)~;~\varepsilon}
			\longrightarrow
		\config{\mu'~;~e_1'.\pi(e_2)~;~\varepsilon'}}
		{\config{\mu~;~e_1~;~\varepsilon} \longrightarrow \config{\mu'~;~e_1'~;~\varepsilon'}}
~~~~~~
			\infer[\textsc{(E-OperCall2)}]
		{\config{\mu~;~r.\pi(e_2)~;~\varepsilon}
			\longrightarrow
		\config{\mu'~;~r.\pi(e_2')~;~\varepsilon'}}
		{\config{\mu~;~e_2~;~\varepsilon} \longrightarrow \config{\mu'~;~e_2'~;~\varepsilon'}} \\[5ex]
			
			\infer[\textsc{(E-OperCall3)}]
		{\config{\mu~;~r.\pi(v)~;~\varepsilon}
			\longrightarrow
		\config{\mu'~;~\keywadj{Unit}~;~\varepsilon \cup \{(r,m)\}}}
		{r \in R & \pi \in \Pi} \\[5ex]
			
	\infer[\textsc{(E-New$_\sigma$)}]
		{\config{\mu~;~\keywadj{new~x} \Rightarrow \overline{\sigma = e}~;~\varepsilon}
			\longrightarrow
		 \config{\mu, l \mapsto \keywadj{new~x} \Rightarrow \overline{\sigma = e}~;~l~;~\varepsilon}}
		{l \notin dom(\mu)} \\[5ex]
		
\end{array}
\end{array}
\]

\end{document}
