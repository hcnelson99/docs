\documentclass{llncs}

\usepackage{listings}
\usepackage{proof}
\usepackage{amssymb}
\usepackage[margin=.9in]{geometry}
\usepackage{amsmath}
\usepackage[english]{babel}
\usepackage[utf8]{inputenc}
\usepackage{enumitem}
\usepackage{filecontents}
\usepackage{calc}
\usepackage[linewidth=0.5pt]{mdframed}
\usepackage{changepage}
\allowdisplaybreaks

\usepackage{fancyhdr}
\renewcommand{\headrulewidth}{0pt}
\pagestyle{fancy}
 \fancyhf{}
\rhead{\thepage}

\lstset{tabsize=3, basicstyle=\ttfamily\small, commentstyle=\itshape\rmfamily, numbers=left, numberstyle=\tiny, language=java,moredelim=[il][\sffamily]{?},mathescape=true,showspaces=false,showstringspaces=false,columns=fullflexible,xleftmargin=5pt,escapeinside={(@}{@)}, morekeywords=[1]{objtype,module,import,let,in,fn,var,type,rec,fold,unfold,letrec,alloc,ref,application,policy,external,component,connects,to,meth,val,where,return,group,by,within,count,connect,with,attr,html,head,title,style,body,div,keyword,unit,def}}
\lstloadlanguages{Java,VBScript,XML,HTML}

\newcommand{\keywadj}[1]{\mathtt{#1}}
\newcommand{\keyw}[1]{\keywadj{#1}~}
\newcommand{\reftt}{\mathtt{ref}~}
\newcommand{\Reftt}{\mathtt{Ref}~}
\newcommand{\inttt}{\mathtt{int}~}
\newcommand{\Inttt}{\mathtt{Int}~}
\newcommand{\stepsto}{\leadsto}
\newcommand{\todo}[1]{\textbf{[#1]}}
\newcommand{\intuition}[1]{#1}
\newcommand{\hyphen}{\hbox{-}}

%\newcommand{\intuition}[1]{}

\newlist{pcases}{enumerate}{1}
\setlist[pcases]{
  label=\fbox{\textit{Case}}\protect\thiscase\textit{:}~,
  ref=\arabic*,
  align=left,
  labelsep=0pt,
  leftmargin=0pt,
  labelwidth=0pt,
  parsep=0pt
}
\newcommand{\pcase}[1][]{

  \if\relax\detokenize{#1}\relax
    \def\thiscase{}
  \else
    \def\thiscase{~\fbox{#1:}}
  \fi
  \item
}

\newcommand{\thm}[3]{
	\begin{large}
		\bf{#1}
	\end{large} \\\\
	\fbox{Statement.} ~ #2
	\fbox{Proof.}~ #3 \qed
}

\newcommand{\proofcase}[2]{
	\begin{adjustwidth}{1.5em}{0pt}
		\fbox{Case.}~~#1. \\ ~#2
	\end{adjustwidth}
}

\newcommand{\subcase}[1] {
	\begin{adjustwidth}{2.2em}{0pt}
		\underline{Subcase.} #1
	\end{adjustwidth}
}

\newcommand{\stmt}[1] {

\begin{adjustwidth}{2.5em}{0pt}
	#1
\end{adjustwidth}

}
\newcommand{\type}[2]{
	#1~\keyw{with} #2
}

\newcommand{\newd}[0]{
	\keywadj{new}_d~x \Rightarrow \overline{d = e}
}

\newcommand{\newsig}[0]{
	\keywadj{new}_\sigma~x \Rightarrow \overline{\sigma = e}
}

\begin{document}

\today
\section{Effects}

\noindent
R is a fixed set of resources. We define a resource as a language primitive with the authority to perform I/O operations. $\Pi$ is the set of I/O operations. In this document we cannot dynamically create resources or operations on resources.\\

\noindent
Members of $R$ are denoted $r$; members of $\Pi$ are denoted $\pi$. $r.\pi$ is syntactic sugar for $(r,\pi)$ (for example, $\keywadj{FileIO.append}$ instead of $\keywadj{(FileIO, append)}$). An effect is a member of the pairs $R \times \Pi$. A set of effects is denoted by $\varepsilon$. \\

\noindent
We say a piece of code $C$ has the runtime effect $r.\pi$ if $r.\pi$ is called during the execution of $C$. $C$ captures $r.\pi$ if it has the authority to call $r.\pi$ at some point during its execution. The static effects of $C$ is an approximation of the runtime effects by our typing system. Later on we show the static effects of $C$ give a conservative upper-bound on the runtime effects. \\

\noindent
Types in our system are either resources or structural. Structural types are distinguished by what method declarations they have. The type with no methods is called $\keywadj{Unit}$; its unique instance of denoted $\keywadj{unit}$. Although they look similar in form, operations and methods are distinct. Methods can only be invoked by objects; operations can only be invoked by resources. \\

\noindent
We make some simplifying assumptions about methods and operations. Methods take exactly one argument. If the argument is not specified it is assumed to be $\keywadj{unit}$. Invoking some operation $r.\pi$ returns $\varnothing$. We don't model arguments to operations, so all operations are null-ary.


\section{Static Semantics For Fully-Annotated Programs}

\noindent
In this first system every method in the program is explicitly annotated with its set of effects.

\subsection{Grammar}

\[
\begin{array}{lll}
\begin{array}{lllr}

	e & ::= & x & expressions \\
  		& | & r \\
  		& | & \newsig \\
  		& | & e.m(e)\\
  		& | & e.\pi\\
		&&\\

	\tau & ::= & \{ \bar \sigma \} ~ | ~ \{ \bar r \} & types \\
		&&\\

	\sigma & ::= &  \keyw{def} m(x:\tau):\tau~\keyw{with}\varepsilon  & labeled~ decls.\\
		&&\\
		
	\Gamma & ::= & \varnothing \\
		& | & \Gamma,~x : \tau\\
		&&\\

\end{array}
& ~~~~~~
&
\end{array}
\]


\noindent \textbf{Notes:}

\begin{itemize}
	\item All declarations ($\sigma$-terms) are annotated by what effects they have.
	\item All methods take exactly one argument. If a method specifies no argument the argument is assumed to be of type $\keywadj{Unit}$.
	\item The type $\{ \bar r \}$ is a set of resources; there will only be one actual resource at run-time, and it will be one of the resources in the set. This covers the case where e.g. a conditional returns a different resource on either branch.
\end{itemize}

\subsection{Static Semantics}
\noindent
\fbox{$\Gamma \vdash e : \tau~\keyw{with} \varepsilon$}

\[
\begin{array}{c}
\infer[\textsc{($\varepsilon$-Var)}]
  {\Gamma,~x:\tau \vdash x: \type{\tau}{\varnothing} }
  {} 
~~~~~~~~~~
\infer[\textsc{($\varepsilon$-Resource)}]
  {\Gamma,~r :\{\bar r\} \vdash r: \type{   \{ \bar r \}  }{\varnothing}}
  {} \\[5ex]

\infer[\textsc{($\varepsilon$-NewObj)}]
	{\Gamma \vdash \newsig : \{ \bar \sigma \}~\keyw{with} \varnothing}
	{\Gamma,~x : \{ \bar \sigma \} \vdash \overline{\sigma = e}~\keywadj{OK}} ~~~~~~

\infer[\textsc{($\varepsilon$-OperCall)}]
	{\Gamma \vdash e_1.\pi : \keyw{Unit} \keyw{with} \{ \bar r.\pi \} \cup \varepsilon_1}
	{\Gamma \vdash e_1 : \{ \bar r \}~ \keyw{with} \varepsilon_1} \\[5ex]
	
\infer[\textsc{($\varepsilon$-MethCall$_\sigma$)}]
	{\Gamma \vdash e_1.m_i(e_2):\tau_3~\keyw{with} \varepsilon_1 \cup \varepsilon_2 \cup \varepsilon_3}
	{\Gamma \vdash e_1 : \{ \bar \sigma \}~\keyw{with} \varepsilon_1 & \Gamma \vdash e_2 : \tau_2~\keyw{with} \varepsilon_2 & \sigma_i = \keyw{def} m_i(y:\tau_2):\tau_3~\keyw{with} \varepsilon_3}

\end{array}
\]
\noindent
$\fbox{$\Gamma \vdash \sigma = e~\keyw{OK}$}$

\[
\begin{array}{c}
\infer[\textsc{($\varepsilon$-ValidImpl$_\sigma$)}]
	{\Gamma \vdash \sigma = e~\keywadj{OK}}
	{\Gamma,~y:\tau_2 \vdash e: \tau_3~\keyw{with} \varepsilon_3 & \sigma = \keyw{def} m(y:\tau_2):\tau_3~\keyw{with} \varepsilon_3} \\[5ex]
\end{array}
\]

\noindent \textbf{Notes:}

\begin{itemize}
	\item In \textsc{$\varepsilon$-Var}, \textsc{$\varepsilon$-Resource}, and \textsc{$\varepsilon$-NewObj} the consequent has an expression typed with no effect. In these rules we may be gaining an authority for an effect but we must use it in some code for that effect to happen.
	\item \textsc{$\varepsilon$-ValidImpl} says that the return type and effects of the body of a method must be exactly the same as its declaration.
\end{itemize}



\subsection{Dynamic Semantics}
~\\
\noindent
\fbox{$e \longrightarrow e~|~\varepsilon$}

\[
\begin{array}{c}

	\infer[\textsc{(E-MethCall1$_{\sigma}$)}]
		{e_1.m(e_2) \longrightarrow e_1'.m(e_2)~|~\varepsilon}
		{e_1 \longrightarrow e_1'~|~\varepsilon}

		~~~~~~

	\infer[\textsc{(E-MethCall2$_{\sigma}$)}]
		{v_1.m(e_2) \longrightarrow v_1.m(e_2')~|~\varepsilon}
		{v_1 = \keywadj{new}_{\sigma}~x \Rightarrow \overline{\sigma = e} & e_2 \longrightarrow e_2'~|~\varepsilon}\\[5ex]
				
		
	\infer[\textsc{(E-MethCall3$_{\sigma}$)}]
		{v_1.m(v_2)
			\longrightarrow
		 [v_1/x, v_2/y]e~|~\varnothing}
  		{v_1 = \keywadj{new}_{\sigma}~x \Rightarrow \overline{\sigma = e} & \keywadj{def~m}(y: \tau_1) : \tau_2~\keyw{with} \varepsilon = e \in \overline {\sigma = e}} \\[5ex]

	\infer[\textsc{(E-OperCall1)}]
		{e_1.\pi
			\longrightarrow
		 e_1'.\pi~|~\varepsilon}
		{e_1 \longrightarrow e_1'~|~\varepsilon}
~~~~~~

			\infer[\textsc{(E-OperCall2)}]
		{r.\pi
			\longrightarrow
		 \keywadj{unit}~|~\{r.\pi\}}
		{} \\[5ex]
			
\end{array}
\]
\noindent
\fbox{$e \longrightarrow_* e~|~\varepsilon$}

\[
\begin{array}{c}

	\infer[\textsc{(E-MultiStep1)}]
	{e \longrightarrow_* e~|~\varnothing}
	{} ~~~~~~~~
	
	\infer[\textsc{(E-MultiStep2)}]
	{e \longrightarrow_* e'~|~\varepsilon}
	{e \longrightarrow e'~|~\varepsilon} \\[5ex]

	\infer[\textsc{(E-MultiStep3)}]
	{e \longrightarrow_* e''~|~\varepsilon_1 \cup \varepsilon_2}
	{e \longrightarrow_* e'~|~\varepsilon_1 & e' \longrightarrow_* e''~|~\varepsilon_2}

\end{array}
\]

\noindent \textbf{Notes:}
\begin{itemize}
	\item A multi-step involves \textit{zero} or more applications of a small-step.
	\item Multi-step rules accumulate the run-time effects produced by the individual small-steps.
	\item The only rule which produces effects is \textsc{E-OperCall2} (the rule for evaluating operations on resources).
	\item Method calls are evaluated by performing substitution on the body of the method. There is no store.
\end{itemize}






%%%%%%%%%%%%%%%%%%%%%%%%%%%%%%%%%%%%%%
\subsection{Substitution Function}
%%%%%%%%%%%%%%%%%%%%%%%%%%%%%%%%%%%%%%

~
\begin{large}
	\bf{Definition 2.4.1. (Substitution)}
\end{large}

$[e'/z]e$ means 'substitute every free occurrence of $z$ in $e$ for $e'$. Here's a definition over rules in the grammar.

\begin{itemize}
	\item $[e'/z]z = e'$
	\item $[e'/z]y = y$, if $y \neq z$
	\item $[e'/z]r = r$
	\item $[e'/z](e_1.m(e_2)) = ([e'/z]e_1).m([e'/z]e_2)$
	\item $[e'/z](e_1.\pi) = ([e'/z]e_1).\pi$
	\item $[e'/z](\newsig) = \keywadj{new}_\sigma~x \Rightarrow \overline{ \sigma = [e'/z]e }$, if $z \neq x$ and $z \notin \keywadj{freevars}(e_i)$
\end{itemize}

\noindent
We use the convention of \textit{alpha-conversion} to make substitution capture-avoiding. In practice, this means that whenever substitution is undefined because of variable capture, we rename variables to meet the side conditions (see Benjamin Pierce, ``Types and Programming Languages'' p. 71 for more details). \\

\noindent
Substitution of multiple variables is written $[e_1/z_1, ..., e_n/z_n]e$, which is shorthand for $[e_n/z_n]...[e_1/z_1]e$ \\

\thm{Lemma 2.4.2. (Substitution Lemma)}
{If $\Gamma, z : \tau' \vdash e : \tau~\keyw{with} \varepsilon$, and $\Gamma \vdash e' : \tau'~\keyw{with} \varepsilon'$, then $\Gamma \vdash [e'/z]e : \tau~\keyw{with} \varepsilon$. \\\\
\fbox{Intuition} If you substitute $z$ for something of the same type, the type of the whole expression stays the same after substitution.
\\\\}
{By structural induction on possible derivations of $\Gamma \vdash e : \tau~\keyw{with} \varepsilon$. First, if $z$ does not appear in $e$ then $[e'/z]e = e$, so the statement holds vacuously. So without loss of generality we assume $z$ appears somewhere in $e$ and consider the last rule used in the derivation, and then the location of $z$. \\

	\proofcase{\textsc{$\varepsilon$-Var}} {
		Then $[e'/z]z = e_1$. By assumption $\Gamma \vdash e' : \tau~\keyw{with} \varepsilon$, so $\Gamma \vdash [e'/z]z = e$. \\
	}
		
	\proofcase{\textsc{$\varepsilon$-Resource}} {
		Then $e = r$. $\keywadj{freevars}(r) = \varnothing$, so the statement holds vacuously.\\
	}
		
	\proofcase{\textsc{$\varepsilon$-NewObj}} {
		Then $e = \newsig$. $z$ appears in some method body $e_i$. By inversion we know $\Gamma, x : \{ \bar \sigma \} \vdash \overline {\sigma = e}~\keywadj{OK}$. The only rule with this conclusion is \textsc{$\varepsilon$-ValidImpl$_\sigma$}; by inversion on that we have:
		 \begin{itemize}
		 	\item $\sigma_i = \keyw{def} m_i(y : \tau_1) : \tau_2~\keyw{with} \varepsilon$
		 	\item $\Gamma, y : \tau_1 \vdash e_i : \tau_2~\keyw{with} \varepsilon$
		\end{itemize}
	$\Gamma, z : \tau \vdash \sigma_i = e_i~\keywadj{OK}$ via the inductive assumption. We can use \textsc{$\varepsilon$-ValidImpl$_\sigma$} to conclude $\overline{\sigma = [e'/z]e}~\keywadj{OK}$. Then by \textsc{$\varepsilon$-NewObj} we type $[e'/z]e$ to the same as the original.\\
	}
	
	\proofcase{\textsc{$\varepsilon$-OperCall}} {
		Then $e = e_1.\pi$. The variable $z$ must appear in $e_1$. By rule inversion we have a sub-derivation for the type of both sub-expressions so applying the inductive assumption we know $e_1$ and $[e'/z]e_1$ have the same types. Since $e$ and $[e'/z]e$ have the same syntactic structure, and their corresponding subexpressions have the same types, then \textsc{$\varepsilon$-OperCall} will type $[e'/z]e$ to the same thing as $e$. \\
	}

	\proofcase{\textsc{$\varepsilon$-MethCall$_\sigma$}} {
		Then $e =e _1.m_i(e_2)$. The variable $z$ must appear in $e_1$ or $e_2$. By rule inversion we have a sub-derivation for both so applying the inductive hypothesis we know the types of $e_1$ and $[e'/z]e_1$ are the same, and the types of $e_2$ and $[e'/z]e_2$ are the same. Since $e$ and $[e'/z]e$ have the same syntactic structure, and their corresponding subexpressions have the same types, then \textsc{$\varepsilon$-MethCall} types $[e'/z](e_1.m_i(e_2))$ to the same as $e_1.m_i(e_2)$.
	}
}

\noindent
\fbox{Corollary.} If $\Gamma, z_i : \tau'_i \vdash e : \tau~\keyw{with} \varepsilon$, and $\Gamma \vdash e'_i : \tau'_i~\keyw{with} \varepsilon'_i$, then $\Gamma \vdash [e'_1/z_1, ..., e'_n/z_n]e : \tau~\keyw{with} \varepsilon$. This follows by the definition $[e'_1/z_1, ..., e'_n/z_n]e = [e'_n/z_n]...[e'_1/z_1]e$ and induction on the length $n$.


%%%%%%%%%%%%%%%%%%%%%%%%%%%%%%%%%%%%%%
\subsection{Soundness Theorem}
%%%%%%%%%%%%%%%%%%%%%%%%%%%%%%%%%%%%%%

\noindent
In this section we build up to the soundness theorem for our typing system. We do this by proving progress and two preservation theorems about types and static effects. The two preservation theorems directly give us the soundness statement.\\
 
 
 
 
 
 
 
 
 
 
\thm{Lemma 2.5.1. (Canonical Forms)}
{Suppose $e$ is a value. The following are true:
	\begin{itemize}
	\item If $\Gamma \vdash e : \type{ \{ \bar r \} }{\varepsilon} $, then $e = r$ for some $r \in R$.
	\item If $\Gamma \vdash e : \type{ \{ \bar \sigma \}}{\varepsilon}$, then $e = \newsig$.
	\end{itemize}
Furthermore, $\varepsilon = \varnothing$ in each case. \\\\
}
{ These typing judgements each appear exactly once in the conclusion of different rules. The result follows by inversion of \textsc{$\varepsilon$-Resource} and \textsc{$\varepsilon$-NewObj} respectively. \\
}










\thm{Theorem 2.5.2. (Progress Theorem)}
{If $\Gamma \vdash e_A : \tau_A~\keyw{with} \varepsilon_A$, either $e_A$ is a value or a small-step $e_A \longrightarrow e_B~|~\varepsilon$ can be applied. \\\\}
{By induction on possible derivations of $\Gamma \vdash e_A : \tau_A~\keyw{with} \varepsilon_A$. Consider the last rule used. \\

	\proofcase{\textsc{$\varepsilon$-Var}, \textsc{$\varepsilon$-Resource}, \textsc{$\varepsilon$-NewObj}} {
		Then $e_A$ is a value. \\
	}
		
	\proofcase{\textsc{$\varepsilon$-MethCall$_\sigma$}} {
		Then $e_A = e_1.m_i(e_2)$ and the following are known:
		\begin{itemize}
			\item~$e_A : \tau_3~\keyw{with} \varepsilon_1 \cup \varepsilon_2 \cup \varepsilon_3$
			\item~$e_1  : \{ \overline \sigma \}~ \keyw{with} \varepsilon_1$
			\item~$e_2 : \tau_2~\keyw{with} \varepsilon_2$
			\item~$\sigma_i = \keyw{def} m_i(y : \tau_2) : \tau_3~ \keyw{with} \varepsilon_3$
		\end{itemize}
		We look at the cases for when $e_1$ and $e_2$ are values.
		
			\subcase{ $e_1$ is not a value. The derivation of $e_A : \tau~\keyw{with} \varepsilon_A$ includes the subderivation $e_1 : \{ \bar \sigma \}~\keyw{with} \varepsilon_1$. By the inductive hypothesis $e_1 \longrightarrow e_1'~|~\varepsilon$. Then \textsc{E-MethCall1} gives the reduction $e_A \longrightarrow e_1'.m_i(e_2)~|~\varepsilon$. }
	
			\subcase{ $e_2$ is not a value. Without loss of generality, $e_1 = v_1$ is a value. Also,
		$e_2 : \tau_2~\keyw{with} \varepsilon_2$ is a subderivation. By inductive hypothesis, $e_2 \longrightarrow e_2'~|~\varepsilon$. Then \textsc{E
		MethCall2$_{\sigma}$} gives the reduction $e_A \longrightarrow
		v_1.m_i(e_2')~|~\varepsilon$.}
		
			\subcase{ $e_1 = v_1$ and $e_2 = v_2$ are values. By Canonical Forms, $e_1 = \keywadj{new}_{\sigma}~x \Rightarrow \overline{\sigma = e}$. Also, $\keyw{def} m_i(y: \tau_2) : \tau_3~\keyw{with} \varepsilon_3 = e_i \in \overline{ \sigma = e }$. By the assumption of the typing rule used, the receiver and argument are well-typed for the method $m_i$. Then \textsc{E-MethCall3$_{\sigma}$} gives the reduction $e_1.m_i(e_2) \longrightarrow [v_1/x, v_2/y]e_i~|~\varnothing$.\\}

	}
	
	\proofcase{\textsc{$\varepsilon$-OperCall}} {
Then $e_A = e_1.\pi$ and the following are known:
\begin{itemize}
	\item~$e_A : \keyw{Unit~with} \{ r.\pi \} \cup \varepsilon_1$
	\item~$e_1 : \{ \bar r \}~\keyw{with} \varepsilon_1$
\end{itemize}
We look at the cases for when $e_1$ is a value.
	
		\subcase { $e_1$ is not a value. Then $e_1 : \{ \bar r \}~\keyw{with} \varepsilon_1$ is a subderivation. Applying inductive assumption, we have $e_1 \longrightarrow e_1'~|~\varepsilon$. Then \textsc{E-OperCall1} gives the reduction $e_1.\pi \longrightarrow e_1'.\pi~|~\varepsilon$.
		}
		
		\subcase { $e_1$ is a value. By Canonical Forms, $\exists r \in R \mid e_1 = r$. Then \textsc{E-OperCall3} gives the reduction $r.\pi \longrightarrow \keywadj{unit}~|~\{ r.\pi \}$.}
		
	}
}




\thm{Theorem 2.5.3. (Preservation Theorem)}
{Suppose the following hold:
	\begin{itemize}
		\item $\Gamma_A \vdash e_A : \tau_A~\keyw{with} \varepsilon_A$
		\item $e_A \longrightarrow e_B~|~\varepsilon$
		\item $\Gamma_B \vdash e_B : \tau_B~\keyw{with} \varepsilon_B$
	\end{itemize}
Then $\tau_B = \tau_A$. Also, $\varepsilon \subseteq \varepsilon_A$ and $\varepsilon_B \subseteq \varepsilon_A$  and $\forall r.\pi \in \varepsilon_A \setminus \varepsilon_B \mid r.\pi \in \varepsilon$.\\

\noindent
\fbox{Intuition.} Reduction preserves the relevant static effects in the sense that you only lose static effects during a computation if they actually happen. So you can't gain static effects during reduction, and every lost static effect is ``accounted for''.
\\\\}
{By induction on possible derivations of $\Gamma \vdash e_A : \tau_A~\keyw{with} \varepsilon_A$. Consider the last rule used.\\

	\proofcase{ \textsc{$\varepsilon$-Resource}, \textsc{$\varepsilon$-Var}, \textsc{$\varepsilon$-NewObj}} {
		$e_A$ is a value so no reduction can be applied to it. The theorem statement is vacuously satisfied.\\ }
	
	\proofcase{\textsc{$\varepsilon$-MethCall$_{\sigma}$}}
	{Then $e_A = e_1.m_i(e_2)$ and the following are true:
\begin{itemize}
	\item $e_A : \tau_3~\keyw{with} \varepsilon_1 \cup \varepsilon_2 \cup \varepsilon_3$
	\item $e_1  : \{ \overline \sigma \}~ \keyw{with} \varepsilon_1$
	\item $e_2 : \tau_2~\keyw{with} \varepsilon_2$
	\item $\sigma_i = \keyw{def} m_i(y : \tau_2) : \tau_3~ \keyw{with} \varepsilon_3$
\end{itemize}
The type to be preserved is $\tau_3$, so we have to show $\Gamma_B$ must have ascribed the type $\tau_3$ to $e_B$ in its judgement. We do a case analysis on the reductions applicable to $e_1.m_i(e_2)$.

		\subcase{ \textsc{E-MethCall1$_\sigma$.} Then $e_1 \longrightarrow e_1'~|~\varepsilon$. By inductive assumption $e_1' : \{ \bar \sigma \}~\keyw{with} \varepsilon_1'$. Then by \textsc{$\varepsilon$-MethCall} we have $e_B = e_1'.m_i(e_2) : \tau_3~\keyw{with} \varepsilon_1' \cup \varepsilon_2 \cup \varepsilon_3$. Then $\varepsilon_B = \varepsilon_1' \cup \varepsilon_2 \cup \varepsilon_3$ and $\varepsilon_A \setminus \varepsilon_B = \varepsilon_1 \setminus \varepsilon_1'$. For every $r.\pi \in \varepsilon_1 \setminus \varepsilon_1'$ we know $r.\pi \in \varepsilon$ by the inductive assumption. Since $e_B$ has the type $\tau_3$ the type is preserved.}
		
		\subcase{ \textsc{E-MethCall2$_\sigma$.} 	Then $e_1 = v_1 = \newsig$, and $e_2 \longrightarrow e_2'~|~\varepsilon$. By inductive assumption $e_2' : \tau_2~\keyw{with} \varepsilon_2$. Then by \textsc{$\varepsilon$-MethCall} we have $e_B = v_1.m_i(e_2) : \tau_3~\keyw{with} \varepsilon_1 \cup \varepsilon_2' \cup \varepsilon_3$. Then $\varepsilon_B = \varepsilon_1 \cup \varepsilon_2' \cup \varepsilon_3$ and $\varepsilon_A \setminus \varepsilon_B = \varepsilon_2 \setminus \varepsilon_2'$. For every $r.\pi \in \varepsilon_2 \setminus \varepsilon_2'$ we know $r.\pi \in \varepsilon$ by the inductive assumption. Since $e_B$ has the type $\tau_3$ the type is preserved.}
		
		\subcase{ \textsc{E-MethCall3$_\sigma$} Then $e_1 = v_1 = \newsig$, and $\keyw{def} m_i(y : \tau_2) : \tau_3~\keyw{with} \varepsilon_3 = e' \in \overline{\sigma = e}$, and $e_2 = v_2$ is a value, and $v_1.m_i(v_2) \longrightarrow [v_1/x, v_2/y]e'~|~\varnothing$. \\
	
	\noindent	
We already know $e_1  : \{ \overline \sigma \}~ \keyw{with} \varepsilon_1$. The only rule with this conclusion is \textsc{$\varepsilon$-NewObj}. By inversion, $\overline{ \sigma = e }~\keywadj{OK}$. The only rule with this conclusion is \textsc{$\varepsilon$-ValidImpl$_{\sigma}$}. By inversion, $e' : \tau_3~\keyw{with} \varepsilon_3$. \\

		\noindent
Now $e_B = [v_1/x, v_2/y]e'$ because the rule \textsc{E-MethCall3} was used. \textbf{We know $v_1 = e_1$ and $x$ have the same type, which is $\{ \overline \sigma \}~ \keyw{with} \varepsilon_1$. We also know $v_2 = e_2$ and $y$ have the same type, which is $\tau_2~\keyw{with} \varepsilon_2$.} By the substitution lemma, the type of $e'$ and $e_B = [v_1/x, v_2/y]e'$ is the same. So $e_B : \tau_3~\keyw{with} \varepsilon_3$, and namely $\varepsilon_B = \varepsilon_3$.\\

		\noindent
Since $e_1 = v_1$ and $e_2 = v_2$ are values, by Canonical Forms $\varepsilon_1 = \varepsilon_2 = \varnothing$. So $\varepsilon_A = \varepsilon_3$. Then $\varepsilon_A \setminus \varepsilon_B = \varnothing$, so there are no lost effects to account for. Since $e_B$ has the type $\tau_3$ the type is preserved.
\\ }
}


	\proofcase{\textsc{$\varepsilon$-OperCall}} {Then $e_A = e_1.\pi : \keyw{Unit}$, and we know:
	\begin{itemize}
		\item $e_A :  \{ r, \pi \} \cup \varepsilon_1$
		\item $e_1 : \{ \bar r \}~\keyw{with} \varepsilon_1$
	\end{itemize}
The type being preserved is $\keywadj{Unit}$, so we have to show $\Gamma_B$ must have ascribed $\keywadj{Unit}$ to $e_B$. There are two reduction rules applicable to terms of the form $e_1.\pi$. 

	\subcase { \textsc{E-OperCall1}. Then $e_1 \longrightarrow e_1'~|~\varepsilon$. By inductive assumption, $e_1' : \{ \bar r \} ~ \keyw{with} \varepsilon_1'$. Then by \textsc{$\varepsilon$-OperCall} we have $e_B = e_1'.\pi : \type{ \keywadj{Unit} }{\{r.\pi\} \cup \varepsilon_1'}$. Then $\varepsilon_B = \{ r.\pi \} \cup \varepsilon_1'$ and $\varepsilon_A \setminus \varepsilon_B = \varepsilon_1 \setminus \varepsilon_1'$. For every $r.\pi \in \varepsilon_1 \setminus \varepsilon_1'$ we know $r.\pi \in \varepsilon$ by the inductive assumption. Since $e_B$ has the type $\keywadj{Unit}$ the type is preserved.
	}
	
	\subcase{ \textsc{E-OperCall2}. Then $r.\pi \longrightarrow \keywadj{unit}~|~\{ r.\pi \}$. By Canonical Forms, $\varepsilon_1 = \varnothing$, so $e_A : \keyw{Unit~with} \{ r.\pi \}$. By a degenerate case of \textsc{$\varepsilon$-NewObj}, $e_B = \keywadj{unit} : \keyw{Unit~with} \varnothing$. Then $\varepsilon_A \setminus \varepsilon_B = \{ r.\pi \}$, which is exactly the set of runtime effects $\varepsilon$; our only lost effect is accounted for. Since $e_B$ has the type $\keywadj{Unit}$ the type is preserved.}	
}
}

\thm{Theorem 2.5.4. (Soundness Theorem)}
{If $\Gamma_A \vdash e_A : \tau_A~\keyw{with} \varepsilon_A$ either $e_A$ is a value or $e_A \longrightarrow e_B~|~\varepsilon$ and $\exists \Gamma_B \mid \Gamma_B \vdash e_B : \tau_B~\keyw{with} \varepsilon_B$, where $\tau_A = \tau_B$ and $\varepsilon \subseteq \varepsilon_A$ .\\\\}
{By the progress theorem we know either $e_A$ is a value or a small-step $e_A \longrightarrow e_B~|~\varepsilon$ can be applied. If a typing judgement $\Gamma_B \vdash e_B : \type{\tau_B}{\varepsilon_B}$ then we know $\tau_B = \tau_A$ and $\varepsilon \subseteq \varepsilon_A$. It is sufficient to show the existence of $\Gamma_B$. We proceed by induction on the typing judgement for $\Gamma_A \vdash e_A : \type{\tau_A}{\varepsilon_A}$.\\


	\proofcase{\textsc{$\varepsilon$-OperCall}} {
Then $e_A = e_1.\pi$, and we know:
	\begin{itemize}
		\item $e_A :  \{ r, \pi \} \cup \varepsilon_1$
		\item $e_1 : \{ \bar r \}~\keyw{with} \varepsilon_1$
	\end{itemize}
Consider the reduction rule used.

	\subcase { \textsc{E-OperCall1}. Then $e_1 \longrightarrow e_1'~|~\varepsilon$. Since $\Gamma_A \vdash e_1 : \tau_1~\keyw{with} \varepsilon_1$, then by inductive assumption $\exists \Gamma_A' \mid \Gamma_A' \vdash e_1' : \tau_1~\keyw{with} \varepsilon_1'$.
	}
	
	\subcase{ \textsc{E-OperCall2}. Then $r.\pi \longrightarrow \keywadj{unit}~|~\{ r.\pi \}$. Then $\Gamma_A \vdash \keywadj{unit} : \type{\keywadj{Unit}}{\varnothing}$ by a degenerate case of the \textsc{C-NewObj} rule, so choose $\Gamma_B = \Gamma_A$. }	
}
}

\thm{Theorem 2.5.4. (Soundness Theorem)}
{If $\Gamma_A \vdash e_A : \tau_A~\keyw{with} \varepsilon_A$ and $e \longrightarrow_* e_B~|~\varepsilon$ then $\exists \Gamma_B \mid \Gamma_B \vdash e_B : \tau_B~\keyw{with} \varepsilon_B$, where $\tau_A = \tau_B$ and $\varepsilon \subseteq \varepsilon_A$.\\\\}
{If the multi-step involves zero steps then the theorem is vacuously satisfied. Otherwise the multi-step involves more than one step. Induct on the number of steps. \\

\begin{adjustwidth}{1.5em}{0pt}
	\fbox{Base Case.}~ The length is 1. The theorem holds by Small-Step Soundness For Mini Calculus. \\
\end{adjustwidth}

\begin{adjustwidth}{1.5em}{0pt}
	\fbox{Inductive Case.}~ If there is a multi-step of length n+1 then by \textsc{E-MultiStep3} it can be decomposed into a multi-step of length n; $e_1 \longrightarrow_* e_n~|~\varepsilon_n$; and a multi-step of length 1; $e_n \longrightarrow_* e_{n+1}~|~\varepsilon$. Then $e_1 \longrightarrow_* e_{n+1}~|~\varepsilon_n \cup \varepsilon$ is the entire multi-step. \\
	
	\noindent
	Let the type derivations be $\Gamma \vdash e_1 : \tau_1~\keyw{with} \varepsilon_1$ and $\Gamma \vdash e_{n}~\keyw{with} \varepsilon_{n}$ \\
	
	\noindent
	 By inductive assumption on the smaller multi-steps, we have that $\varepsilon \subseteq \varepsilon_n$ and $\varepsilon_n \subseteq \varepsilon_1$. Then $\varepsilon_1 \cup \varepsilon_n \subseteq \varepsilon_1$. $\varepsilon_1$ is the static effect information at the start of the multi-step, so we're done.
\end{adjustwidth}
}










% ========================================
% Partially-Annotated Programs
% ========================================

\newpage

\section{Static Semantics For Partly-Annotated Programs}

\noindent
In the next system we allow objects which have no effect-annotated methods. If an object has no annotations on its methods, the extra rules below can give a conservative effect inference on what it captures. If an object is fully annotated, the rules from the previous section can be used. There is no in-between; either every method of an object is annotated, or none are.

\subsection{Grammar}

\[
\begin{array}{lll}

\begin{array}{lllr}

e & ::= & x & expressions \\
  		& | & r \\
		& | & \keywadj{new}_\sigma~x \Rightarrow \overline{\sigma = e} \\
 		& | & \keywadj{new}_d~x \Rightarrow \overline{d = e} \\
 		& | & e.m(e)\\
 		& | & e.\pi\\
		&&\\

\tau & ::= & \{ \bar \sigma \} & types \\
		& | & \{ \bar r \} \\
		& | & \{ \bar d \} \\
		& | & \{ \bar d ~\keyw{captures} \varepsilon \} \\
		&&\\
		
\sigma & ::= & d~\keyw{with}\varepsilon  & labeled~ decls.\\
		&&\\

d & ::= & \keyw{def} m(x:\tau):\tau & unlabeled~decls.\\
		&&\\

\end{array}
& ~~~~~~
&
\end{array}
\]

\noindent \textbf{Notes:}
\begin{itemize}
	\item $\sigma$ denotes a declaration with effect labels; $d$ a declaration without effect labels.
	\item $\keywadj{new}_\sigma$ is for creating annotated objects; $\keywadj{new}_d$ for unannotated objects.
	\item $\{ \bar \sigma \}$ is the type of an annotated object. $\{ \bar d \}$ is the type of an unannotated object.
	\item $\{ \bar d ~\keyw{captures} \varepsilon \}$ is a special kind of type that doesn't appear in source programs but may be assigned by the new rules in this section. Intuitively, $\varepsilon$ is an upper-bound on the effects captured by $\{ \bar d \}$.
\end{itemize}

\subsection{Rules}

$\fbox{$\Gamma \vdash e : \tau$}$

\[
\begin{array}{c}
\infer[\textsc{(T-Var)}]
	{\Gamma,~x : \tau \vdash x : \tau}
	{}
~~~~~~~~~~
\infer[\textsc{(T-Resource)}]
	{\Gamma,~r : \{ \bar r \} \vdash r : \{ \bar r \}}
	{} \\[5ex]
~~~~~~~~~~
\infer[\textsc{(T-OperCall)}]
	{\Gamma \vdash e_1.\pi : \keywadj{Unit}}
	{\Gamma \vdash e_1 : \{ \bar r \}} \\[5ex]
~~~~~~~~~~
\infer[\textsc{(T-MethCall$_\sigma$)}]
	{\Gamma \vdash e_1.m(e_2) : \tau_3}
	{\Gamma \vdash e_1 : \{ \bar \sigma \} & \keyw{def} m(y : \tau_2) : \tau_3~\keyw{with} \varepsilon_3 \in \{ \bar \sigma \} &  \Gamma \vdash e_2 : \tau_2 }\\[5ex]
~~~~~~~~~~
\infer[\textsc{(T-MethCall$_d$)}]
	{\Gamma \vdash e_1.m(e_2) : \tau_3}
	{\Gamma \vdash e_1 : \{ \bar d\} & \keyw{def} m(y : \tau_2) : \tau_3 \in \{ \bar d \} &  \Gamma \vdash e_2 : \tau_2 }\\[5ex]
~~~~~~~~~~

~~~~~~~~~~
\infer[\textsc{(T-New$_\sigma$)}]
	{\Gamma \vdash~\keywadj{new}_\sigma~x \Rightarrow \overline{\sigma = e} : \{ \bar \sigma \}}
	{\Gamma \vdash \sigma_i = e_i~\keywadj{OK}}
~~~~~~~~~~
\infer[\textsc{(T-New$_d$)}]
	{\Gamma \vdash~\keywadj{new}_d~x \Rightarrow \overline{d = e} : \{ \bar d \}}
	{\Gamma \vdash d_i = e_i~\keywadj{OK}}\\[5ex]
\end{array}
\]

$\fbox{$\Gamma \vdash d = e~\keyw{OK}$}$
\[
\begin{array}{c}
\infer[\textsc{($\varepsilon$-ValidImpl$_d$)}]
	{\Gamma \vdash d = e~\keywadj{OK}}
	{d = \keyw{def} m(y : \tau_2) : \tau_3 & \Gamma, y : \tau_2 \vdash e : \tau_3}
	\\[5ex]
\end{array}
\]

\fbox{$\Gamma \vdash e : \tau~\keyw{with} \varepsilon$}

\[
\begin{array}{c}
\infer[\textsc{(C-NewObj)}]
	{\Gamma \vdash~\keywadj{new}_d~x \Rightarrow \overline{d = e} : \{  \bar d~\keyw{captures} \varepsilon_c \} ~\keyw{with} \varnothing }
	{\varepsilon_c = \keywadj{effects}(\Gamma') & \Gamma' \subseteq \Gamma & \Gamma', x : \{ \bar d~\keyw {captures} \varepsilon_c \} \vdash \overline{d = e}~\keyw{OK}} \\[5ex]
\end{array}
\]

\[
\begin{array}{c}
\infer[\textsc{(C-MethCall)}]
	{\Gamma \vdash e_1.m_i(e_2) : \tau_3~ \keyw{with} \varepsilon_1 \cup \varepsilon_2 \cup \keywadj{effects}(\tau_2) \cup \varepsilon_c}
	{\Gamma \vdash e_1 : \{ \bar d~\keyw{captures} \varepsilon_c \}~\keyw{with} \varepsilon_1 & \Gamma \vdash e_2:\tau_2~\keyw{with} \varepsilon_2&d_i =~\keyw{def} m_i(y:\tau_2):\tau_3 } \\[5ex]
\end{array}
\]

\[
\begin{array}{c}
\infer[\textsc{(C-ConservativeInference)}]
	{\Gamma \vdash e : \tau ~\keyw{with} \keywadj{effects}(\Gamma') }
	{\Gamma' \subseteq \Gamma & \Gamma' \vdash e : \tau } \\[5ex]
\end{array}
\]

\noindent \textbf{Notes:}

\begin{itemize}
	\item This system also includes the rules from the fully-annotated system.
	\item The \textsc{T} rules do standard typing of objects, without any effect analysis. Their sole purpose is so $\varepsilon$-ValidImpl$_d$ can be applied.
	\item In applying \textsc{C-NewObj}, $\Gamma$ is the current context. The variable $\Gamma'$ is some sensible choice of sub-context. A good choice of sub-context is $\Gamma$ restricted to the free variables in the method-body being type-checked. This tightens the upper-bound to exclude resources never used in the program.
	\item To perform effect analysis on an unannotated object $\{ \bar d \}$, figure out what it captures using $\textsc{C-NewObj}$, yielding the type $\{ \bar d~\keyw{captures} \varepsilon \}$. Then, when a method is called on that object, \textsc{C-MethCall} can be applied.
	\item \textsc{C-ConservativeInference} is used to effect-type unannotated portions of code. It essentially says that in the absence of any effect information, you can assume its effects are $\keywadj{effects(\Gamma')}$, a safe upper-bound for some relevant choice of $\Gamma'$.
\end{itemize}

\subsection{Effects Function}

\noindent
The $\keywadj{effects}$ function returns the set of effects in a particular context. \\

\noindent
A method $m$ can return a resource $r$ (directly or via some enclosing object). Returning a resource isn't an effect but it means any unannotated program using $m$ also captures $r$. To account for this, when the $\keywadj{effects}$ function is operating on a type $\tau$ it must analyse the return type of the method declarations in $\tau$. Since the resource might be itself enclosed by an object, we do a recursive analysis.

\begin{itemize}
	\item $\keywadj{effects}(\varnothing) = \varnothing$
	\item $\keywadj{effects}(\Gamma, x : \tau) = \keywadj{effects}(\Gamma) \cup \keywadj{effects}(\tau)$
	\item $\keywadj{effects}(\{\bar r\}) = \{ (r, \pi) \mid r \in \bar r, \pi \in \Pi \}$
	\item $\keywadj{effects}(\{\bar \sigma\}) = \bigcup_{\sigma \in \bar \sigma}~\keywadj{effects}(\sigma)$
	\item $\keywadj{effects}(\{\bar d\}) = \bigcup_{d \in \bar d}~\keywadj{effects}(d)$
	\item $\keywadj{effects}(d~\keyw{with} \varepsilon) = \varepsilon \cup \keywadj{effects}(d)$
	\item $\keywadj{effects}(\keywadj{def~m}(x : \tau_1) : \tau_2) = \keywadj{effects}(\tau_2)$
	\item $\keywadj{effects}(\{\bar d ~\keyw{captures} \varepsilon\}) = \varepsilon$
\end{itemize}

\noindent \textbf{Notes:}
\begin{itemize}
	\item In the last case, it is not necessary to recurse to sub-declarations; the type $\{ \bar d~\keyw{captures} \varepsilon \}$ can only result from \textsc{C-NewObj}. In that case, $\varepsilon$ is already an estimate of what the object captures in some context $\Gamma_0$; furthermore, the current $\Gamma$ may be bigger than $\Gamma_0$, in which case re-annotating the object would introduce more effects and lose precision.
\end{itemize}
\newpage


\section{Dynamic Semantics}

\subsection{Terminology}

\begin{itemize}
	\item The runtime effects are what actually happens when the program is executed; the static effect annotations are an estimate of what will happen at runtime (our eventual aim is to show the static annotations are safe).
	\item If $e$ is an expression then $[e_1/x_1]e$ is a new expression, the same as e, with every free occurrence of $x_1$ replaced by $e_1$. $[e_1/x_1, ..., e_n/x_n]e$ is syntactic sugar for repeated one-variable substitution: $[e_1/x_1]...[e_n/x_n]e$.
	\item $\varnothing$ is the empty set. The empty type is denoted $\keywadj{Unit}$. Its single instance is $\keywadj{unit}$.
\end{itemize}

\subsection{Grammar}

\[
\begin{array}{lll}
\begin{array}{lllr}
	e & ::= & x & expressions \\
  		& | & e.m(e)\\
  		& | & e.\pi(e)\\
		& | & v \\
		&&\\
		
	v & ::= & r & values \\
  		& | & \keywadj{new}_{\sigma}~x \Rightarrow \overline{\sigma = e} \\
  		& | & \keywadj{new}_d~x \Rightarrow \overline{d = e} \\
		&&\\

\end{array}
& ~~~~~~
&

\begin{array}{lllr}

	d & ::= & \keyw{def} m(x:\tau):\tau & unlabeled~decls.\\		&&\\
		
	\sigma & ::= & d~\keyw{with}\varepsilon  & labeled~ decls.\\
		&&\\
	\tau & ::= & \{ \bar \sigma \}  & types \\
		& | & \{ \bar r \} \\
		&&\\

	\Gamma & :: = & \varnothing & contexts\\
		& | & \Gamma,~x : \tau\\
		&&\\

&&\\

\end{array}
\end{array}
\]

\subsection{Rules}


\fbox{$e \longrightarrow e~|~\varepsilon$}


\[
\begin{array}{lll}

	\infer[\textsc{(E-MethCall1)}]
		{e_1.m(e_2) \longrightarrow e_1'.m(e_2)~|~\varepsilon}
		{e_1 \longrightarrow e_1'~|~\varepsilon}\\[5ex]

	\infer[\textsc{(E-MethCall2$_{\sigma}$)}]
		{v_1.m(e_2) \longrightarrow v_1.m(e_2')~|~\varepsilon}
		{v_1 = \keywadj{new}_{\sigma}~x \Rightarrow \overline{\sigma = e} & e_2 \longrightarrow e_2'~|~\varepsilon}
				
		~~~~~~
	\infer[\textsc{(E-MethCall2$_d$)}]
		{v_1.m(e_2) \longrightarrow v_1.m(e_2')~|~\varepsilon}
		{v_1 = \keywadj{new}_{d}~x \Rightarrow \overline{d = e} & e_2 \longrightarrow e_2'~|~\varepsilon}
		\\[5ex]
		
	\infer[\textsc{(E-MethCall3$_{\sigma}$)}]
		{v_1.m(v_2)
			\longrightarrow
		 [v_1/x, v_2/y]e~|~\varnothing}
  		{v_1 = \keywadj{new}_{\sigma}~x \Rightarrow \overline{\sigma = e} & \keywadj{def~m}(y: \tau_1) : \tau_2~\keyw{with} \varepsilon = e \in \overline {\sigma = e}} \\[5ex]

	\infer[\textsc{(E-MethCall3$_{d}$)}]
		{v_1.m(v_2)
			\longrightarrow
		 [v_1/x, v_2/y]e~|~\varnothing}
  		{v_1 = \keywadj{new}_{d}~x \Rightarrow \overline{d = e} & \keywadj{def~m}(y: \tau_1) : \tau_2 = e \in \overline {d = e}} \\[5ex]
			
	\infer[\textsc{(E-OperCall1)}]
		{e_1.\pi
			\longrightarrow
		 e_1'.\pi~|~\varepsilon}
		{e_1 \longrightarrow e_1'~|~\varepsilon}
~~~~~~

			\infer[\textsc{(E-OperCall3)}]
		{r.\pi
			\longrightarrow
		 \keywadj{unit}~|~\{r.\pi\}}
		{} \\[5ex]
			
\end{array}
\]



\fbox{$e \longrightarrow_* e~|~\varepsilon$}

\[
\begin{array}{lll}

	\infer[\textsc{(E-MultiStep1)}]
	{e \longrightarrow_* e~|~\varnothing}
	{} ~~~~~~~~
	
	\infer[\textsc{(E-MultiStep2)}]
	{e \longrightarrow_* e'~|~\varepsilon}
	{e \longrightarrow e'~|~\varepsilon} \\[5ex]

	\infer[\textsc{(E-MultiStep3)}]
	{e \longrightarrow_* e''~|~\varepsilon_1 \cup \varepsilon_2}
	{e \longrightarrow_* e'~|~\varepsilon_1 & e' \longrightarrow_* e''~|~\varepsilon_2}

\end{array}
\]

\noindent \textbf{Notes:}
\begin{itemize}
	\item A multi-step involves \textit{zero} or more applications of a small-step.
	\item Multi-step rules accumulate the run-time effects produced by the individual small-steps.
	\item The only rule which produces effects is \textsc{E-OperCall3} (the rule for evaluating operations on resources).
	\item Method calls are evaluated by performing substitution on the body of the method, and evaluating that.
\end{itemize}

\newpage

\newpage

\section{Theorems}






\ 
\thm{Lemma 5.1. (Canonical Forms)}
{Suppose $e$ is a value. The following are true:
	\begin{itemize}
	\item If $\Gamma \vdash e : \type{ \{ \bar r \} }{\varepsilon} $, then $e = r$ for some resource $r$.
	\item If $\Gamma \vdash e : \type{ \{ \bar \sigma \}}{\varepsilon}$, then $e = \newsig$.
	\item If $\Gamma \vdash e : \{ \bar d~\keyw{captures} \varepsilon_c \}~ \keyw{with} \varepsilon$, then $e = \keywadj{new}_{d}~x \Rightarrow \overline{ d = e }$.
	\end{itemize}
Furthermore, $\varepsilon = \varnothing$ in each case. \\\\
}
{ These typing judgements each appear exactly once in the conclusion of different rules. The result follows by inversion of \textsc{$\varepsilon$-Resource}, \textsc{$\varepsilon$-NewObj}, and \textsc{C-NewObj} respectively. \\
}















\thm{Theorem 5.10. (Preservation Under Labeling)}
{If $\Gamma \vdash e : \tau$ then $\Gamma \vdash \hat e : \tau$. \\\\}
{By induction on the complexity of the expression $e$ and considering the possible forms of $e$. If $e = r$ or $e = x$ then it holds trivially. Consider the other cases.

\proofcase{$e_1.m(e_2)$}{ 
By induction we know $\keywadj{type}(e_1) = \keywadj{type}(\hat e_1)$ and $\keywadj{type}(e_2) = \keywadj{type}(\hat e_2)$. The result holds by applying \textsc{T-MethCall$_\sigma$}.
}

\proofcase{$e_1.\pi(e_2)$}{
Same as above but use \textsc{T-MethCall$_r$}.
}

\proofcase{$\keyw{new_\sigma} x \Rightarrow \overline{\sigma = e}$}{ From induction we know that since $\Gamma \vdash \overline{\sigma = e}~\keywadj{OK}$ then $\Gamma \vdash \overline{\sigma = \hat e}~ \keywadj{OK}$. The result holds by applying \textsc{T-NewObj$_\sigma$}.
}

\proofcase{$\keyw{new_d} x \Rightarrow \overline{d = e}$}{ Same as above, but $\Gamma \vdash \overline{d = \hat e}$ lets us apply \textsc{T-NewObj$_d$}.
}
}

\thm{Theorem 5.11. (Small-Step Soundness Of Labeled Programs)}
{If $e_A \longrightarrow e_B~|~\varepsilon_B$ and $e_A : \tau_A$ then $\hat e_A : \hat \tau~\keyw{with ~\hat \varepsilon}$ where $\varepsilon_B \subseteq \hat \varepsilon$.\\\\}
{By induction on the evaluation rule used. In all cases, the type of $\hat e_A$ is the same as $e_A$ by the previous theorem so we omit its derivation.

\proofcase{\textsc{E-MethCall1}}
{Then the following are known.
\begin{itemize}
	\item $e_A = e_1.m(e_2)$
	\item $e_1 \rightarrow e_1'~|~\varepsilon$
	\item $e_B = e_1'.m(e_2)~|~\varepsilon$
\end{itemize}
By inductive assumption $\hat e_1 : \tau~ \keyw{with} \hat \varepsilon_1$ where $\varepsilon \subseteq \hat \varepsilon_1$. By applying the rule \textsc{$\varepsilon$-MethCall} we type $\hat e_A$ to something with the effect-set $\hat \varepsilon = \hat \varepsilon_1 \cup \hat \varepsilon_2 \cup \hat \varepsilon_3 \supseteq \hat \varepsilon_1 \supseteq \varepsilon$.
}

\proofcase{ \textsc{E-OperCall1}}
{
	Same as previous case, but use the rule \textsc{$\varepsilon$-OperCall}.
}

\proofcase{\textsc{E-MethCall2$_\sigma$}, \textsc{E-MethCall2$_d$}}
{Then the following are known.
\begin{itemize}
	\item $e_A = v_1.m(e_2)$
	\item $e_2 \longrightarrow e_2'~|~\varepsilon$
\end{itemize}
By inductive assumption $\hat e_2 : \tau~ \keyw{with} \hat \varepsilon_2$ where $\varepsilon \subseteq \hat \varepsilon_2$. By applying the rule \textsc{$\varepsilon$-MethCall} we type $\hat e_1.m(\hat e_2)$ to something with the effect-set $\hat \varepsilon = \hat \varepsilon_1 \cup \hat \varepsilon_2 \cup \hat \varepsilon_3 \supseteq \hat \varepsilon_2 \supseteq \varepsilon$.
}

\proofcase{\textsc{E-OperCall2}} {
	Same as previous case, but use the rule
}

\proofcase{\textsc{E-MethCall3$_\sigma$}, \textsc{E-MethCall3$_d$}}
{In this case $\varepsilon = \varnothing$ so the statement holds trivially.
}

\proofcase{\textsc{E-OperCall3}}
{Then $e_A = r.\pi(v)$. $\hat r = r$ so $\hat e_A = r.\pi(\hat v)$. By application of \textsc{$\varepsilon$-OperCall} we have $\hat e_A : \keyw{Unit~with} \{ r.\pi \} \cup \hat \varepsilon_1 \cup \hat \varepsilon_2$, where $\hat \varepsilon_2$ is the set of effects for $\hat v$ and $\hat \varepsilon_1 = \varnothing$ for $\hat r = r$. By inspection of the rule for $\textsc{E-OperCall3}$, $\varepsilon = r.\pi \subseteq \{ r.\pi \} \cup \hat \varepsilon_2 = \hat \varepsilon_A$.
}
}


\thm{Theorem 5.12. (Soundness Theoerm)}
{If $\Gamma \vdash e_A : \tau_A~\keyw{with} \varepsilon_A$ and $e_A \longrightarrow e_B~|~\varepsilon$ then $\Gamma \vdash e_B : \tau_B~\keyw{with} \varepsilon_B$ and $\varepsilon \subseteq \varepsilon_A$.\\\\}
{Induct on the typing judgement for $\Gamma \vdash e_A : \tau_A~\keyw{with} \varepsilon_A$ and then on the evaluation rule used for $e_A \longrightarrow e_B~|~\varepsilon$. First note that the theorem statement has already been proven for typing judgements from the calculus for fully-labeled programs. Therefore we'll only consider the extra typing rules in the calculus that allows unlabeled objects.

\proofcase{\textsc{C-NewObj}}
{Then $e_A$ is of the form $\keywadj{new}_d$ so it is a value. Then no reduction can be applied; the theorem statement holds vacuously.
}

\proofcase{\textsc{C-MethCall}}
{
Then $e_A = e_1.m(e_2)$ where $e_1 = \keyw{new_d} x \Rightarrow \overline{d = e} : \{ \bar d~\keyw{captures} \varepsilon_c \}~ \keyw{with} \varnothing$, where $\varepsilon_c = \keywadj{effects}(\Gamma_A')$, for $\Gamma_A' \subseteq \Gamma_A$. Consider the extension $\hat e_A = \keywadj{label}(e_A)$. Since $\keywadj{label}$ only changes type-information, the runtime effects of $e_A$ are invariant under the $\keywadj{label}$ function. Furthermore, $e_A$ is a value if and only if $\hat e_A$ is a value.\\

\noindent
Now, $\hat e_A$ is not a value because $e_A$ is not a value. Therefore we can reduce $\hat e_A$. Then $\hat e_A \longrightarrow \hat e_B~|~\varepsilon$, the same $\varepsilon$ we get when reducing $e_A$. \\

\noindent
By small-step soundness of labeled programs, $\varepsilon \subseteq \hat \varepsilon_A$. By refinement, $\hat \varepsilon_A \subseteq \varepsilon_A$. So $\varepsilon \subseteq \varepsilon_A$.
}

\proofcase{\textsc{C-ConservativeTyping}}
{
Then $\varepsilon_A = \keywadj{effects}(\Gamma')$ for some $\Gamma'$. Runtime effects must be captured by the environment; therefore $\varepsilon \subseteq \varepsilon_A$.
}
}

%
%C-MethCall Case For Progress
%
%	\proofcase{\textsc{C-MethCall}} {
%Then $e_A = e_1.m_i(e_2)$ and the following are known:

%\begin{itemize}
%	\item ~$e_A : \tau_3~\keyw{with} \varepsilon_1 \cup \varepsilon_2 \cup \keywadj{effects}(\tau_2) %\cup \varepsilon_c$
%	\item ~$e_1 : \{ \bar d~\keyw{captures} \varepsilon_c \}~\keywadj{with}~\varepsilon_1$
%	\item ~$e_2 : \tau_2~\keyw{with} \varepsilon_2$
%	\item ~$d_i = \keyw{def} m_i(y : \tau_2) : \tau_3$
%\end{itemize}
%	We look at the cases for when $e_1$ and $e_2$ are values.
	
%		\subcase { $e_1$ is not a value. Also, $e_1 : \{ \bar d~\keyw{captures} \varepsilon_c \}~\keywadj{with}~\varepsilon_1$ is a subderivation. By inductive hypothesis, $e_1 \longrightarrow e_1'~|~\varepsilon$. Then \textsc{E-MethCall1} gives the reduction $e_1.m_i(e_2) \longrightarrow e_1'.m_i(e_2)~|~\varepsilon$.}
		
%		\subcase { $e_2$ is not a value. Without loss of generality, $e_1 = v_1$ is a value. Also, $e_2 :  \tau_2~\keyw{with} \varepsilon_2$ is a subderivation. By inductive hypothesis, $e_2 \longrightarrow e_2'~|~\varepsilon$. Then \textsc{E-MethCall2$_d$} gives the reduction $v_1.m_i(e_2) \longrightarrow v_1.m_i(e_2')~|~\varepsilon$. }
		
%		\subcase{ $e_1$ and $e_2$ are values. By Canonical Forms, $e_1 = \keywadj{new}_d~x \Rightarrow \overline{ d = e }$. Also, $\keyw{def} m_i(y : \tau_2) : \tau_3 = e_i \in \overline {d = e}$. By assumption of the typing rule used, the receiver and argument are well-typed for method $m_i$. Then \textsc{E-MethCall3$_d$} gives the reduction $v_1.m_i(v_2) \longrightarrow [v_1/x, v_2/y]e_i~|~\varnothing$ }
%	}


\end{document}
