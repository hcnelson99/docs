\vspace{-0.3cm}
\section{Capability Calculus ($\epscalc$)}
\vspace{-0.3cm}
\label{s:cc}

Allowing a mix of annotated [with effects] and unannotated code helps
reduce the cognitive overhead on developers, allowing them to rapidly
prototype in the unannotated sublanguage and incrementally add
annotations as they are needed. However, reasoning about unannotated
code is difficult in general. Figure \ref{fig:unannotated_reasoning}
demonstrates why: $\kwa{someMethod}$ takes a function $f$ as input and
executes it, but the effects of $f$ depend on its
implementation. Without more information, there is no way to know what
effects might be incurred by $\kwa{someMethod}$.

\begin{figure}
\vspace{-0.8cm}
\begin{lstlisting}
def someMethod(f: Unit $\rightarrow$ Unit):
   f()
\end{lstlisting}
\vspace{-0.5cm}
\caption{What effects can $\kwa{someMethod}$ incur?}
\vspace{-0.5cm}
\label{fig:unannotated_reasoning}
\end{figure}

The key insight of our paper is that capability-safe design can help us:
because the only authority code can exercise is that which is
explicitly given to it, the only capabilities that the unannotated
code can use must be passed into it. If these capabilities are being
passed in from an annotated environment, we know what effects they
capture. These effects are therefore a conservative upper bound on
what can happen in the unannotated code. To demonstrate, consider a
developer who wants to decide whether to use the $\kwa{logger}$
functor in Figure \ref{fig:cc_motivation}. It must be instantiated
with two capabilities, $\kwa{File}$ and $\kwa{Socket}$, and provides
an unannotated function $\kwa{log}$.

\begin{figure}
\begin{lstlisting}
module def logger(f:{File},s:{Socket}):Logger

def log(x: Unit): Unit
   ...
\end{lstlisting}
\vspace{-0.5cm}
\caption{In a capability-safe setting, $\kwa{logger}$ can only exercise authority over the $\kwa{File}$ and $\kwa{Socket}$ capabilities given to it.}
\vspace{-0.5cm}
\label{fig:cc_motivation}
\end{figure}

What effects will be incurred if $\kwa{Logger.log}$ is invoked? One
approach is to manually\footnote{or automatically---but if the
  automation produces an unexpected result we must fall back to manual
  reasoning to understand why.} examine its source code, but this is
tedious and error-prone. In many real-world situations, the source
code may be obfuscated or unavailable. A capability-based argument can
do better: the only authority which $\kwa{Logger}$ can exercise is
that which it has been explicitly given. Here, the $\kwa{Logger}$
requires a $\kwa{File}$ and a $\kwa{Socket}$, so
$\kwa{ \{ \kwa{File.*, Socket.*} \} }$ is an upper bound on the
effects of $\kwa{Logger}$. Knowing $\kwa{Logger}$ could be performing
arbitrary reads and writes to $\kwa{File}$, or arbitrary communication
with the $\kwa{Socket}$, the developer decides this implementation
cannot be trusted and does not use it.

The reasoning we employed only required us to examine the interface of
the unannotated code for the capabilities passed into it. To model
this situation in $\epscalc$, we add a new $\kwa{import}$ expression
that selects what authority $\varepsilon$ the unannotated code may
exercise. In the above example, the expected least authority of
$\kwa{Logger}$ is $\{ \kwa{File.append} \}$, so that is what the
corresponding $\kwa{import}$ would select. The type system can then
check if the capabilities being passed into the unannotated code
exceed its selected authority. If it accepts, then $\varepsilon$
safely approximates the effects of the unannotated code. This is the
key result: when unannotated code is nested inside annotated code,
capability-safety enables us to make a safe inference about its
effects by examining what capabilities are being passed in by the
annotated code.

\vspace{-0.5cm}
\subsection{Grammar ($\epscalc$)}
\vspace{-0.2cm}

The grammar of $\epscalc$ is split into rules for annotated code and
analogous rules for unannotated code. To distinguish the two, we put a
hat above annotated types, expressions, and contexts: $\hat e$,
$\hat \tau$, and $\hat \Gamma$ are annotated, while $e$, $\tau$, and
$\Gamma$ are unannotated. The rules for unannotated programs and their
types are given in Figure
\ref{fig:epscalc_unannotated_grammar}. Unannotated types $\tau$ are
built using $\rightarrow$ and sets of resources $\{ \bar r \}$. An
unannotated context $\Gamma$ maps variables to unannotated
types. Rules for annotated programs and their types are in
Figure \ref{fig:epscalc_annotated_grammar}.

\begin{figure}
\[
\begin{array}{lll}

\begin{array}{lllr}
e & ::= & ~ & exprs: \\
	& | & x & variable \\
	& | & v & value \\
	& | & e ~ e & application \\
	& | & e.\pi & operation \\
	&&\\

v & ::= & ~ & values: \\
	& | & r & resource~literal \\
	& | & \lambda x: \tau.e & abstraction \\
	&&\\
\end{array}

\hspace{5ex}

\begin{array}{lllr}

\tau & ::= & ~ & types: \\
		& | & \{ \bar r \} \\
		& | & \tau \rightarrow \tau \\ 
		&&\\

\Gamma & ::= & ~ & type~ctx: \\
				& | & \varnothing \\
				& | & \Gamma, x: \tau \\
				&&\\
				
\varepsilon & ::= & ~ & effects: \\
		& | & \{ \overline{r.\pi} \} & effect~set \\
		&&\\

				
\end{array}

\end{array}
\]
\vspace{-0.5cm}
\caption{Unannotated programs and types in $\epscalc$.}
\vspace{-0.5cm}
\label{fig:epscalc_unannotated_grammar}
\end{figure}

\begin{figure}
\[
\begin{array}{lll}
7
\begin{array}{lllr}

\hat e & ::= & ~ & labeled~exprs: \\
	& | & x \\
	& | & \hat v \\
	& | & \hat e ~ \hat e \\
	& | & \hat e.\pi \\
	& | & \kwa{import}(\varepsilon_s)~x = \hat e~\kwa{in}~e & import \\
	&&\\

\hat v & ::= & ~ & labeled~values: \\
	& | & r \\
	& | & \lambda x: \hat \tau.\hat e \\
	&&\\

\end{array}

& ~~~~~~~~&

\begin{array}{lllr}

\hat \tau & ::= & ~ & annotated ~types: \\
		& | & \{ \bar r \} \\
		& | & \hat \tau \rightarrow_{\varepsilon} \hat \tau \\
		&&\\

\hat \Gamma & ::= & ~ & annotated~type~ctx:\\
				& | & \varnothing \\
				& | & \hat \Gamma, x: \hat \tau \\
				&&\\

\varepsilon & ::= & ~ & effects: \\
		& | & \{ \overline{r.\pi} \} & effect~set \\
		&&\\

\end{array}

\end{array}
\]
\vspace{-0.5cm}
\caption{Annotated programs and types in $\epscalc$.}
\vspace{-0.5cm}
\label{fig:epscalc_annotated_grammar}
\end{figure}

The new form is $\import{\varepsilon_s}{x}{\hat e}{e}$, modelling the
points at which capabilities are passed from annotated code into
unannotated code. $e$ is the unannotated code. $\hat e$ is the
capability being given to it; we call $\hat e$ an import. For
simplicity, we assume only one capability is being passed into
$e$. $\hat e$ is associated with the name $x$ inside
$e$. $\varepsilon_s$ is the maximum authority that $e$ is allowed to
exercise (its ``selected authority''). As an example, suppose an
unannotated $\kwa{Logger}$, which requires $\kwa{File}$, is expected
to only $\kwa{append}$ to a file, but has an implementation that
writes. This would be modelled by the expression
$\import{\kwa{File.append}}{x}{\kwa{File}}{\lambda y:
  \Unit.~\kwa{x.write}}$.

Observe that $\kwa{import}$ is the only way to mix annotated and
unannotated code, because it is the only situation in which we can say
something interesting about the unannotated code. For the rest of our
discussion on $\epscalc$, we will only be interested in unannotated
code when it is encapsulated by an $\kwa{import}$ expression.

One of the requirements of capability safety is there be no ambient
authority.  This requirement is met by forbidding resource literals
$r$ from being used directly inside an \kwat{import} statement (they
can still be passed in as a capability via the \kwat{import}'s binding
variable $x$).  We could enforce this syntactically, by removing $r$
from the language of unannotated expressions, but we choose to do it
instead using the typing rule for \kwat{import}, given below.

\vspace{-0.2cm}
\subsection{Semantics ($\epscalc$)}
\vspace{-0.2cm}

Reductions are defined on annotated expressions and are natural
reduction rules from extended lambda calculus. We omit them for
brevity. If unannotated code $e$ is wrapped inside annotated code
$\import{\varepsilon_s}{x}{\hat e}{e}$, we transform it into annotated
code by recursively annotating its parts with $\varepsilon_s$. In
practice, it is meaningful to execute purely unannotated code --- but
our only interest is when that code is wrapped inside an
$\kwa{import}$ expression, so we do not bother to give rules for
it. There are two additional rules for reducing $\kwa{import}$
expressions, given in Figure \ref{fig:epscalc_reductions}:
\textsc{E-Import1} reduces the capability being imported, while
\textsc{E-Import2} first annotates $e$ with its selected authority
$\varepsilon$ --- this is $\annot{e}{\varepsilon}$ --- and then
substitutes the import $\hat v$ for its name $x$ in $e$ --- this is
$[\hat v/x]\annot{e}{\varepsilon}$.

\begin{figure}

\fbox{$\hat e \longrightarrow \hat e~|~\varepsilon$}

\[
\begin{array}{c}
\infer[\textsc{(E-Import1)}]
	{\kwa{import}(\varepsilon_s)~x = \hat e~\kw{in} e \longrightarrow \kwa{import}(\varepsilon_s)~x = \hat e'~\kw{in} e~|~\varepsilon'}
	{\hat e \longrightarrow \hat e'~|~\varepsilon'}\\[4ex]

\infer[\textsc{(E-Import2)}]
	{\kwa{import}(\varepsilon_s)~x = \hat v~\kw{in} e \longrightarrow [\hat v/x]\kwa{annot}(e, \varepsilon_s)~|~\varnothing}
	{}

\end{array}
\]
\caption{New single-step reductions in $\epscalc$.}
\label{fig:epscalc_reductions}
\end{figure}

$\annot{e}{\varepsilon}$ produces the expression obtained by
recursively annotating the parts of $e$ with the set of effects
$\varepsilon$. A definition is given in Figure
\ref{fig:annot_defn}. There are versions of $\kwa{annot}$ defined for
expressions and types. Later we shall need to annotate contexts, so
the definition is given here. It is worth mentioning that
$\kwa{annot}$ operates on a purely syntatic level --- nothing prevents
us from annotating a program with something unsafe, so any use of
$\kwa{annot}$ must be justified.

\begin{figure}
\vspace{-0.2cm}

$\kwa{annot} :: e \times \varepsilon \rightarrow \hat e$

\begin{itemize}
	\setlength\itemsep{-0.2em}
	\item[] $\annot{r}{\_} = r$
	\item[] $\annot{\lambda x: \tau_1 . e}{\varepsilon} = \lambda x: \annot{\tau_1}{\varepsilon} . \annot{e}{\varepsilon}$
	\item[] $\annot{e_1~e_2}{\varepsilon} = \kwa{annot}(e_1, \varepsilon)~\kwa{annot}(e_2, \varepsilon)$
	\item[] $\annot{e_1.\pi}{\varepsilon} = \annot{e_1}{\varepsilon}.\pi$
\end{itemize}
	
$\kwa{annot} :: \tau \times \varepsilon \rightarrow \hat \tau$

\begin{itemize}
	\setlength\itemsep{-0.2em}
	\item[] $\annot{\{ \bar r \}}{\_} = \{ \bar r \}$
	\item[] $\annot{\tau_1 \rightarrow \tau_2}{\varepsilon} = \annot{\tau_1}{\varepsilon} \rightarrow_{\varepsilon} \annot{\tau_2}{\varepsilon}$.	
\end{itemize}

$\kwa{annot} :: \Gamma \times \varepsilon \rightarrow \hat \Gamma$

\begin{itemize}
	\setlength\itemsep{-0.2em}
	\item[] $\annot{\varnothing}{\_} = \varnothing$
	\item[] $\annot{\Gamma, x: \tau}{\varepsilon} = \annot{\Gamma}{\varepsilon}, x: \annot{\tau}{\varepsilon}$
\end{itemize}
\vspace{-0.5cm}
\caption{Definition of $\kwa{annot}$.}
\vspace{-0.5cm}
\label{fig:annot_defn}
\end{figure}

\subsection{Static Rules ($\epscalc$)}

A term can be annotated or unannotated, so we need to be able to
recognise when either is well-typed. We do not reason about the
effects of unannotated code directly, so judgements about them have
the form $\Gamma \vdash e: \tau$. Subtyping judgements have the form
$\tau <: \tau$. A summary of the rules for unannotated judgements is
given in Figure \ref{fig:unannotated_static_rules}.

\begin{figure}
\vspace{-5pt}

\fbox{$\Gamma \vdash e: \tau$}

\[
\begin{array}{c}


\infer[\textsc{(T-Var)}]
	{\Gamma, x: \tau \vdash x: \tau}
	{}
\hspace{5ex}
\infer[\textsc{(T-Resource)}]
	{\Gamma, r: \{ r \} \vdash r : \{ r \}}
	{}

\hspace{5ex}
\infer[\textsc{(T-Abs)}]
	{\Gamma \vdash \lambda x: \tau_1.e : \tau_1 \rightarrow \tau_2}
	{\Gamma, x: \tau_1 \vdash e: \tau_2}\\[4ex]
	
\infer[\textsc{(T-App)}]
	{\Gamma \vdash e_1~e_2: \tau_3}
	{\Gamma \vdash e_1: \tau_2 \rightarrow \tau_3 & \Gamma \vdash e_2: \tau_2}
\hspace{5ex}
\infer[\textsc{(T-OperCall)}]
	{\Gamma \vdash e.\pi: \kwa{Unit}}
	{\Gamma \vdash e: \{ \bar r \}}

\end{array}
\]

\fbox{$\tau <: \tau$}

\[
\begin{array}{c}

\infer[\textsc{(S-Arrow)}]
	{ \tau_1 \rightarrow \tau_2 <: \tau_1' \rightarrow \tau_2' }
	{ \tau_1' <: \tau_1 & \tau_2 <: \tau_2' }
\hspace{5ex}
\infer[\textsc{(S-Resources)}]
	{ \{ \bar r_1 \} <: \{ \bar r_2 \} }
	{ \{ \bar r_1 \} \subseteq \{ \bar r_2 \} }

\end{array}
\]

\vspace{-0.3cm}
\caption{(Sub)typing judgements for the unannotated sublanguage of $\epscalc$}
\vspace{-0.3cm}
\label{fig:unannotated_static_rules}
\end{figure}

The annotated static rules are effectively the same only involving the
annotated expressions. The interesting rule is
\textsc{$\varepsilon$-Import}, given in Figure \ref{fig:import_rule},
which gives the type and approximate effects of an $\kwa{import}$
expression. This is the only way to reason about what effects might be
incurred by some unannotated code.

The rule is complicated, so to explain it we shall start with a
simplified version and spend the rest of this section building up to
the final version of \textsc{$\varepsilon$-Import}.

To begin, typing $\import{\varepsilon_s}{x}{\hat e}{e}$ in a context
$\hat \Gamma$ requires us to know that the import $\hat e$ is
well-typed, so we add the premise
$\hat \Gamma \vdash \hat e: \hat \tau~\kw{with} \varepsilon_1$. Since
$x = \hat e$ is an import, it can be used throughout $e$. We do not
want $e$ to exercise authority it hasn't explicitly selected, so
whatever capabilities it uses must be selected by the $\kwa{import}$
expression; therefore, we require that $e$ can be typechecked using
only the binding $x: \hat \tau$. There is a problem though: $e$ is
unannotated and $\hat \tau$ is annotated, and there is no rule for
typechecking unannotated code in an annotated context. To get around
this, we define a function $\kwa{erase}$ in Figure
\ref{fig:erase_defn} which removes the annotations from a type. We
then add $x: \erase{\hat \tau} \vdash e: \tau$ as a premise.

\begin{figure}
$\kwa{erase} :: \hat \tau \rightarrow \tau$
\begin{itemize}
	\setlength\itemsep{-0.2em}
	\item[] $\erase{\{ \bar r \}} = \{ \bar r \}$
	\item[] $\erase{\hat \tau_1 \rightarrow_{\varepsilon} \hat \tau_2} = \erase{\hat \tau_1} \rightarrow \erase{\hat \tau_2}$
\end{itemize}

\vspace{-0.5cm}
\caption{Definition of $\kwa{erase}$.}
\vspace{-0.5cm}
\label{fig:erase_defn}
\end{figure}

Note that, since the environment $\Gamma$ for $e$ has only one binding
(for $x$), it cannot contain any bindings of resource literals---and
the rule \textsc{T-Resource} requires a binding in the environment in
order to type a resource literal in an expression. Typing $e$ in the
restricted environment given by $\kwa{import}$ thus prohibits ambient
authority.

The first version of \textsc{$\varepsilon$-Import} is given in Figure
\ref{fig:import_rule_1}. Since
$\import{\varepsilon_s}{x}{\hat v}{e} \longrightarrow [\hat
v/x]\annot{e}{\varepsilon_s}$ by \textsc{E-Import2}, the ascribed type
is $\annot{\tau}{\varepsilon}$, which is the type of the unannotated
code, annotated with its selected authority $\varepsilon_s$. The
effects of the $\kwa{import}$ are $\varepsilon_1 \cup \varepsilon_s$
--- the former comes from reducing the imported capability, which
happens before the body of the $\kwa{import}$ is annotated and
executed, and the latter contains all the effects which the
unannotated code might incur.

\begin{figure}[h]
\vspace{-0.5cm}
\[
\begin{array}{c}

\infer[\textsc{($\varepsilon$-Import1-Bad)}]
	{ \hat \Gamma \vdash \import{\varepsilon_s}{x}{\hat e}{e}: \kwa{annot}(\tau, \varepsilon_s)~\kw{with} \varepsilon_s \cup \varepsilon_1 }
	{ \hat \Gamma \vdash \hat e: \hat \tau~\kw{with} \varepsilon_1 & x: \kwa{erase}(\hat \tau) \vdash e: \tau }

\end{array}
\]
\vspace{-0.5cm}
\caption{A first (incorrect) rule for type-and-effect checking $\kwa{import}$ expressions.}
\vspace{-0.5cm}
\label{fig:import_rule_1}
\end{figure}

At the moment there is no relation between the selected authority
$\varepsilon$ and those effects captured by the imported capability
$\hat e$. Consider
$\hat e' = \import{\varnothing}{x}{\File}{\kwa{x.write}}$, which
imports a $\File$ and writes to it, but declares its authority as
$\varnothing$. According to \textsc{$\varepsilon$-Import1},
$\vdash \hat e': \Unit~\kw{with} \varnothing$, but this is clearly
wrong since $\hat e'$ writes to $\kwa{File}$.  An $\kwa{import}$
should only be well-typed if the capability being imported only
captures effects contained in the unannotated code's selected
authority $\varepsilon$. In this case, $\kwa{File}$ captures
$\kwa{ \{ File.* \}}$, which is not contained in the selected
authority $\varnothing$, so it should be rejected for that reason.  To
this end we define a function $\kwa{effects}$, which collects the set
of effects that an annotated type captures. A first (but not yet
correct) definition is given in Figure \ref{fig:fx_defn}. We can then
add the premise $\kwa{effects}(\hat \tau) \subseteq \varepsilon_s$ to
require that any imported capability must not capture authority beyond
that selected in $\varepsilon_s$. The updated rule is given in Figure
\ref{fig:import_rule_2}.

\begin{figure}

$\kwa{effects} :: \hat \tau \rightarrow \varepsilon$
\begin{itemize}
	\setlength\itemsep{-0.2em}
	\item[] $\fx{\{ \bar r \}} = \{ r.\pi \mid r \in \bar r, \pi \in \Pi \}$
	\item[] $\fx{\hat \tau_1 \rightarrow_{\varepsilon} \hat \tau_2} = \fx{\hat \tau_1} \cup \varepsilon \cup \fx{\hat \tau_2}$
\end{itemize}
\vspace{-0.3cm}
\caption{A first (incorrect) definition of $\kwa{effects}$.}
\vspace{-0.3cm}
\label{fig:fx_defn}
\end{figure}

\begin{figure}

\[
\begin{array}{c}

\infer[\textsc{($\varepsilon$-Import2-Bad)}]
	{ \hat \Gamma \vdash \import{\varepsilon_s}{x}{\hat e}{e}: \kwa{annot}(\tau, \varepsilon_s)~\kw{with} \varepsilon \cup \varepsilon_1 }
	{ \hat \Gamma \vdash \hat e: \hat \tau~\kw{with} \varepsilon_1 & x: \kwa{erase}(\hat \tau) \vdash e: \tau & \kwa{effects}(\hat \tau) \subseteq \varepsilon_s}

\end{array}
\]
\vspace{-0.3cm}
\caption{A second (still incorrect) rule for type-and-effect checking $\kwa{import}$ expressions.}
\vspace{-0.5cm}
\label{fig:import_rule_2}
\end{figure}

The counterexample from before is now rejected by
\textsc{$\varepsilon$-Import2}, but there are still issues: the
annotations on one import can be broken by another import. To
illustrate, consider Figure \ref{fig:rule_import2_counterexample}
where two\footnote{Our formalisation only permits a single capability
  to be imported, but this discussion leads to a generalisation needed
  for the rules to be safe when multiple capabilities can be imported.
  In any case, importing multiple capabilities can be handled with an
  encoding of pairs.} capabilities are imported. This program imports
a function $\kwa{go}$ which, when given a
$\Unit \rightarrow_{\varnothing} \Unit$ function with no effects, will
execute it. The other import is $\kwa{File}$. The unannotated code
creates a $\Unit \rightarrow \Unit$ function which writes to
$\kwa{File}$ and passes it to $\kwa{go}$, which subsequently incurs
$\kwa{File.write}$.

\begin{figure}[h]
\vspace{-0.5cm}

\begin{lstlisting}
import({File.*})
   go = $\lambda$x: Unit $\rightarrow_{\varnothing}$ Unit. x unit
   f = File
in
   go ($\lambda$y: Unit. f.write)

\end{lstlisting}

\vspace{-0.5cm}
\caption{Permitting multiple imports will break \textsc{$\varepsilon$-Import2}.}
\vspace{-0.5cm}
\label{fig:rule_import2_counterexample}
\end{figure}

In the world of annotated code it is not possible to pass a
file-writing function to $\kwa{go}$, but because the judgement
$x: \erase{\hat \tau} \vdash e: \tau$ discards the annotations on
$\kwa{go}$, and since the file-writing function has type
$\unit \rightarrow \unit$, the unannotated world accepts it. The
approximation is actually safe at the top-level, because the
$\kwa{import}$ selects $\{ \kwa{File.*} \}$, which contains
$\kwa{File.write}$ --- but it contains code that violates the type
signature of $\kwa{go}$. We want to prevent this.

If $\kwa{go}$ had the type
$\Unit \rightarrow_{\{ \kwa{File.write} \}} \Unit$ the above example
would be safe, but a modified version where a file-reading function is
passed to $\kwa{go}$ would have the same issue. $\kwa{go}$ is only
safe when it expects every effect that the unannotated code might pass
to it: if $\kwa{go}$ had the type
$\Unit \rightarrow_{\{ \kwa{File.*} \}} \Unit$, then the unannotated
code cannot pass it a capability with an effect it isn't already
expecting, so the annotation on $\kwa{go}$ cannot be
violated. Therefore, we require imported capabilities to have
authority to incur the effects in $\varepsilon$. To achieve greater
control in how we say this, the definition of $\kwa{effects}$ is split
into two separate functions called $\kwa{effects}$ and
$\kwa{ho \hyphen effects}$. The latter is for higher-order effects,
i.e. the effects that are not captured within a function, but rather
are possible because of what it is passed as an argument. If values of
$\hat \tau$ possess a capability that can be used to incur the effect
$r.\pi$, then $r.\pi \in \fx{\hat \tau}$. If values of $\hat \tau$ can
incur an effect $r.\pi$, but need to be given the capability (as a
function argument) by someone else in order to do it, then
$r.\pi \in \hofx{\hat \tau}$. Definitions are given in Figure
\ref{fig:fx_defns}.

\begin{figure}
\vspace{-0.5cm}

$\kwa{effects} :: \hat \tau \rightarrow \varepsilon$

\begin{itemize}
	\setlength\itemsep{-0.2em}
	\item[] $\fx{\{ \bar r \}} = \{ r.\pi \mid r \in \bar r, \pi \in \Pi \}$
	\item[] $\fx{\hat \tau_1 \rightarrow_{\varepsilon} \hat \tau_2} = \hofx{\hat \tau_1} \cup \varepsilon \cup \fx{\hat \tau_2}$
\end{itemize}

$\kwa{ho \hyphen effects} :: \hat \tau \rightarrow \varepsilon$

\begin{itemize}
	\setlength\itemsep{-0.2em}
	\item[] $\hofx{\{ \bar r \}} = \varnothing$
	\item[] $\hofx{\hat \tau_1 \rightarrow_{\varepsilon} \hat \tau_2} = \fx{\hat \tau_1} \cup \hofx{\hat \tau_2}$
\end{itemize}

\vspace{-0.5cm}
\caption{Effect functions (corrected).}
\vspace{-0.5cm}
\label{fig:fx_defns}
\end{figure}

Both $\kwa{effects}$ and $\kwa{ho \hyphen effects}$ are mutually recursive,
with base cases for resource types. Any effect can be directly
incurred by a resource on itself, hence
$\fx{\{ \bar r \}} = \{ r.\pi \mid r \in \bar r, \pi \in \Pi \}$. A
resource cannot be used to indirectly invoke some other effect, so
$\hofx{\{ \bar r \}} = \varnothing$. The mutual recursion echoes the
subtyping rule for functions. Recall that functions are contravariant
in their input type and covariant in their output; likewise, both
functions recurse on the input-type using the other function, and
recurse on the output-type using the same function.

In light of these new definitions, we still require
$\fx{\hat \tau} \subseteq \varepsilon_s$ --- unannotated code must
select any effect its capabilities can incur --- but we add a new
premise $\varepsilon_s \subseteq \hofx{\hat \tau}$, stipulating that
imported capabilities must know about every effect they could be given
by the unannotated code (which is at most $\varepsilon$). The
counterexample from Figure \ref{fig:rule_import2_counterexample} is
now rejected, because
$\hofx{(\Unit \rightarrow_{\varnothing} \Unit)
  \rightarrow_{\varnothing} \Unit} = \varnothing$, but
$\{ \kwa{File.*} \} \not\subseteq \varnothing$.  However, this is
\textit{still} not sufficient! Consider
$\varepsilon_s \subseteq \hofx{ \hat \tau_1 \rightarrow_{\varepsilon'}
  \hat \tau_2 }$. We want \textit{every} higher-order capability
involved to be expecting $\varepsilon_s$. Expanding the definition of
$\kwa{ho \hyphen effects}$, this is the same as
$\varepsilon_s \subseteq \fx{\hat \tau_1} \cup \hofx{\hat
  \tau_2}$. Let $r.\pi \in \varepsilon_s$ and suppose
$r.\pi \in \fx{\hat \tau_1}$, but $r.\pi \notin \hofx{\hat
  \tau_2}$. Then
$\varepsilon_s \subseteq \fx{\hat \tau_1} \cup \hofx{\hat \tau_2}$ is
still true, but $\hat \tau_2$ is not expecting $r.\pi$. Unannotated
code could then violate the annotations on $\hat \tau_2$ by passing it
a capability for $r.\pi$, using the same trickery as before. The cause
of the issue is that $\subseteq$ does not distribute over $\cup$. We
want a relation like
$\varepsilon_s \subseteq \fx{\hat \tau_1} \cup \hofx{\hat \tau_2}$,
which also implies $\varepsilon_s \subseteq \fx{\hat \tau_1}$ and
$\varepsilon_s \subseteq \fx{\hat \tau_2}$. Figure
\ref{fig:safe_defns} defines this: $\kwa{safe}$ is a distributive
version of $\varepsilon_s \subseteq \fx{\hat \tau}$ and
$\kwa{ho \hyphen safe}$ is a distributive version of
$\varepsilon_s \subseteq \hofx{\hat \tau}$.

\begin{figure}

\noindent
$\fbox{$\safe{\hat \tau}{\varepsilon}$}$

\[
\begin{array}{c}

\infer[\textsc{(Safe-Resource)}]
	{ \kwa{safe}(\{ \bar r \}, \varepsilon) }
	{} 
\hspace{5ex}
	
\infer[\textsc{(Safe-Arrow)}]
	{\kwa{safe}(\hat \tau_1 \rightarrow_{\varepsilon'} \hat \tau_2, \varepsilon)}
	{\varepsilon \subseteq \varepsilon' & \kwa{ho \hyphen safe}(\hat \tau_1, \varepsilon) & \kwa{safe}(\hat \tau_2, \varepsilon)} \\[3ex]

\end{array}
\]

\noindent
$\fbox{$\hosafe{\hat \tau}{\varepsilon}$}$

\[
\begin{array}{c}

\infer[\textsc{(HOSafe-Resource)}]
	{ \kwa{ho \hyphen safe}( \{ \bar r \}, \varepsilon)} 
	{}
\hspace{5ex}

\infer[\textsc{(HOSafe-Arrow)}]
	{ \kwa{ho \hyphen safe}( \hat \tau_1 \rightarrow_{\varepsilon'} \hat \tau_2, \varepsilon ) }
	{ \kwa{safe}(\hat \tau_1, \varepsilon)  & \kwa{ho \hyphen safe}(\hat \tau_2, \varepsilon) }\\[3ex]

\end{array}
\]

\vspace{-0.5cm}
\caption{Safety judgements in $\epscalc$.}
\vspace{-0.5cm}
\label{fig:safe_defns}
\end{figure}

\begin{figure}
\vspace{-0.5cm}

\[
\begin{array}{c}

\infer[\textsc{($\varepsilon$-Import3-Bad)}]
	{ \hat \Gamma \vdash \kwa{import}(\varepsilon_s)~x = \hat e~\kw{in} e: \kwa{annot}(\tau, \varepsilon_s)~\kw{with} \varepsilon \cup \varepsilon_1 }
{{\def\arraystretch{1.4}
  \begin{array}{c}
\hat \Gamma \vdash \hat e: \hat \tau~\kw{with} \varepsilon_1
~~~~~~
\kwa{effects}(\hat \tau) \subseteq \varepsilon_s \\
\hosafe{\hat \tau}{\varepsilon_s} ~~~~~~ x: \kwa{erase}(\hat \tau) \vdash e: \tau
  \end{array}}} 
 
\end{array}
\]

\vspace{-0.2cm}
\caption{A third (still incorrect) rule for type-and-effect checking $\kwa{import}$ expressions.}
\vspace{-0.8cm}
\label{fig:import_rule3}
\end{figure}

An amended version of \textsc{$\varepsilon$-Import} is given in Figure
\ref{fig:import_rule3}. It contains a new premise
$\hosafe{\hat \tau}{\varepsilon_s}$ which formalises the notion that
every capability which could be given to a value of $\hat \tau$ --- or
any of its constituent pieces --- must be expecting the effects
$\varepsilon_s$ it might be given by the unannotated code.
The premises so far restrict what authority can be selected by
unannotated code, but what about authority passed as a function
argument? Consider the example
$\hat e = \import{\varnothing}{x}{\unit}{\lambda f: { \File
  }.~\kwa{f.write}}$. The unannotated code selects no capabilities and
returns a function which, when given $\kwa{File}$, incurs
$\kwa{File.write}$. This satisfies the premises in
\textsc{$\varepsilon$-Import3}, but its annotated type is
$\{ \File \} \rightarrow_{\varnothing} \Unit$: not good!

Suppose the unannotated code defines a function $f$, which gets
annotated with $\varepsilon_s$ to produce
$\annot{f}{\varepsilon_s}$. Suppose $\annot{f}{\varepsilon_s}$ is
invoked at a later point in the annotated world and incurs the effect
$r.\pi$. What is the source of $r.\pi$? If $r.\pi$ was selected by the
$\kwa{import}$ expression surrounding $f$, it is safe for
$\annot{f}{\varepsilon_s}$ to incur this effect. Otherwise,
$\annot{f}{\varepsilon_s}$ may have been passed an argument which can
be used to incur $r.\pi$, in which case $r.\pi$ is a higher-order
effect of $\annot{f}{\varepsilon_s}$. If the argument is a function,
then $r.\pi \in \varepsilon_s$ by the soundness of our calculus (or it
would not typecheck). If the argument is a resource $r$, then
$\annot{f}{\varepsilon_s}$ may exercise $r.\pi$ without declaring it
--- this is the case we do not yet account for.

We want $\varepsilon_s$ to contain every effect captured by resources
passed into $\annot{f}{\varepsilon_s}$ as arguments. We can do this by
inspecting the (unannotated) type of $f$ for resource sets. For
example, if the unannotated type is
$\kwa{ \{ File \} \rightarrow \Unit}$, then we need
$\kwa{ \{ File.* \} }$ in $\varepsilon_s$. To do this, we add a new
premise $\hofx{\annot{\tau}{\varnothing}} \subseteq
\varepsilon_s$. $\kwa{ho \hyphen effects}$ is only defined on
annotated types, so we first annotate $\tau$ with $\varnothing$. We
are only inspecting the resources passed into $f$ as arguments, so the
annotations are not relevant -- annotating $\tau$ with $\varnothing$
is therefore a good choice. We can now handle the example from
before. The unannotated code types via the judgement
$x: \Unit \vdash \lambda f: \{ \File \}.~\kwa{f.write}: \{ \File \}
\rightarrow \Unit$. Its higher-order effects are
$\hofx{\annot{ \{ \File \} \rightarrow \Unit}{\varnothing}} = \{
\kwa{File.*} \}$, but $\{ \kwa{File.*} \} \not\subseteq \varnothing$,
so the example is safely rejected.

The final version of \textsc{$\varepsilon$-Import} is given in Figure
\ref{fig:import_rule}. With it, we can now model the example from the
beginning of this section, where the $\kwa{Logger}$ selects the
$\kwa{File}$ capability and exposes an unannotated function
$\kwa{log}$ with type $\Unit \rightarrow \Unit$ and implementation
$e$. The expected least authority of $\kwa{Logger}$ is
$\{ \kwa{File.append} \}$, so its corresponding $\kwa{import}$
expression would be
$\import{\kwa{File.append}}{f}{\kwa{File}}{\lambda x: \Unit.~e}$. The
imported capability is $ f = \kwa{File}$, and
$\fx{\{\File\}} = \{ \kwa{File.*} \} \not\subseteq \{
\kwa{File.append} \}$, so this example is safely rejected:
$\kwa{Logger.log}$ has authority to do anything with $\kwa{File}$, and
its implementation $e$ might be violating its stipulated least
authority $\{ \kwa{File.append} \}$.

\vspace{-0.5cm}
\subsection{Soundness ($\epscalc$)} 
\vspace{-0.5cm}

Only annotated programs can be reduced and have their effects
approximated, so the soundness theorem only applies to annotated
judgements. Its statement is given below.

\begin{figure}[t]

\[
\begin{array}{c}

\infer[\textsc{($\varepsilon$-Import)}]
	{ \hat \Gamma \vdash \kwa{import}(\varepsilon_s)~x = \hat e~\kw{in} e: \kwa{annot}(\tau, \varepsilon_s)~\kw{with} \varepsilon_s \cup \varepsilon_1 }
{{\def\arraystretch{1.4}
  \begin{array}{c}
\kwa{effects}(\hat \tau) \cup \hofx{\annot{\tau}{\varnothing}}\subseteq \varepsilon_s \\
\hat \Gamma \vdash \hat e: \hat \tau~\kw{with} \varepsilon_1 ~~~~~~ \kwa{ho \hyphen safe}(\hat \tau, \varepsilon_s) ~~~~~~ x: \kwa{erase}(\hat \tau) \vdash e: \tau
  \end{array}}} 
 
\end{array}
\]

\vspace{-0.2cm}
\caption{The final rule for typing imports.}
\vspace{-0.5cm}
\label{fig:import_rule}
\end{figure}

\begin{theorem}[$\epscalc$ Single-step Soundness]
If $\hat \Gamma \vdash \hat e_A: \hat \tau_A~\kw{with} \varepsilon_A$ and $\hat e_A$ is not a value, then $\hat e_A \longrightarrow \hat e_B~|~\varepsilon$, where $\hat \Gamma \vdash \hat e_B: \hat \tau_B~\kw{with} \varepsilon_B$ and $\hat \tau_B <: \hat \tau_A$ and $\varepsilon_B \cup \varepsilon \subseteq \varepsilon_A$, for some $\hat e_B, \varepsilon, \hat \tau_B, \varepsilon_B$.
\end{theorem}

Due to small page limit we refer the interested readers to the Technical Report with complete proofs including multi-step soundness for the system presented here~\cite{ecs:2018:aaron-tr}.