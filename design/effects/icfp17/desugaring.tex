\section{Desugaring}

In this section we develop notation and techniques so our calculi can express the practical examples of the next section. To do this we show how to encode $\unit$ and $\kwa{let}$ in $\epscalc$, make some simplifying assumptions, and show how to express the Wyvern examples in $\epscalc$.

\subsection{Unit, Let}

The $\unit$ literal is defined as $\unit \defn \lambda x: \varnothing.~x$. It is the same in both annotated and unannotated code. In annotated code, it has the type $\Unit \defn \varnothing \rightarrow_{\varnothing} \varnothing$, while in unannotated code it has the type $\Unit \defn \varnothing \rightarrow \varnothing$. These are technically two separate types, but we will not distinguish between them. Note that $\unit$ is a value, and because $\varnothing$ is uninhabited (there is no empty resource literal), $\unit$ cannot be applied to anything. Furthermore, $\vdash \unit: \Unit~\kw{with} \varnothing$ by \textsc{$\varepsilon$-Abs}, and $\vdash \unit: \Unit$ by \textsc{T-Abs}. We use $\Unit$ to represent the absence of information, such as when a function takes no input or returns no value

The expression $\letxpr{x}{\hat e_1}{\hat e_2}$ reduces $\hat e_1$ to a value $\hat v_1$, binds it to the name $x$ in $\hat e_2$, and then executes $[\hat v_1/x]\hat e_2$. If $\hat \Gamma \vdash \hat e_1: \hat \tau_1~\kw{with} \varepsilon_1$, then $\letxpr{x}{\hat e_1}{\hat e_2} \defn (\lambda x: \hat \tau_1 . \hat e_2) \hat e_1$\footnote{We could also define an unannotated version of $\kwa{let}$, but we only need the annotated version.}. If $\hat e_1$ is a non-value, we can reduce the $\kwa{let}$ by \textsc{E-App2}. If $\hat e_1$ is a value, we may apply \textsc{E-App3}, which binds $\hat e_1$ to $x$ in $\hat e_2$. $\kwa{let}$ expressions can be typed using \textsc{$\varepsilon$-App}.

\subsection{Modules}

Wyvern's modules are first-class, desugaring into objects --- invoking a module's function is no different from invoking an object's method. There are two kinds of modules: pure and resourceful. For our purposes, a pure module is one with no (transitive) authority over any resources, while a resource module has (transitive) authority over some resource. A pure module may still be given a capability, for example as a function argument, but it may not possess or capture the capability for longer than the duration of the method call. Figure \ref{fig:wyv_modules} shows an example of two modules, one pure and one resourceful, each declared in a separate file. Pure modules are declared with the $\kwa{module}$ keyword, while resource modules are declared with $\kwa{resource~module}$.

\begin{figure}[h]

\begin{lstlisting}
module PureMod

def tick(f: {File}):Unit with {File.append}
   f.append

\end{lstlisting}

\begin{lstlisting}
resource module ResourceMod
require File

def tick():Unit with {File.append}
   File.append
\end{lstlisting}

\caption{Definition of two modules, one pure and the other resourceful.}
\label{fig:wyv_modules}
\end{figure}

Resource modules, like objects, must be instantiated. When they are instantiated they are given the capabilities they require. In Figure \ref{fig:wyv_modules}, $\kwa{ResourceMod}$ requests the use of a $\kwa{File}$ capability. Figure \ref{fig:wyv_module_instantiation} demonstrates how the two modules above would be instantiated and used. To prevent infinite regress the $\kwa{File}$ must, at some point, be introduced into the program. This happens in a special main module. When the program begins execution, the $\kwa{File}$ capability is passed into the program from the system environment. $\kwa{Main}$ then instantiates all the other modules in the program with their capabilities. If a module is annotated, its function signatures will have effect annoations. For example, in Figure \ref{fig:wyv_modules}, $\kwa{PureMod.tick}$ has the $\kwa{File.append}$ annotation, meaning it should typecheck as $\kwa{ \{ File \} \rightarrow_{\{\kwa{File.append}\}} \Unit }$. Both $\kwa{PureMod}$ and $\kwa{ResourceMod}$ are annotated. 


\begin{figure}[h]

\begin{lstlisting}
resource module Main
require File
instantiate PureMod
instantiate ResourceMod(File)

PureMod.tick(File)
\end{lstlisting}

\caption{The $\kwa{Main}$ module which instantiates $\kwa{PureMod}$ and $\kwa{ResourceMod}$ and then invokes $\kwa{PureMod.tick}$.}
\label{fig:wyv_module_instantiation}
\end{figure}

Our Wyvern examples are simplified in several ways so they can be expressed in $\epscalc$. The only objects used are modules. The modules only ever contain one function and the capabilities they require; they have no mutable fields. There are no self-referencing modules or recursive functions. Modules do not reference each other cyclically. These simplifications enable us to model each module as a function. Applying the function will be equivalent to applying the single function defined by the module. A collection of modules is desugared into $\epscalc$ as follows. First, a sequence of let-bindings are used to name constructor functions which, when given the capabilities requested by a module, will return the function representing an instance of that module. The constructor for $\kwa{M}$ is called $\kwa{MakeM}$. If the module does not require any capabilities it takes $\Unit$ as its argument. A function is then defined which represents the body of code in the $\kwa{Main}$ module. When invoked, this function will instantiate all the modules by invoking their constructors and execute the code in $\kwa{Main}$. Finally, the function representing $\kwa{Main}$ is invoked with the primitive capabilities that are passed from the system environment into $\kwa{Main}$.

Figure \ref{fig:wyv_tutorial_desugaring} shows how the examples above desugar. Lines 1-3 define the constructor for $\kwa{PureMod}$. Since $\kwa{PureMod}$ requires no capabilities, the constructor takes $\Unit$ as an argument on line 2. Lines 5-7 define the constructor for $\kwa{ResourceMod}$. It requires a $\kwa{File}$ capability, so the constructor takes $\kwa{\{File\}}$ as its input type on line 6. The constructor for $\kwa{Main}$ is defined on lines 9-14, which instantiates the other modules and runs the code inside $\kwa{Main}$. Line 16 starts execution by invoking $\kwa{MakeMain}$ with the initial set of capabilities, which in this case is just $\kwa{File}$.

\begin{figure}[h]

\begin{lstlisting}
let MakePureMod =
   $\lambda$x:Unit.
      $\lambda$f:{File}. f.append in

let MakeResourceMod =
   $\lambda$f:{File}.
      $\lambda$x:Unit. f.append in

let MakeMain =
   $\lambda$f:{File}.
      $\lambda$x: Unit.
         let PureMod = (MakePureMod unit) in
         let ResourceMod = (MakeResourceMod f) in
         (ResourceMod unit) in

(MakeMain File) unit
\end{lstlisting}

\caption{Desugaring of $\kwa{PureMod}$ and $\kwa{ResourceMod}$ into $\epscalc$.}
\label{fig:wyv_tutorial_desugaring}
\end{figure}

When an unannotated module is translated into $\epscalc$, the desugared contents will be encapsulated with an $\kwa{import}$ expression. The selected authority on the $\kwa{import}$ expression will be that we expect of the unannotated code according to the principle of least authority in the particular example under consideration. For example, if the client only expects the unannotated code to have the $\kwa{File.append}$ effect, the corresponding $\kwa{import}$ expression will select $\kwa{\{File.append\}}$.