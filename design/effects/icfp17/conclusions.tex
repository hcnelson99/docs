
\section{Conclusions}

We have explored $\opercalc$, which extends the simply-typed lambda calculus with a notion of primitive capabilities and their effects. $\opercalc$ programs are fully annotated with their effects; relaxing this requirement, we obtained $\epscalc$, which allows unannotated code to be nested inside annotated code with a new $\kwa{import}$ construct. The capability-safe design of $\epscalc$ allows us to make a safe inference about the effects of the unannotated code by inspecting what capabilities are passed into it by its annotated surroundings. Such an approach allows code to be incrementally annotated, giving developers a balance between safety and convenience and alleviating the verbosity that has discouraged widespread adoption of effect systems \cite{rytz2012}.

\subsection{Related Work}

Capabilities were introduced by Dennis and Van Horn as a way to control which processes in an operating system had permission to access certain parts of memory \cite{dennis66}. An \textit{access control list} would declare what permissions a program may exercise. These early ideas are considerably different to the object capability model introduced by Mark Miller \cite{miller06}, which imposes constraint on how permissions can proliferate. Maffeis et. al. formalised the notion of a capability-safe language and showed that a subset of Caja (a Javascript implementation) is capability-safe \cite{maffeis10}. Miller's model has also has been applied to more heavyweight formal systems: Drossopoulou et. al. combined Hoare logic with capabilities to determine whether components of a system can be trusted \cite{drossopoulou07}. Other capability-safe languages include Wyvern \cite{nistor13} and Newspeak \cite{bracha10}.

The original effect system by Lucassen and Gifford was used to determine if two expressions could safely run in parallel \cite{lucassen88}. Subsequent applications include determining what functions a program might invoke \cite{tang94} and what regions in memory might be accessed or updated during execution \cite{talpin94}. In these systems, ``effects'' are performed upon ``regions''; in ours, ``operations'' are performed upon ``resources''. An important difference in $\epscalc$ is the distinction between annotated and unannotated code: only the former will type-and-effect-check. This approach allows for an effect discipline to be incrementally imposed on an otherwise effect-unconscious system.

Fengyun Liu has also combined capability-safety and effect systems, with applications to purity analysis in Scala \cite{liu16}. If a function is known to be pure then optimisations such as inlining and parallelisation can be made. Liu's work is motivated by achieving such optimisations for Scala compilers. It distinguishes between free and stoic functions: free functions may exercise ambient authority whereas stoic functions may not. Stoic functions are therefore capability-safe pockets whose purity can be determined by examining what capabilities are passed into them. Liu's System F-Impure does not track effects, whereas $\epscalc$, by distinguishing between regular effects and higher-order effects, gives more fine-grained detail about what a piece of code will do when executed.

The systems by Lucassen and Liu have effect polymorphism, whereas $\epscalc$ does not. Another capability-safe effect-system is the one by Devriese et. al., who use effect polymorphism and possible world semantics to guarantee behavioural invariants on data structures \cite{devriese16}. Our approach is not as expressive, based only on a topological analysis of how capabilities can be passed around the program, but the formalism is much more lightweight.

\subsection{Future Work}

Our conception of effects is quite specific, modelling only the invocation of operations on a primitive capability as an effect. This definition could be generalised to allow for other sorts of effects, such as accessing or writing mutable state. Resources and operations are also fixed throughout runtime; it would be interesting to consider the theory in a setting which allows dynamic resource creation and destruction.

The current theory contains no effect polymorphism. This would allow the type of a function to be parameterised by a set of effects. For an example of such a function, consider $\kwa{map}$: given a function $f$ and a list $l$, map applies $f$ to every element of $l$ to produce a new list $l'$. The effects of $\kwa{map}$ are dependent on the effects of $f$. The only way to define $\kwa{map}$ in $\epscalc$ would be to conservatively approximate it as having every effect, in which case all precision has been lost. A polymorphic effect system which considers the type of $\kwa{map}$ as being parameterised by a set of effects could give more meaningful approximations.

Lastly, the ideas in this paper might be extended and developed to the point where they can be used in real-world situations. Implementing these ideas in an existing, general-purpose language would do much towards that end.

%Many believe in the real and practical value of the object capability model, but we do not fully understand its formal benefits. 






