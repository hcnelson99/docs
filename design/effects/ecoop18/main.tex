\documentclass[a4paper,UKenglish]{lipics-v2016}
%This is a template for producing LIPIcs articles. 
%See lipics-manual.pdf for further information.
%for A4 paper format use option "a4paper", for US-letter use option "letterpaper"
%for british hyphenation rules use option "UKenglish", for american hyphenation rules use option "USenglish"
% for section-numbered lemmas etc., use "numberwithinsect"
 
\usepackage{microtype}%if unwanted, comment out or use option "draft"

%\graphicspath{{./graphics/}}%helpful if your graphic files are in another directory

\bibliographystyle{plainurl}% the recommended bibstyle

% Author macros::begin %%%%%%%%%%%%%%%%%%%%%%%%%%%%%%%%%%%%%%%%%%%%%%%%
\title{Capabilities: Effects for Free}
%\titlerunning{Capabilities: Effects for Free} %optional, in case that the title is too long; the running title should fit into the top page column

%% Please provide for each author the \author and \affil macro, even when authors have the same affiliation, i.e. for each author there needs to be the  \author and \affil macros
\author{Authors omitted for double-blind review.}
%\author[1]{Aaron Craig}
%\author[1]{Alex Potanin}
%\author[1]{Lindsay Groves}
%\author[2]{Jonathan Aldrich}
%\affil[1]{ECS, VUW\\
%  \texttt{aaroncraig@protonmail.ch, alex@ecs.vuw.ac.nz, lindsay@ecs.vuw.ac.nz}}
%\affil[2]{ISR, CMU\\
%  \texttt{jonathan.aldrich@cs.cmu.edu}}
%\authorrunning{A. Craig et al} %mandatory. First: Use abbreviated first/middle names. Second (only in severe cases): Use first author plus 'et. al.'

%\Copyright{Aaron Craig, Alex Potanin, Lindsay Groves, and Jonathan Aldrich}%mandatory, please use full first names. LIPIcs license is "CC-BY";  http://creativecommons.org/licenses/by/3.0/

\subjclass{300 Software and its engineering, Semantics}
%Dummy classification -- please refer to \url{http://www.acm.org/about/class/ccs98-html}}% mandatory: Please choose ACM 1998 classifications from http://www.acm.org/about/class/ccs98-html . E.g., cite as "F.1.1 Models of Computation". 
\keywords{capabilities, effects}% mandatory: Please provide 1-5 keywords
% Author macros::end %%%%%%%%%%%%%%%%%%%%%%%%%%%%%%%%%%%%%%%%%%%%%%%%%

%Editor-only macros:: begin (do not touch as author)%%%%%%%%%%%%%%%%%%%%%%%%%%%%%%%%%%
\EventEditors{John Q. Open and Joan R. Acces}
\EventNoEds{2}
\EventLongTitle{42nd Conference on Very Important Topics (CVIT 2016)}
\EventShortTitle{CVIT 2016}
\EventAcronym{CVIT}
\EventYear{2016}
\EventDate{December 24--27, 2016}
\EventLocation{Little Whinging, United Kingdom}
\EventLogo{}
\SeriesVolume{42}
\ArticleNo{23}
% Editor-only macros::end %%%%%%%%%%%%%%%%%%%%%%%%%%%%%%%%%%%%%%%%%%%%%%%

%% Some recommended packages.
\usepackage{booktabs}   %% For formal tables:
                        %% http://ctan.org/pkg/booktabs
\usepackage{subcaption} %% For complex figures with subfigures/subcaptions
                        %% http://ctan.org/pkg/subcaption


\usepackage{bm}
\usepackage{color}
\usepackage{ebproof} % For proof trees
\usepackage{listings} % For code snippets
\usepackage{proof} % For inference rules.
\usepackage[ruled]{algorithm2e}


\definecolor{grey}{gray}{0.92}

\lstset{
tabsize=3,
basicstyle=\ttfamily\small, commentstyle=\itshape\rmfamily, 
backgroundcolor=\color{grey},
numbers=left,
numberstyle=\tiny,
language=java,
moredelim=[il][\sffamily]{?},
mathescape=true,
showspaces=false,
showstringspaces=false,
columns=fullflexible,
escapeinside={(@}{@)}, morekeywords=[1]{def, if, then, else, with, val, module, instantiate}}
\lstloadlanguages{Java,VBScript,XML,HTML}

%using \kwa outside math mode
\newcommand{\kwat}[1]{$\kwa{#1}$}

% Hyphens
\newcommand{\hyphen}{\hbox{-}}

% For defining derived forms.
\newcommand\defn{\mathrel{\overset{\makebox[0pt]{\mbox{\normalfont\tiny\sffamily def}}}{=}}}

% Constants, types.
\newcommand{\unit}{\kwa{unit}}
\newcommand{\Unit}{\kwa{Unit}}
\newcommand{\File}{\kwa{File}}
\newcommand{\Socket}{\kwa{Socket}}

% Keywords.
\newcommand{\kwa}[1]{\mathtt{#1}}
\newcommand{\kw}[1]{\mathtt{#1}~}

% Expressions.
\newcommand{\import}[4]{\kwa{import}(#1)~#2 = #3~\kw{in} #4}
\newcommand{\letxpr}[3]{\kw{let} #1 = #2~\kw{in} #3}	

% Functions in the type theory.
\newcommand{\annot}[2]{\kwa{annot}(#1, #2)}
\newcommand{\erase}[1]{\kwa{erase}(#1)}
\newcommand{\fx}[1]{\kwa{effects}(#1)}
\newcommand{\hofx}[1]{\kwa{ho \hyphen effects}(#1)}

% Safety predicates in the type theory.
\newcommand{\safe}[2]{\kwa{safe}(#1, #2)}
\newcommand{\hosafe}[2]{\kwa{ho \hyphen safe}(#1, #2)}

% Names of the calculi.
\newcommand{\opercalc}{\kwa{OC}}
\newcommand{\epscalc}{\kwa{CC}}


\renewcommand{\algorithmcfname}{ALGORITHM}
\SetAlFnt{\small}
\SetAlCapFnt{\small}
\SetAlCapNameFnt{\small}
\SetAlCapHSkip{0pt}
\IncMargin{-\parindent}


\makeatletter\if@ACM@journal\makeatother
%% Journal information (used by PACMPL format)
%% Supplied to authors by publisher for camera-ready submission
\acmJournal{PACMPL}
\acmVolume{1}
\acmNumber{1}
\acmArticle{1}
\acmYear{2017}
\acmMonth{1}
\acmDOI{10.1145/nnnnnnn.nnnnnnn}
\startPage{1}
\else\makeatother
%% Conference information (used by SIGPLAN proceedings format)
%% Supplied to authors by publisher for camera-ready submission
\acmConference[PL'17]{ACM SIGPLAN Conference on Programming Languages}{January 01--03, 2017}{New York, NY, USA}
\acmYear{2017}
\acmISBN{978-x-xxxx-xxxx-x/YY/MM}
\acmDOI{10.1145/nnnnnnn.nnnnnnn}
\startPage{1}
\fi


%% Copyright information
%% Supplied to authors (based on authors' rights management selection;
%% see authors.acm.org) by publisher for camera-ready submission
\setcopyright{none}             %% For review submission
%\setcopyright{acmcopyright}
%\setcopyright{acmlicensed}
%\setcopyright{rightsretained}
%\copyrightyear{2017}           %% If different from \acmYear


%% Bibliography style
\bibliographystyle{ACM-Reference-Format}
%% Citation style
%% Note: author/year citations are required for papers published as an
%% issue of PACMPL.
\citestyle{acmauthoryear}   %% For author/year citations % Contains the packages and other data common to both paper (main.tex) and supplementary material (proofs.tex).

% Document starts
\begin{document}

\maketitle

%% Abstract
%% Note: \begin{abstract}...\end{abstract} environment must come
%% before \maketitle command
\begin{abstract}
Capabilities are increasingly used to reason informally about the properties of secure systems. Can capabilities also aid in \textit{formal} reasoning? To answer this question, we examine a calculus that uses effects to capture resource use and extend it with a rule that captures the essence of capability-based reasoning. We demonstrate that capabilities provide a way to reason for free about effects: we can bound the effects of an expression based on the capabilities to which it has access.  This reasoning is ``free'' in that it relies only on type-checking (not effect-checking); does not require the programmer to add effect annotations within the expression; nor does it require the expression to be analysed for its effects. Our result sheds light on the essence of what capabilities provide and suggests useful ways of integrating lightweight capability-based reasoning into languages.
\end{abstract}

\chapter{Introduction}\label{C:intro}

Good software is distinguished from bad software by design qualities such as security, maintainability, and performance. We are interested in how the design of a programming language and its type system can make it easier to write secure software.

There are different situations where we may not trust code. One example is in any development environment adhering to ideas of \textit{code ownership}, wherein developers may be responsible for particular components in the system \cite{bird_ownership}. When a developer writes code to interact with a component outside their domain of responsibility, they can make false assumptions or interact with the other component incorrectly, potentially introducing security bugs. Another setting involves applications which allow third-party plug-ins, in which case third-party code could be written by anyone, including the untrustworthy. One kind of application which does this is the web mash-up, which brings together several existing, disparate services into one system. In both cases we want the entire system to function securely, despite the existence of untrustworthy components.

It is difficult to determine if a piece of code is trustworthy, but a range of techniques might be used. One approach is to \textit{sandbox} the untrusted code inside a virtual environment. If anything goes wrong, damage is theoretically limited to the virtual environment, but in practice, this approach has many vulnerabilities \cite{coker15, maass16, watson07, schreuders13}. On the other hand, verification techniques allow for a robust analysis of the behaviour of code, but are heavyweight and require the developers using them to have a deep understanding of the techniques being employed \cite{kneuper97}. Furthermore, verification requires one to supply a complete specification of the system, which may itself be an undefined or evolving artifact during the development process. Lightweight analyses, such as type systems, are easy for the developer to use, but existing languages lack adequate controls for detecting and isolating untrustworthy components \cite{chen07, ter-louw08}. A qualitative approach might instead be employed, where software is developed according to best-practice guidelines. One such guideline is the \textit{principle of least authority}: that software components should only have access to the information and resources necessary for their purpose \cite{saltzer74}. For example, a logger module, which needs only to append to a file, should not have arbitrary read-write access. Another is \textit{privilege separation}, where the division of a program into components is informed by what resources are needed and how they are to be propagated \cite{saltzer75}. This report is interested in the class of lightweight analyses, and in particular how type systems could be used to reject unsafe programs or put developers in a more informed position to make qualitative assessments about their code.

One approach to privilege separation is the capability model. A \textit{capability} is an unforgeable token granting its bearer permission to perform some operation \cite{dennis66}. For example, a system resource like a file or socket can only be used through a capability granting operations on it. Capabilities also encapsulate the source of \textit{effects}, which describe intensional details about the way in which a program executes \cite{nielson99}. For example, a logger might $\kwa{append}$ to a $\kwa{File}$, and so executing its code would incur the $\kwa{File.append}$ effect. In the capability model, this would require the logger to possess a capability granting it the ability to append to files.

Although the idea of a capability is an old one in the access literature, there has been recent interest in the application of the idea to programming language design. Miller has identified the ways in which capabilities should proliferate to encourage \textit{robust composition} --- a set of ideas summarised as ``only connectivity begets connectivity'' \cite{miller06}. In this paradigm, actors in a program are explicitly parametrised by what capabilities they use. This enables one to reason about what privileges a component might exercise by examining its interface. Building on these ideas, Maffeis et. al. formalised the notion of a \textit{capability-safe} language, showing a subset of Caja (a JavaScript implementation) is capability-safe \cite{maffeis10}.

Effect systems were introduced by Lucassen and Gifford for the purposes of optimising pure code \cite{lucassen88}. They have also been applied to problems such as determining which functions might be invoked in a program \cite{tang94}, or determining which regions in memory may be accessed or updated \cite{talpin94}. Knowing what effects a piece of code might incur allows a developer to determine if code is trustworthy before executing it. This can be qualitatively assessed by comparing the static approximation of its effects to its expected least authority --- a ``logger'' implementation which writes to a $\kwa{Socket}$ is not to be trusted!

Despite these benefits, effect systems have seen little use in mainstream programming languages. Rytz et. al. believe verbosity is the main reason \cite{rytz2012}. Successive works have focussed on reducing the developer overhead through techniques such as effect-inference, but the benefit of capabilities for enabling effect-inference has not received much attention. Because capabilities encapsulate the source of effects, and because capability-safety impose constraints on how they propagate through a system, the task of determining what effects might be incurred by a piece of code is simplified. This is the key contribution of this report: the idea that capability-safety facilitates a low-cost effect system with minimal user overhead. 

We begin this report by discussing preliminary concepts involving the formal definition of programming languages, effect systems, and Miller's capability model. Chapter 3 introduces the Operation Calculus $\opercalc$, a typed, effect-annotated lambda calculus with a simple notion of capabilities and effect. Dropping the requirement that all code in a program must be effect-annotated, we develop the Capability Calculus $\epscalc$, which permits the nesting of unannotated code inside annotated code in a controlled, capability-safe manner with a new $\kwa{import}$ construct. A safe inference about the unannotated code can be made at these junctions. In chapter 4 we demonstrate how $\epscalc$ can model practical examples, finishing with a summary and comparison of some of the existing work in this area.

\section{Operation Calculus ($\opercalc$)}

$\opercalc$ extends the simply-typed lambda calculus with a notion of primitive capabilities and their operations. Every function is annotated with the effects it may incur. Its static rules associate a type and a set of effects to well-formed programs. Defining and proving $\opercalc$ sound will introduce the notations and concepts needed to understand $\epscalc$, which is the more general, interesting language.

In a capability-safe language, ``only connectivity begets connectivity'' \cite{miller06}: all access to a capability must derive from previous access. To prevent an infinite regress, there are a set of primitive capabilities passed into the program by the system environment. These primitive capabilities present operations to their bearer for manipulating resources in the system environment. For example, $\kwa{File}$ might provide read/write operations on a particular file in the file system. For convenience, we often conflate primitive capabilities with the resources they manipulate, referring to both as resources. An effect in $\opercalc$ is a particular operation invoked on some resource; for example, $\kwa{File.write}$. The pieces of an $\opercalc$ program are (conservatively) annotated with the effects they may incur at runtime. Annotations might be given in accordance with the principle of least authority to specify the maximum authority a component may exercise. When this authority is being exceeded, an effect system like that of $\opercalc$ will reject the program, signalling an unsafe implementation. For example, consider the pair of modules in Figure \ref{fig:opercalc_motivating}: the $\kwa{Logger}$ module possesses a $\File$ capability and exposes a single function $\kwa{log}$. The $\kwa{Client}$ has a single function $\kwa{run}$ which, when passed a $\kwa{Logger}$, will invoke $\kwa{Logger.log}$.

\begin{figure}[h]
\vspace{-5pt}

\begin{lstlisting}
resource module Logger
require File

def log(): Unit with {File.append} =
    File.read
\end{lstlisting}

\begin{lstlisting}
module Client

def run(l: Logger): Unit with {File.append} =
    l.log()
\end{lstlisting}

\vspace{-7pt}
\caption{The implementation of $\kwa{Logger.log}$ exceeds its specified authority.}
\label{fig:opercalc_motivating}
\end{figure}

$\kwa{Client.run}$ and $\kwa{Logger.log}$ are both annotated with $\{ \kwa{File.append} \}$, but the (potentially malicious) implementation of $\kwa{Logger.log}$ incurs the $\kwa{File.read}$ effects. By the end of this section, we will have developed rules for $\opercalc$ that can determine such mismatches between specification and implementation in annotated code.

$\opercalc$ makes some simplifying assumptions. The semantics of particular operations are not modelled --- our only interest is in what operations could be invoked, and by whom. Therefore, we assume all operations are null-ary and return a dummy $\unit$ value; $\kwa{File.write(``hello, world!'')}$ becomes $\kwa{File.write}$. Primitive capabilities and operations are fixed throughout the runtime and cannot be created or destroyed.

\subsection{Grammar ($\opercalc$)}

A grammar for $\opercalc$ programs is given in Figure \ref{fig:opercalc_grammar}. In addition to those from the lambda calculus, there are two new forms. A resource literal $r$ is a variable drawn from a fixed set $R$. Resources model those primitive capabilities into the program from the system environment. $\kwa{File}$ and $\kwa{Socket}$ are examples of resource literals. An operation call $e.\pi$ is the invocation of an operation $\pi$ on $e$. For example, invoking the $\kwa{open}$ operation on the $\kwa{File}$ resource would be $\kwa{File.open}$. Operations are drawn from a fixed set $\Pi$.

\begin{figure}[h]
\vspace{-5pt}

\[
\begin{array}{lll}

\begin{array}{lllr}

e & ::= & ~ & exprs: \\
	& | & x & variable \\
	& | & v & value \\
	& | & e ~ e & application \\
	& | & e.\pi & operation~call \\
	&&\\

\end{array}

\begin{array}{lllr}

v & ::= & ~ & values: \\ 
	& | & r & resource~literal \\
	& | & \lambda x: \tau.e & abstraction \\
	&&\\

\end{array}

\end{array}
\]

\vspace{-7pt}
\caption{Grammar for $\opercalc$ programs.}
\label{fig:opercalc_grammar}
\end{figure}

An effect is a pair $(r, \pi) \in R \times \Pi$. Sets of effects are denoted $\varepsilon$. As a shorthand, we write $r.\pi$ instead of $(r, \pi)$. Effects should be distinguished from operation calls: an operation call is the invocation of a particular operation on a particular resource in a program, while an effect is a mathematical object describing this behaviour. The notation $r.*$ is a short-hand for the set $\{ r.\pi \mid \pi \in \Pi \}$, which contains every effect on $r$. Sometimes we abuse notation by conflating the effect $r.\pi$ with the singleton $\{ r.\pi \}$. We may also write things like $\{ r_1.*, r_2.* \}$, which should be understood as the set of all operations on $r_1$ and $r_2$.

\subsection{Substitution ($\opercalc$)}

Figure \ref{fig:opercalc_sub_defn} defines substitution on the new forms of $\opercalc$. We also impose an extra restriction that a variable can only be substituted for a value. This restriction is imposed because if a variable can be replaced with an arbitrary expression, we might also be introducing arbitrary effects, violating the type-and-effect safety of the static rules. The semantics are call-by-value, so this restriction is no problem.

\begin{figure}[h]

$\kwa{substitution :: e \times v \times v \rightarrow e}$

\begin{itemize}
	\item[] $[v/y]r = r$
	\item[] $[v/y](e_1.\pi) = ([v/y]e_1).\pi$
\end{itemize}

\vspace{-7pt}
\caption{Extra cases for $\kwa{substitution}$ in $\opercalc$.}
\label{fig:opercalc_sub_defn}
\end{figure}


\subsection{Semantics ($\opercalc$)}

During reduction an operation call may be evaluated. When this happens a runtime effect is said to have taken place. Reflecting this, the form of the single-step reduction judgement is now $e \longrightarrow e~|~\varepsilon$, which means that $e$ reduces to $e'$, incurring the set of effects $\varepsilon$ in the process. In the case of single-step reduction, $\varepsilon$ is at most a single effect. Judgements for single-step reductions are given in Figure \ref{fig:opercalc_singlestep}.

\begin{figure}[h]

\noindent
\fbox{$e \longrightarrow e~|~\varepsilon$}

\[
\begin{array}{c}

\infer[\textsc{(E-App1)}]
	{e_1 e_2 \longrightarrow e_1'~ e_2~|~\varepsilon}
	{e_1 \longrightarrow e_1'~|~\varepsilon}
	~~~~~~
\infer[\textsc{(E-App2)}]
	{v_1 ~ e_2 \longrightarrow v_1 ~ e_2'~|~\varepsilon} 
	{e_2 \longrightarrow e_2'~|~\varepsilon}
~~~~~~
\infer[\textsc{(E-App3)}]
	{ (\lambda x: \tau. e) v_2 \longrightarrow [ v_2/x] e~|~\varnothing }
	{}\\[4ex]
	
\infer[\textsc{(E-OperCall1)}]
	{ e.\pi \longrightarrow  e'.\pi~|~\varepsilon }
	{ e \rightarrow  e'~|~\varepsilon}
		
	~~~~~~
	
\infer[\textsc{(E-OperCall2)}]
	{r.\pi \longrightarrow \kwa{unit}~|~\{ r.\pi \} }
	{}
	 \\[4ex]
	 
\end{array}
\]


\vspace{-7pt}
\caption{Single-step reductions in $\opercalc$.}
\label{fig:opercalc_singlestep}
\end{figure}

The first three rules are analogous to reductions in the lambda calculus. \textsc{E-App1} and \textsc{E-App2} incur the effects of reducing their subexpressions. \textsc{E-App3} replaces the formal name $x$ with the actual value $v_2$ being passed as an argument, which incurs no effects. The first new rule is \textsc{E-OperCall1}, which reduces the receiver of an operation call; the effects incurred are the effects incurred by reducing the receiver. When an operation $\pi$ is invoked on a resource literal $r$, \textsc{E-OperCall2} will reduce it to $\unit$, incurring $\{ r.\pi \}$ as a result. For example, $\kwa{File.write} \longrightarrow \unit~|~\{ \kwa{File.write} \}$ by \textsc{E-OperCall2}. $\unit$ can be treated as a derived form; an explanation is given in section 4.

A multi-step reduction is a sequence of zero\footnote{We permit multi-step reductions of length zero to be consistent with Pierce, who defines multi-step reduction as a reflexive relation \cite[p. 39]{pierce02}.} or more single-step reductions. The resulting set of runtime effects is the union of all the runtime effects from the intermediate single-steps. Judgements for multi-step reductions are given in Figure \ref{fig:opercalc_multistep_defn}. By \textsc{E-MultiStep1}, any expression can ``reduce'' to itself with no runtime effects. By \textsc{E-MultiStep2}, any single-step reduction is also a multi-step reduction. If $e \longrightarrow e'~|~\varepsilon_1$ and $e' \longrightarrow e''~|~\varepsilon_2$ are sequences of reductions, then so is $e \longrightarrow e''~|~\varepsilon_1 \cup \varepsilon_2$, by \textsc{E-MultiStep3}.

\begin{figure}[h]

\noindent
\fbox{$ e \longrightarrow^{*}  e~|~\varepsilon$}

\[
\begin{array}{c}

\infer[\textsc{(E-MultiStep1)}]
	{ e \rightarrow^{*}  e~|~\varnothing}
	{}
~~~
\infer[\textsc{(E-MultiStep2)}]
	{ e \rightarrow^{*}  e'~|~\varepsilon}
	{ e \rightarrow  e'~|~\varepsilon}
~~~	
\infer[\textsc{(E-MultiStep3)}]
	{ e \rightarrow^{*}  e''~|~\varepsilon_1 \cup \varepsilon_2}
	{ e \rightarrow^{*}  e'~|~\varepsilon_1 &  e' \rightarrow^{*}  e''~|~\varepsilon_2}
\end{array}
\]

\vspace{-7pt}
\caption{Multi-step reductions in $\opercalc$.}
\label{fig:opercalc_multistep_defn}
\end{figure}

\subsection{Static Rules ($\opercalc$)}

A grammar for types and type contexts is given in Figure \ref{fig:opercalc_types}. The base types of $\opercalc$ are sets of resources, denoted $\{ \bar r\}$. If an expression $e$ is associated with type $\{ \bar r \}$, it means $e$ will reduce to one of the literals in $\bar r$ (assuming $e$ terminates). The set of empty resources (denoted $\varnothing$) is also a valid type, but has no inhabitants. There is a single type constructor $\rightarrow_{\varepsilon}$, where $\varepsilon$ is a concrete set of effects. $\tau_1 \rightarrow_{\varepsilon} \tau_2$ is the type of a function which takes a $\tau_1$ as input, returns a $\tau_2$ as output, and whose body incurs no more than those effects in $\varepsilon$. $\varepsilon$ is a conservative bound: if an effect $r.\pi \in \varepsilon$, it is not guaranteed to happen at runtime, but if $r.\pi \notin \varepsilon$, it cannot happen at runtime. A typing context $\Gamma$ maps variables to types. 

\begin{figure}[h]
\vspace{-5pt}

\[
\begin{array}{lll}

\begin{array}{lllr}

\tau & ::= & ~ & types: \\
		& | & \{ \bar r \} & resource~set \\
		& | & \tau \rightarrow_{\varepsilon} \tau & annotated~arrow \\ 
		&&\\
\end{array}

\begin{array}{lllr}

\Gamma & ::= & ~ & type~ctx: \\
				& | & \varnothing & empty~ctx. \\
				& | & \Gamma, x: \tau & var.~binding \\
				&&\\
\end{array}

\end{array}
\]

\vspace{-7pt}
\caption{Grammar for types in $\opercalc$.}
\label{fig:opercalc_types}
\end{figure}

To illustrate the types of some functions, if $\kwa{log_1}$ has the type $\{ \kwa{File} \} \rightarrow_{\{\kwa{File.append}\}} \Unit$, then invoking $\kwa{log_1}$ will either incur $\kwa{File.append}$ or no effects. If $\kwa{log_2}$ has the type $\{ \kwa{File} \} \rightarrow_{\{\kwa{File.*}\}} \Unit$, then invoking $\kwa{log_2}$ could incur any effect on $\kwa{File}$.

Knowing approximately what effects a piece of code may incur helps a developer determine whether it can be trusted. For example, consider $\kwa{log_3} = \lambda f: \kwa{File}.~e$, which is a logging function that takes a $\kwa{File}$ as an argument and then executes $e$. Suppose this function were to typecheck as $\{ \File \} \rightarrow_{\{ \kwa{File.*} \}} \Unit$ --- seeing that invoking this function could incur any effect on $\kwa{File}$, and not just its expected least authority $\kwa{File.append}$, a developer may therefore decide this implementation cannot be trusted and choose not to execute it. In this spirit, the static rules of $\opercalc$ associate well-typed programs with a type and a set of effects: the judgement $\Gamma \vdash e: \tau~\kw{with} \varepsilon$, means $e$ will reduce to a term of type $\tau$ (assuming it terminates), incurring no more effects than those in $\varepsilon$. Judgements are given in Figure \ref{fig:opercalc_static_rules}.

\begin{figure}[h]

\noindent
\fbox{$\Gamma \vdash e: \tau~\kw{with} \varepsilon$}

\[
\begin{array}{c}

\infer[\textsc{($\varepsilon$-Var)}]
	{ \Gamma, x:\tau \vdash x: \tau~\kw{with} \varnothing }
	{}
	
	~~~
	
\infer[\textsc{($\varepsilon$-Resource)}]
 	{ \Gamma, r: \{ r \} \vdash r : \{ r \}~\kw{with} \varnothing }
 	{} \\[3ex]
 	
 	~~~
	\infer[\textsc{($\varepsilon$-Abs)}]
	{ \Gamma \vdash \lambda x:\tau_2 . e : \tau_2 \rightarrow_{\varepsilon_3} \tau_3~\kw{with} \varnothing }
	{ \Gamma, x: \tau_2 \vdash e: \tau_3~\kw{with} \varepsilon_3 }
	
	~~~
	
\infer[\textsc{($\varepsilon$-App)}]
	{ \Gamma \vdash e_1~e_2 : \tau_3~\kw{with} \varepsilon_1 \cup \varepsilon_2 \cup \varepsilon  }
	{ \Gamma \vdash e_1: \tau_2 \rightarrow_{\varepsilon} \tau_3~\kw{with} \varepsilon_1 & \Gamma \vdash e_2: \tau_2~\kw{with} \varepsilon_2 } \\[3ex]
	
\infer[\textsc{($\varepsilon$-OperCall)}]
	{ \Gamma \vdash e.\pi: \kw{Unit} \kw{with} \{ \bar r.\pi \} }
	{ \Gamma \vdash e: \{ \bar r \}}

\infer[\textsc{($\varepsilon$-Subsume)}]
	{ \Gamma \vdash e: \tau' ~\kw{with} \varepsilon'}
	{ \Gamma \vdash e: \tau ~\kw{with} \varepsilon & \tau <: \tau' & \varepsilon \subseteq \varepsilon'}\\[3ex]
	
\end{array}
\]


\vspace{-7pt}
\caption{Type-with-effect judgements in $\opercalc$.}
\label{fig:opercalc_static_rules}
\end{figure}



\textsc{$\varepsilon$-Var} approximates the runtime effects of a variable as $\varnothing$. \textsc{$\varepsilon$-Resource} does the same for resource literals. Though a resource captures several effects (namely, every possible operation on itself), attempting to ``reduce'' a resource will incur no effects; something must be done with the resource, such as an operation call, in order to have an effect. For a similar reason, \textsc{$\varepsilon$-Abs} approximates the effects of a function literal as $\varnothing$, and ascribes an arrow type annotated with those effects captured by the function. \textsc{$\varepsilon$-App} approximates a lambda application as incurring those effects from evaluating the subexpressions and the effects incurred by executing the body of the function to which the left-hand side evaluates. The effects of the function body are taken from the function's arrow type. An operation call on a resource literal reduces to $\unit$, so \textsc{$\varepsilon$-OperCall} ascribes its type as $\Unit$. The approximate effects of an operation call are: the effects of reducing the subexpression, and then the operation $\pi$ on every possible resource which that subexpression to which that subexpression might reduce. For example, consider $e.\pi$, where $\Gamma \vdash e: \{ \kwa{File, Socket} \}~\kw{with} \varnothing$. Then $e$ could evaluate to $\kwa{File}$, in which case the actual runtime effect is $\kwa{File}.\pi$, or it could evaluate to $\kwa{Socket}$, in which case the actual runtime effect is $\kwa{Socket.\pi}$. Determining which will happen is, in general, undecidable; the safe approximation is to treat them both as happening. The last rule \textsc{$\varepsilon$-Subsume} produces a new judgement by widening the type or approximate effects on an existing one. Subtyping judgement are given in Figure \ref{fig:opercalc_static_rules}.


\begin{figure}[h]
\vspace{-5pt}

\fbox{$\tau <: \tau$}

\[
\begin{array}{c}

\infer[\textsc{(S-Arrow)}]
	{ \tau_1 \rightarrow_{\varepsilon} \tau_2 <: \tau_1' \rightarrow_{\varepsilon'} \tau_2' }
	{ \tau_1' <: \tau_1 & \tau_2 <: \tau_2' & \varepsilon \subseteq \varepsilon' }
~~~~~~
\infer[\textsc{(S-Resource)}]
	{ \{ \bar r_1 \} <: \{ \bar r_2 \} }
	{ r \in r_1 \implies r \in r_2 }

\end{array}
\]

\vspace{-7pt}
\caption{Subtyping judgements of $\opercalc$.}
\label{fig:opercalc_static_rules}
\end{figure}

\textsc{S-Arrow} is the standard rule for arrow types, but also stipulates that the effects on the arrow of the subtype must be contained in the effects on the arrow of the supertype; a valid subtype should not invoke any effects the supertype does not already know about. \textsc{S-Resource} says that a subset of resourcse is a subtype: consider $\{ \bar r_1 \} <: \{ \bar r_2 \}$. Any value with type $\{ \bar r_1 \}$ can reduce to any resource literal in $\bar r_1$, so to be compatible with an interface $\{ \bar r_2 \}$, the resource literals in $\bar r_1$ must also be in $\bar r_2$.

These rules let us determine what sort of effects might be incurred when a piece of code is executed. For example, consider $rw = \lambda x: \{ \kwa{File, Socket} \}.~\kwa{x.write}$, which takes either a $\Socket$ or a $\File$ and writes to it. If $rw$ is applied, it could incur $\kwa{Socket.write}$ or $\kwa{File.read}$, depending on what had been passed. In general, there is no way to statically determine what this will be, so the safe approximation is $\{ \kwa{File.write, Socket.write} \}$. This is the approximation given in a judgement like $\vdash rw~\File: \Unit~\kw{with} \{ \kwa{File.write, Socket.write} \}$. A derivation of this judgement is given in Figure\ref{fig:opercalc_tree}. To fit on the page, all resources and operations have been abbreviated to their first letter. A developer who only expects $rw$ to be incurring $\kwa{File.write}$ can typecheck $rw$, see that it could also be writing to $\kwa{Socket}$, and decide it should not be used. If client code was annotated with $\kwa{ \{ \kwa{File.write} \} }$ and tried to use this function, the type system would reject it.

\begin{figure}[h]


    \begin{prooftree*}

    		\Infer0[\textsc{($\varepsilon$-Var)}]{x: \{ \kwa{F}, \kwa{S} \} \vdash x: \{ \kwa{F}, \kwa{S} \}}
    		
    		\Infer1[\textsc{($\varepsilon$-OperCall)}]{x: \{ \kwa{F}, \kwa{S} \} \vdash \kwa{x.w} : \Unit~\kw{with} \{ \kwa{F.w, S.w} \} }
    		
    		\Infer1[\textsc{($\varepsilon$-Abs)}]{ \lambda x: \{ \kwa{F}, \kwa{S} \}.~\kwa{x.w} : \{ \kwa{F, S} \} \rightarrow_{\{\kwa{F.w, S.w}\}} \Unit~\kw{with} \varnothing }
    		
    
       \Infer0[\textsc{($\varepsilon$-Resource)}]{\vdash \kwa{F}: \{ \kwa{F} \}~\kw{with} \varnothing}
    
    		\Infer2[\textsc{($\varepsilon$-App)}]{ \vdash (\lambda x: \{ \kwa{F, S} \}. ~\kwa{x.w})~\kwa{F} : \Unit~\kw{with} \{ \kwa{F.w, S.w} \}  }
    		
 	\end{prooftree*}
 	
\vspace{-12pt}
\caption{Derivation tree for $\vdash rw~\File: \Unit~\kw{with} \{ \kwa{File.write, Socket.write} \}$.}
\label{fig:opercalc_tree}
\end{figure}

\subsubsection{Soundness ($\opercalc$)}~\\

To show the rules of $\opercalc$ are sound requires an appropriate notion of static approximations being safe with respect to the reductions. If a static judgement like $\Gamma \vdash e: \tau~\kw{with} \varepsilon$ were correct, successive reductions on $e$ should never incur effects not in $\varepsilon$. Furthermore, as $e$ is reduced, we learn more about what it is, so approximations on the reduced forms can only get more specific; compare this with how the type of reduced expressions can only get more specific. Adding this to the standard definition of soundness yields the following theorem statement.

\begin{theorem}[$\opercalc$ Single-step Soundness]
If $ \Gamma \vdash  e_A:  \tau_A~\kw{with} \varepsilon_A$ and $ e_A$ is not a value, then $e_A \longrightarrow e_B~|~\varepsilon$, where $ \Gamma \vdash e_B:  \tau_B~\kw{with} \varepsilon_B$ and $ \tau_B <:  \tau_A$ and $\varepsilon_B \cup \varepsilon \subseteq \varepsilon_A$, for some $e, \varepsilon, \tau_B, \varepsilon_B$.
\end{theorem}

Our approach to proving soundness is to show progress and preservation. Noting that the rules for values give $\varnothing$ as their approximate effects, the proof of the progress theorem is routine.

\begin{theorem}[$\opercalc$ Progress]
If $ \Gamma \vdash  e:  \tau~\kw{with} \varepsilon$ and $ e$ is not a value or variable, then $ e \longrightarrow  e'~|~\varepsilon$, for some $e', \varepsilon$.
\end{theorem}

\begin{proof} By induction on derivations of $ \Gamma \vdash  e:  \tau~\kw{with} \varepsilon$.
\end{proof}

To show preservation we need to know that effect safety is preserved by the substitution in \textsc{E-App3}. The semantics are call-by-value, so the name of a function argument is only ever replaced with a value, and we know that the approximate effects of values are $\varnothing$, so the substitution does not introduce more effects. Beyond this observation, the proof is routine.

\begin{theorem}[$\opercalc$ Preservation]
If $\Gamma \vdash e_A: \tau_A~\kw{with} \varepsilon_A$ and $e_A \longrightarrow e_B~|~\varepsilon$, then $\tau_B <: \tau_A$ and $\varepsilon_B \cup \varepsilon \subseteq \varepsilon_A$, for some $e_B, \varepsilon, \tau_B, \varepsilon_B$.
\end{theorem}

\begin{proof}  By induction on the derivations of $\Gamma \vdash e_A: \tau_A~\kw{with} \varepsilon_A$ and $e_A \longrightarrow e_B~|~\varepsilon$.
\end{proof}

Single-step soundness theorem now holds by combining progress and preservation. The soundness of multi-step reductions follows by inducting on the length of a multi-step and appealing to single-step soundness.

\begin{theorem}[$\opercalc$ Single-step Soundness]
If $ \Gamma \vdash  e_A:  \tau_A~\kw{with} \varepsilon_A$ and $ e_A$ is not a value, then $e_A \longrightarrow e_B~|~\varepsilon$, where $ \Gamma \vdash e_B:  \tau_B~\kw{with} \varepsilon_B$ and $ \tau_B <:  \tau_A$ and $\varepsilon_B \cup \varepsilon \subseteq \varepsilon_A$, for some $e_B, \varepsilon, \tau_B, \varepsilon_B$.
\end{theorem}
\begin{proof}
If $ e_A$ is not a value then the reduction exists by the progress theorem. The rest follows by the preservation theorem.
\end{proof}

\begin{theorem}[$\opercalc$ Multi-step Soundness]
If $ \Gamma \vdash  e_A:  \tau_A~\kw{with} \varepsilon_A$ and $e_A \longrightarrow^{*} e_B~|~\varepsilon$, where $\Gamma \vdash e_B: \tau_B~\kw{with} \varepsilon_B$ and $ \tau_B <: \tau_A$ and $\varepsilon_B \cup \varepsilon \subseteq \varepsilon_A$.
\end{theorem}

\begin{proof} By induction on the length of the multi-step reduction.
\end{proof}


\vspace{-0.3cm}
\section{Capability Calculus ($\epscalc$)}
\vspace{-0.3cm}
\label{s:cc}

Allowing a mix of annotated [with effects] and unannotated code helps
reduce the cognitive overhead on developers, allowing them to rapidly
prototype in the unannotated sublanguage and incrementally add
annotations as they are needed. However, reasoning about unannotated
code is difficult in general. Figure \ref{fig:unannotated_reasoning}
demonstrates why: $\kwa{someMethod}$ takes a function $f$ as input and
executes it, but the effects of $f$ depend on its
implementation. Without more information, there is no way to know what
effects might be incurred by $\kwa{someMethod}$.

\begin{figure}
\vspace{-0.8cm}
\begin{lstlisting}
def someMethod(f: Unit $\rightarrow$ Unit):
   f()
\end{lstlisting}
\vspace{-0.5cm}
\caption{What effects can $\kwa{someMethod}$ incur?}
\vspace{-0.5cm}
\label{fig:unannotated_reasoning}
\end{figure}

The key insight of our paper is that capability-safe design can help us:
because the only authority code can exercise is that which is
explicitly given to it, the only capabilities that the unannotated
code can use must be passed into it. If these capabilities are being
passed in from an annotated environment, we know what effects they
capture. These effects are therefore a conservative upper bound on
what can happen in the unannotated code. To demonstrate, consider a
developer who wants to decide whether to use the $\kwa{logger}$
functor in Figure \ref{fig:cc_motivation}. It must be instantiated
with two capabilities, $\kwa{File}$ and $\kwa{Socket}$, and provides
an unannotated function $\kwa{log}$.

\begin{figure}
\begin{lstlisting}
module def logger(f:{File},s:{Socket}):Logger

def log(x: Unit): Unit
   ...
\end{lstlisting}
\vspace{-0.5cm}
\caption{In a capability-safe setting, $\kwa{logger}$ can only exercise authority over the $\kwa{File}$ and $\kwa{Socket}$ capabilities given to it.}
\vspace{-0.5cm}
\label{fig:cc_motivation}
\end{figure}

What effects will be incurred if $\kwa{Logger.log}$ is invoked? One
approach is to manually\footnote{or automatically---but if the
  automation produces an unexpected result we must fall back to manual
  reasoning to understand why.} examine its source code, but this is
tedious and error-prone. In many real-world situations, the source
code may be obfuscated or unavailable. A capability-based argument can
do better: the only authority which $\kwa{Logger}$ can exercise is
that which it has been explicitly given. Here, the $\kwa{Logger}$
requires a $\kwa{File}$ and a $\kwa{Socket}$, so
$\kwa{ \{ \kwa{File.*, Socket.*} \} }$ is an upper bound on the
effects of $\kwa{Logger}$. Knowing $\kwa{Logger}$ could be performing
arbitrary reads and writes to $\kwa{File}$, or arbitrary communication
with the $\kwa{Socket}$, the developer decides this implementation
cannot be trusted and does not use it.

The reasoning we employed only required us to examine the interface of
the unannotated code for the capabilities passed into it. To model
this situation in $\epscalc$, we add a new $\kwa{import}$ expression
that selects what authority $\varepsilon$ the unannotated code may
exercise. In the above example, the expected least authority of
$\kwa{Logger}$ is $\{ \kwa{File.append} \}$, so that is what the
corresponding $\kwa{import}$ would select. The type system can then
check if the capabilities being passed into the unannotated code
exceed its selected authority. If it accepts, then $\varepsilon$
safely approximates the effects of the unannotated code. This is the
key result: when unannotated code is nested inside annotated code,
capability-safety enables us to make a safe inference about its
effects by examining what capabilities are being passed in by the
annotated code.

\vspace{-0.5cm}
\subsection{Grammar ($\epscalc$)}
\vspace{-0.2cm}

The grammar of $\epscalc$ is split into rules for annotated code and
analogous rules for unannotated code. To distinguish the two, we put a
hat above annotated types, expressions, and contexts: $\hat e$,
$\hat \tau$, and $\hat \Gamma$ are annotated, while $e$, $\tau$, and
$\Gamma$ are unannotated. The rules for unannotated programs and their
types are given in Figure
\ref{fig:epscalc_unannotated_grammar}. Unannotated types $\tau$ are
built using $\rightarrow$ and sets of resources $\{ \bar r \}$. An
unannotated context $\Gamma$ maps variables to unannotated
types. Rules for annotated programs and their types are in
Figure \ref{fig:epscalc_annotated_grammar}.

\begin{figure}
\[
\begin{array}{lll}

\begin{array}{lllr}
e & ::= & ~ & exprs: \\
	& | & x & variable \\
	& | & v & value \\
	& | & e ~ e & application \\
	& | & e.\pi & operation \\
	&&\\

v & ::= & ~ & values: \\
	& | & r & resource~literal \\
	& | & \lambda x: \tau.e & abstraction \\
	&&\\
\end{array}

\hspace{5ex}

\begin{array}{lllr}

\tau & ::= & ~ & types: \\
		& | & \{ \bar r \} \\
		& | & \tau \rightarrow \tau \\ 
		&&\\

\Gamma & ::= & ~ & type~ctx: \\
				& | & \varnothing \\
				& | & \Gamma, x: \tau \\
				&&\\
				
\varepsilon & ::= & ~ & effects: \\
		& | & \{ \overline{r.\pi} \} & effect~set \\
		&&\\

				
\end{array}

\end{array}
\]
\vspace{-0.5cm}
\caption{Unannotated programs and types in $\epscalc$.}
\vspace{-0.5cm}
\label{fig:epscalc_unannotated_grammar}
\end{figure}

\begin{figure}
\[
\begin{array}{lll}
7
\begin{array}{lllr}

\hat e & ::= & ~ & labeled~exprs: \\
	& | & x \\
	& | & \hat v \\
	& | & \hat e ~ \hat e \\
	& | & \hat e.\pi \\
	& | & \kwa{import}(\varepsilon_s)~x = \hat e~\kwa{in}~e & import \\
	&&\\

\hat v & ::= & ~ & labeled~values: \\
	& | & r \\
	& | & \lambda x: \hat \tau.\hat e \\
	&&\\

\end{array}

& ~~~~~~~~&

\begin{array}{lllr}

\hat \tau & ::= & ~ & annotated ~types: \\
		& | & \{ \bar r \} \\
		& | & \hat \tau \rightarrow_{\varepsilon} \hat \tau \\
		&&\\

\hat \Gamma & ::= & ~ & annotated~type~ctx:\\
				& | & \varnothing \\
				& | & \hat \Gamma, x: \hat \tau \\
				&&\\

\varepsilon & ::= & ~ & effects: \\
		& | & \{ \overline{r.\pi} \} & effect~set \\
		&&\\

\end{array}

\end{array}
\]
\vspace{-0.5cm}
\caption{Annotated programs and types in $\epscalc$.}
\vspace{-0.5cm}
\label{fig:epscalc_annotated_grammar}
\end{figure}

The new form is $\import{\varepsilon_s}{x}{\hat e}{e}$, modelling the
points at which capabilities are passed from annotated code into
unannotated code. $e$ is the unannotated code. $\hat e$ is the
capability being given to it; we call $\hat e$ an import. For
simplicity, we assume only one capability is being passed into
$e$. $\hat e$ is associated with the name $x$ inside
$e$. $\varepsilon_s$ is the maximum authority that $e$ is allowed to
exercise (its ``selected authority''). As an example, suppose an
unannotated $\kwa{Logger}$, which requires $\kwa{File}$, is expected
to only $\kwa{append}$ to a file, but has an implementation that
writes. This would be modelled by the expression
$\import{\kwa{File.append}}{x}{\kwa{File}}{\lambda y:
  \Unit.~\kwa{x.write}}$.

Observe that $\kwa{import}$ is the only way to mix annotated and
unannotated code, because it is the only situation in which we can say
something interesting about the unannotated code. For the rest of our
discussion on $\epscalc$, we will only be interested in unannotated
code when it is encapsulated by an $\kwa{import}$ expression.

One of the requirements of capability safety is there be no ambient
authority.  This requirement is met by forbidding resource literals
$r$ from being used directly inside an \kwat{import} statement (they
can still be passed in as a capability via the \kwat{import}'s binding
variable $x$).  We could enforce this syntactically, by removing $r$
from the language of unannotated expressions, but we choose to do it
instead using the typing rule for \kwat{import}, given below.

\vspace{-0.2cm}
\subsection{Semantics ($\epscalc$)}
\vspace{-0.2cm}

Reductions are defined on annotated expressions and are natural
reduction rules from extended lambda calculus. We omit them for
brevity. If unannotated code $e$ is wrapped inside annotated code
$\import{\varepsilon_s}{x}{\hat e}{e}$, we transform it into annotated
code by recursively annotating its parts with $\varepsilon_s$. In
practice, it is meaningful to execute purely unannotated code --- but
our only interest is when that code is wrapped inside an
$\kwa{import}$ expression, so we do not bother to give rules for
it. There are two additional rules for reducing $\kwa{import}$
expressions, given in Figure \ref{fig:epscalc_reductions}:
\textsc{E-Import1} reduces the capability being imported, while
\textsc{E-Import2} first annotates $e$ with its selected authority
$\varepsilon$ --- this is $\annot{e}{\varepsilon}$ --- and then
substitutes the import $\hat v$ for its name $x$ in $e$ --- this is
$[\hat v/x]\annot{e}{\varepsilon}$.

\begin{figure}

\fbox{$\hat e \longrightarrow \hat e~|~\varepsilon$}

\[
\begin{array}{c}
\infer[\textsc{(E-Import1)}]
	{\kwa{import}(\varepsilon_s)~x = \hat e~\kw{in} e \longrightarrow \kwa{import}(\varepsilon_s)~x = \hat e'~\kw{in} e~|~\varepsilon'}
	{\hat e \longrightarrow \hat e'~|~\varepsilon'}\\[4ex]

\infer[\textsc{(E-Import2)}]
	{\kwa{import}(\varepsilon_s)~x = \hat v~\kw{in} e \longrightarrow [\hat v/x]\kwa{annot}(e, \varepsilon_s)~|~\varnothing}
	{}

\end{array}
\]
\caption{New single-step reductions in $\epscalc$.}
\label{fig:epscalc_reductions}
\end{figure}

$\annot{e}{\varepsilon}$ produces the expression obtained by
recursively annotating the parts of $e$ with the set of effects
$\varepsilon$. A definition is given in Figure
\ref{fig:annot_defn}. There are versions of $\kwa{annot}$ defined for
expressions and types. Later we shall need to annotate contexts, so
the definition is given here. It is worth mentioning that
$\kwa{annot}$ operates on a purely syntatic level --- nothing prevents
us from annotating a program with something unsafe, so any use of
$\kwa{annot}$ must be justified.

\begin{figure}
\vspace{-0.2cm}

$\kwa{annot} :: e \times \varepsilon \rightarrow \hat e$

\begin{itemize}
	\setlength\itemsep{-0.2em}
	\item[] $\annot{r}{\_} = r$
	\item[] $\annot{\lambda x: \tau_1 . e}{\varepsilon} = \lambda x: \annot{\tau_1}{\varepsilon} . \annot{e}{\varepsilon}$
	\item[] $\annot{e_1~e_2}{\varepsilon} = \kwa{annot}(e_1, \varepsilon)~\kwa{annot}(e_2, \varepsilon)$
	\item[] $\annot{e_1.\pi}{\varepsilon} = \annot{e_1}{\varepsilon}.\pi$
\end{itemize}
	
$\kwa{annot} :: \tau \times \varepsilon \rightarrow \hat \tau$

\begin{itemize}
	\setlength\itemsep{-0.2em}
	\item[] $\annot{\{ \bar r \}}{\_} = \{ \bar r \}$
	\item[] $\annot{\tau_1 \rightarrow \tau_2}{\varepsilon} = \annot{\tau_1}{\varepsilon} \rightarrow_{\varepsilon} \annot{\tau_2}{\varepsilon}$.	
\end{itemize}

$\kwa{annot} :: \Gamma \times \varepsilon \rightarrow \hat \Gamma$

\begin{itemize}
	\setlength\itemsep{-0.2em}
	\item[] $\annot{\varnothing}{\_} = \varnothing$
	\item[] $\annot{\Gamma, x: \tau}{\varepsilon} = \annot{\Gamma}{\varepsilon}, x: \annot{\tau}{\varepsilon}$
\end{itemize}
\vspace{-0.5cm}
\caption{Definition of $\kwa{annot}$.}
\vspace{-0.5cm}
\label{fig:annot_defn}
\end{figure}

\subsection{Static Rules ($\epscalc$)}

A term can be annotated or unannotated, so we need to be able to
recognise when either is well-typed. We do not reason about the
effects of unannotated code directly, so judgements about them have
the form $\Gamma \vdash e: \tau$. Subtyping judgements have the form
$\tau <: \tau$. A summary of the rules for unannotated judgements is
given in Figure \ref{fig:unannotated_static_rules}.

\begin{figure}
\vspace{-5pt}

\fbox{$\Gamma \vdash e: \tau$}

\[
\begin{array}{c}


\infer[\textsc{(T-Var)}]
	{\Gamma, x: \tau \vdash x: \tau}
	{}
\hspace{5ex}
\infer[\textsc{(T-Resource)}]
	{\Gamma, r: \{ r \} \vdash r : \{ r \}}
	{}

\hspace{5ex}
\infer[\textsc{(T-Abs)}]
	{\Gamma \vdash \lambda x: \tau_1.e : \tau_1 \rightarrow \tau_2}
	{\Gamma, x: \tau_1 \vdash e: \tau_2}\\[4ex]
	
\infer[\textsc{(T-App)}]
	{\Gamma \vdash e_1~e_2: \tau_3}
	{\Gamma \vdash e_1: \tau_2 \rightarrow \tau_3 & \Gamma \vdash e_2: \tau_2}
\hspace{5ex}
\infer[\textsc{(T-OperCall)}]
	{\Gamma \vdash e.\pi: \kwa{Unit}}
	{\Gamma \vdash e: \{ \bar r \}}

\end{array}
\]

\fbox{$\tau <: \tau$}

\[
\begin{array}{c}

\infer[\textsc{(S-Arrow)}]
	{ \tau_1 \rightarrow \tau_2 <: \tau_1' \rightarrow \tau_2' }
	{ \tau_1' <: \tau_1 & \tau_2 <: \tau_2' }
\hspace{5ex}
\infer[\textsc{(S-Resources)}]
	{ \{ \bar r_1 \} <: \{ \bar r_2 \} }
	{ \{ \bar r_1 \} \subseteq \{ \bar r_2 \} }

\end{array}
\]

\vspace{-0.3cm}
\caption{(Sub)typing judgements for the unannotated sublanguage of $\epscalc$}
\vspace{-0.3cm}
\label{fig:unannotated_static_rules}
\end{figure}

The annotated static rules are effectively the same only involving the
annotated expressions. The interesting rule is
\textsc{$\varepsilon$-Import}, given in Figure \ref{fig:import_rule},
which gives the type and approximate effects of an $\kwa{import}$
expression. This is the only way to reason about what effects might be
incurred by some unannotated code.

The rule is complicated, so to explain it we shall start with a
simplified version and spend the rest of this section building up to
the final version of \textsc{$\varepsilon$-Import}.

To begin, typing $\import{\varepsilon_s}{x}{\hat e}{e}$ in a context
$\hat \Gamma$ requires us to know that the import $\hat e$ is
well-typed, so we add the premise
$\hat \Gamma \vdash \hat e: \hat \tau~\kw{with} \varepsilon_1$. Since
$x = \hat e$ is an import, it can be used throughout $e$. We do not
want $e$ to exercise authority it hasn't explicitly selected, so
whatever capabilities it uses must be selected by the $\kwa{import}$
expression; therefore, we require that $e$ can be typechecked using
only the binding $x: \hat \tau$. There is a problem though: $e$ is
unannotated and $\hat \tau$ is annotated, and there is no rule for
typechecking unannotated code in an annotated context. To get around
this, we define a function $\kwa{erase}$ in Figure
\ref{fig:erase_defn} which removes the annotations from a type. We
then add $x: \erase{\hat \tau} \vdash e: \tau$ as a premise.

\begin{figure}
$\kwa{erase} :: \hat \tau \rightarrow \tau$
\begin{itemize}
	\setlength\itemsep{-0.2em}
	\item[] $\erase{\{ \bar r \}} = \{ \bar r \}$
	\item[] $\erase{\hat \tau_1 \rightarrow_{\varepsilon} \hat \tau_2} = \erase{\hat \tau_1} \rightarrow \erase{\hat \tau_2}$
\end{itemize}

\vspace{-0.5cm}
\caption{Definition of $\kwa{erase}$.}
\vspace{-0.5cm}
\label{fig:erase_defn}
\end{figure}

Note that, since the environment $\Gamma$ for $e$ has only one binding
(for $x$), it cannot contain any bindings of resource literals---and
the rule \textsc{T-Resource} requires a binding in the environment in
order to type a resource literal in an expression. Typing $e$ in the
restricted environment given by $\kwa{import}$ thus prohibits ambient
authority.

The first version of \textsc{$\varepsilon$-Import} is given in Figure
\ref{fig:import_rule_1}. Since
$\import{\varepsilon_s}{x}{\hat v}{e} \longrightarrow [\hat
v/x]\annot{e}{\varepsilon_s}$ by \textsc{E-Import2}, the ascribed type
is $\annot{\tau}{\varepsilon}$, which is the type of the unannotated
code, annotated with its selected authority $\varepsilon_s$. The
effects of the $\kwa{import}$ are $\varepsilon_1 \cup \varepsilon_s$
--- the former comes from reducing the imported capability, which
happens before the body of the $\kwa{import}$ is annotated and
executed, and the latter contains all the effects which the
unannotated code might incur.

\begin{figure}[h]
\vspace{-0.5cm}
\[
\begin{array}{c}

\infer[\textsc{($\varepsilon$-Import1-Bad)}]
	{ \hat \Gamma \vdash \import{\varepsilon_s}{x}{\hat e}{e}: \kwa{annot}(\tau, \varepsilon_s)~\kw{with} \varepsilon_s \cup \varepsilon_1 }
	{ \hat \Gamma \vdash \hat e: \hat \tau~\kw{with} \varepsilon_1 & x: \kwa{erase}(\hat \tau) \vdash e: \tau }

\end{array}
\]
\vspace{-0.5cm}
\caption{A first (incorrect) rule for type-and-effect checking $\kwa{import}$ expressions.}
\vspace{-0.5cm}
\label{fig:import_rule_1}
\end{figure}

At the moment there is no relation between the selected authority
$\varepsilon$ and those effects captured by the imported capability
$\hat e$. Consider
$\hat e' = \import{\varnothing}{x}{\File}{\kwa{x.write}}$, which
imports a $\File$ and writes to it, but declares its authority as
$\varnothing$. According to \textsc{$\varepsilon$-Import1},
$\vdash \hat e': \Unit~\kw{with} \varnothing$, but this is clearly
wrong since $\hat e'$ writes to $\kwa{File}$.  An $\kwa{import}$
should only be well-typed if the capability being imported only
captures effects contained in the unannotated code's selected
authority $\varepsilon$. In this case, $\kwa{File}$ captures
$\kwa{ \{ File.* \}}$, which is not contained in the selected
authority $\varnothing$, so it should be rejected for that reason.  To
this end we define a function $\kwa{effects}$, which collects the set
of effects that an annotated type captures. A first (but not yet
correct) definition is given in Figure \ref{fig:fx_defn}. We can then
add the premise $\kwa{effects}(\hat \tau) \subseteq \varepsilon_s$ to
require that any imported capability must not capture authority beyond
that selected in $\varepsilon_s$. The updated rule is given in Figure
\ref{fig:import_rule_2}.

\begin{figure}

$\kwa{effects} :: \hat \tau \rightarrow \varepsilon$
\begin{itemize}
	\setlength\itemsep{-0.2em}
	\item[] $\fx{\{ \bar r \}} = \{ r.\pi \mid r \in \bar r, \pi \in \Pi \}$
	\item[] $\fx{\hat \tau_1 \rightarrow_{\varepsilon} \hat \tau_2} = \fx{\hat \tau_1} \cup \varepsilon \cup \fx{\hat \tau_2}$
\end{itemize}
\vspace{-0.3cm}
\caption{A first (incorrect) definition of $\kwa{effects}$.}
\vspace{-0.3cm}
\label{fig:fx_defn}
\end{figure}

\begin{figure}

\[
\begin{array}{c}

\infer[\textsc{($\varepsilon$-Import2-Bad)}]
	{ \hat \Gamma \vdash \import{\varepsilon_s}{x}{\hat e}{e}: \kwa{annot}(\tau, \varepsilon_s)~\kw{with} \varepsilon \cup \varepsilon_1 }
	{ \hat \Gamma \vdash \hat e: \hat \tau~\kw{with} \varepsilon_1 & x: \kwa{erase}(\hat \tau) \vdash e: \tau & \kwa{effects}(\hat \tau) \subseteq \varepsilon_s}

\end{array}
\]
\vspace{-0.3cm}
\caption{A second (still incorrect) rule for type-and-effect checking $\kwa{import}$ expressions.}
\vspace{-0.5cm}
\label{fig:import_rule_2}
\end{figure}

The counterexample from before is now rejected by
\textsc{$\varepsilon$-Import2}, but there are still issues: the
annotations on one import can be broken by another import. To
illustrate, consider Figure \ref{fig:rule_import2_counterexample}
where two\footnote{Our formalisation only permits a single capability
  to be imported, but this discussion leads to a generalisation needed
  for the rules to be safe when multiple capabilities can be imported.
  In any case, importing multiple capabilities can be handled with an
  encoding of pairs.} capabilities are imported. This program imports
a function $\kwa{go}$ which, when given a
$\Unit \rightarrow_{\varnothing} \Unit$ function with no effects, will
execute it. The other import is $\kwa{File}$. The unannotated code
creates a $\Unit \rightarrow \Unit$ function which writes to
$\kwa{File}$ and passes it to $\kwa{go}$, which subsequently incurs
$\kwa{File.write}$.

\begin{figure}[h]
\vspace{-0.5cm}

\begin{lstlisting}
import({File.*})
   go = $\lambda$x: Unit $\rightarrow_{\varnothing}$ Unit. x unit
   f = File
in
   go ($\lambda$y: Unit. f.write)

\end{lstlisting}

\vspace{-0.5cm}
\caption{Permitting multiple imports will break \textsc{$\varepsilon$-Import2}.}
\vspace{-0.5cm}
\label{fig:rule_import2_counterexample}
\end{figure}

In the world of annotated code it is not possible to pass a
file-writing function to $\kwa{go}$, but because the judgement
$x: \erase{\hat \tau} \vdash e: \tau$ discards the annotations on
$\kwa{go}$, and since the file-writing function has type
$\unit \rightarrow \unit$, the unannotated world accepts it. The
approximation is actually safe at the top-level, because the
$\kwa{import}$ selects $\{ \kwa{File.*} \}$, which contains
$\kwa{File.write}$ --- but it contains code that violates the type
signature of $\kwa{go}$. We want to prevent this.

If $\kwa{go}$ had the type
$\Unit \rightarrow_{\{ \kwa{File.write} \}} \Unit$ the above example
would be safe, but a modified version where a file-reading function is
passed to $\kwa{go}$ would have the same issue. $\kwa{go}$ is only
safe when it expects every effect that the unannotated code might pass
to it: if $\kwa{go}$ had the type
$\Unit \rightarrow_{\{ \kwa{File.*} \}} \Unit$, then the unannotated
code cannot pass it a capability with an effect it isn't already
expecting, so the annotation on $\kwa{go}$ cannot be
violated. Therefore, we require imported capabilities to have
authority to incur the effects in $\varepsilon$. To achieve greater
control in how we say this, the definition of $\kwa{effects}$ is split
into two separate functions called $\kwa{effects}$ and
$\kwa{ho \hyphen effects}$. The latter is for higher-order effects,
i.e. the effects that are not captured within a function, but rather
are possible because of what it is passed as an argument. If values of
$\hat \tau$ possess a capability that can be used to incur the effect
$r.\pi$, then $r.\pi \in \fx{\hat \tau}$. If values of $\hat \tau$ can
incur an effect $r.\pi$, but need to be given the capability (as a
function argument) by someone else in order to do it, then
$r.\pi \in \hofx{\hat \tau}$. Definitions are given in Figure
\ref{fig:fx_defns}.

\begin{figure}
\vspace{-0.5cm}

$\kwa{effects} :: \hat \tau \rightarrow \varepsilon$

\begin{itemize}
	\setlength\itemsep{-0.2em}
	\item[] $\fx{\{ \bar r \}} = \{ r.\pi \mid r \in \bar r, \pi \in \Pi \}$
	\item[] $\fx{\hat \tau_1 \rightarrow_{\varepsilon} \hat \tau_2} = \hofx{\hat \tau_1} \cup \varepsilon \cup \fx{\hat \tau_2}$
\end{itemize}

$\kwa{ho \hyphen effects} :: \hat \tau \rightarrow \varepsilon$

\begin{itemize}
	\setlength\itemsep{-0.2em}
	\item[] $\hofx{\{ \bar r \}} = \varnothing$
	\item[] $\hofx{\hat \tau_1 \rightarrow_{\varepsilon} \hat \tau_2} = \fx{\hat \tau_1} \cup \hofx{\hat \tau_2}$
\end{itemize}

\vspace{-0.5cm}
\caption{Effect functions (corrected).}
\vspace{-0.5cm}
\label{fig:fx_defns}
\end{figure}

Both $\kwa{effects}$ and $\kwa{ho \hyphen effects}$ are mutually recursive,
with base cases for resource types. Any effect can be directly
incurred by a resource on itself, hence
$\fx{\{ \bar r \}} = \{ r.\pi \mid r \in \bar r, \pi \in \Pi \}$. A
resource cannot be used to indirectly invoke some other effect, so
$\hofx{\{ \bar r \}} = \varnothing$. The mutual recursion echoes the
subtyping rule for functions. Recall that functions are contravariant
in their input type and covariant in their output; likewise, both
functions recurse on the input-type using the other function, and
recurse on the output-type using the same function.

In light of these new definitions, we still require
$\fx{\hat \tau} \subseteq \varepsilon_s$ --- unannotated code must
select any effect its capabilities can incur --- but we add a new
premise $\varepsilon_s \subseteq \hofx{\hat \tau}$, stipulating that
imported capabilities must know about every effect they could be given
by the unannotated code (which is at most $\varepsilon$). The
counterexample from Figure \ref{fig:rule_import2_counterexample} is
now rejected, because
$\hofx{(\Unit \rightarrow_{\varnothing} \Unit)
  \rightarrow_{\varnothing} \Unit} = \varnothing$, but
$\{ \kwa{File.*} \} \not\subseteq \varnothing$.  However, this is
\textit{still} not sufficient! Consider
$\varepsilon_s \subseteq \hofx{ \hat \tau_1 \rightarrow_{\varepsilon'}
  \hat \tau_2 }$. We want \textit{every} higher-order capability
involved to be expecting $\varepsilon_s$. Expanding the definition of
$\kwa{ho \hyphen effects}$, this is the same as
$\varepsilon_s \subseteq \fx{\hat \tau_1} \cup \hofx{\hat
  \tau_2}$. Let $r.\pi \in \varepsilon_s$ and suppose
$r.\pi \in \fx{\hat \tau_1}$, but $r.\pi \notin \hofx{\hat
  \tau_2}$. Then
$\varepsilon_s \subseteq \fx{\hat \tau_1} \cup \hofx{\hat \tau_2}$ is
still true, but $\hat \tau_2$ is not expecting $r.\pi$. Unannotated
code could then violate the annotations on $\hat \tau_2$ by passing it
a capability for $r.\pi$, using the same trickery as before. The cause
of the issue is that $\subseteq$ does not distribute over $\cup$. We
want a relation like
$\varepsilon_s \subseteq \fx{\hat \tau_1} \cup \hofx{\hat \tau_2}$,
which also implies $\varepsilon_s \subseteq \fx{\hat \tau_1}$ and
$\varepsilon_s \subseteq \fx{\hat \tau_2}$. Figure
\ref{fig:safe_defns} defines this: $\kwa{safe}$ is a distributive
version of $\varepsilon_s \subseteq \fx{\hat \tau}$ and
$\kwa{ho \hyphen safe}$ is a distributive version of
$\varepsilon_s \subseteq \hofx{\hat \tau}$.

\begin{figure}

\noindent
$\fbox{$\safe{\hat \tau}{\varepsilon}$}$

\[
\begin{array}{c}

\infer[\textsc{(Safe-Resource)}]
	{ \kwa{safe}(\{ \bar r \}, \varepsilon) }
	{} 
\hspace{5ex}
	
\infer[\textsc{(Safe-Arrow)}]
	{\kwa{safe}(\hat \tau_1 \rightarrow_{\varepsilon'} \hat \tau_2, \varepsilon)}
	{\varepsilon \subseteq \varepsilon' & \kwa{ho \hyphen safe}(\hat \tau_1, \varepsilon) & \kwa{safe}(\hat \tau_2, \varepsilon)} \\[3ex]

\end{array}
\]

\noindent
$\fbox{$\hosafe{\hat \tau}{\varepsilon}$}$

\[
\begin{array}{c}

\infer[\textsc{(HOSafe-Resource)}]
	{ \kwa{ho \hyphen safe}( \{ \bar r \}, \varepsilon)} 
	{}
\hspace{5ex}

\infer[\textsc{(HOSafe-Arrow)}]
	{ \kwa{ho \hyphen safe}( \hat \tau_1 \rightarrow_{\varepsilon'} \hat \tau_2, \varepsilon ) }
	{ \kwa{safe}(\hat \tau_1, \varepsilon)  & \kwa{ho \hyphen safe}(\hat \tau_2, \varepsilon) }\\[3ex]

\end{array}
\]

\vspace{-0.5cm}
\caption{Safety judgements in $\epscalc$.}
\vspace{-0.5cm}
\label{fig:safe_defns}
\end{figure}

\begin{figure}
\vspace{-0.5cm}

\[
\begin{array}{c}

\infer[\textsc{($\varepsilon$-Import3-Bad)}]
	{ \hat \Gamma \vdash \kwa{import}(\varepsilon_s)~x = \hat e~\kw{in} e: \kwa{annot}(\tau, \varepsilon_s)~\kw{with} \varepsilon \cup \varepsilon_1 }
{{\def\arraystretch{1.4}
  \begin{array}{c}
\hat \Gamma \vdash \hat e: \hat \tau~\kw{with} \varepsilon_1
~~~~~~
\kwa{effects}(\hat \tau) \subseteq \varepsilon_s \\
\hosafe{\hat \tau}{\varepsilon_s} ~~~~~~ x: \kwa{erase}(\hat \tau) \vdash e: \tau
  \end{array}}} 
 
\end{array}
\]

\vspace{-0.2cm}
\caption{A third (still incorrect) rule for type-and-effect checking $\kwa{import}$ expressions.}
\vspace{-0.8cm}
\label{fig:import_rule3}
\end{figure}

An amended version of \textsc{$\varepsilon$-Import} is given in Figure
\ref{fig:import_rule3}. It contains a new premise
$\hosafe{\hat \tau}{\varepsilon_s}$ which formalises the notion that
every capability which could be given to a value of $\hat \tau$ --- or
any of its constituent pieces --- must be expecting the effects
$\varepsilon_s$ it might be given by the unannotated code.
The premises so far restrict what authority can be selected by
unannotated code, but what about authority passed as a function
argument? Consider the example
$\hat e = \import{\varnothing}{x}{\unit}{\lambda f: { \File
  }.~\kwa{f.write}}$. The unannotated code selects no capabilities and
returns a function which, when given $\kwa{File}$, incurs
$\kwa{File.write}$. This satisfies the premises in
\textsc{$\varepsilon$-Import3}, but its annotated type is
$\{ \File \} \rightarrow_{\varnothing} \Unit$: not good!

Suppose the unannotated code defines a function $f$, which gets
annotated with $\varepsilon_s$ to produce
$\annot{f}{\varepsilon_s}$. Suppose $\annot{f}{\varepsilon_s}$ is
invoked at a later point in the annotated world and incurs the effect
$r.\pi$. What is the source of $r.\pi$? If $r.\pi$ was selected by the
$\kwa{import}$ expression surrounding $f$, it is safe for
$\annot{f}{\varepsilon_s}$ to incur this effect. Otherwise,
$\annot{f}{\varepsilon_s}$ may have been passed an argument which can
be used to incur $r.\pi$, in which case $r.\pi$ is a higher-order
effect of $\annot{f}{\varepsilon_s}$. If the argument is a function,
then $r.\pi \in \varepsilon_s$ by the soundness of our calculus (or it
would not typecheck). If the argument is a resource $r$, then
$\annot{f}{\varepsilon_s}$ may exercise $r.\pi$ without declaring it
--- this is the case we do not yet account for.

We want $\varepsilon_s$ to contain every effect captured by resources
passed into $\annot{f}{\varepsilon_s}$ as arguments. We can do this by
inspecting the (unannotated) type of $f$ for resource sets. For
example, if the unannotated type is
$\kwa{ \{ File \} \rightarrow \Unit}$, then we need
$\kwa{ \{ File.* \} }$ in $\varepsilon_s$. To do this, we add a new
premise $\hofx{\annot{\tau}{\varnothing}} \subseteq
\varepsilon_s$. $\kwa{ho \hyphen effects}$ is only defined on
annotated types, so we first annotate $\tau$ with $\varnothing$. We
are only inspecting the resources passed into $f$ as arguments, so the
annotations are not relevant -- annotating $\tau$ with $\varnothing$
is therefore a good choice. We can now handle the example from
before. The unannotated code types via the judgement
$x: \Unit \vdash \lambda f: \{ \File \}.~\kwa{f.write}: \{ \File \}
\rightarrow \Unit$. Its higher-order effects are
$\hofx{\annot{ \{ \File \} \rightarrow \Unit}{\varnothing}} = \{
\kwa{File.*} \}$, but $\{ \kwa{File.*} \} \not\subseteq \varnothing$,
so the example is safely rejected.

The final version of \textsc{$\varepsilon$-Import} is given in Figure
\ref{fig:import_rule}. With it, we can now model the example from the
beginning of this section, where the $\kwa{Logger}$ selects the
$\kwa{File}$ capability and exposes an unannotated function
$\kwa{log}$ with type $\Unit \rightarrow \Unit$ and implementation
$e$. The expected least authority of $\kwa{Logger}$ is
$\{ \kwa{File.append} \}$, so its corresponding $\kwa{import}$
expression would be
$\import{\kwa{File.append}}{f}{\kwa{File}}{\lambda x: \Unit.~e}$. The
imported capability is $ f = \kwa{File}$, and
$\fx{\{\File\}} = \{ \kwa{File.*} \} \not\subseteq \{
\kwa{File.append} \}$, so this example is safely rejected:
$\kwa{Logger.log}$ has authority to do anything with $\kwa{File}$, and
its implementation $e$ might be violating its stipulated least
authority $\{ \kwa{File.append} \}$.

\vspace{-0.5cm}
\subsection{Soundness ($\epscalc$)} 
\vspace{-0.5cm}

Only annotated programs can be reduced and have their effects
approximated, so the soundness theorem only applies to annotated
judgements. Its statement is given below.

\begin{figure}[t]

\[
\begin{array}{c}

\infer[\textsc{($\varepsilon$-Import)}]
	{ \hat \Gamma \vdash \kwa{import}(\varepsilon_s)~x = \hat e~\kw{in} e: \kwa{annot}(\tau, \varepsilon_s)~\kw{with} \varepsilon_s \cup \varepsilon_1 }
{{\def\arraystretch{1.4}
  \begin{array}{c}
\kwa{effects}(\hat \tau) \cup \hofx{\annot{\tau}{\varnothing}}\subseteq \varepsilon_s \\
\hat \Gamma \vdash \hat e: \hat \tau~\kw{with} \varepsilon_1 ~~~~~~ \kwa{ho \hyphen safe}(\hat \tau, \varepsilon_s) ~~~~~~ x: \kwa{erase}(\hat \tau) \vdash e: \tau
  \end{array}}} 
 
\end{array}
\]

\vspace{-0.2cm}
\caption{The final rule for typing imports.}
\vspace{-0.5cm}
\label{fig:import_rule}
\end{figure}

\begin{theorem}[$\epscalc$ Single-step Soundness]
If $\hat \Gamma \vdash \hat e_A: \hat \tau_A~\kw{with} \varepsilon_A$ and $\hat e_A$ is not a value, then $\hat e_A \longrightarrow \hat e_B~|~\varepsilon$, where $\hat \Gamma \vdash \hat e_B: \hat \tau_B~\kw{with} \varepsilon_B$ and $\hat \tau_B <: \hat \tau_A$ and $\varepsilon_B \cup \varepsilon \subseteq \varepsilon_A$, for some $\hat e_B, \varepsilon, \hat \tau_B, \varepsilon_B$.
\end{theorem}

Due to small page limit we refer the interested readers to the Technical Report with complete proofs including multi-step soundness for the system presented here~\cite{ecs:2018:aaron-tr}.
\section{Desugaring}

In this section we develop notation and techniques so our calculi can express the practical examples of the next section. To do this we show how to encode $\unit$ and $\kwa{let}$ in $\epscalc$, make some simplifying assumptions, and show how to express the Wyvern examples in $\epscalc$.

\subsection{Unit, Let}

The $\unit$ literal is defined as $\unit \defn \lambda x: \varnothing.~x$. It is the same in both annotated and unannotated code. In annotated code, it has the type $\Unit \defn \varnothing \rightarrow_{\varnothing} \varnothing$, while in unannotated code it has the type $\Unit \defn \varnothing \rightarrow \varnothing$. These are technically two separate types, but we will not distinguish between them. Note that $\unit$ is a value, and because $\varnothing$ is uninhabited (there is no empty resource literal), $\unit$ cannot be applied to anything. Furthermore, $\vdash \unit: \Unit~\kw{with} \varnothing$ by \textsc{$\varepsilon$-Abs}, and $\vdash \unit: \Unit$ by \textsc{T-Abs}. We use $\Unit$ to represent the absence of information, such as when a function takes no input or returns no value

The expression $\letxpr{x}{\hat e_1}{\hat e_2}$ reduces $\hat e_1$ to a value $\hat v_1$, binds it to the name $x$ in $\hat e_2$, and then executes $[\hat v_1/x]\hat e_2$. If $\hat \Gamma \vdash \hat e_1: \hat \tau_1~\kw{with} \varepsilon_1$, then $\letxpr{x}{\hat e_1}{\hat e_2} \defn (\lambda x: \hat \tau_1 . \hat e_2) \hat e_1$\footnote{We could also define an unannotated version of $\kwa{let}$, but we only need the annotated version}. If $\hat e_1$ is a non-value, we can reduce the $\kwa{let}$ by \textsc{E-App2}. If $\hat e_1$ is a value, we may apply \textsc{E-App3}, which binds $\hat e_1$ to $x$ in $\hat e_2$. $\kwa{let}$ expressions can be typed using \textsc{$\varepsilon$-App}.

\subsection{Modules}

Wyvern's modules are first-class, desugaring into objects --- invoking a module's function is no different from invoking an object's method. There are two kinds of modules: pure and resourceful. For our purposes, a pure module is one with no (transitive) authority over any resources, while a resource module has (transitive) authority over some resource. A pure module may still be given a capability, for example as a function argument, but it may not possess or capture the capability for longer than the duration of the method call. Figure \ref{fig:wyv_modules} shows an example of two modules, one pure and one resourceful, each declared in a separate file. Pure modules are declared with the $\kwa{module}$ keyword, while resource modules are declared with $\kwa{resource~module}$.

\begin{figure}[h]

\begin{lstlisting}
module PureMod

def tick(f: {File}):Unit with {File.append}
   f.append

\end{lstlisting}

\begin{lstlisting}
resource module ResourceMod
require File

def tick():Unit with {File.append}
   File.append
\end{lstlisting}

\caption{Definition of two modules, one pure and the other resourceful.}
\label{fig:wyv_modules}
\end{figure}

Resource modules, like objects, must be instantiated. When they are instantiated they are given the capabilities they require. In Figure \ref{fig:wyv_modules}, $\kwa{ResourceMod}$ requests the use of a $\kwa{File}$ capability. Figure \ref{fig:wyv_module_instantiation} demonstrates how the two modules above would be instantiated and used. To prevent infinite regress the $\kwa{File}$ must, at some point, be introduced into the program. This happens in a special main module. When the program begins execution, the $\kwa{File}$ capability is passed into the program from the system environment. $\kwa{Main}$ then instantiates all the other modules in the program with their capabilities. If a module is annotated, its function signatures will have effect annoations. For example, in Figure \ref{fig:wyv_modules}, $\kwa{PureMod.tick}$ has the $\kwa{File.append}$ annotation, meaning it should typecheck as $\kwa{ \{ File \} \rightarrow_{\{\kwa{File.append}\}} \Unit }$. Both $\kwa{PureMod}$ and $\kwa{ResourceMod}$ are annotated. 


\begin{figure}[h]

\begin{lstlisting}
resource module Main
require File
instantiate PureMod
instantiate ResourceMod(File)

PureMod.tick(File)
\end{lstlisting}

\caption{The $\kwa{Main}$ module which instantiates $\kwa{PureMod}$ and $\kwa{ResourceMod}$ and then invokes $\kwa{PureMod.tick}$.}
\label{fig:wyv_module_instantiation}
\end{figure}

Our Wyvern examples are simplified in several ways so they can be expressed in $\epscalc$. The only objects used are modules. The modules only ever contain one function and the capabilities they require; they have no mutable fields. There are no self-referencing modules or recursive functions. Modules do not reference each other cyclically. These simplifications enable us to model each module as a function. Applying the function will be equivalent to applying the single function defined by the module. A collection of modules is desugared into $\epscalc$ as follows. First, a sequence of let-bindings are used to name constructor functions which, when given the capabilities requested by a module, will return the function representing an instance of that module. The constructor for $\kwa{M}$ is called $\kwa{MakeM}$. If the module does not require any capabilities it takes $\Unit$ as its argument. A function is then defined which represents the body of code in the $\kwa{Main}$ module. When invoked, this function will instantiate all the modules by invoking their constructors and execute the code in $\kwa{Main}$. Finally, the function representing $\kwa{Main}$ is invoked with the primitive capabilities that are passed from the system environment into $\kwa{Main}$.

Figure \ref{fig:wyv_tutorial_desugaring} shows how the examples above desugar. Lines 1-3 define the constructor for $\kwa{PureMod}$. Since $\kwa{PureMod}$ requires no capabilities, the constructor takes $\Unit$ as an argument on line 2. Lines 5-7 define the constructor for $\kwa{ResourceMod}$. It requires a $\kwa{File}$ capability, so the constructor takes $\kwa{\{File\}}$ as its input type on line 6. The constructor for $\kwa{Main}$ is defined on lines 9-14, which instantiates the other modules and runs the code inside $\kwa{Main}$. Line 16 starts execution by invoking $\kwa{MakeMain}$ with the initial set of capabilities, which in this case is just $\kwa{File}$.

\begin{figure}[h]

\begin{lstlisting}
let MakePureMod =
   $\lambda$x:Unit.
      $\lambda$f:{File}. f.append in

let MakeResourceMod =
   $\lambda$f:{File}.
      $\lambda$x:Unit. f.append in

let MakeMain =
   $\lambda$f:{File}.
      $\lambda$x: Unit.
         let PureMod = (MakePureMod unit) in
         let ResourceMod = (MakeResourceMod f) in
         (ResourceMod unit) in

(MakeMain File) unit
\end{lstlisting}

\caption{Desugaring of $\kwa{PureMod}$ and $\kwa{ResourceMod}$ into $\epscalc$.}
\label{fig:wyv_tutorial_desugaring}
\end{figure}

When an unannotated module is translated into $\epscalc$, the desugared contents will be encapsulated with an $\kwa{import}$ expression. The selected authority on the $\kwa{import}$ expression will be that we expect of the unannotated code according to the principle of least authority in the particular example under consideration. For example, if the client only expects the unannotated code to have the $\kwa{File.append}$ effect, the corresponding $\kwa{import}$ expression will select $\kwa{\{File.append\}}$.

\chapter{Applications}

In this chapter we show how $\epscalc$ can be used in practice, and show how its rules can enable effect reasoning in existing capability-safe languages. This will take the form of writing a program in a high-level, capability-safe language, translating it to an equivalent $\epscalc$ program, and demonstrating how the rules of $\epscalc$ enable reasoning about the use of effects.

In this section the high-level programs will be written in a version of Wyvern. Wyvern is a pure, object-oriented, capability-safe language. It has a first-class module system, in which modules and objects are treated uniformly. Although $\epscalc$ does not have objects, the example Wyvern programs can be expressed using functions. This does not mean the examples given aren't demonstrating useful and realistic situations --- we simply do not need the added expressiveness given by self-referential objects.

In section 4.1. we discuss how the translation from Wyvern to $\epscalc$ will work, and what simplifying assumptions are made in our examples. This also serves as a gentle introduction to Wyvern's syntax. A variety of scenarios are then explored in 4.2. to show how the rules of $\epscalc$ can help developers in practice.

\section{Translations and Encodings}

Our aim is to develop some notation to help us translate Wyvern programs into $\epscalc$. Our approach will be to encode these additional rules and forms into the base language of $\epscalc$; essentially, to give common patterns and forms a short-hand, so they can be easily named and recalled. This is called \textit{sugaring}. When these derived forms are collapsed into their underlying representation, it is called \textit{desugaring}. We are going to introduce several rules to show a Wyvern program might be considered syntactic sugar for an $\epscalc$ program, and translate examples by desugaring according to our rules.

\subsection{Unit}

$\kwa{Unit}$ is a type inhabited by exactly one value. It conveys the absence of information; in $\epscalc$ an operation call on a resource literal reduces to $\unit$ for this reason. We define $\unit \defn \lambda x: \varnothing. x$. The $\unit$ literal is the same in both annotated and naked code. In annotated code, it has the type $\Unit \defn \varnothing \rightarrow_{\varnothing} \varnothing$, while in naked code it has the type $\Unit \defn \varnothing \rightarrow \varnothing$. While these are technically two seperate types, we will not distinguish between the annotated and naked versions, simply referring to them both as $\Unit$.

Note that $\unit$ is a value, and because $\varnothing$ is uninhabited (there is no empty resource literal), $\unit$ cannot be applied to anything. Furthermore, $\vdash \unit: \Unit~\kw{with} \varnothing$ by \textsc{$\varepsilon$-Abs}, and $\vdash \unit: \Unit$ by \textsc{T-Abs}. This leads to the derived rules in \ref{fig:unit_rules}.

\begin{figure}[h]


\fbox{$\Gamma \vdash e: \tau$} \\
\fbox{$\hat \Gamma \vdash \hat e: \hat \tau~\kw{with} \varepsilon$}


\[
\begin{array}{c}

\infer[\textsc{(T-Unit)}]
	{\Gamma \vdash \unit : \Unit}
	{} ~~~~

\infer[(\textsc{$\varepsilon$-Unit})]
	{\hat \Gamma \vdash \unit : \Unit~\kw{with} \varnothing}
	{}

\end{array}
\]

\caption{Derived $\kwa{Unit}$ rules.}
\label{fig:unit_rules}
\end{figure}

Since $\unit$ represents the absence of information, we also use it as the type when a function either takes no argument, or returns nothing. \ref{fig:unit_sugaring} shows the definition of a Wyvern function which takes no argument and returns nothing, and its corresponding representation in $\epscalc$.

\begin{figure}[h]

\begin{lstlisting}
def method():Unit
   unit
\end{lstlisting}

\begin{lstlisting}
$\lambda$x:Unit. unit
\end{lstlisting}

\caption{Desugaring of functions which take no arguments or return nothing.}
\label{fig:unit_sugaring}
\end{figure}

\subsection{Let}

\noindent
The expression $\letxpr{x}{\hat e_1}{\hat e_2}$ first binds the value $\hat e_1$ to the name $x$ and then evaluates $\hat e_2$. We can generalise by allowing $\hat e_1$ to be a non-value, in which case it must first be reduced to a value. If $\Gamma \vdash \hat e_1: \hat \tau_1$, then $\letxpr{x}{\hat e_1}{\hat e_2} \defn (\lambda x: \hat \tau_1 . \hat e_2) \hat e_1$. Note that if $\hat e_1$ is a non-value, we can reduce the $\kwa{let}$ by \textsc{E-App2}. If $\hat e_1$ is a value, we may apply \textsc{E-App3}, which binds $\hat e_1$ to $x$ in $\hat e_2$. This is fundamentally a lambda application, so it can be typed using \textsc{$\varepsilon$-App} (or \textsc{T-App}, if the terms involved are unlabelled). The new rules in \ref{fig:let_rules} capture these derivations.

\begin{figure}[h]

\fbox{$\Gamma \vdash e: \tau$} \\
\fbox{$\hat \Gamma \vdash \hat e: \hat \tau~\kw{with} \varepsilon$} \\
\fbox{$\hat e \rightarrow \hat e ~|~ \varepsilon$}

\[
\begin{array}{c}

	~~~
	
	\infer[\textsc{($\varepsilon$-Let)}]
	{\Gamma \vdash \letxpr{x}{e_1}{e_2}: \tau_2}
	{\Gamma \vdash e_1: \tau_1 & \Gamma, x: \tau_1 \vdash e_2: \tau_2} \\[2ex]

\infer[\textsc{($\varepsilon$-Let)}]
	{\hat \Gamma \vdash \letxpr{x}{\hat e_1}{\hat e_2} : \hat \tau_2~\kw{with} \varepsilon_1 \cup \varepsilon_2}
	{\hat \Gamma \vdash \hat e_1 : \hat \tau_1~\kw{with} \varepsilon_1 & \hat \Gamma, x: \hat \tau_1 \vdash \hat e_2: \hat \tau_2~\kw{with} \varepsilon_2} \\[2ex]
	
\infer[\textsc{(E-Let1)}]
	{\letxpr{x}{\hat e_1}{\hat e_2} \longrightarrow \letxpr{x}{\hat e_1'}{\hat e_2}~|~\varepsilon_1}
	{\hat e_1 \longrightarrow \hat e_1'~|~\varepsilon_1} \\[2ex]
	
\infer[\textsc{(E-Let2)}]
	{\letxpr{x}{\hat v}{\hat e} \longrightarrow [\hat v/x]\hat e~|~\varnothing}
	{} 

\end{array}
\]

\caption{Derived $\kwa{let}$ rules.}
\label{fig:let_rules}
\end{figure}

$\kwa{let}$ expressions can be used to sequence computations. Intuitively, the $\kwa{let}$ expression simply names the results of the intemediate steps and then ignores them in its body. When we ignore the result of a computation we shall bind it to $\_$ instead of a real name, to suggest the result isn't important and prevent the naming of unused variables. \ref{fig:let_rules} shows how this is done.

\begin{figure}[h]

\begin{lstlisting}
def method(f: {File}):Unit with {File.open, File.write, File.close}
   f.open
   f.write(``hello, world!'')
   f.close
\end{lstlisting}

\begin{lstlisting}
$\lambda$f: {File}.
   let _ = f.open in
   let _ = f.write in
   f.close
\end{lstlisting}

\caption{Desugaring of a sequence of computations.}
\label{fig:let_rules}
\end{figure}

\subsection{Modules and Objects}

Wyvern's modules are first-class and desugar into objects; invoking a method inside a module is no different from invoking an object's method. There are two kinds of modules: pure and resourceful. For our purposes, a pure module is one with no (transitive) authority over any resources, while a resource module has (transitive) authority over some resource. A pure module may still be given a capability, for example by requesting it in a function signature, but it may not possess or capture the capability for longer than the duration of the method call. \ref{fig:wyv_modules} shows an example of two modules, one pure and one resourceful, each declared in a seperate file. Note how pure modules are declared with the $\kwa{module}$ keyword, while resource modules are declared with the $\kwa{resource~module}$ keywords.

\begin{figure}[h]

\begin{lstlisting}
module PureMod

def tick(f: {File}):Unit
   f.append

\end{lstlisting}

\begin{lstlisting}
resource module ResourceMod
require File

def tick():Unit with {File.append}
   File.append
\end{lstlisting}

\caption{Definition of two modules, one pure and the other resourceful.}
\label{fig:wyv_modules}
\end{figure}

Wyvern is capability-safe, so resource modules must be instantiated with the capabilities they require. In \ref{fig:wyv_modules}, $\kwa{ResourceMod}$ requests the use of a $\kwa{File}$ capability, which must be supplied to it from someone already possessing it. Modules are behaving like objects in this way, because they require explicit instantiation. \ref{fig:wyv_module_instantiation} demonstrates how the two modules above would be instantiated and used.

To prevent infinite regress the $\kwa{File}$ must, at some point, be introduced into the program. This happens in a special main module. When the program begins execution, the $\kwa{File}$ capability is passed into the program from the system environment. All these initial capabilities are modelled in $\epscalc$ as resource literals. They are then propagated by the top-level entry point.

\begin{figure}[h]

\begin{lstlisting}
require File
instantiate PureMod
instantiate ResourceMod(File)

def main():Unit
   PureMod.tick(File)
   ResourceMod.tick()
\end{lstlisting}

\caption{Definition of two modules, one pure and the other resourceful.}
\label{fig:wyv_module_instantiation}
\end{figure}

Before explaining our translation of Wyvern programs into $\epscalc$, we must explain several simplifications made in all of our examples which enable our particular desugaring.

Objects are only ever used in the form of modules. Modules only ever contain functions and other modules, and have no mutable fields. The examples contain no recursion or self-reference, including a module invoking its own functions. Modules will not reference each other cyclically. Lastly, modules only contain one function definition. Despite these simplifications, the chosen examples will highlight the essential aspects of $\epscalc$.

Because modules do not exercise self-reference and only contain one function definition, they will be modelled as functions in $\epscalc$. Applying this function will be equivalent to applying the single function definition in the module.

A collection of modules is desugared into $\epscalc$ as follows. First, a sequence of let-bindings are used to name constructor functions which, when given the capabilities requested by a module, will return an instance of the module. If the module does not require any capabilities then it will take $\Unit$ as its argument. The constructor function for $\kwa{M}$ is called $\kwa{MakeM}$. A function is then defined which represents the $\kwa{main}$ function, which is the entry point into the program. This $\kwa{main}$ function will instantiate all the modules by invoking the constructor functions, and then execute the body of code in main. Finally, the main function is invoked with the primitive capabilities it needs.

To demonstrate this process, \ref{fig:wyv_tutorial_desugaring} shows how the examples above desugar. Lines 1-3 define the constructor for $\kwa{PureMod}$; since $\kwa{PureMod}$ requires no capabilities, the constructor takes $\Unit$ as an argument on line 2. Lines 6-8 define the constructor for $\kwa{ResourceMod}$; it requires a $\kwa{File}$ capability, so the constructor takes $\kwa{\{File\}}$ as its input type on line 7. The entry point to the program is defiend on lines 11-15, which invokes the constructors and then runs the body of the $\kwa{main}$ method. Lastly, line 17 starts everything off by invoking $\kwa{Main}$ with the initial set of capabilities, which in this case is just $\kwa{File}$.

\begin{figure}[h]

\begin{lstlisting}
let MakePureMod =
   $\lambda$x:Unit.
      $\lambda$f:{File}. f.append
in

let MakeResourceMod =
   $\lambda$f:{File}.
      $\lambda$x:Unit. f.append
in

let MakeMain =
   $\lambda$f:{File}.
      $\lambda$x: Unit.
         let PureMod = (MakePureMod unit) in
         let ResourceMod = (MakeResourceMod f) in
         let _ = (PureMod f) in (ResourceMod unit) in

(MakeMain File) unit
\end{lstlisting}

\caption{Desugaring of $\kwa{PureMod}$ and $\kwa{ResourceMod}$ into $\epscalc$.}
\label{fig:wyv_tutorial_desugaring}
\end{figure}




\section{Examples}

In this section we present several scenarios where a developer may be forced to reason about the use of effects, and show how the capability-based reasoning of effects can assist them. In some scenarios, a program exhibits a certain nefarious behaviour, in which case capability-based reasoning can automatically detect this behaviour and reject it. Other scenarios are more qualitative; perhaps a developer must make a design choice and none of the alternatives \textit{prima facie} stand out. In such cases, capability-based reasoning might supply them with useful information, enabling tehm to make more informed design choices. We also hope to convince the reader that the rules of $\epscalc$ have practical worth, and could be used to enrich existing capability-safe languages.

The format of each section is as follows. A program is introduced which exhibits some bad behaviour or demonstrates a particular story about software development. The language used is \textit{Wyvern}; a pure, object-oriented, capability-safe language with first-class modules-as-objects. We show how the Wyvern program can be written as a corresponding $\epscalc$ program and sketch a derivation showing how the rules of $\epscalc$ and a sketch a derivation showing how the rules of $\epscalc$ would solve the relevant problem.

We take some shortcuts with the translation of Wyvern into $\epscalc$. Our ``objects'' are really records of functions; the difference between the two is self-reference. The particular examples chosen do not require self-reference, so no important properties are lost by treating Wyvern objects as records.

\subsection{Unannotated Client}

In Figure \ref{fig:eg1} an annotated $\kwa{Logger}$ module provides its client the ability to append to a particular $\kwa{File}$ resource. $\kwa{File}$ is a primitive capability passed into the program when it begins execution, perhaps from the system environment or a virtual machine. The $\kwa{Logger}$ module presents a controlled subset of operations on the $\kwa{File}$ viz. $\kwa{File.append}$. The program consists of an unannotated client which instantiates the $\kwa{Logger}$ module with the capability it selects ($\kwa{File}$) and then attempts to log.

If the client code is executed, what effects will it have? The answer is not immediately clear from the client's source-code, but a capability-based argument goes as follows: because the client code can typecheck needing only $\kwa{Logger}$, then whatever effects presented by $\kwa{Logger}$ are an upper-bound on the effects of the client.

\begin{figure}[h]

\begin{lstlisting}
resource module Logger
require File

def log(): Unit with File.append =
    File.append(``message logged'')
\end{lstlisting}

\begin{lstlisting}
module Client
require Logger

def run(): Unit =
   Logger.log()
\end{lstlisting}

\begin{lstlisting}
resource module Main
require File
instantiate Logger(File)
instantiate Client(Logger)

Client.run()
\end{lstlisting}

\caption{A $\kwa{logger}$ client doesn't need to add effect labels; these can be inferred.}
\label{fig:eg1}
\end{figure}

The desugaring first creates two functions, $\kwa{MakeLogger}$ and $\kwa{MakeClient}$, which instantiate the $\kwa{Logger}$ and $\kwa{Client}$ modules; the client code is treated as an implicit module. Lines 1-4 define a function which, given a $\kwa{File}$, returns a record containing a single $\kwa{log}$ function. Lines 6-8 define a function which, given a $\kwa{Logger}$, returns the unannotated client code, wrapped inside an $\kwa{import}$ expression selecting its needed authority. Lines 10-14 are the meat of the program; this function, when given a $\kwa{File}$ capability, creates the modules and then runs the client code. Program execution begins on line $16$, where $\kwa{Main}$ is given its initial set of capabilities --- which, in this case, is just $\kwa{File}$.

\begin{figure}[h]

\begin{lstlisting}
let MakeLogger =
   ($\lambda$f: File.
      $\lambda$x: Unit. f.append) in
          
let MakeClient =
   ($\lambda$logger: Logger.
      import(File.append) logger = logger in
         $\lambda$x: Unit. logger unit) in
          
let MakeMain =
   ($\lambda$f: File.
      $\lambda$x: Unit.
         let LoggerModule = MakeLogger f in
         let ClientModule = MakeClient LoggerModule in
         ClientModule unit) in

(MakeMain File) unit
\end{lstlisting}

\caption{Desugared version of Figure \ref{fig:eg1}.}
\label{fig:eg1_desugared}
\end{figure}

The interesting part  is on lines 7-8, where the unannotated code selects $\kwa{File.append}$ as its authority. This is exactly the effects of the logger, i.e. $\kwa{effects}(\Unit \rightarrow_{\kwa{File.append}} \Unit) = \{ \kwa{File.append} \}$. The code also satisfies the higher-order safety predicates, and the body of the $\kwa{import}$ expression typechecks in the empty context. Therefore, the unannotated code typechecks by \textsc{$\varepsilon$-Import}.

In such a small example the client could simply inspect the source code of $\kwa{Logger}$ to determine what effects it might have. Several situations can make this impossible or tedious. First, the manual approach loses efficiency when the system involves many modules of large size across code-ownership boundaries; capability-based reasoning tells you automatically. Second, the source code of $\kwa{Logger}$ might be obfuscated or unavailable, and the only useful information is that given by its signature. Lastly, the client may not care about effects in this situation; the program may be a quick-and-dirty throwaway, in which case it is nice that the capability-based reasoning still accepts the client code without annotations..

\subsection{API Violation}

Figure \ref{fig:eg2} inverts the roles of the last scenario: now, the annotated $\kwa{Client}$ wants to use the unannotated $\kwa{Logger}$. The $\kwa{Logger}$ module captures the $\kwa{File}$ capability, and exposes a single function $\kwa{log}$ with the $\kwa{File.append}$ effect. However, the $\kwa{Client}$ has a function $\kwa{run}$ which executes $\kwa{Logger.log}$, incurring the effect of $\kwa{File.append}$, but declares its set of effects as $\varnothing$. The implementation and the signature of $\kwa{Client.run}$ are inconsistent --- does the type system recognise this?

\begin{figure}[h]

\begin{lstlisting}
resource module Logger
require File

def log(): Unit =
    File.append(``message logged'')
\end{lstlisting}

\begin{lstlisting}
resource module Client
require Logger

def run(): Unit with $\varnothing$ =
   Logger.log()
\end{lstlisting}

\begin{lstlisting}
resource module Main
require File
instantiate Logger(File)
instantiate Client(Logger)

Client.run()
\end{lstlisting}

\caption{The unlabelled code in $\kwa{Logger}$ exercises authority exceeding that selected by $\kwa{Client}$.}
\label{fig:eg2}
\end{figure}

A desugaring is given in Figure \ref{fig:eg2_desugared}. Lines 1-3 define the function which instantiates the $\kwa{Logger}$ module. Lines 5-8 define the function which instantiates the $\kwa{Client}$ module. Lines 10-15 define the function which instantiates the $\kwa{Main}$ module. Line 17 initiates the program, supplying $\kwa{File}$ to the $\kwa{Main}$ module and invoking its main method. On lines 3-4, the unannotated code is modelled using an $\kwa{import}$ expression which selects $\varnothing$ as its authority. So far this coheres to the expectations of $\kwa{Client}$. However, \textsc{$\varepsilon$-Import} cannot be applied because the name being bound, $f$, has the type $\{ \File \}$, and $\fx{\{ \File \}} = \{ \kwa{File}.* \}$, which is inconsistent with the declared effects $\varnothing$.

\begin{figure}[h]

\begin{lstlisting}
let MakeLogger =
   ($\lambda$f: File.
      import($\varnothing$) f = f in
         $\lambda$x: Unit. f.append) in

let MakeClient =
	($\lambda$logger: Logger.
	   $\lambda$x: Unit. logger unit) in

let MakeMain =
   ($\lambda$f: File.
      let LoggerModule = MakeLogger f in
      let ClientModule = MakeClient LoggerModule in
      ClientModule unit) in

(MakeMain File) unit
\end{lstlisting}

\caption{Desugared version of Figure \ref{fig:eg2}.}
\label{fig:eg2_desugared}
\end{figure}

The only way for this to typecheck would be to annotate $\kwa{Client.run}$ as having every effect on $\File$. This demonstrates how the effect-system of $\epscalc$ approximates unlabelled code: it simply considers it as having every effect which could be incurred on those resources in scope, which here is $\kwa{File}.*$.

\subsection{API Violation}

Figure \ref{fig:eg3} is a variation of the last example, but now $\kwa{Logger.log}$ is passed the $\kwa{File}$ capability, rather than possessing it. $\kwa{Logger.log}$ still incurs $\kwa{File.append}$ inside unannotated code, which causes the implementation of $\kwa{Client.run}$ to violate its signature.

\begin{figure}[h]

\begin{lstlisting}
module Logger

def log(f: {File}): Unit
    f.append(``message logged'')
\end{lstlisting}

\begin{lstlisting}
module Client
instantiate Logger(File)

def run(f: {File}): Unit with $\varnothing$
   Logger.log(File)
   
\end{lstlisting}

A desugared version is given in Figure \ref{fig:eg3}, which is largely the same as in the previous example, except 

\begin{lstlisting}
resource module Main
require File
instantiate Client

Client.run(File)
\end{lstlisting}

\caption{The unlabelled code in $\kwa{Logger}$ exercises authority exceeding that selected by $\kwa{Client}$.}
\label{fig:eg3}
\end{figure}




\begin{figure}[h]

\begin{lstlisting}
let MakeClient =
	($\lambda$x: Unit.
	   let MakeLogger =
	      ($\lambda$x: Unit.
	         import($\varnothing$) x=x in
	            $\lambda$f: {File}. f.append) in
      let LoggerModule = MakeLogger unit in
      $\lambda$f: {File}. LoggerModule f) in
	
let MakeMain =
   ($\lambda$f: {File}.
      $\lambda$x: Unit.
         let ClientModule = MakeClient unit in
         ClientModule f) in

(MakeMain File) unit
\end{lstlisting}

\caption{Desugared version of \ref{fig:eg3}.}
\label{fig:eg3_desugared}
\end{figure}


\subsection{API Violation}


\begin{figure}[h]

\begin{lstlisting}
resource module Logger
require File

def log(): Unit with {File.append, File.write} =
    File.append(``message logged'')
    File.write(``message written'')
\end{lstlisting}

\begin{lstlisting}
module Client

def run(l: Logger): Unit with {File.append} =
    l.log()
\end{lstlisting}

\begin{lstlisting}
resource module Main
require File
instantiate Logger(File)

Client.run(Logger)
\end{lstlisting}

\caption{This won't type because of a mismatch between the effects of $\kwa{Client}$ and the effects of $\kwa{Logger}$.}
\label{fig:eg4}
\end{figure}


\begin{figure}[h]

\begin{lstlisting}
let MakeLogger =
   ($\lambda$f: File.
      $\lambda$x: Unit. let _ = f.append in f.write) in
           
let MakeClient =
   ($\lambda$x: Unit.
      $\lambda$logger: Logger. logger unit) in
                  
let MakeMain =
   ($\lambda$f: File.
      $\lambda$x: Unit.
         let LoggerModule = MakeLogger f in
         let ClientModule = MakeClient unit in
         ClientModule.run LoggerModule) in

(MakeMain File) unit
\end{lstlisting}

\caption{Desugared version of Figure \ref{fig:eg4}.}
\label{fig:eg4_desugared}
\end{figure}


\subsection{Hidden Authority}

\begin{figure}[h]

\begin{lstlisting}
module Malicious

def stealData(f: {File}):Unit with {File.read} =
   f.read
\end{lstlisting}

\begin{lstlisting}
module Plugin
instantiate Malicious

def run(f: {File}): Unit with $\varnothing$ =
   Malicious.stealData(f)
\end{lstlisting}

\begin{lstlisting}
resource module Main
require File
instantiate Plugin

Plugin.run(File)
\end{lstlisting}

\caption{The $\kwa{Main}$ module transitively invokes a $\kwa{File.read}$ effect, violating its selected authority.}
\label{fig:eg5}
\end{figure}

\begin{figure}[h]

\begin{lstlisting}
let MakePlugin =
   ($\lambda$x: Unit.
      let MakeMalicious =
         ($\lambda$x: Unit. $\lambda$f: {File}. f.read) in
      let MaliciousModule = (MakeMalicious unit) in
      $\lambda$f: {File}. MaliciousModule f) in
      
let MakeMain =
   ($\lambda$f: File.
      $\lambda$x: Unit.
         let PluginModule = MakePlugin unit in
         PluginModule.run f) in

(MakeMain File) unit
\end{lstlisting}

\caption{Desugared version of Figure \ref{fig:eg5}.}
\label{fig:eg5_desugared}
\end{figure}

\subsection{Hidden Authority 2} 

\begin{figure}[h]

\begin{lstlisting}
module Malicious

def stealData(f: {File}):Unit =
   f.read
\end{lstlisting}

\begin{lstlisting}
module Plugin
instantiate Malicious

def run(f: {File}): Unit with $\varnothing$ =
   Malicious.stealData(f)
\end{lstlisting}

\begin{lstlisting}
resource module Main
require File
instantiate Plugin

Plugin.run(File)
\end{lstlisting}

\caption{The transitive invocation of $\kwa{File.read}$ now happens inside unannotated code, but the type system will still reject this program.}
\label{fig:eg6}
\end{figure}

\begin{figure}[h]

\begin{lstlisting}
let MakePlugin =
   ($\lambda$x: Unit.
      let MakeMalicious =
         ($\lambda$x: Unit.
            import($\varnothing$) x=x in
               $\lambda$f: {File}. f.read) in
      let MaliciousModule = (MakeMalicious unit) in
      $\lambda$f: {File}. MaliciousModule f) in
      
let MakeMain =
   ($\lambda$f: File.
      $\lambda$x: Unit.
         let PluginModule = MakePlugin unit in
         PluginModule.run f) in

(MakeMain File) unit
\end{lstlisting}

\caption{Desugared version of Figure \ref{fig:eg6}.}
\label{fig:eg6_desugared}
\end{figure}

\subsection{Hidden Authority 2} 

\begin{figure}[h]

\begin{lstlisting}
module Malicious

def log(f: Unit $\rightarrow$ Unit):Unit
   f()
\end{lstlisting}

\begin{lstlisting}
module Plugin
instantiate Malicious

def run(f: {File}): Unit with $\varnothing$
   Malicious.log($\lambda$x:Unit. f.read)
\end{lstlisting}

\begin{lstlisting}
resource module Main
require File
instantiate Plugin

Plugin.run(File)
\end{lstlisting}

\caption{The transitive invocation of $\kwa{File.read}$ happens when the unannotated code executes the function given to it.}
\label{This is the label.}
\end{figure}


\subsection{Resource Leak}

\begin{figure}[h]

\begin{lstlisting}
module Malicious

def log(f: Unit $\rightarrow$ File):Unit
   f().read
\end{lstlisting}

\begin{lstlisting}
module Plugin
instantiate Malicious

def run(f: {File}): Unit with $\varnothing$
   Malicious.log($\lambda$x:Unit. f)
\end{lstlisting}

\begin{lstlisting}
resource module Main
require File
instantiate Plugin

Plugin.run(File)
\end{lstlisting}

\caption{A resource leak allows $\kwa{Malicious}$ to gain access to the $\kwa{File}$ capability directly.}
\label{This is the label.}
\end{figure}


\section{Conclusions}

We introduced $\opercalc$, a lambda calculus with primitive capabilities and their effects. $\opercalc$ programs are fully annotated with their effects. Relaxing this requirement, we obtained $\epscalc$, which allows unannotated code to be nested inside annotated code with a new $\kwa{import}$ construct. The capability-safe design of $\epscalc$ allows us to safely infer the effects of unannotated code by inspecting what capabilities are passed into it by its annotated surroundings. Such an approach allows code to be incrementally annotated, giving developers a balance between safety and convenience and alleviating the verbosity that has discouraged widespread adoption of previous effect systems \cite{rytz2012}.

\subsection{Related Work}

Capabilities were introduced by Dennis and Van Horn as a way to control which processes in an operating system had permission to access certain parts of memory \cite{dennis66}. An \textit{access control list} would declare what permissions a program may exercise. These early ideas are considerably different to the object capability model introduced by Mark Miller \cite{miller06}, which imposes constraint on how permissions can proliferate. Maffeis et. al. formalised the notion of a capability-safe language and showed that a subset of Caja (a Javascript implementation) is capability-safe \cite{maffeis10}. Miller's model has also has been applied to more heavyweight formal systems: Drossopoulou et. al. combined Hoare logic with capabilities to determine whether components of a system can be trusted \cite{drossopoulou07}. Other capability-safe languages include Wyvern \cite{nistor13} and Newspeak \cite{bracha10}.

The original effect system by Lucassen and Gifford was used to determine if two expressions could safely run in parallel \cite{lucassen88}. Subsequent applications include determining what functions a program might invoke \cite{tang94} and what regions in memory might be accessed or updated during execution \cite{talpin94}. In these systems, ``effects'' are performed upon ``regions''; in ours, ``operations'' are performed upon ``resources''. An important difference in $\epscalc$ is the distinction between annotated and unannotated code: only the former will type-and-effect-check. This approach allows for an effect discipline to be incrementally imposed on an otherwise effect-unconscious system.

Fengyun Liu has also combined capability-safety and effect systems, with applications to purity analysis in Scala \cite{liu16}. If a function is known to be pure then optimisations such as inlining and parallelisation can be made. Liu's work is motivated by achieving such optimisations for Scala compilers. It distinguishes between free and stoic functions: free functions may exercise ambient authority whereas stoic functions may not. Stoic functions are therefore capability-safe pockets whose purity can be determined by examining what capabilities are passed into them. Liu's System F-Impure does not track effects, whereas $\epscalc$, by distinguishing between regular effects and higher-order effects, gives more fine-grained detail about what a piece of code will do when executed.

The systems by Lucassen and Liu have effect polymorphism, whereas $\epscalc$ does not. Another capability-safe effect-system is the one by Devriese et. al., who use effect polymorphism and possible world semantics to guarantee behavioural invariants on data structures \cite{devriese16}. Our approach is not as expressive, based only on a topological analysis of how capabilities can be passed around the program, but the formalism is much more lightweight.

\subsection{Future Work}

Our conception of effects is quite specific, modelling only the invocation of operations on a primitive capability as an effect. This definition could be generalised to allow for other sorts of effects, such as accessing or writing mutable state. Resources and operations are also fixed throughout runtime; it would be interesting to consider the theory in a setting which allows dynamic resource creation and destruction.

The current theory contains no effect polymorphism. This would allow the type of a function to be parameterised by a set of effects. For an example of such a function, consider $\kwa{map}$: given a function $f$ and a list $l$, map applies $f$ to every element of $l$ to produce a new list $l'$. The effects of $\kwa{map}$ are dependent on the effects of $f$. The only way to define $\kwa{map}$ in $\epscalc$ would be to conservatively approximate it as having every effect, in which case all precision has been lost. A polymorphic effect system which considers the type of $\kwa{map}$ as being parameterised by a set of effects could give more meaningful approximations.

Lastly, the ideas in this paper might be extended and developed to the point where they can be used in real-world situations. Implementing these ideas in an existing, general-purpose language would do much towards that end.

%Many believe in the real and practical value of the object capability model, but we do not fully understand its formal benefits. 








%%% Acknowledgments
%\begin{acks}                            %% acks environment is optional
%                                        %% contents suppressed with 'anonymous'
%  %% Commands \grantsponsor{<sponsorID>}{<name>}{<url>} and
%  %% \grantnum[<url>]{<sponsorID>}{<number>} should be used to
%  %% acknowledge financial support and will be used by metadata
%  %% extraction tools.
%  This material is based upon work supported by the
%  \grantsponsor{GS100000001}{National Science
%    Foundation}{http://dx.doi.org/10.13039/100000001} under Grant
%  No.~\grantnum{GS100000001}{nnnnnnn} and Grant
%  No.~\grantnum{GS100000001}{mmmmmmm}.  Any opinions, findings, and
%  conclusions or recommendations expressed in this material are those
%  of the author and do not necessarily reflect the views of the
%  National Science Foundation.
%\end{acks}

%% Bibliography
\bibliography{biblio}

%% Appendix
\appendix
% Uncomment to put the proofs at the end as an appendix.
% %% For double-blind review submission
\documentclass[acmlarge,review,anonymous]{acmart}\settopmatter{printfolios=true}
%% For single-blind review submission
%\documentclass[acmlarge,review]{acmart}\settopmatter{printfolios=true}
%% For final camera-ready submission
%\documentclass[acmlarge]{acmart}\settopmatter{}

%% Note: Authors migrating a paper from PACMPL format to traditional
%% SIGPLAN proceedings format should change 'acmlarge' to
%% 'sigplan,10pt'.


%% Some recommended packages.
\usepackage{booktabs}   %% For formal tables:
                        %% http://ctan.org/pkg/booktabs
\usepackage{subcaption} %% For complex figures with subfigures/subcaptions
                        %% http://ctan.org/pkg/subcaption


\usepackage{bm}
\usepackage{color}
\usepackage{ebproof} % For proof trees
\usepackage{listings} % For code snippets
\usepackage{proof} % For inference rules.
\usepackage[ruled]{algorithm2e}


\definecolor{grey}{gray}{0.92}

\lstset{
tabsize=3,
basicstyle=\ttfamily\small, commentstyle=\itshape\rmfamily, 
backgroundcolor=\color{grey},
numbers=left,
numberstyle=\tiny,
language=java,
moredelim=[il][\sffamily]{?},
mathescape=true,
showspaces=false,
showstringspaces=false,
columns=fullflexible,
escapeinside={(@}{@)}, morekeywords=[1]{def, if, then, else, with, val, module, instantiate}}
\lstloadlanguages{Java,VBScript,XML,HTML}

%using \kwa outside math mode
\newcommand{\kwat}[1]{$\kwa{#1}$}

% Hyphens
\newcommand{\hyphen}{\hbox{-}}

% For defining derived forms.
\newcommand\defn{\mathrel{\overset{\makebox[0pt]{\mbox{\normalfont\tiny\sffamily def}}}{=}}}

% Constants, types.
\newcommand{\unit}{\kwa{unit}}
\newcommand{\Unit}{\kwa{Unit}}
\newcommand{\File}{\kwa{File}}
\newcommand{\Socket}{\kwa{Socket}}

% Keywords.
\newcommand{\kwa}[1]{\mathtt{#1}}
\newcommand{\kw}[1]{\mathtt{#1}~}

% Expressions.
\newcommand{\import}[4]{\kwa{import}(#1)~#2 = #3~\kw{in} #4}
\newcommand{\letxpr}[3]{\kw{let} #1 = #2~\kw{in} #3}	

% Functions in the type theory.
\newcommand{\annot}[2]{\kwa{annot}(#1, #2)}
\newcommand{\erase}[1]{\kwa{erase}(#1)}
\newcommand{\fx}[1]{\kwa{effects}(#1)}
\newcommand{\hofx}[1]{\kwa{ho \hyphen effects}(#1)}

% Safety predicates in the type theory.
\newcommand{\safe}[2]{\kwa{safe}(#1, #2)}
\newcommand{\hosafe}[2]{\kwa{ho \hyphen safe}(#1, #2)}

% Names of the calculi.
\newcommand{\opercalc}{\kwa{OC}}
\newcommand{\epscalc}{\kwa{CC}}


\renewcommand{\algorithmcfname}{ALGORITHM}
\SetAlFnt{\small}
\SetAlCapFnt{\small}
\SetAlCapNameFnt{\small}
\SetAlCapHSkip{0pt}
\IncMargin{-\parindent}


\makeatletter\if@ACM@journal\makeatother
%% Journal information (used by PACMPL format)
%% Supplied to authors by publisher for camera-ready submission
\acmJournal{PACMPL}
\acmVolume{1}
\acmNumber{1}
\acmArticle{1}
\acmYear{2017}
\acmMonth{1}
\acmDOI{10.1145/nnnnnnn.nnnnnnn}
\startPage{1}
\else\makeatother
%% Conference information (used by SIGPLAN proceedings format)
%% Supplied to authors by publisher for camera-ready submission
\acmConference[PL'17]{ACM SIGPLAN Conference on Programming Languages}{January 01--03, 2017}{New York, NY, USA}
\acmYear{2017}
\acmISBN{978-x-xxxx-xxxx-x/YY/MM}
\acmDOI{10.1145/nnnnnnn.nnnnnnn}
\startPage{1}
\fi


%% Copyright information
%% Supplied to authors (based on authors' rights management selection;
%% see authors.acm.org) by publisher for camera-ready submission
\setcopyright{none}             %% For review submission
%\setcopyright{acmcopyright}
%\setcopyright{acmlicensed}
%\setcopyright{rightsretained}
%\copyrightyear{2017}           %% If different from \acmYear


%% Bibliography style
\bibliographystyle{ACM-Reference-Format}
%% Citation style
%% Note: author/year citations are required for papers published as an
%% issue of PACMPL.
\citestyle{acmauthoryear}   %% For author/year citations % Contains the packages and other data common to both paper (main.tex) and supplementary material (proofs.tex).


% Document starts
\begin{document}


\title{Capability-Flavoured Effects (Supplementary Material with Proofs)}


\maketitle


\section{$\opercalc$ Proofs}


\begin{lemma}[$\opercalc$ Canonical Forms]
Unless the rule used is \textsc{$\varepsilon$-Subsume}, the following are true:
\begin{enumerate}
	\item If $\Gamma \vdash x: \tau~\kw{with} \varepsilon$ then $\varepsilon = \varnothing$.
	\item If $ \Gamma \vdash  v:  \tau~\kw{with} \varepsilon$ then $\varepsilon = \varnothing$.
	\item If $ \Gamma \vdash v: \{ \bar r \}~\kw{with} \varepsilon$ then $ v = r$ and $\{ \bar r \} = \{ r \}$.
	\item If $\Gamma \vdash v: \tau_1 \rightarrow_{\varepsilon'} \tau_2~\kw{with} \varepsilon$ then $v = \lambda x:\tau. e$.
\end{enumerate}
\end{lemma}


\begin{proof}
~
\begin{enumerate}
	\item The only rule that applies to variables is \textsc{$\varepsilon$-Var} which ascribes the type $\varnothing$.
	\item By definition a value is either a resource literal or a lambda. The only rules which can type values are \textsc{$\varepsilon$-Resource} and \textsc{$\varepsilon$-Abs}. In the conclusions of both, $\varepsilon = \varnothing$.
	\item The only rule ascribing the type $\{ \bar r \}$ is \textsc{$\varepsilon$-Resource}. Its premises imply the result.
	\item The only rule ascribing the type $\tau_1 \rightarrow_{\varepsilon'} \tau_2$ is \textsc{$\varepsilon$-Abs}. Its premises imply the result.
\end{enumerate}
\end{proof}


\hrulefill


\begin{theorem}[$\opercalc$ Progress]
If $ \Gamma \vdash  e:  \tau~\kw{with} \varepsilon$ and $ e$ is not a value or variable, then $ e \longrightarrow  e'~|~\varepsilon$, for some $e', \varepsilon$.
\end{theorem}


\begin{proof} By induction on $ \Gamma \vdash  e:  \tau~\kw{with} \varepsilon$. \\

Case: \textsc{$\varepsilon$-Var}, \textsc{$\varepsilon$-Resource}, or  \textsc{$\varepsilon$-Abs}. Then $e$ is a value or variable and the theorem statement holds vacuously.\\

Case: \textsc{$\varepsilon$-App}. Then $ e =  e_1~ e_2$. If $ e_1$ is not a value or variable it can be reduced $e_1 \longrightarrow e_1'~|~\varepsilon$ by inductive assumption, so $ e_1~ e_2 \longrightarrow  e_1'~ e_2~|~\varepsilon$ by \textsc{E-App1}. If $ e_1 =  v_1$ is a value and $ e_2$ a non-value, then $e_2$ can be reduced $e_2 \longrightarrow e_2'~|~\varepsilon$ by inductive assumption, so $ e_1~ e_2 \longrightarrow  v_1~ e_2'~|~\varepsilon$ by \textsc{E-App2}. Otherwise $ e_1 = v_1$ and $ e_2 = v_2$ are both values. By inversion on \textsc{$\varepsilon$-App} and canonical forms, $\Gamma \vdash v_1: \tau_2 \rightarrow_{\varepsilon'} \tau_3~\kw{with} \varnothing$, and $v_1 = \lambda x: \tau_2. e_{body}$. Then $(\lambda x:  \tau.  e_{body})  v_2 \longrightarrow [ v_2/x]e_{body}~|~\varnothing$ by \textsc{E-App3}.\\

Case: \textsc{$\varepsilon$-OperCall}. Then $ e =  e_1.\pi$. If $ e_1$ is a non-value it can be reduced $e_1 \longrightarrow e_1'~|~\varepsilon$ by inductive assumption, so $ e_1.\pi \longrightarrow  e_1'.\pi~|~\varepsilon$ by \textsc{E-OperCall1}. Otherwise $ e_1 =  v_1$ is a value. By inversion on \textsc{$\varepsilon$-OperCall} and canonical forms, $\Gamma \vdash v_1: \{ r \}~\kw{with} \{ r.\pi \}$, and $v_1 = r$. Then $r.\pi \longrightarrow \kwa{unit}~|~\{ r.\pi \}$ by \textsc{E-OperCall2}.\\

Case: \textsc{$\varepsilon$-Subsume}. If $e$ is a value or variable, the theorem holds vacuously. Otherwise by inversion on \textsc{$\varepsilon$-Subsume},  $ \Gamma \vdash e:  \tau'~\kw{with} \varepsilon'$, and $e \longrightarrow e'~|~\varepsilon$ by inductive assumption.

\end{proof}


\hrulefill


\begin{lemma}[$\opercalc$ Substitution]
If $\Gamma, x: \tau' \vdash e: \tau~\kw{with} \varepsilon$ and $\Gamma \vdash v: \tau'~\kw{with} \varnothing$ then $\Gamma \vdash [v/x]e: \tau~\kw{with} \varepsilon$.
\end{lemma}


\begin{proof} By induction on the derivation of $ \Gamma, x:  \tau' \vdash e:  \tau~\kw{with} \varepsilon$. \\

\textit{Case}: \textsc{$\varepsilon$-Var}. Then $ e = y$ is a variable. Either $y = x$ or $y \neq x$. Suppose $y=x$. By applying canonical Forms to the theorem assumption $\Gamma, x: \tau' \vdash e: \tau'~\kw{with} \varnothing$, hence $\tau' = \tau$. $[v/x]y = [v/x]x = v$, and by assumption, $\Gamma \vdash v: \tau'~\kw{with} \varnothing$, so $\Gamma \vdash [v/x]y: \tau~\kw{with} \varnothing$.

Otherwise $y \neq x$. By applying canonical forms to the theorem assumption $\Gamma, x: \tau' \vdash y: \tau~\kw{with} \varnothing$, so $y: \tau \in \Gamma$. Since $[v/x]y = y$, then $\Gamma \vdash y: \tau~\kw{with} \varnothing$ by \textsc{$\varepsilon$-Var}. \\

\textit{Case}: \textsc{$\varepsilon$-Resource}. Because $ e = r$ is a resource literal then $ \Gamma \vdash r:  \{ r \}~\kw{with} \varnothing$ by canonical forms. By definition $[ v/x]r = r$, so $ \Gamma \vdash [ v/x]r:  \{ \bar r\}~\kw{with} \varnothing$. \\

\textit{Case:} \textsc{$\varepsilon$-App}. By inversion $ \Gamma, x:  \tau' \vdash  e_1: \tau_2 \rightarrow_{\varepsilon_3}  \tau_3~\kw{with} \varepsilon_A$ and $ \Gamma, x:  \tau' \vdash  e_2:  \tau_2~\kw{with} \varepsilon_B$, where $\varepsilon = \varepsilon_A \cup \varepsilon_B \cup \varepsilon_3$ and $ \tau =  \tau_3$. From inversion on \textsc{$\varepsilon$-App} and inductive assumption, $ \Gamma \vdash [ v/x] e_1:  \tau_2 \rightarrow_{\varepsilon_3}  \tau_3~\kw{with} \varepsilon_A$ and $ \Gamma \vdash [ v/x] e_2:  \tau_2~\kw{with} \varepsilon_B$. By \textsc{$\varepsilon$-App}  $ \Gamma \vdash ([ v/x] e_1) ([ v/x] e_2) :  \tau_3~\kw{with} \varepsilon_A \cup \varepsilon_B \cup \varepsilon_3$. By simplifying and applying the definition of $\kwa{substitution}$, this is the same as $ \Gamma \vdash [ v/x]( e_1~ e_2):  \tau~\kw{with} \varepsilon$. \\

\textit{Case:} \textsc{$\varepsilon$-OperCall}. By inversion $ \Gamma, x:  \tau' \vdash  e_1: \{ \bar r \}~\kw{with} \varepsilon_1$ and $\tau = \Unit$ and $\varepsilon = \varepsilon_1 \cup \{ r.\pi \mid r \in \bar r, \pi \in \Pi \}$. By inductive assumption, $ \Gamma \vdash [ v/x] e_1 : \{ \bar r \} ~\kw{with} \varepsilon_1$. Then by \textsc{$\varepsilon$-OperCall}, $ \Gamma \vdash ([ v/x] e_1).\pi: \Unit~\kw{with} \varepsilon_1 \cup \{ r.\pi \mid r.\pi \in \bar r \times \Pi \}$. By simplifying and applying the definition of $\kwa{substitution}$, this is the same as $ \Gamma \vdash [ v/x]( e_1.\pi):  \tau~\kw{with} \varepsilon$.\\

\textit{Case:} \textsc{$\varepsilon$-Subsume}. By inversion, $ \Gamma, x:  \tau' \vdash  e:  \tau_2~\kw{with} \varepsilon_2$, where $ \tau_2 <:  \tau$ and $\varepsilon_2 \subseteq \varepsilon$. By inductive hypothesis, $ \Gamma \vdash [ v/x] e:  \tau_2~\kw{with} \varepsilon_2$. Then $ \Gamma \vdash [ v/x] e:  \tau~\kw{with} \varepsilon$ by \textsc{$\varepsilon$-Subsume}.

\end{proof}


\hrulefill


\begin{theorem}[$\opercalc$ Preservation]
If $\Gamma \vdash e_A: \tau_A~\kw{with} \varepsilon_A$ and $e_A \longrightarrow e_B~|~\varepsilon$, then $\tau_B <: \tau_A$ and $\varepsilon_B \cup \varepsilon \subseteq \varepsilon_A$, for some $e_B, \varepsilon, \tau_B, \varepsilon_B$.
\end{theorem}


\begin{proof}
By induction on the derivation of $ \Gamma \vdash  e_A:  \tau_A~\kw{with} \varepsilon_A$ and then the derivation of $e_A \longrightarrow e_B~|~\varepsilon$.  \\

\textit{Case:} \textsc{$\varepsilon$-Var}, \textsc{$\varepsilon$-Resource}, \textsc{$\varepsilon$-Unit}, \textsc{$\varepsilon$-Abs}. Then $e_A$ is a value and cannot be reduced, so the theorem holds vacuously.\\

\textit{Case:} \textsc{$\varepsilon$-App}. Then $e_A =  e_1~ e_2$ and $\Gamma \vdash e_1:  \tau_2 \longrightarrow_{\varepsilon_3}  \tau_3~\kw{with} \varepsilon_1$ and $ \Gamma \vdash  e_2:  \tau_2~\kw{with} \varepsilon_2$ and $\tau_B = \tau_3$ and $\varepsilon_A = \varepsilon_1 \cup \varepsilon_2 \cup \varepsilon_3$.  In each case we choose $\tau_B = \tau_A$ and $\varepsilon_B \cup \varepsilon = \varepsilon_A$.

\textbf{Subcase:} \textsc{E-App1}. Then $e_1~e_2 \longrightarrow e_1'~e_2~|~\varepsilon$. By inversion on \textsc{E-App1}, $e_1 \longrightarrow e_1'~|~\varepsilon$. By inductive hypothesis and \textsc{$\varepsilon$-Subsume} $\Gamma \vdash v_1: \tau_2 \longrightarrow_{\varepsilon_3} \tau_3~\kw{with} \varepsilon_1$. Then $\Gamma \vdash e_1'~e_2: \tau_3~\kw{with} \varepsilon_1 \cup \varepsilon_2 \cup \varepsilon_3$ by \textsc{$\varepsilon$-App}.

\textbf{Subcase:} \textsc{E-App2}. Then $e_1 = v_1$ is a value and $e_2 \longrightarrow e_2'~|~\varepsilon$. By inversion on \textsc{E-App2}, $e_2 \longrightarrow e_2'~|~\varepsilon$. By inductive hypothesis and \textsc{$\varepsilon$-Subsume} $\Gamma \vdash e_2': \tau_2~\kw{with} \varepsilon_2$. Then $\Gamma \vdash v_1~e_2': \tau_3~\kw{with} \varepsilon_1 \cup \varepsilon_2 \cup \varepsilon_3$ by \textsc{$\varepsilon$-App}.

\textbf{Subcase:} \textsc{E-App3}. Then $e_1 = \lambda x: \tau_2.e_{body}$ and $e_2 = v_2$ are values and $(\lambda x: \tau_2.e_{body})~v_2 \longrightarrow [v_2/x]e_{body}~|~\varnothing$. By inversion on the rule \textsc{$\varepsilon$-App} used to type $\lambda x:  \tau_2. e_{body}$, we know $\Gamma, x:  \tau_2 \vdash e_{body}: \tau_3~\kw{with} \varepsilon_3$. $e_1 = v_1$ and $e_2 = v_2$ are values, so $\varepsilon_1 = \varepsilon_2 = \varnothing$ by canonical forms . Then by the substitution lemma, $ \Gamma \vdash [ v_2/x] e_{body} :  \tau_3~\kw{with} \varepsilon_3$ and $\varepsilon_A = \varepsilon_B = \varepsilon$. \\

\textit{Case:}  \textsc{$\varepsilon$-OperCall}. Then $e_A = e_1.\pi$ and $\Gamma \vdash e_1: \{ \bar r \}~\kw{with} \varepsilon_1$ and $\tau_A = \Unit$ and $\varepsilon_A = \varepsilon_1 \cup \{ r.\pi \mid r \in \bar r, \pi \in \Pi \}$.

\textbf{Subcase:} \textsc{E-OperCall1}. Then $e_1.\pi \longrightarrow e_1'.\pi~|~\varepsilon$. By inversion on \textsc{E-OperCall1}, $e_1 \longrightarrow e_1'~|~\varepsilon$. By inductive hypothesis and application of \textsc{$\varepsilon$-Subsume}, $\Gamma \vdash e_1': \{ \bar r \}~\kw{with} \varepsilon_1$. Then $\Gamma \vdash e_1'.\pi: \{ \bar r \}~\kw{with} \varepsilon_1 \cup \{ r.\pi \mid r \in \bar r, \pi \in \Pi \}$ by \textsc{$\varepsilon$-OperCall}.

\textbf{Subcase:} \textsc{E-OperCall2}. Then $e_1 = r$ is a resource literal and $r.\pi \longrightarrow \kwa{unit}~|~\{ r.\pi \}$. By canonical forms, $\varepsilon_1 = \varnothing$. By \textsc{$\varepsilon$-Unit}, $ \Gamma \vdash \kwa{unit}: \kwa{Unit}~\kw{with} \varnothing$. Therefore $\tau_B = \tau_A$ and $\varepsilon \cup \varepsilon_B = \{ r.\pi \} = \varepsilon_A$.
\end{proof}


\hrulefill


\begin{theorem}[$\opercalc$ Single-step Soundness]
If $ \Gamma \vdash  e_A:  \tau_A~\kw{with} \varepsilon_A$ and $ e_A$ is not a value, then $e_A \longrightarrow e_B~|~\varepsilon$, where $ \Gamma \vdash e_B:  \tau_B~\kw{with} \varepsilon_B$ and $ \tau_B <:  \tau_A$ and $\varepsilon_B \cup \varepsilon \subseteq \varepsilon_A$, for some $e_B, \varepsilon, \tau_B, \varepsilon_B$.
\end{theorem}


\begin{proof}
If $ e_A$ is not a value then the reduction exists by the progress theorem. The rest follows by the preservation theorem.
\end{proof}


\hrulefill


\begin{theorem}[$\opercalc$ Multi-step Soundness]
If $ \Gamma \vdash  e_A:  \tau_A~\kw{with} \varepsilon_A$ and $e_A \longrightarrow^{*} e_B~|~\varepsilon$, where $\Gamma \vdash e_B: \tau_B~\kw{with} \varepsilon_B$ and $ \tau_B <: \tau_A$ and $\varepsilon_B \cup \varepsilon \subseteq \varepsilon_A$.
\end{theorem}


\begin{proof} By induction on the length of the multi-step reduction.\\

\textit{Case:} Length $0$. Then $e_A = e_B$ and $\tau_A = \tau_B$ and $\varepsilon = \varnothing$ and $\varepsilon_A = \varepsilon_B$.\\

\textit{Case:} Length $n+1$. By inversion the multi-step can be split into a multi-step of length $n$, which is $ e_A \longrightarrow^{*}  e_C~|~\varepsilon'$, and a single-step of length $1$, which is $e_C \longrightarrow e_B~|~\varepsilon''$, where $\varepsilon = \varepsilon' \cup \varepsilon''$. By inductive assumption and preservation theorem, $ \Gamma \vdash  e_C:  \tau_C~\kw{with} \varepsilon_C$ and $ \Gamma \vdash  e_B:  \tau_B~\kw{with} \varepsilon_B$, where $ \tau_C <:  \tau_A$ and $ \varepsilon_C \cup \varepsilon' \subseteq \varepsilon_A$. By single-step soundness, $ \tau_B <:  \tau_C$ and $ \varepsilon_B \cup \varepsilon'' \subseteq \varepsilon_C$. Then by transitivity, $ \tau_B <:  \tau$ and $ \varepsilon_B \cup \varepsilon' \cup \varepsilon'' = \varepsilon_B \cup \varepsilon \subseteq \varepsilon_A$.
\end{proof}


\section{$\epscalc$ Proofs}


\begin{lemma}[$\epscalc$ Canonical Forms]
Unless the rule used is \textsc{$\varepsilon$-Subsume}, the following are true:
\begin{enumerate}
	\item If $\hat \Gamma \vdash x: \hat \tau~\kw{with} \varepsilon$ then $\varepsilon = \varnothing$.
	\item If $\hat \Gamma \vdash \hat v: \hat \tau~\kw{with} \varepsilon$ then $\varepsilon = \varnothing$.
	\item If $\hat \Gamma \vdash \hat v: \{ \bar r \}~\kw{with} \varepsilon$ then $\hat v = r$ and $\{ \bar r \} = \{ r \}$.
	\item If $\hat \Gamma \vdash \hat v: \hat \tau_1 \rightarrow_{\varepsilon'} \hat \tau_2~\kw{with} \varepsilon$ then $\hat v = \lambda x:\tau. \hat e$.
\end{enumerate}
\end{lemma}


\begin{proof}
Same as for $\opercalc$.
\end{proof}


\hrulefill


\begin{theorem}[$\epscalc$ Progress]
If $\hat \Gamma \vdash \hat e: \hat \tau~\kw{with} \varepsilon$ and $\hat e$ is not a value, then $\hat e \longrightarrow \hat e'~|~\varepsilon$, for some $\hat e', \varepsilon$.
\end{theorem}


\begin{proof} By induction on the derivation of $\hat \Gamma \vdash \hat e: \hat \tau~\kw{with} \varepsilon$.\\

\textit{Case}: \textsc{$\varepsilon$-Module}. Then $\hat e = \import{\varepsilon_{s}}{x}{\hat e_{i}}{e}$. If $\hat e_i$ is a non-value then $\hat e_i \longrightarrow \hat e_i'~|~\varepsilon$ by inductive assumption and $\import{\varepsilon_{s}}{x}{\hat e_i}{e} \longrightarrow \import{\varepsilon_{s}}{x}{\hat e_i'}{e}~|~\varepsilon$ by \textsc{E-Module1}. Otherwise $\hat e_i = \hat v_i$ is a value and $\import{\varepsilon_{s}}{x}{\hat v_i}{e} \longrightarrow [\hat v_i/x]\kwa{annot}(e, \varepsilon_{s})~|~\varnothing$ by \textsc{E-Module2}.
\end{proof}


\hrulefill


\begin{lemma}[$\epscalc$ Substitution]
If $\hat \Gamma, x: \hat \tau' \vdash \hat e: \hat \tau~\kw{with} \varepsilon$ and $\hat \Gamma \vdash \hat v: \hat \tau'~\kw{with} \varnothing$ then $\hat \Gamma \vdash [\hat v/x]\hat e_A: \hat \tau~\kw{with} \varepsilon$.
\end{lemma}


\begin{proof} By induction on the derivation of $\hat \Gamma, x: \hat \tau' \vdash \hat e: \hat \tau~\kw{with} \varepsilon$. \\

\textit{Case:} \textsc{$\varepsilon$-Module}. Then the following are true.

\begin{enumerate}
	\item $\hat e = \import{\varepsilon_{s}}{x}{\hat e_i}{e}$
	\item $\hat \Gamma, y: \hat \tau' \vdash \hat e_i: \hat \tau_i~\kw{with} \varepsilon_i$
	\item $y: \erase{\hat \tau_i} \vdash e: \tau$
	\item $\hat \Gamma, y: \hat \tau' \vdash \import{\varepsilon_s}{x}{\hat e_i}{e} : \kwa{annot}(\tau, \varepsilon_s)~\kw{with} \varepsilon_s \cup \varepsilon_i$
	\item $\varepsilon_s = \fx{\hat \tau_i} \cup \hofx{\annot{\tau}{\varnothing}}$
	\item $\hat \tau_A = \annot{\tau}{\varepsilon}$
	\item $\hat \varepsilon_A = \varepsilon_s \cup \varepsilon_i$
\end{enumerate}

By applying inductive assumption to (2) $\hat \Gamma \vdash [\hat v/x]\hat e_i: \hat \tau_i~\kw{with} \varepsilon_i$.
 Then by \textsc{$\varepsilon$-Module} $\hat \Gamma \vdash \import{\varepsilon_s}{y}{[\hat v/x]\hat e_i}{e}: \kwa{annot}(\tau_i, \varepsilon_s)~\kw{with} \varepsilon_s \cup \varepsilon_i$. By definition of $\kwa{substitution}$, the form in this judgement is the same as $[\hat v/x]\hat e$.
\end{proof}


\hrulefill


\begin{lemma}[$\epscalc$ Approximation 1]
If $\kwa{effects}(\hat \tau) \subseteq \varepsilon$ and $\kwa{ho \hyphen safe}(\hat \tau, \varepsilon)$ then $\hat \tau <: \kwa{annot}(\kwa{erase}(\hat \tau), \varepsilon)$.
\end{lemma}

\begin{lemma}[$\epscalc$ Approximation 2]
If $\kwa{ho \hyphen effects}(\hat \tau) \subseteq \varepsilon$ and $\safe{\hat \tau}{\varepsilon}$ then $\kwa{annot(erase}(\hat \tau), \varepsilon) <: \hat \tau$.
\end{lemma}


\begin{proof}
By simultaneous induction on derivations of $\kwa{safe}$ and $\kwa{ho \hyphen safe}$.\\

\textit{Case:} $\hat \tau = \{ \bar r \}$ Then $\hat \tau = \kwa{annot(erase}(\hat \tau), \varepsilon)$ and the results for both lemmas hold immediately. \\

\textit{Case: $\hat \tau = \hat \tau_1 \rightarrow_{\varepsilon'} \hat \tau_2$, $\fx{\hat \tau} \subseteq \varepsilon$, $\hosafe{\hat \tau}{\varepsilon}$} It is sufficient to show $\hat \tau_2 <: \kwa{annot(erase}(\hat \tau_2), \varepsilon)$ and $\kwa{annot(erase}(\hat \tau_1), \varepsilon) <: \hat \tau_1$, because the result will hold by \textsc{S-Effects}. To achieve this we shall inductively apply lemma 1 to $\hat \tau_2$ and lemma 2 to $\hat \tau_1$. 

From $\fx{\hat \tau} \subseteq \varepsilon$ we have $\hofx{\hat \tau_1} \cup \varepsilon' \cup \fx{\hat \tau_2} \subseteq \varepsilon$ and therefore $\fx{\hat \tau_2} \subseteq \varepsilon$. From $\hosafe{\hat \tau}{\varepsilon}$ we have $\hosafe{\hat \tau_2}{\varepsilon}$. Therefore we can apply lemma 1 to $\hat \tau_2$.

From $\fx{\hat \tau} \subseteq \varepsilon$ we have $\hofx{\hat \tau_1} \cup \varepsilon' \cup \fx{\hat \tau_2} \subseteq \varepsilon$ and therefore $\hofx{\hat \tau_1} \subseteq \varepsilon$. From $\hosafe{\hat \tau}{\varepsilon}$ we have $\hosafe{\hat \tau_1}{\varepsilon}$. Therefore we can apply lemma 2 to $\hat \tau_1$.\\

\textit{Case: $\hat \tau = \hat \tau_1 \rightarrow_{\varepsilon'} \hat \tau_2$, $\hofx{\hat \tau} \subseteq \varepsilon$, $\safe{\hat \tau}{\varepsilon}$ } It is sufficient to show $\kwa{annot(erase}(\hat \tau_2), \varepsilon) <: \hat \tau_2$ and $\hat \tau_1 <: \kwa{annot(erase}(\hat \tau_1), \varepsilon)$, because the result will hold by \textsc{S-Effects}. To achieve this we shall inductively apply lemma 2 to $\hat \tau_2$ and lemma 1 to $\hat \tau_1$.

From $\hofx{\hat \tau} \subseteq \varepsilon$ we have $\fx{\hat \tau_1} \cup \hofx{\hat \tau_2} \subseteq \varepsilon$ and therefore $\hofx{\hat \tau_2} \subseteq \varepsilon$. From $\safe{\hat \tau}{\varepsilon}$ we have $\safe{\hat \tau_2}{\varepsilon}$. Therefore we can apply lemma 2 to $\hat \tau_2$.

From $\hofx{\hat \tau} \subseteq \varepsilon$ we have $\fx{\hat \tau_1} \cup \hofx{\hat \tau_2} \subseteq \varepsilon$ and therefore $\fx{\hat \tau_1} \subseteq \varepsilon$. From $\safe{\hat \tau}{\varepsilon}$ we have $\hosafe{\hat \tau_1}{\varepsilon}$. Therefore we can apply lemma 1 to $\hat \tau_1$.

\end{proof}


\hrulefill


\begin{lemma}[$\epscalc$ Annotation]
If the following are true:

\begin{enumerate}
	\item $\hat \Gamma \vdash \hat v_i : \hat \tau_i~\kw{with} \varnothing$
	\item $\Gamma, y: \kwa{erase}(\hat \tau_i) \vdash e: \tau$
	\item $\kwa{effects}(\hat \tau_i) \cup \hofx{\annot{\tau}{\varnothing}} \cup \fx{\annot{\Gamma}{\varnothing}} \subseteq \varepsilon_{s}$
	\item $\kwa{ho \hyphen safe}(\hat \tau_i, \varepsilon_s)$
\end{enumerate}

Then $\hat \Gamma, \kwa{annot}(\Gamma, \varepsilon_s), y: \hat \tau_i \vdash \kwa{annot}(e, \varepsilon_s) : \kwa{annot}(\tau, \varepsilon_s)~\kw{with} \varepsilon_s$.
\end{lemma}


\begin{proof}
By induction on the derivation of $\Gamma, y: \kwa{erase}(\hat \tau_i) \vdash e: \tau$. When applying the inductive assumption, $e$, $\tau$, and $\Gamma$ may vary, but the other variables are fixed. \\

\textit{Case: \textsc{T-Var}}. Then $e=x$ and $\Gamma, y: \kwa{erase}(\hat \tau_i) \vdash x: \tau$. Either $x=y$ or $x \neq y$. \\

\textbf{Subcase 1: $x = y$}. Then $y: \erase{\hat \tau_i} \vdash y: \tau$ so $\tau = \erase{\hat \tau_i}$. By \textsc{$\varepsilon$-Var}, $y: \hat \tau_i \vdash x: \hat \tau_i~\kw{with} \varnothing$. By definition $\annot{x}{\varepsilon_s} = x$, so (5) $y: \hat \tau_i \vdash \annot{x}{\varepsilon_s}: \hat \tau_i~\kw{with} \varnothing$. By (3) and (4) we know $\fx{\hat \tau_i} \subseteq \varepsilon_s$ and $\hosafe{\hat \tau_i}{\varepsilon_s}$. By the approximation lemma, $\hat \tau_i <: \annot{\erase{\hat \tau_i}}{\varepsilon_s}$. We know $\erase{\hat \tau_i} = \tau$, so this judgement can be rewritten as $\hat \tau_i <: \annot{\tau}{\varepsilon_s}$. From this we can use \textsc{$\varepsilon$-Subsume} to narrow the type of (5) and widen the approximate effects of (5) from $\varnothing$ to $\varepsilon_s$, giving $y: \hat \tau_i \vdash \annot{x}{\varepsilon_s}: \annot{\tau}{\varepsilon_s}~\kw{with} \varepsilon_s$. Finally, by widening the context, $\hat \Gamma, \annot{\Gamma}{\varepsilon_s}, \hat \tau_i \vdash \annot{x}{\varepsilon_s}: \annot{\tau}{\varepsilon_s}~\kw{with} \varepsilon_s$.\\

\textbf{Subcase 2: $x \neq y$}. Because $\Gamma, y: \erase{\hat \tau_i} \vdash x: \tau$ and $x \neq y$ then $x: \tau \in \Gamma$. Then $x: \annot{\tau}{\varepsilon_s} \in \annot{\Gamma}{\varepsilon_s}$ so $\annot{\Gamma}{\varepsilon_s} \vdash x: \annot{\tau}{\varepsilon_s}~\kw{with} \varnothing$ by \textsc{$\varepsilon$-Var}. By definition $\annot{x}{\varepsilon_s} = x$, so $\annot{\Gamma}{\varepsilon_s} \vdash \annot{x}{\varepsilon_s}: \annot{\tau}{\varepsilon_s}~\kw{with} \varnothing$. Applying \textsc{$\varepsilon$-Subsume} gives $\annot{\Gamma}{\varepsilon_s} \vdash \annot{x}{\varepsilon_s}: \annot{\tau}{\varepsilon_s}~\kw{with} \varepsilon_s$. By widening the context $\hat \Gamma, \annot{\Gamma}{\varepsilon_s}, y: \hat \tau_i \vdash \annot{\tau}{\varepsilon_s}~\kw{with} \varepsilon'$.\\

\textit{Case: \textsc{T-Resource}}. Then $\Gamma, y: \kwa{erase}(\hat \tau_i) \vdash r : \{ r \}$. By \textsc{$\varepsilon$-Resource}, $\hat \Gamma, \kwa{annot}(\Gamma, \varepsilon), y: \hat \tau_i \vdash r: \{ r \}~\kw{with} \varnothing$. Applying definitions, $\kwa{annot}(r, \varepsilon) = r$ and $\annot{\{ r \}}{\varepsilon_s} = \{ r \}$, so this judgement can be rewritten as $\hat \Gamma, \kwa{annot}(\Gamma, \varepsilon), y: \hat \tau_i \vdash \annot{e}{\varepsilon_s}: \annot{\tau}{\varepsilon_s}~\kw{with} \varnothing$. By \textsc{$\varepsilon$-Subsume}, $\hat \Gamma, \kwa{annot}(\Gamma, \varepsilon_s), y: \hat \tau_i \vdash \annot{e}{\varepsilon_s}: \annot{\tau}{\varepsilon_s}~\kw{with} \varepsilon_s$.\\

\textit{Case: \textsc{T-Abs}}. Then $\Gamma, y: \erase{\hat \tau_i} \vdash \lambda x: \tau_2.e_{body}: \tau_2 \rightarrow \tau_3$. Applying definitions, (5) $\kwa{annot}(e, \varepsilon_s) = \kwa{annot}(\lambda x: \tau_2.e_{body}, \varepsilon_s) = \lambda x: \annot{\tau_2}{\varepsilon_s}.\annot{e_{body}}{\varepsilon_s}$ and $\annot{\tau}{\varepsilon_s} = \annot{\tau_2 \rightarrow \tau_3}{\varepsilon_s} = \kwa{annot}(\tau_2, \varepsilon_s) \rightarrow_{\varepsilon_s} \kwa{annot}(\tau_3, \varepsilon_s)$. By inversion on \textsc{T-Abs}, we get the sub-derivation (6) $\Gamma, y: \erase{\hat \tau_i}, x: \tau_2 \vdash e_{body}: \tau_2$. We shall apply the inductive assumption to this judgement with an unannotated context consisting of $\Gamma, x: \tau_2$. To be a valid application of the lemma, it is required that $\fx{\annot{\Gamma, x: \tau_2}{\varnothing} \subseteq \varepsilon_s$. We already know $\fx{\annot{\Gamma}{\varnothing}} \subseteq \varepsilon_s$ by assumption (3). Also by assumption (3), $\hofx{\annot{\tau_2 \rightarrow \tau_3}{\varnothing}} \subseteq \varepsilon_s$; then by definition of $\kwa{ho \hyphen effects}$, $\fx{\annot{\tau_2}{\varnothing}} \subseteq \hofx{\annot{\tau_2 \rightarrow \tau_3}{\varnothing}}$, so $\fx{\annot{x: \tau_2}}{\varepsilon_s}} \subseteq \varepsilon_s$ by transitivity. Then by applying the inductive assumption to (6), $\hat \Gamma, \annot{\Gamma}{\varepsilon_s}, \annot{x: \tau_2}{\varepsilon_s}, y: \hat \tau_i \vdash \annot{e_{body}}{\varepsilon_s}: \annot{\tau_3}{\varepsilon_s}~\kw{with} \varepsilon_s$. By \textsc{$\varepsilon$-Abs}, $\hat \Gamma, \annot{\Gamma}{\varepsilon_s}, y: \hat \tau_i \vdash \lambda x: \annot{\hat \tau_2}{\varepsilon_s}. \annot{e_{body}}{\varepsilon_s} : \annot{\hat \tau_2}{\varepsilon_s} \rightarrow_{\varepsilon_s} \annot{\hat \tau_3}{\varepsilon_s}~\kw{with} \varnothing$. By applying the identities from (5), this judgement can be rewritten as $\hat \Gamma, \annot{\Gamma}{\varepsilon_s}, y: \hat \tau_i \vdash \annot{e}{\varepsilon_s} : \annot{\tau}{\varepsilon_s} ~\kw{with} \varnothing$. Finally, by applying \textsc{$\varepsilon$-Subsume}, $\hat \Gamma, \annot{\Gamma}{\varepsilon_s}, y: \hat \tau_i \vdash \annot{e}{\varepsilon_s} : \annot{\tau}{\varepsilon_s} ~\kw{with} \varepsilon_s$. \\

\textit{Case: \textsc{T-App}}. Then $\Gamma, y: \kwa{erase}(\hat \tau_i) \vdash e_1~e_2: \tau_3$ and by inversion $\Gamma, y:\kwa{erase}(\hat \tau_i) \vdash e_1: \tau_2 \rightarrow \tau_3$ and $\Gamma, y: \kwa{erase}(\hat \tau_i) \vdash e_2: \tau_2$. By applying the inductive assumption to these judgements, $\hat \Gamma, \kwa{annot}(\Gamma, \varepsilon_s), y: \hat \tau_i \vdash \annot{e_1}{\varepsilon_2}: \annot{\tau_2}{\varepsilon_s} \rightarrow_{\varepsilon_s} \annot{\tau_3}{\varepsilon_s}~\kw{with} \varepsilon_s$ and $\hat \Gamma, \kwa{annot}(\Gamma, \varepsilon_s), y: \hat \tau \vdash \kwa{annot}(e_2, \varepsilon_s): \kwa{annot}(\tau_2, \varepsilon_s)~\kw{with} \varepsilon_s$. Then by \textsc{$\varepsilon$-App}, we get $\hat \Gamma, \kwa{annot}(\Gamma, \varepsilon_s), y: \hat \tau \vdash \annot{e_1}{\varepsilon_s}~\annot{e_2}{\varepsilon_s} :  \kwa{annot}(\tau_3, \varepsilon)~\kw{with} \varepsilon$. Unfolding the definition of  $\kwa{annot}$ , this judgement can be rewritten as $\hat \Gamma, \kwa{annot}(\Gamma, \varepsilon_s), y: \hat \tau \vdash \annot{e_1~e_2}{\varepsilon_s} :  \kwa{annot}(\tau_3, \varepsilon)~\kw{with} \varepsilon$. Finally, because $e = e_1~e_2$ and $\tau = \tau_3$, this is the same as $\hat \Gamma, \kwa{annot}(\Gamma, \varepsilon_s), y: \hat \tau \vdash \annot{e}{\varepsilon_s} :  \kwa{annot}(\tau, \varepsilon)~\kw{with} \varepsilon$.
\\

\noindent
\textit{Case: \textsc{T-OperCall}}. Then $\Gamma, y: \kwa{erase}(\hat \tau_i) \vdash e_1.\pi : \Unit$. By inversion we get the sub-derivation $\Gamma, y: \kwa{erase}(\hat \tau_i) \vdash e_1: \{ \bar r \}$. Applying the inductive assumption, $\hat \Gamma, \kwa{annot}(\Gamma, \varepsilon), y: \hat \tau_i \vdash \annot{e_1}{\varepsilon_s}: \annot{\{ \bar r \}}{\varepsilon_s}~\kw{with} \varepsilon_s$. By definition, $\annot{\{ \bar r \}}{\varepsilon_s} = \{ \bar r \}$, so this judgement can be rewritten as $\hat \Gamma, \kwa{annot}(\Gamma, \varnothing), y: \hat \tau_i \vdash e_1: \{ \bar r \}~\kw{with} \varepsilon_s$. By \textsc{$\varepsilon$-OperCall}, $\hat \Gamma, \kwa{annot}(\Gamma, \varnothing), y: \hat \tau \vdash \annot{e_1.\pi}{\varepsilon_s}: \{ \bar r \} ~\kw{with} \varepsilon_s \cup \{ \bar r.\pi \}$. All that remains is to show $\{ \bar r.\pi \} \subseteq \varepsilon$. We shall do this by considering which subcontext left of the turnstile is capturing $\{ \bar r \}$. Technically, $\hat \Gamma$ may not have a binding for every $r \in \bar r$: the judgement for $e_1$ might be derived using \textsc{S-Resources} and \textsc{$\varepsilon$-Subsume}. However, at least one binding for some $r \in \bar r$ must be present in $\hat \Gamma$ to get the original typing judgement being subsumed, so we shall assume without loss of generality that $\hat \Gamma$ contains a binding for every $r \in \bar r$. \\

\textbf{Subcase 1:} $\{ \bar r \} = \hat \tau$. By assumption (3), $\fx{\hat \tau} \subseteq \varepsilon_s$, so $\bar r.\pi \subseteq \{ r.\pi \mid r \in \bar r, \pi \in \Pi \} = \fx{\{\bar r\}} \subseteq \varepsilon_s$. \\

\textbf{Subcase 2:} $r: \{ \bar r \} \in \annot{\Gamma}{\varepsilon_s}$. Then $\bar r.\pi \in \fx{\{ \bar r \}} \subseteq \fx{\annot{\Gamma}{\varnothing}}$, and by assumption (3) $\fx{\annot{\Gamma}{\varnothing}} \subseteq \varepsilon_s$, so $\bar r.\pi \in \varepsilon_s$.\\


\textbf{Subcase 3:} $r: \{ \bar r \} \in \hat \Gamma$. Because $\Gamma, y: \erase{\hat \tau} \vdash e_1: \{ \bar r \}$, then $\bar r \in \Gamma$ or $r = \tau$. If $r \in \annot{\Gamma}{\varnothing}$ then subcase 2 holds. Else $r = \erase{\hat \tau}$. Because $\hat \tau = \{ \bar r \}$, then $\erase{\{ \bar r \}} = \{ \bar r \}$, so $\hat \tau = \tau$; therefore subcase 1 holds.
\end{proof}


\hrulefill


\begin{theorem}[$\epscalc$ Preservation]
If $\hat \Gamma \vdash \hat e_A: \hat \tau_A~\kw{with} \varepsilon_A$ and $\hat e_A \longrightarrow \hat e_B~|~\varepsilon$, then $\hat \Gamma \vdash \hat e_B: \hat \tau_B~\kw{with} \varepsilon_B$, where $\hat e_B <: \hat e_A$ and $\varepsilon \cup \varepsilon_B \subseteq \varepsilon_A$, for some $\hat e_B, \varepsilon, \hat \tau_B, \varepsilon_B$.
\end{theorem}

\begin{proof}
By induction on the derivation of $\hat \Gamma \vdash \hat e_A: \hat \tau_A~\kw{with} \varepsilon_A$ and then the derivation of $\hat e_A \longrightarrow \hat e_B~|~\varepsilon$. \\

\textit{Case:} \textsc{$\varepsilon$-Import}. Then by inversion on the rules used, the following are true:

\begin{enumerate}
	\item $\hat e_A = \kwa{import}(\varepsilon_s)~x = \hat v_i~\kw{in} e$
	\item $x: \kwa{erase}(\hat \tau_i) \vdash e: \tau$
	\item $\hat \Gamma \vdash \hat e_i: \hat \tau_i~\kw{with} \varepsilon_1$
	\item $\hat \Gamma \vdash \hat e_A: \kwa{annot}(\tau, \varepsilon_s)~\kw{with} \varepsilon_s \cup \varepsilon_1$
	\item $\kwa{effects}(\hat \tau_i) \cup \hofx{\annot{\tau}{\varnothing}} \subseteq \varepsilon_s$
	\item $\kwa{ho \hyphen safe}(\hat \tau_i, \varepsilon_s)$
\end{enumerate}

\textbf{Subcase 1:} \textsc{E-Import1}. Then $\import{\varepsilon_s}{x}{\hat e_i}{e} \longrightarrow \import{\varepsilon_s}{x}{\hat e_i'}{e}~|~\varepsilon$ and by inversion, $\hat e_i \longrightarrow \hat e_i'~|~\varepsilon$. By inductive assumption and subsumption, $\hat \Gamma \vdash \hat e_i': \hat \tau_i'~\kw{with} \varepsilon_1$. Then by \textsc{$\varepsilon$-Import}, $\hat \Gamma \vdash \import{\varepsilon_s}{x}{\hat e_i'}{e}: \annot{\tau}{\varepsilon_s}~\kw{with} \varepsilon_s$. \\

\textbf{Subcase 2:} \textsc{E-Import2}. Then $\hat e_i = \hat v_i$ is a value and $\varepsilon_1 = \varnothing$ by canonical forms. Apply the annotation lemma with $\Gamma = \varnothing$ to get $\hat \Gamma, x: \hat \tau_i \vdash \kwa{annot}(e, \varepsilon_s): \kwa{annot}(\tau, \varepsilon_s)~\kw{with} \varepsilon_s$. From assumption (4) and canonical forms we have $\hat \Gamma \vdash \hat v: \hat \tau_i~\kw{with} \varnothing$. Applying the substitution lemma, $\hat \Gamma \vdash [\hat v_i/x]\kwa{annot}(e, \varepsilon): \kwa{annot}(\tau, \varepsilon_s)~\kw{with} \varepsilon_s$. Then $\varepsilon \cup \varepsilon_B = \varepsilon_A = \varepsilon_s$ and $\tau_A = \tau_B = \annot{\tau}{\varepsilon_s}$.

\end{proof}


\hrulefill


\begin{theorem}[$\epscalc$ Single-step Soundness]
If $\hat \Gamma \vdash \hat e_A: \hat \tau_A~\kw{with} \varepsilon_A$ and $\hat e_A$ is not a value, then $\hat e_A \longrightarrow \hat e_B~|~\varepsilon$, where $\hat \Gamma \vdash \hat e_B: \hat \tau_B~\kw{with} \varepsilon_B$ and $\hat \tau_B <: \hat \tau_A$ and $\varepsilon_B \cup \varepsilon \subseteq \varepsilon_A$, for some $\hat e_B$, $\varepsilon$, $\hat \tau_B$, and $\varepsilon_B$.
\end{theorem}


\begin{theorem}[$\epscalc$ Multi-step Soundness]
If $\hat \Gamma \vdash \hat e_A: \hat \tau_A~\kw{with} \varepsilon_A$ and $\hat e_A \longrightarrow^{*} e_B~|~\varepsilon$, then $\hat \Gamma \vdash \hat e_B: \hat \tau_B~\kw{with} \varepsilon_B$, where $\hat \tau_B <: \hat \tau_A$ and $\varepsilon_B \cup \varepsilon \subseteq \varepsilon_A$, for some $\hat \tau_B$, $\varepsilon_B$.
\end{theorem}

\begin{proof}
The same as for $\opercalc$.
\end{proof}


\end{document}

\end{document}