\documentclass{llncs}

\usepackage{listings}
\usepackage{proof}
\usepackage{amssymb}
\usepackage[margin=.9in]{geometry}
\usepackage{amsmath}
\usepackage[english]{babel}
\usepackage[utf8]{inputenc}
\usepackage{enumitem}
\usepackage{filecontents}
\usepackage{calc}
\usepackage[linewidth=0.5pt]{mdframed}
\usepackage{changepage}
\usepackage{tabto}
\allowdisplaybreaks

\usepackage{fancyhdr}
\renewcommand{\headrulewidth}{0pt}
\pagestyle{fancy}
 \fancyhf{}
\rhead{\thepage}

\lstset{tabsize=3, basicstyle=\ttfamily\small, commentstyle=\itshape\rmfamily, numbers=left, numberstyle=\tiny, language=java,moredelim=[il][\sffamily]{?},mathescape=true,showspaces=false,showstringspaces=false,columns=fullflexible,xleftmargin=5pt,escapeinside={(@}{@)}, morekeywords=[1]{objtype,module,import,let,in,fn,var,type,rec,fold,unfold,letrec,alloc,ref,application,policy,external,component,connects,to,meth,val,where,return,group,by,within,count,connect,with,attr,html,head,title,style,body,div,keyword,unit,def}}
\lstloadlanguages{Java,VBScript,XML,HTML}

\newcommand{\keywadj}[1]{\mathtt{#1}}
\newcommand{\keyw}[1]{\keywadj{#1}~}

\newcommand{\kw}[1]{\keyw{ #1 }}
\newcommand{\kwa}[1]{\keywadj{ #1 }}
\newcommand{\reftt}{\mathtt{ref}~}
\newcommand{\Reftt}{\mathtt{Ref}~}
\newcommand{\inttt}{\mathtt{int}~}
\newcommand{\Inttt}{\mathtt{Int}~}
\newcommand{\stepsto}{\leadsto}
\newcommand{\todo}[1]{\textbf{[#1]}}
\newcommand{\intuition}[1]{#1}
\newcommand{\hyphen}{\hbox{-}}

%\newcommand{\intuition}[1]{}

\newlist{pcases}{enumerate}{1}
\setlist[pcases]{
  label=\fbox{\textit{Case}}\protect\thiscase\textit{:}~,
  ref=\arabic*,
  align=left,
  labelsep=0pt,
  leftmargin=0pt,
  labelwidth=0pt,
  parsep=0pt
}
\newcommand{\pcase}[1][]{

  \if\relax\detokenize{#1}\relax
    \def\thiscase{}
  \else
    \def\thiscase{~\fbox{#1:}}
  \fi
  \item
}

\newcommand{\thm}[3]{
	\begin{large}
		\bf{#1}
	\end{large} \\\\
	\fbox{Statement.} ~ #2
	\fbox{Proof.}~ #3 \qed
}

\newcommand{\proofcase}[2]{
	\begin{adjustwidth}{1.5em}{0pt}
		\fbox{Case.}~~#1. \\ ~#2
	\end{adjustwidth}
}

\newcommand{\subcase}[1] {
	\begin{adjustwidth}{2.2em}{0pt}
		\underline{Subcase.} #1
	\end{adjustwidth}
}

\newcommand{\stmt}[1] {

\begin{adjustwidth}{2.5em}{0pt}
	#1
\end{adjustwidth}

}
\newcommand{\type}[2]{
	#1~\keyw{with} #2
}

\newcommand{\newd}[0]{
	\keywadj{new}_d~x \Rightarrow \overline{d = e}
}

\newcommand{\newsig}[0]{
	\keywadj{new}_\sigma~x \Rightarrow \overline{\sigma = e}
}


\begin{document}

\section{Conservative Example}

\subsection{Sugared}

In this example, calls to a particular class of functions are first routed through a logging function $\kwa{loggerProxy}$ which records their use. For simplicity we just assume that the user wants to apply $0$ to every function (practically you'd want to have $\kwa{loggerProxy}$ take an extra argument for the input to $\kwa{f}$, but this complicates desugaring and isn't essential to the point being made).

Note that although $\kwa{id}$ does not have a capability for $\kwa{File.log}$, and $\kwa{loggerProxy}$ does, the implementation of $\kwa{loggerProxy}$ is being sensible (so it is type-and-effect safe).

\begin{lstlisting}
def id(a: Int): Int with $\varnothing$ =
    a

def loggerProxy(f: Int $\rightarrow$ Int with $\varnothing$): Unit with File.log =
    File.log(``Log: called function '' + f)
    id(0)
    
def main(): Unit =
	proxy(id)

\end{lstlisting}

\subsection{Desugared}

A multi-variable function is desugared into an object with a method which returns an object with a method.

\begin{lstlisting}
let $x_1$ = new$_\sigma$x $\Rightarrow$ {
    def id(a: Int): Int with $\varnothing$ =
       a
} in

let $x_2$ = new$_\sigma$x $\Rightarrow$ {
    def fix(obj: {id: Int $\rightarrow$ Int with $\varnothing$}): Unit with Sock.read =
        new$_\sigma$x $\Rightarrow$ {
            def m(): Unit with $\varnothing$ =
               obj.id(0)
        }.m(File.log)
} in

let $x_3$ = new$_d$x $\Rightarrow$ {
    def main(): Unit =
       $x_2$.fix($x_1$)
} in

$x_3$.main()

\end{lstlisting}

\subsection{Typing}

To type $x_3$ using \textsc{C-NewObj} you need a $\Gamma'$ containing $x_2$ and $x_1$. $\kwa{capture}(x_2) = File.log$ and $\kwa{capture}(x_1) = \varnothing$. So $\varepsilon_c = \kwa{capture}(\Gamma') = \kwa{File.log}$. To meet the premises of \textsc{C-NewObj} we need $\kwa{capture}(\tau) \supset File.log$, for every $\tau$ which is the argument of a higher-order function in scope. \\

\noindent
The only higher-order function in scope is $\kwa{x_2.fix}$. Its formal parameter has the type $\kwa{id:~Int \rightarrow Int~with~\varnothing}$. The capture of this type is $\varnothing$ and $\varnothing \not\supset \kwa{File.log}$, so this program is rejected.

\section{$\kwa{arg \hyphen types}$}

\subsection{$\keywadj{arg \hyphen types}$ Function}

This function examines the declaration of every method which could be (directly) invoked inside a particular $\Gamma$. It returns a set of the types of the arguments of those methods.

\begin{itemize}
	\item $\kwa{arg \hyphen types}(\varnothing) = \varnothing$
	\item $\kwa{arg \hyphen types}(\Gamma, x : \tau) = \kwa{arg \hyphen types}(\Gamma) \cup \kwa{arg \hyphen types}(\tau)$
	\item $\kwa{arg \hyphen types}(\{r\}) = \varnothing$
	\item $\kwa{arg \hyphen types}(\{\bar \sigma\}) = \bigcup_{\sigma \in \bar \sigma}~\kwa{arg \hyphen types}(\sigma)$
	\item $\kwa{arg \hyphen types}(\{\bar d\}) = \bigcup_{d \in \bar d}~\kwa{arg \hyphen types}(d)$
	\item $\kwa{arg \hyphen types}(\{\bar d ~\keyw{captures} \varepsilon_c\}) =
 \kwa{arg \hyphen types}(\{ \bar d \})$
 	\item $\kwa{arg \hyphen types}(d ~\kw{with} \varepsilon) = \kwa{arg \hyphen types}(d)$
 	\item $\kwa{arg \hyphen types}(\kw{def} m(y: \tau_2): \tau_3) = \{ \tau_2 \} \cup \kwa{arg \hyphen types}(\tau_3) \cup \kwa{arg \hyphen types}(\tau_2)$ (is $\kwa{arg \hyphen types}(\tau_2)$ necessary?)
\end{itemize}

\subsection{$\keywadj{higher \hyphen order \hyphen args}$ Function}

\[
\begin{array}{c}

\infer[\textsc{(HigherOrderArgs)}]
	{\tau \in \kwa{higher \hyphen order \hyphen args}(\Gamma)}
	{\tau \in \kwa{arg \hyphen types}(\Gamma) ~~~~ \kwa{is \hyphen higher \hyphen order}(\tau)}

\end{array}
\]

\subsection{$\keywadj{is \hyphen obj}$ Predicate}

\noindent
The $\kwa{is \hyphen obj}$ predicate says whether or not a particular type $\tau$ is an object.

\[
\begin{array}{c}

\infer[\textsc{(IsObj$_d$)}]
	{\kwa{is \hyphen obj}(\{ \bar d \})}
	{}
	
	~~~~
	
	\infer[\textsc{(IsObj$_\sigma$)}]
	{\kwa{is \hyphen obj}(\{ \bar \sigma \})}
	{}
	
	~~~~
	
	\infer[\textsc{(IsObjSummary)}]
	{\kwa{is \hyphen obj}(\{ \bar d~\kw{captures} \varepsilon_c\})}
	{}

\end{array}
\]

\subsection{$\kwa{is \hyphen higher \hyphen order}$ Predicate}

A type is higher-order if it has a method accepting another object as an argument.


\[
\begin{array}{c}

\infer[\textsc{(HigherOrder$_d$)}]
	{\kwa{is \hyphen higher \hyphen order}(\{ \bar d \})}
	{d_i = \kwa{def}~m(y: \tau_2): \tau_3 ~~~\kwa{is \hyphen obj}(\tau_2)}
	\\[5ex]
	\infer[\textsc{(HigherOrder$_\sigma$)}]
	{\kwa{is \hyphen higher \hyphen order}(\{ \bar \sigma \})}
	{\sigma_i = \kwa{def}~m(y: \tau_2): \tau_3 ~\kw{with} \varepsilon ~~~~ \kwa{is \hyphen obj}(\tau_2)}

\end{array}
\]



\end{document}
